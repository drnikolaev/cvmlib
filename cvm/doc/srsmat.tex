\bigskip
\noindent
\verb"template <typename TR>"\\
\verb"class srsmatrix : public srmatrix <TR> {"\\
\verb"public:"\\
\verb"    "\GOT{srsmatrix}{srsmatrix.srsmatrix ()}\verb" ();"\\
\verb"    explicit "\GOT{srsmatrix}{srsmatrix.srsmatrix (int)}\verb" (int nMN);"\\
\verb"    "\GOT{srsmatrix}{srsmatrix.srsmatrix (TR*,int)}\verb" (TR* pD, int nMN, TR tol = cvmMachSp());"\\
\verb"    "\GOT{srsmatrix}{srsmatrix.srsmatrix (const TR*,int)}\verb" (const TR* pD, int nMN, TR tol = cvmMachSp());"\\
\verb"    "\GOT{srsmatrix}{srsmatrix.srsmatrix (const srsmatrix&)}\verb" (const srsmatrix& m);"\\
\verb"    explicit "\GOT{srsmatrix}{srsmatrix.srsmatrix (const rmatrix&)}\verb" (const rmatrix& m, TR tol = cvmMachSp());"\\
\verb"    explicit "\GOT{srsmatrix}{srsmatrix.srsmatrix (const rvector&)}\verb" (const rvector& v);"\\
\verb"    "\GOT{srsmatrix}{srsmatrix.submatrixctr}\verb" (srsmatrix& m, int nRowCol, int size);"\\
\verb"    TR "\GOT{operator ()}{srsmatrix.operator (,)}\verb" (int im, int in) const throw (cvmexception);"\\
\verb"    const rvector "\GOT{operator ()}{srsmatrix.operator ()}\verb" (int i) const throw (cvmexception);"\\
\verb"    const rvector "\GOT{operator []}{srsmatrix.operator []}\verb" (int i) const throw (cvmexception);"\\
\verb"    const rvector "\GOT{diag}{srsmatrix.diag}\verb" (int i) const throw (cvmexception);"\\
\verb"    srsmatrix& "\GOT{operator =}{srsmatrix.operator = (const srsmatrix&)}\verb" (const srsmatrix& m) throw (cvmexception);"\\
\verb"    srsmatrix& "\GOT{assign}{srsmatrix.assign}\verb" (const rvector& v, TR tol = cvmMachSp())"\\
\verb"                       throw (cvmexception);"\\
\verb"    srsmatrix& "\GOT{assign}{srsmatrix.assign}\verb" (const TR* pD, TR tol = cvmMachSp())"\\
\verb"                       throw (cvmexception);"\\
\verb"    srsmatrix& "\GOT{assign}{srsmatrix.assign (int, const srsmatrix&)}\verb" (int nRowCol, const srsmatrix& m)"\\
\verb"                       throw (cvmexception);"\\
\verb"    srsmatrix& "\GOT{set}{srsmatrix.set}\verb" (TR x);"\\
\verb"    srsmatrix& "\GOT{set}{srsmatrix.set(int,int,TR)}\verb" (int nRow, int nCol, TR x);"\\
\verb"    srsmatrix& "\GOT{set\_diag}{srsmatrix.setdiag}\verb" (int i, const rvector& v) throw (cvmexception);"\\
\verb"    srsmatrix& "\GOT{resize}{srsmatrix.resize}\verb" (int nNewMN) throw (cvmexception);"\\
\verb"    bool "\GOT{operator ==}{srsmatrix.operator ==}\verb" (const srsmatrix& m) const;"\\
\verb"    bool "\GOT{operator !=}{srsmatrix.operator !=}\verb" (const srsmatrix& m) const;"\\
\verb"    srsmatrix& "\GOT{operator <{}<}{srsmatrix.operator <<}\verb" (const srsmatrix& m) throw (cvmexception);"\\
\verb"    srsmatrix "\GOT{operator +}{srsmatrix.operator +}\verb" (const srsmatrix& m) const"\\
\verb"                          throw (cvmexception);"\\
\verb"    srsmatrix "\GOT{operator -}{srsmatrix.operator -}\verb" (const srsmatrix& m) const"\\
\verb"                          throw (cvmexception);"\\
\verb"    srsmatrix& "\GOT{sum}{srsmatrix.sum}\verb" (const srsmatrix& m1,"\\
\verb"                    const srsmatrix& m2) throw (cvmexception);"\\
\verb"    srsmatrix& "\GOT{diff}{srsmatrix.diff}\verb" (const srsmatrix& m1,"\\
\verb"                     const srsmatrix& m2) throw (cvmexception);"\\
\verb"    srsmatrix& "\GOT{operator +=}{srsmatrix.operator +=}\verb" (const srsmatrix& m) throw (cvmexception);"\\
\verb"    srsmatrix& "\GOT{operator -=}{srsmatrix.operator -=}\verb" (const srsmatrix& m) throw (cvmexception);"\\
\verb"    srsmatrix "\GOT{operator -}{srsmatrix.operator - ()}\verb" () const;"\\
\verb"    srsmatrix& "\GOT{operator ++}{srsmatrix.operator ++}\verb" ();"\\
\verb"    srsmatrix& "\GOT{operator ++}{srsmatrix.operator ++}\verb" (int);"\\
\verb"    srsmatrix& "\GOT{operator -{}-}{srsmatrix.operator --}\verb" ();"\\
\verb"    srsmatrix& "\GOT{operator -{}-}{srsmatrix.operator --}\verb" (int);"\\
\verb"    srsmatrix "\GOT{operator *}{srsmatrix.operator * (TR)}\verb" (TR d) const;"\\
\verb"    srsmatrix "\GOT{operator /}{srsmatrix.operator / (TR)}\verb" (TR d) const throw (cvmexception);"\\
\verb"    srsmatrix& "\GOT{operator *=}{srsmatrix.operator *= (TR)}\verb" (TR d);"\\
\verb"    srsmatrix& "\GOT{operator /=}{srsmatrix.operator /= (TR)}\verb" (TR d) throw (cvmexception);"\\
\verb"    srsmatrix& "\GOT{normalize}{srsmatrix.normalize}\verb" ();"\\
\verb"    srsmatrix "\GOT{operator \TildaT}{srsmatrix.transposition}\verb" () const throw (cvmexception);"\\
\verb"    srsmatrix& "\GOT{transpose}{srsmatrix.transposition}\verb" (const srsmatrix& m) throw (cvmexception);"\\
\verb"    srsmatrix& "\GOT{transpose}{srsmatrix.transposition}\verb" ();"\\
\verb"    rvector "\GOT{operator *}{srsmatrix.operator * (const rvector&)}\verb" (const rvector& v) const throw (cvmexception);"\\
\verb"    rmatrix "\GOT{operator *}{srsmatrix.operator * (const rmatrix&)}\verb" (const rmatrix& m) const throw (cvmexception);"\\
\verb"    srmatrix "\GOT{operator *}{srsmatrix.operator * (const srmatrix&)}\verb" (const srmatrix& m) const throw (cvmexception);"\\
\verb"    rvector "\GOT{operator /}{srsmatrix.operator / (rvector)}\verb" (const rvector& vB) const throw (cvmexception);"\\
\verb"    srsmatrix& "\GOT{syrk}{srsmatrix.syrk}\verb" (TR alpha,"\\
\verb"                     const rvector& v, TR beta) throw (cvmexception);"\\
\verb"    srsmatrix& "\GOT{syrk}{srsmatrix.syrk}\verb" (bool bTransp, TR alpha,"\\
\verb"                     const rmatrix& m, TR beta) throw (cvmexception);"\\
\verb"    srsmatrix& "\GOT{syr2k}{srsmatrix.syr2k}\verb" (TR alpha,"\\
\verb"                     const rvector& v1, const rvector& v2, TR beta)"\\
\verb"                     throw (cvmexception);"\\
\verb"    srsmatrix& "\GOT{syr2k}{srsmatrix.syr2k}\verb" (bool bTransp, TR alpha,"\\
\verb"                     const rmatrix& m1, const rmatrix& m2, TR beta)"\\
\verb"                     throw (cvmexception);"\\
\verb"    srsmatrix& "\GOT{inv}{srsmatrix.inv}\verb" (const srsmatrix& mArg) throw (cvmexception);"\\
\verb"    srsmatrix "\GOT{inv}{srsmatrix.inv}\verb" () const throw (cvmexception);"\\
\verb"    srsmatrix& "\GOT{exp}{srsmatrix.exp}\verb" (const srsmatrix& m,"\\
\verb"                   TR tol = cvmMachSp()) throw (cvmexception);"\\
\verb"    srsmatrix "\GOT{exp}{srsmatrix.exp}\verb" (TR tol = cvmMachSp()) const throw (cvmexception);"\\
\verb"    srsmatrix& "\GOT{polynom}{srsmatrix.polynom}\verb" (const srsmatrix& m, const rvector& v)"\\
\verb"                        throw (cvmexception);"\\
\verb"    srsmatrix "\GOT{polynom}{srsmatrix.polynom}\verb" (const rvector& v) const throw (cvmexception);"\\
\verb"    rvector "\GOT{eig}{srsmatrix.eig}\verb" (srmatrix& mEigVect) const throw (cvmexception);"\\
\verb"    rvector "\GOT{eig}{srsmatrix.eig}\verb" () const throw (cvmexception);"\\
\verb"    srmatrix "\GOT{cholesky}{srsmatrix.cholesky}\verb" () const throw (cvmexception);"\\
\verb"    srmatrix "\GOT{bunch\_kaufman}{srsmatrix.bunch_kaufman}\verb" () const throw (cvmexception);"\\
\verb"    srsmatrix& "\GOT{identity}{srsmatrix.identity}\verb" ();"\\
\verb"    srsmatrix& "\GOT{vanish}{srsmatrix.vanish}\verb" ();"\\
\verb"    srsmatrix& "\GOT{randomize}{srsmatrix.randomize}\verb" (TR dFrom, TR dTo);"\\
\verb"};"
\newpage



\subsubsection{srsmatrix ()}
Constructor%
\pdfdest name {srsmatrix.srsmatrix ()} fit
\begin{verbatim}
srsmatrix::srsmatrix ();
\end{verbatim}
creates  empty \verb"srsmatrix" object.
See also \GOT{srsmatrix}{srsmatrix}.
Example:
\begin{Verbatim}
using namespace cvm;

srsmatrix m;
std::cout << m.msize() << " " << m.nsize() << " " << m.size()
          << std::endl << std::endl;
m.resize (3);
m.set(1.);
std::cout << m;
\end{Verbatim}
prints
\begin{Verbatim}
0 0 0

1 1 1
1 1 1
1 1 1
\end{Verbatim}
\newpage




\subsubsection{srsmatrix (int)}
Constructor%
\pdfdest name {srsmatrix.srsmatrix (int)} fit
\begin{verbatim}
explicit srsmatrix::srsmatrix (int nMN);
\end{verbatim}
creates  $n\times n$ \verb"srsmatrix" object where $n$ is passed in
\verb"nMN" parameter. 
Constructor throws  \GOT{cvmexception}{cvmexception}
in case of non-positive size passed or memory allocation failure.
See also \GOT{srsmat\-rix}{srsmatrix}.
Example:
\begin{Verbatim}
using namespace cvm;

srsmatrix m(4);
std::cout << m.msize() << " " << m.nsize() << " " << m.size()
          << std::endl << std::endl;
m.set(1.);
std::cout << m;
\end{Verbatim}
prints
\begin{Verbatim}
4 4 16

1 1 1 1
1 1 1 1
1 1 1 1
1 1 1 1
\end{Verbatim}
\newpage



\subsubsection{srsmatrix (TR*,int)}
Constructor%
\pdfdest name {srsmatrix.srsmatrix (TR*,int)} fit
\begin{verbatim}
srsmatrix::srsmatrix (TR* pD, int nMN, TR tol = cvmMachSp());
\end{verbatim}
creates  $n\times n$ \verb"srsmatrix" object where $n$ is passed in
\verb"nMN" parameter.
Unlike others, this constructor \textit{does not allocate  memory}.
It just shares  memory with an array pointed to by \verb"pD".
Constructor throws  \GOT{cvmexception}{cvmexception}
if the matrix created doesn't appear to be symmetric 
or in case of non-positive size passed. Symmetry tolerance is
set by parameter \verb'tol'.
If subsequent application flow would change the array passed so
it becomes not symmetric matrix anymore then results are not predictable.
See also \GOT{srsmatrix}{srsmatrix},
\GOT{srsmatrix (const TR*,int)}{srsmatrix.srsmatrix (const TR*,int)}.
Example:
\begin{Verbatim}
using namespace cvm;
double a[] = {1., 1., 1., 1., 1., 1., 1., 1., 1.};
srsmatrix m (a, 3);
m.set(2,1,5.);
std::cout << m << std::endl;
std::cout << a[0] << " " << a[1] << " " << a[2] << " "
          << a[3] << " " << a[4] << " " << a[5] << " " << std::endl;
\end{Verbatim}
prints
\begin{Verbatim}
1 5 1
5 1 1
1 1 1

1 5 1 5 1 1
\end{Verbatim}
\newpage


\subsubsection{srsmatrix (const TR*,int)}
Constructor%
\pdfdest name {srsmatrix.srsmatrix (const TR*,int)} fit
\begin{verbatim}
srsmatrix::srsmatrix (const TR* pD, int nMN, TR tol = cvmMachSp());
\end{verbatim}
creates  $n\times n$ \verb"srsmatrix" object where $n$ is passed in
\verb"nMN" parameter
and copies $n*n$ elements of an array  \verb"pD" to it \GOT{by colums}{SubSubSectionStorage}.
Constructor throws  \GOT{cvmexception}{cvmexception}
if the matrix created doesn't appear to be symmetric 
or in case of non-positive size passed. Symmetry tolerance is
set by parameter \verb'tol'.
See also \GOT{srsmatrix}{srsmatrix},
\GOT{srsmatrix (TR*,int)}{srsmatrix.srsmatrix (TR*,int)}.
Example:
\begin{Verbatim}
using namespace cvm;
const double a[] = {1., 1., 1., 1., 1., 1., 1., 1., 1.};
srsmatrix m (a, 3);
m.set(2,1,5.);
std::cout << m << std::endl;
std::cout << a[0] << " " << a[1] << " " << a[2] << " "
          << a[3] << " " << a[4] << " " << a[5] << " " << std::endl;
\end{Verbatim}
prints
\begin{Verbatim}
1 5 1
5 1 1
1 1 1

1 1 1 1 1 1
\end{Verbatim}
\newpage


\subsubsection{srsmatrix (const srsmatrix\&)}
Copy constructor%
\pdfdest name {srsmatrix.srsmatrix (const srsmatrix&)} fit
\begin{verbatim}
srsmatrix::srsmatrix (const srsmatrix& m);
\end{verbatim}
creates  \verb"srsmatrix" object as a copy of \verb"m".
Constructor throws  \GOT{cvmexception}{cvmexception}
in case of memory allocation failure.
See also \GOT{srsmatrix}{srsmatrix}.
Example:
\begin{Verbatim}
using namespace cvm;

double a[] = {1., 2., 3., 2., 5., 6., 3., 6., 9.};
srsmatrix m (a, 3);
srmatrix mc(m);
m.set(2,3,7.77);
std::cout << m << std::endl << mc;
\end{Verbatim}
prints
\begin{Verbatim}
1 2 3
2 5 7.77
3 7.77 9

1 2 3
2 5 6
3 6 9
\end{Verbatim}
\newpage



\subsubsection{srsmatrix (const rmatrix\&)}
Constructor%
\pdfdest name {srsmatrix.srsmatrix (const rmatrix&)} fit
\begin{verbatim}
explicit srsmatrix::srsmatrix (const rmatrix& m, TR tol = cvmMachSp());
\end{verbatim}
creates  \verb"srsmatrix" object as a copy of a matrix \verb"m".
It's assumed that $m\times n$ matrix \verb"m" must have equal
sizes, i.e. $m = n$ is satisfied, and must be symmetric.
Symmetry tolerance is set by parameter \verb'tol'.
Constructor throws  \GOT{cvmexception}{cvmexception}
if this is not true or in case of memory allocation failure.
%Please note that this constructor is \emph{not explicit}.
See also \GOT{srsmatrix}{srsmatrix}.
Example:
\begin{Verbatim}
using namespace cvm;

double a[] = {1., 2., 3., 2., 5., 6., 3., 6., 9.};
rmatrix m(a, 3, 3);
srsmatrix ms(m);
std::cout << ms;
\end{Verbatim}
prints
\begin{Verbatim}
1 2 3
2 5 6
3 6 9
\end{Verbatim}
\newpage



\subsubsection{srsmatrix (const rvector\&)}
Constructor%
\pdfdest name {srsmatrix.srsmatrix (const rvector&)} fit
\begin{verbatim}
explicit srsmatrix::srsmatrix (const rvector& v);
\end{verbatim}
creates  \verb"srsmatrix" object
of size \verb"v.size()" by \verb"v.size()"
and assigns vector \verb"v" to its main diagonal.
Constructor throws  \GOT{cvmexception}{cvmexception}
in case of memory allocation failure.
See also \GOT{srsmatrix}{srsmatrix}, \GOT{rvector}{rvector}.
Example:
\begin{Verbatim}
using namespace cvm;

double a[] = {1., 2., 3., 4., 5.};
const rvector v(a, 5);
srsmatrix m(v);
std::cout << m;
\end{Verbatim}
prints
\begin{Verbatim}
1 0 0 0 0
0 2 0 0 0
0 0 3 0 0
0 0 0 4 0
0 0 0 0 5
\end{Verbatim}
\newpage


\subsubsection{submatrix}
Submatrix constructor%
\pdfdest name {srsmatrix.submatrixctr} fit
\begin{verbatim}
srsmatrix::srsmatrix (srsmatrix& m, int nRowCol, int size);
\end{verbatim}
creates  \verb"srmatrix" object as  \emph{submatrix} of 
symmetric matrix \verb"m".
It means that the matrix created shares memory with some part
of \verb"m". This part is defined by its upper left corner (parameter
\verb"nRowCol", \Based)
and its size (parameter \verb"size").
See also \GOT{srsmatrix}{srsmatrix}.
Example:
\begin{Verbatim}
using namespace cvm;

srsmatrix m(5);
srsmatrix subm(m, 2, 2);
subm.set(1.);
std::cout << m;
\end{Verbatim}
prints
\begin{Verbatim}
0 0 0 0 0
0 1 1 0 0
0 1 1 0 0
0 0 0 0 0
\end{Verbatim}
\newpage



\subsubsection{operator (,)}
Indexing operator%
\pdfdest name {srsmatrix.operator (,)} fit
\begin{verbatim}
TR srsmatrix::operator () (int im, int in) const throw (cvmexception);
\end{verbatim}
provides access to a particular element of a calling matrix. 
Unlike indexing operators in other classes,
this operator doesn't return  \emph{l-value}
because this would make the matrix non-symmetric.
%%%%%%%%%%%%%%%%%%%%%%%%%%%%%%%%%%%%%%%%%%%%%%%%%%%%%%%%%%%Use ...  to change pair of symmetric elements to keep the matrix symmetric
Operator is \Based.
It throws \GOT{cvmexception}{cvmexception}
if some of parameters passed
is outside of \verb"[1,msize()]" range.
See also \GOT{srsmatrix}{srsmatrix},
\GOT{Matrix::msize()}{Matrix.msize},
\GOT{Matrix::nsize()}{Matrix.nsize}.
Example:
\begin{Verbatim}
using namespace cvm;

double a[] = {1., 2., 3., 2., 5., 6., 3., 6., 9.};
srsmatrix m(a, 3);
std::cout << m(1,1) << " " << m(2,3) << std::endl;
\end{Verbatim}
prints
\begin{Verbatim}
1 6
\end{Verbatim}
\newpage



\subsubsection{operator ()}
Indexing operator%
\pdfdest name {srsmatrix.operator ()} fit
\begin{verbatim}
const rvector srsmatrix::operator () (int i) const throw (cvmexception);
\end{verbatim}
provides access to \hbox{$i$-th} column of a calling matrix.
Unlike indexing operators in other classes,
this operator doesn't return  \emph{l-value}
because this would make the matrix non-symmetric.
Operator creates an object of class \verb"rvector"
as  \emph{copy} of a column and therefore it's
\emph{not  l-value}.
Operator is \Based.
It throws \GOT{cvmexception}{cvmexception}
if parameter \verb"i" is outside of \verb"[1,nsize()]" range.
See also \GOT{srsmatrix}{srsmatrix}.
Example:
\begin{Verbatim}
using namespace cvm;

double a[] = {1., 2., 3., 2., 5., 6., 3., 6., 9.};
srsmatrix m(a, 3);
std::cout << m(1) << m(2) << m(3);
\end{Verbatim}
prints
\begin{Verbatim}
1 2 3
2 5 6
3 6 9
\end{Verbatim}
\newpage



\subsubsection{operator []}
Indexing operator%
\pdfdest name {srsmatrix.operator []} fit
\begin{verbatim}
const rvector srsmatrix::operator [] (int i) const throw (cvmexception);
\end{verbatim}
provides access to \hbox{$i$-th} row of a calling matrix.
Unlike indexing operators in other classes,
this operator doesn't return  \emph{l-value}
because this would make the matrix non-symmetric.
Operator creates an object of class \verb"rvector"
as  \emph{copy} of a row and therefore it's
\emph{not  l-value}.
Operator is \Based.
It throws \GOT{cvmexception}{cvmexception}
if parameter \verb"i" is outside of \verb"[1,msize()]" range.
See also \GOT{srsmatrix}{srsmatrix}.
Example:
\begin{Verbatim}
using namespace cvm;

double a[] = {1., 2., 3., 2., 5., 6., 3., 6., 9.};
srsmatrix m(a, 3);
std::cout << m[1] << m[2] << m[3];
\end{Verbatim}
prints
\begin{Verbatim}
1 2 3
2 5 6
3 6 9
\end{Verbatim}
\newpage



\subsubsection{diag}
Function%
\pdfdest name {srsmatrix.diag} fit
\begin{verbatim}
const rvector srsmatrix::diag (int i) const throw (cvmexception);
\end{verbatim}
provides access to \hbox{$i$-th} diagonal of a calling matrix,
where $i=0$ for main diagonal, $i<0$ for lower diagonals 
and $i>0$ for upper ones.
Unlike \verb"diag" function in other classes,
this one doesn't return  \emph{l-value}
because this would make the matrix non-symmetric.
Function creates an object of class \verb"rvector"
as  \emph{copy} of a diagonal and therefore it's
\emph{not  l-value}.
Function is \Based.
It throws \GOT{cvmexception}{cvmexception}
if parameter \verb"i" is outside of 
\verb"[-msize()+1,nsize()-1]" range.
See also \GOT{srsmatrix}{srsmatrix}.
Example:
\begin{Verbatim}
using namespace cvm;

double a[] = {1., 2., 3., 2., 5., 6., 3., 6., 9.};
srsmatrix m(a, 3);
std::cout << m << std::endl
          << m.diag(-2) << m.diag(-1) << m.diag(0)
          << m.diag(1) << m.diag(2);
\end{Verbatim}
prints
\begin{Verbatim}
1 2 3
2 5 6
3 6 9

3
2 6
1 5 9
2 6
3
\end{Verbatim}
\newpage



\subsubsection{operator = (const srsmatrix\&)}
Operator%
\pdfdest name {srsmatrix.operator = (const srsmatrix&)} fit
\begin{verbatim}
srsmatrix& srsmatrix::operator = (const srsmatrix& m)
throw (cvmexception);
\end{verbatim}
sets every element of a calling symmetric matrix to a value of
appropriate element of symmetric matrix \verb"m"
and returns a reference to
the matrix changed.
Operator throws  \GOT{cvmexception}{cvmexception}
in case of different sizes of the operands.
See also \GOT{srsmatrix}{srsmatrix}.
Example:
\begin{Verbatim}
using namespace cvm;

double a[] = {1., 2., 3., 2., 5., 6., 3., 6., 9.};
const srsmatrix m1(a, 3);
srsmatrix m2(3);

m2 = m1;
std::cout << m2;
\end{Verbatim}
prints
\begin{Verbatim}
1 2 3
2 5 6
3 6 9
\end{Verbatim}
\newpage



\subsubsection{assign (const TR*)}
Function%
\pdfdest name {srsmatrix.assign} fit
\begin{verbatim}
srsmatrix& srsmatrix::assign (const rvector& v, TR tol = cvmMachSp())
throw (cvmexception);

srsmatrix& srsmatrix::assign (const TR* pD, TR tol = cvmMachSp())
throw (cvmexception);
\end{verbatim}
sets every element of a calling matrix to a value of
appropriate element of  vector~\verb'v'
or  array pointed to by~\verb"pD"
and returns a reference to
the matrix changed.
In first version it's assumed that vector passed is long
enough to fill calling matrix. Function throws \GOT{cvmexception}{cvmexception} 
otherwise.
Function throws  \GOT{cvmexception}{cvmexception}
if the matrix changed doesn't appear to be symmetric.
Symmetry tolerance is set by parameter \verb'tol'.
See also \GOT{srsmatrix}{srsmatrix}.
Example:
\begin{Verbatim}
using namespace cvm;

double a[] = {1., 2., 3., 2., 5., 6., 3., 6., 9.};
srsmatrix m(3);
m.assign(a);
std::cout << m;
\end{Verbatim}
prints
\begin{Verbatim}
1 2 3
2 5 6
3 6 9
\end{Verbatim}
\newpage



\subsubsection{assign (int, const srsmatrix\&)}
Function%
\pdfdest name {srsmatrix.assign (int, const srsmatrix&)} fit
\begin{verbatim}
srsmatrix& srsmatrix::assign (int nRowCol, const srsmatrix& m)
throw (cvmexception);
\end{verbatim}
sets main sub-matrix of a calling symmetric matrix beginning with \Based row
\verb"nRowCol" to  symmetric matrix \verb"m" and
returns a reference to the matrix changed.
Function throws 
 \GOT{cvmexception}{cvmexception} if \verb"nRowCol"
is not positive or matrix \verb"m" doesn't fit.
See also \GOT{srsmatrix}{srsmatrix}.
Example:
\begin{Verbatim}
using namespace cvm;

srsmatrix m1(5);
srsmatrix m2(2);
m1.set(1.);
m2.set(2.);
m1.assign(2,m2);
std::cout << m1;
\end{Verbatim}
prints
\begin{Verbatim}
1 1 1 1 1
1 2 2 1 1
1 2 2 1 1
1 1 1 1 1
1 1 1 1 1
\end{Verbatim}
\newpage







\subsubsection{set (TR)}
Function%
\pdfdest name {srsmatrix.set} fit
\begin{verbatim}
srsmatrix& srsmatrix::set (TR x);
\end{verbatim}
sets every element of a calling matrix to a value of
parameter \verb"x" and returns a reference to
the matrix changed.
See also \GOT{srsmatrix}{srsmatrix}.
Example:
\begin{Verbatim}
using namespace cvm;

srsmatrix m(3);
m.set(3.);
std::cout << m;
\end{Verbatim}
prints
\begin{Verbatim}
3 3 3
3 3 3
3 3 3
\end{Verbatim}
\newpage



\subsubsection{set (int,int,TR)}
Function%
\pdfdest name {srsmatrix.set(int,int,TR)} fit
\begin{verbatim}
srsmatrix& srsmatrix::set (int nRow, int nCol, TR x);
\end{verbatim}
sets both elements located on \verb"nRow"'s row
and \verb"nCol"'s column and on 
\verb"nCol"'s row
and \verb"nRow"'s column to a value of
parameter \verb"x" and returns a reference to
the matrix changed (thus the matrix remains symmetric).
Parameters passed 
are \Based.
Function throws \GOT{cvmexception}{cvmexception}
if any of the parameters passed
is outside of \verb"[1,msize()]" range.
See also \GOT{srsmatrix}{srsmatrix}.
Example:
\begin{Verbatim}
using namespace cvm;

srsmatrix m(3);
m.set(3.);
m.set(1,3,7.);
std::cout << m;
\end{Verbatim}
prints
\begin{Verbatim}
3 3 7
3 3 3
7 3 3
\end{Verbatim}
\newpage



\subsubsection{set\_diag (int,rvector)}
Function%
\pdfdest name {srsmatrix.setdiag} fit
\begin{verbatim}
srsmatrix& srsmatrix::set_diag (int i, const rvector& v)
throw (cvmexception);
\end{verbatim}
assigns vector \verb"v"
to \hbox{$i$-th} diagonal of a calling matrix,
where $i=0$ for main diagonal, $i<0$ for lower diagonals 
and $i>0$ for upper ones.
If $i\not=0$, then  function assigns the vector
to both $i$-th and $-i$-th diagonals
(thus calling matrix remains symmetric).
Function returns a reference to the matrix changed.
Parameter \verb"i" is \emph{not} \Based.
Function throws \GOT{cvmexception}{cvmexception}
if parameter \verb"i" is outside of 
\verb"[-msize()+1,nsize()-1]" range or if
vector \verb"v" passed has  size not equal to
\verb"msize()-abs(i)".
See also \GOT{srsmatrix}{srsmatrix}.
Example:
\begin{Verbatim}
using namespace cvm;

srsmatrix m(3);
rvector v(2);
m.set(3.);
v.set(1.);
m.set_diag(1,v);
std::cout << m;
\end{Verbatim}
prints
\begin{Verbatim}
3 1 3
1 3 1
3 1 3
\end{Verbatim}
\newpage



\subsubsection{resize}
Function%
\pdfdest name {srsmatrix.resize} fit
\begin{verbatim}
srsmatrix& srsmatrix::resize (int nNewMN) throw (cvmexception);
\end{verbatim}
changes  size of a calling matrix to \verb"nNewMN" by \verb"nNewMN"
and returns a reference to
the matrix changed. In case of increasing of its size, calling matrix
is filled up with zeroes.
Function throws  \GOT{cvmexception}{cvmexception}
in case of negative size passed or memory allocation failure.
See also \GOT{srsmatrix}{srsmatrix}.
Example:
\begin{Verbatim}
using namespace cvm;

double a[] = {1., 2., 3., 2., 5., 6., 3., 6., 9.};
srsmatrix m(a,3);
std::cout << m << std::endl;
m.resize(4);
std::cout << m;
\end{Verbatim}
prints
\begin{Verbatim}
1 2 3
2 5 6
3 6 9

1 2 3 0
2 5 6 0
3 6 9 0
0 0 0 0
\end{Verbatim}
\newpage


\subsubsection{operator ==}
Operator%
\pdfdest name {srsmatrix.operator ==} fit
\begin{verbatim}
bool srsmatrix::operator == (const srsmatrix& m) const;
\end{verbatim}
compares  calling  matrix with symmetric matrix \verb"m"
and returns \verb"true" if they have the same sizes
and their appropriate elements differ by not more than the
\GO{smallest normalized positive number}{Utilities.cvmMachMin}.
Returns \verb"false" otherwise.
See also \GOT{srsmatrix}{srsmatrix}.
Example:
\begin{Verbatim}
using namespace cvm;

double a[] = {1., 2., 2., 3.};
srsmatrix m1(a, 2);
srsmatrix m2(2);
m2.set(1,1,1.); 
m2.set(1,2,2.);
m2.set(2,2,3.);

std::cout << (m1 == m2) << std::endl;
\end{Verbatim}
prints
\begin{Verbatim}
1
\end{Verbatim}
\newpage




\subsubsection{operator !=}
Operator%
\pdfdest name {srsmatrix.operator !=} fit
\begin{verbatim}
bool srsmatrix::operator != (const srsmatrix& m) const;
\end{verbatim}
compares  calling matrix with symmetric matrix \verb"m"
and returns \verb"true" if they have different sizes
or at least one of their appropriate elements
differs by more than the
\GO{smallest normalized positive number}{Utilities.cvmMachMin}.
Returns \verb"false" otherwise.
See also \GOT{srsmatrix}{srsmatrix}.
Example:
\begin{Verbatim}
using namespace cvm;

double a[] = {1., 2., 2., 3.};
srsmatrix m1(a, 2);
srsmatrix m2(2);
m2.set(1,1,1.0001); 
m2.set(1,2,2.);
m2.set(2,2,3.);

std::cout << (m1 != m2) << std::endl;
\end{Verbatim}
prints
\begin{Verbatim}
1
\end{Verbatim}
\newpage



\subsubsection{operator <{}<}
Operator%
\pdfdest name {srsmatrix.operator <<} fit
\begin{verbatim}
srsmatrix& srsmatrix::operator << (const srsmatrix& m)
throw (cvmexception);
\end{verbatim}
destroys  calling matrix, creates  new one as a copy of \verb"m"
and returns a reference to
the matrix changed.
Operator throws  \GOT{cvmexception}{cvmexception}
in case of memory allocation failure.
See also \GOT{srsmatrix}{srsmatrix}.
Example:
\begin{Verbatim}
using namespace cvm;

double a[] = {1., 2., 3., 2., 5., 6., 3., 6., 9.};
srsmatrix m(a, 3);
srsmatrix mc(1);
std::cout << m << std::endl << mc << std::endl;
mc << m;
std::cout << mc;
\end{Verbatim}
prints
\begin{Verbatim}
1 2 3
2 5 6
3 6 9

0

1 2 3
2 5 6
3 6 9
\end{Verbatim}
\newpage



\subsubsection{operator +}
Operator%
\pdfdest name {srsmatrix.operator +} fit
\begin{verbatim}
srsmatrix srsmatrix::operator + (const srsmatrix& m) const
throw (cvmexception);
\end{verbatim}
creates an object of type \verb"srsmatrix" as a sum of
a calling symmetric matrix and symmetric matrix \verb"m".
It throws  \GOT{cvmexception}{cvmexception}
in case of different sizes of the operands.
See also \GOT{srsmatrix::sum}{srsmatrix.sum}, \GOT{srsmatrix}{srsmatrix}.
Example:
\begin{Verbatim}
using namespace cvm;

double a[] = {1., 2., 3., 2., 5., 6., 3., 6., 9.};
srsmatrix m1(a, 3);
srsmatrix m2(3);
m2.set(1.);
std::cout << m1 + m2 << std::endl << m1 + m1;
\end{Verbatim}
prints
\begin{Verbatim}
2 3 4
3 6 7
4 7 10

2 4 6
4 10 12
6 12 18
\end{Verbatim}
\newpage




\subsubsection{operator -}
Operator%
\pdfdest name {srsmatrix.operator -} fit
\begin{verbatim}
srsmatrix srsmatrix::operator - (const srsmatrix& m) const
throw (cvmexception);
\end{verbatim}
creates an object of type \verb"srsmatrix" as a difference of
a calling symmetric matrix and symmetric matrix \verb"m".
It throws  \GOT{cvmexception}{cvmexception}
in case of different sizes of the operands.
See also \GOT{srsmatrix::diff}{srsmatrix.diff}, \GOT{srsmatrix}{srsmatrix}.
Example:
\begin{Verbatim}
using namespace cvm;

double a[] = {1., 2., 3., 2., 5., 6., 3., 6., 9.};
srsmatrix m1(a, 3);
srsmatrix m2(3);
m2.set(1.);
std::cout << m1 - m2 << std::endl << m1 - m1;
\end{Verbatim}
prints
\begin{Verbatim}
0 1 2
1 4 5
2 5 8

0 0 0
0 0 0
0 0 0
\end{Verbatim}
\newpage



\subsubsection{sum}
Function%
\pdfdest name {srsmatrix.sum} fit
\begin{verbatim}
srsmatrix& srsmatrix::sum (const srsmatrix& m1, const srsmatrix& m2)
throw (cvmexception);
\end{verbatim}
assigns the result of addition of symmetric
matrices \verb"m1" and \verb"m2"  to a calling symmetric matrix
and returns a reference to
the matrix changed.
It throws  \GOT{cvmexception}{cvmexception}
in case of different sizes of the operands.
See also \GOT{srsmatrix::ope\-ra\-tor~+~}{srsmatrix.operator +},
\GOT{srsmatrix}{srsmatrix}.
Example:
\begin{Verbatim}
using namespace cvm;

double a[] = {1., 2., 3., 2., 5., 6., 3., 6., 9.};
const srsmatrix m1(a, 3);
srsmatrix m2(3);
srsmatrix m(3);
m2.set(1.);

std::cout << m.sum(m1, m2) << std::endl;
std::cout << m.sum(m, m2);
\end{Verbatim}
prints
\begin{Verbatim}
2 3 4
3 6 7
4 7 10

3 4 5
4 7 8
5 8 11
\end{Verbatim}
\newpage



\subsubsection{diff}
Function%
\pdfdest name {srsmatrix.diff} fit
\begin{verbatim}
srsmatrix& srsmatrix::diff (const srsmatrix& m1, const srsmatrix& m2)
throw (cvmexception);
\end{verbatim}
assigns the result of subtraction of
symmetric matrices \verb"m1" and \verb"m2" to a calling symmetric matrix
and returns a reference to
the matrix changed.
It throws  \GOT{cvmexception}{cvmexception}
in case of different sizes of the operands.
See also \GOT{srsmatrix::ope\-ra\-tor~-~}{srsmatrix.operator -},
\GOT{srsmatrix}{srsmatrix}.
Example:
\begin{Verbatim}
using namespace cvm;

double a[] = {1., 2., 3., 2., 5., 6., 3., 6., 9.};
const srsmatrix m1(a, 3);
srsmatrix m2(3);
srsmatrix m(3);
m2.set(1.);

std::cout << m.diff(m1, m2) << std::endl;
std::cout << m.diff(m, m2);
\end{Verbatim}
prints
\begin{Verbatim}
0 1 2
1 4 5
2 5 8

-1 0 1
0 3 4
1 4 7
\end{Verbatim}
\newpage



\subsubsection{operator +=}
Operator%
\pdfdest name {srsmatrix.operator +=} fit
\begin{verbatim}
srsmatrix& srsmatrix::operator += (const srsmatrix& m)
throw (cvmexception);
\end{verbatim}
adds symmetric matrix \verb"m" to a calling symmetric matrix 
and returns a reference to
the matrix changed.
It throws  \GOT{cvmexception}{cvmexception}
in case of different sizes of the operands.
See also \GOT{srsmatrix::ope\-ra\-tor~+~}{srsmatrix.operator +},
\GOT{srsmatrix::sum}{srsmatrix.sum},
\GOT{srsmatrix}{srsmatrix}.
Example:
\begin{Verbatim}
using namespace cvm;

srsmatrix m1(3);
srsmatrix m2(3);
m1.set(1.);
m2.set(2.);

m1 += m2;
std::cout << m1 << std::endl;

// well, you can do this too, but temporary object would be created
m2 += m2; 
std::cout << m2;
\end{Verbatim}
prints
\begin{Verbatim}
3 3 3
3 3 3
3 3 3

4 4 4
4 4 4
4 4 4
\end{Verbatim}
\newpage




\subsubsection{operator -=}
Operator%
\pdfdest name {srsmatrix.operator -=} fit
\begin{verbatim}
srsmatrix& srsmatrix::operator -= (const srsmatrix& m)
throw (cvmexception);
\end{verbatim}
subtracts symmetric matrix \verb"m" from  calling symmetric matrix
and returns a reference to
the matrix changed.
It throws  \GOT{cvmexception}{cvmexception}
in case of different sizes of the operands.
See also \GOT{srsmatrix::ope\-ra\-tor~-~}{srsmatrix.operator -},
\GOT{srsmatrix::diff}{srsmatrix.diff},
\GOT{srsmatrix}{srsmatrix}.
Example:
\begin{Verbatim}
using namespace cvm;

srsmatrix m1(3);
srsmatrix m2(3);
m1.set(1.);
m2.set(2.);

m1 -= m2;
std::cout << m1 << std::endl;

// well, you can do this too, but temporary object would be created
m2 -= m2; 
std::cout << m2;
\end{Verbatim}
prints
\begin{Verbatim}
-1 -1 -1
-1 -1 -1
-1 -1 -1

0 0 0
0 0 0
0 0 0
\end{Verbatim}
\newpage



\subsubsection{operator - ()}
Operator%
\pdfdest name {srsmatrix.operator - ()} fit
\begin{verbatim}
srsmatrix srsmatrix::operator - () const throw (cvmexception);
\end{verbatim}
creates an object of type \verb"srsmatrix" as
a calling symmetric matrix multiplied by $-1$.
See also \GOT{srsmatrix}{srsmatrix}.
Example:
\begin{Verbatim}
using namespace cvm;

double a[] = {1., 2., 3., 2., 5., 6., 3., 6., 9.};
srsmatrix m(a, 3);
std::cout << -m;
\end{Verbatim}
prints
\begin{Verbatim}
-1 -2 -3
-2 -5 -6
-3 -6 -9
\end{Verbatim}
\newpage



\subsubsection{operator ++}
Operator%
\pdfdest name {srsmatrix.operator ++} fit
\begin{verbatim}
srsmatrix& srsmatrix::operator ++ ();
srsmatrix& srsmatrix::operator ++ (int);
\end{verbatim}
adds identity matrix to a calling symmetric matrix
and returns a reference to
the matrix changed.
See also \GOT{srsmatrix}{srsmatrix}.
Example:
\begin{Verbatim}
using namespace cvm;

srsmatrix m(4);
m.set(4.);
m++;
std::cout << m << std::endl;
std::cout << ++m;
\end{Verbatim}
prints
\begin{Verbatim}
5 4 4 4
4 5 4 4
4 4 5 4
4 4 4 5

6 4 4 4
4 6 4 4
4 4 6 4
4 4 4 6
\end{Verbatim}
\newpage



\subsubsection{operator -{}-}
Operator%
\pdfdest name {srsmatrix.operator --} fit
\begin{verbatim}
srsmatrix& srsmatrix::operator -- ();
srsmatrix& srsmatrix::operator -- (int);
\end{verbatim}
subtracts identity matrix from  calling symmetric matrix
and returns a reference to
the matrix changed.
See also \GOT{srsmatrix}{srsmatrix}.
Example:
\begin{Verbatim}
using namespace cvm;

srsmatrix m(4);
m.set(4.);
m--;
std::cout << m << std::endl;
std::cout << --m;
\end{Verbatim}
prints
\begin{Verbatim}
3 4 4 4
4 3 4 4
4 4 3 4
4 4 4 3

2 4 4 4
4 2 4 4
4 4 2 4
4 4 4 2
\end{Verbatim}
\newpage



\subsubsection{operator * (TR)}
Operator%
\pdfdest name {srsmatrix.operator * (TR)} fit
\begin{verbatim}
srsmatrix srsmatrix::operator * (TR d) const;
\end{verbatim}
creates an object of type \verb"srsmatrix" as a product of
a calling symmetric matrix and  number~\verb"d".
See also \GOT{srsmatrix::ope\-ra\-tor~*=~}{srsmatrix.operator *= (TR)},
\GOT{srsmatrix}{srsmatrix}.
Example:
\begin{Verbatim}
using namespace cvm;

double a[] = {1., 2., 3., 2., 5., 6., 3., 6., 9.};
srsmatrix m(a,3);
std::cout << m * 5.;
\end{Verbatim}
prints
\begin{Verbatim}
5 10 15
10 25 30
15 30 45
\end{Verbatim}
\newpage



\subsubsection{operator / (TR)}
Operator%
\pdfdest name {srsmatrix.operator / (TR)} fit
\begin{verbatim}
srsmatrix srsmatrix::operator / (TR d) const throw (cvmexception);
\end{verbatim}
creates an object of type \verb"srsmatrix" as a quotient of
a calling symmetric matrix and  number~\verb"d". It throws
 \GOT{cvmexception}{cvmexception}
if \verb"d" has  absolute value equal or less than the
\GO{smallest normalized positive number}{Utilities.cvmMachMin}.
See also \GOT{srsmatrix::operator~/=~}{srsmatrix.operator /= (TR)},
\GOT{srsmatrix}{srsmatrix}.
Example:
\begin{Verbatim}
using namespace cvm;

double a[] = {1., 2., 3., 2., 5., 6., 3., 6., 9.};
srsmatrix m(a,3);
std::cout << m / 4.;
\end{Verbatim}
prints
\begin{Verbatim}
0.25 0.5 0.75
0.5 1.25 1.5
0.75 1.5 2.25
\end{Verbatim}
\newpage



\subsubsection{operator *= (TR)}
Operator%
\pdfdest name {srsmatrix.operator *= (TR)} fit
\begin{verbatim}
srsmatrix& srsmatrix::operator *= (TR d);
\end{verbatim}
multiplies  calling symmetric matrix by  number~\verb"d"
and returns a reference to
the matrix changed.
See also \GOT{srsmatrix::ope\-ra\-tor~*~}{srsmatrix.operator * (TR)},
\GOT{srsmatrix}{srsmatrix}.
Example:
\begin{Verbatim}
using namespace cvm;

double a[] = {1., 2., 3., 2., 5., 6., 3., 6., 9.};
srsmatrix m(a,3);
m *= 2.;
std::cout << m;
\end{Verbatim}
prints
\begin{Verbatim}
2 4 6
4 10 12
6 12 18
\end{Verbatim}
\newpage



\subsubsection{operator /= (TR)}
Operator%
\pdfdest name {srsmatrix.operator /= (TR)} fit
\begin{verbatim}
srsmatrix& srsmatrix::operator /= (TR d) throw (cvmexception);
\end{verbatim}
divides  calling symmetric matrix by  number~\verb"d"
and returns a reference to
the matrix changed.
It throws  \GOT{cvmexception}{cvmexception}
if \verb"d" has  absolute value equal or less
than the 
\GO{smallest normalized positive number}{Utilities.cvmMachMin}.
See also \GOT{srsmatrix::operator~/~}{srsmatrix.operator / (TR)},
\GOT{srsmatrix}{srsmatrix}.
Example:
\begin{Verbatim}
using namespace cvm;

double a[] = {1., 2., 3., 2., 5., 6., 3., 6., 9.};
srsmatrix m(a,3);
m /= 2.;
std::cout << m;
\end{Verbatim}
prints
\begin{Verbatim}
0.5 1 1.5
1 2.5 3
1.5 3 4.5
\end{Verbatim}
\newpage


\subsubsection{normalize}
Function%
\pdfdest name {srsmatrix.normalize} fit
\begin{verbatim}
srsmatrix& srsmatrix::normalize ();
\end{verbatim}
normalizes  calling symmetric matrix so its \GO{Euclidean norm}{Array.norm}
becomes equal to $1$ if it was greater than the 
\GO{smallest normalized positive number}{Utilities.cvmMachMin}
before the call (otherwise function does nothing).
See also \GOT{srsmatrix}{srsmatrix}.
Example:
\begin{Verbatim}
using namespace cvm;

std::cout.setf (std::ios::scientific | std::ios::left);
std::cout.precision (3);
double a[] = {1., 2., 3., 2., 5., 6., 3., 6., 9.};
srsmatrix m(a,3);
m.normalize();
std::cout << m << m.norm() << std::endl;
\end{Verbatim}
prints
\begin{Verbatim}
6.984e-002 1.397e-001 2.095e-001
1.397e-001 3.492e-001 4.191e-001
2.095e-001 4.191e-001 6.286e-001
1.000e+000
\end{Verbatim}
\newpage



\subsubsection{transposition}
Operator and functions%
\pdfdest name {srsmatrix.transposition} fit
\begin{verbatim}
srsmatrix srsmatrix::operator ~ () const throw (cvmexception);
srsmatrix& srsmatrix::transpose (const srsmatrix& m) throw (cvmexception);
srsmatrix& srsmatrix::transpose ();
\end{verbatim}
do nothing since calling matrix is symmetric.
They are provided to reimplement similar member functions and operator
of the class \GOT{srmatrix}{srmatrix}.
See also \GOT{srsmatrix}{srsmatrix}, \GOT{srmatrix}{srmatrix}.
Example:
\begin{Verbatim}
using namespace cvm;

double a[] = {1., 2., 3., 2., 5., 6., 3., 6., 9.};
srsmatrix m(a,3);
std::cout << m - ~m;
\end{Verbatim}
prints
\begin{Verbatim}
0 0 0
0 0 0
0 0 0
\end{Verbatim}
\newpage




\subsubsection{operator * (const rvector\&)}
Operator%
\pdfdest name {srsmatrix.operator * (const rvector&)} fit
\begin{verbatim}
rvector srsmatrix::operator * (const rvector& v) const
throw (cvmexception);
\end{verbatim}
creates an object of type \verb"rvector"
as a product of a calling symmetric matrix and a vector \verb"v".
It throws  \GOT{cvmexception}{cvmexception}
if  number of columns of a calling matrix
differs from  size of a vector \verb"v".
Use \GOT{rvector::mult}{rvector.mult (const rmatrix&, const rvector&)}
in order to avoid new object creation.
See also
\GOT{srsmatrix}{srsmatrix} and \GOT{rvector}{rvector}.
Example:
\begin{Verbatim}
using namespace cvm;

try {
    srsmatrix m (4);
    rvector v(4);
    m.set(1.);
    v.set(1.);

    std::cout << m * v;
}
catch (std::exception& e) {
    std::cout << "Exception: " << e.what () << std::endl;
}
\end{Verbatim}
prints
\begin{Verbatim}
4 4 4 4
\end{Verbatim}
\newpage



\subsubsection{operator * (const rmatrix\&)}
Operator%
\pdfdest name {srsmatrix.operator * (const rmatrix&)} fit
\begin{verbatim}
rmatrix srsmatrix::operator * (const rmatrix& m) const
throw (cvmexception);
\end{verbatim}
creates an object of type \verb"rmatrix"
as a product of a calling symmetric matrix and a matrix \verb"m".
It throws  \GOT{cvmexception}{cvmexception}
if  number of columns of a calling matrix
differs from  number of rows of a matrix \verb"m".
Use \GOT{rmatrix::mult}{rmatrix.mult} in order to avoid
 new object creation.
See also
\GOT{rmatrix}{rmatrix} and \GOT{srsmatrix}{srsmatrix}.
Example:
\begin{Verbatim}
using namespace cvm;

try {
    srsmatrix ms(4);
    rmatrix m(4,2);
    ms.set(1.);
    m.set(2.);
    std::cout << ms * m;
}
catch (std::exception& e) {
    std::cout << "Exception: " << e.what () << std::endl;
}
\end{Verbatim}
prints
\begin{Verbatim}
8 8
8 8
8 8
8 8
\end{Verbatim}
\newpage



\subsubsection{operator * (const srmatrix\&)}
Operator%
\pdfdest name {srsmatrix.operator * (const srmatrix&)} fit
\begin{verbatim}
srmatrix srsmatrix::operator * (const srmatrix& m) const
throw (cvmexception);
\end{verbatim}
creates an object of type \verb"srmatrix"
as a product of a calling symmetric matrix and a matrix \verb"m".
It throws  \GOT{cvmexception}{cvmexception}
if  operands have different sizes.
Use \GOT{rmatrix::mult}{rmatrix.mult} in order to avoid
 new object creation.
See also
\GOT{srmatrix}{srmatrix} and \GOT{srsmatrix}{srsmatrix}.
Example:
\begin{Verbatim}
using namespace cvm;

try {
    srsmatrix ms(3);
    srmatrix m(3);
    ms.set(1.);
    m.set(2.);
    std::cout << ms * m << std::endl;

    double a[] = {1., 2., 3., 2., 5., 6., 3., 6., 9.};
    const srsmatrix ms2(a, 3);
    std::cout << ms2 * ms;
}
catch (std::exception& e) {
    std::cout << "Exception: " << e.what () << std::endl;
}
\end{Verbatim}
prints
\begin{Verbatim}
6 6 6
6 6 6
6 6 6

6 6 6
13 13 13
18 18 18
\end{Verbatim}
\newpage





\subsubsection{operator / (const rvector\&)}
Operator%
\pdfdest name {srsmatrix.operator / (rvector)} fit
\begin{verbatim}
rvector operator / (const rvector& vB) const throw (cvmexception);
\end{verbatim}
returns solution $x$ of linear equation
$A*x=b$ where calling matrix is square symmetric matrix $A$
and a vector $b$ is passed in parameter \verb"vB".
This operator throws exception 
of type \GOT{cvmexception}{cvmexception}
in case of inappropriate sizes
of the objects or when  matrix $A$ is close to singular.
See also \GOT{rvector::solve}{rvector.solve}, 
\GOT{srmatrix::solve}{srmatrix.solve}, 
\GOT{rvector.operator~\%}{rvector.operator percent (srmatrix)}, 
\GOT{rvector}{rvector}, \GOT{srsmatrix}{srsmatrix}.
Example:
\begin{Verbatim}
using namespace cvm;
std::cout.setf (std::ios::scientific | std::ios::showpos);
std::cout.precision (12);

try {
    double m[] = {3., 2., 3., 2., 5., 6., 3., 6., 9.};
    double b[] = {1., 2., 3.};
    srsmatrix ma(m, 3);
    rvector  vb(b, 3);
    rvector  vx(3);

    vx = ma / vb;

    std::cout << ma * vx - vb;
}
catch (std::exception& e) {
    std::cout << "Exception " << e.what () << std::endl;
}
\end{Verbatim}
prints
\begin{Verbatim}
-1.110223024625e-016 -2.220446049250e-016 +4.440892098501e-016
\end{Verbatim}
\newpage






\subsubsection{syrk}
Functions%
\pdfdest name {srsmatrix.syrk} fit
\begin{verbatim}
srsmatrix& 
srsmatrix::syrk (TR alpha, const rvector& v, TR beta) 
throw (cvmexception);

srsmatrix& 
srsmatrix::syrk (bool bTransp, TR alpha, const rmatrix& m, TR beta)
throw (cvmexception);
\end{verbatim}
call one of \verb"?SYRK" routines of the
\URL{BLAS library}{http://www.netlib.org/blas}
performing  matrix-vector operation defined for the first version as
rank-1 update operation
\begin{equation*}
A=\alpha\,\begin{pmatrix}
v_1 \\
v_2 \\
\vdots \\
v_n
\end{pmatrix}
\begin{pmatrix}
v_1 & v_2 & \cdots & x_n
\end{pmatrix} + \beta A,
\end{equation*}
and for the second version as
\begin{equation*}
A=\alpha MM^T + \beta A\quad\text{or}\quad A=\alpha M^TM + \beta A.
\end{equation*}
Here $\alpha$ and $\beta$ are real numbers 
(parameters \verb"alpha" and \verb"beta"),
$M$ is real matrix (parameter \verb"m"),
$A$ is  calling symmetric matrix
and $v$ is real vector (parameter \verb"v").
First operation for the second version 
is performed if \verb"bTransp" passed
is \verb"false" and second one otherwise.
Function
returns a reference to the matrix changed and throws
 \GOT{cvmexception}{cvmexception}
in case of inappropriate sizes of the operands.
See also
\GOT{rvector}{rvector},
\GOT{rmatrix}{rmatrix} and \GOT{srsmatrix}{srsmatrix}.

Example:
\begin{Verbatim}
using namespace cvm;
const treal alpha = 2.12;
const treal beta = -3.07;
rvector v(3);
srsmatrix ms(3), ms2(3);
v.randomize(-3.,2.);
ms.randomize(-1.,2.);
ms2 = ms;
ms.syrk (alpha, v, beta);
ms2 = alpha * srsmatrix(v.rank1update(v)) + beta * ms2;
std::cout << ms - ms2;
\end{Verbatim}
prints
\begin{Verbatim}
0 0 0
0 0 0
0 0 0
\end{Verbatim}
Example:
\begin{Verbatim}
using namespace cvm;
const treal alpha = 2.12;
const treal beta = -3.07;
rmatrix m(3,3);
srsmatrix ms(3), ms2(3);
m.randomize(-1.,2.);
ms.randomize(-1.,2.);
ms2 = ms;
ms.syrk (false, alpha, m, beta);
ms2 = alpha * srsmatrix (m * ~m) + beta * ms2;
std::cout << ms - ms2;
\end{Verbatim}
prints
\begin{Verbatim}
0 0 0
0 0 0
0 0 0
\end{Verbatim}
Example:
\begin{Verbatim}
using namespace cvm;
const treal alpha = 2.12;
const treal beta = -3.07;
rmatrix m(3,3);
srsmatrix ms(3), ms2(3);
m.randomize(-1.,2.);
ms.randomize(-1.,2.);
ms2 = ms;
ms.syrk (true, alpha, m, beta);
ms2 = alpha * srsmatrix (~m * m) + beta * ms2;
std::cout << ms - ms2;
\end{Verbatim}
prints
\begin{Verbatim}
0 0 0
0 0 0
0 0 0
\end{Verbatim}
\newpage




\subsubsection{syr2k}
Functions%
\pdfdest name {srsmatrix.syr2k} fit
\begin{verbatim}
srsmatrix& 
srsmatrix::syr2k (TR alpha, const rvector& v1, 
                  const rvector& v2, TR beta) throw (cvmexception);
srsmatrix& 
srsmatrix::syr2k (bool bTransp, TR alpha, const rmatrix& m1, 
                  const rmatrix& m2, TR beta) throw (cvmexception);
\end{verbatim}
call one of \verb"?SYR2K" routines of the
\URL{BLAS library}{http://www.netlib.org/blas}
performing  matrix-vector operation defined for the first version as
rank-1 update operation
\begin{equation*}
A=\alpha v_1 v_2' + \alpha v_2 v_1' + \beta A,
\end{equation*}
and for the second version as
\begin{equation*}
A=\alpha M_1 M_2^T + \alpha M_2 M_1^T + \beta A\quad
\text{or}\quad 
A=\alpha M_1^T M_2 + \alpha M_2^T M_1 + \beta A.
\end{equation*}
Here $\alpha$ and $\beta$ are real numbers 
(parameters \verb"alpha" and \verb"beta"),
$M_1$ and $M_2$ are real matrices (parameters \verb"m1" and \verb"m2"),
$A$ is  calling symmetric matrix
and $v_1$ and $v_2$ are real vectors (parameters \verb"v1" and \verb"v2").
First operation for the second version 
is performed if \verb"bTransp" passed
is \verb"false" and second one otherwise.
Function
returns a reference to the matrix changed and throws
 \GOT{cvmexception}{cvmexception}
in case of inappropriate sizes of the operands.
See also
\GOT{rvector}{rvector},
\GOT{rmatrix}{rmatrix} and \GOT{srsmatrix}{srsmatrix}.
Example:
\begin{Verbatim}
using namespace cvm;
std::cout.setf (std::ios::scientific | std::ios::showpos); 
std::cout.precision (2);
const treal alpha = 2.12;
const treal beta = -3.07;
rvector v1(4);
rvector v2(4);
srsmatrix ms(4), ms2(4);
v1.randomize(-1.,3.);
v2.randomize(-1.,3.);
ms.randomize(-1.,3.);
ms2 = ms;
ms.syr2k (alpha, v1, v2, beta);
ms2 = alpha * srsmatrix(v1.rank1update(v2) + v2.rank1update(v1)) + beta * ms2;
std::cout << ms - ms2;
\end{Verbatim}
prints
\begin{Verbatim}
+0.00e+000 -1.33e-015 +0.00e+000 +0.00e+000
-1.33e-015 +0.00e+000 +4.44e-016 +0.00e+000
+0.00e+000 +4.44e-016 -5.55e-017 +0.00e+000
+0.00e+000 +0.00e+000 +0.00e+000 -8.88e-016
\end{Verbatim}
Example:
\begin{Verbatim}
const treal alpha = 2.12;
const treal beta = -3.07;
rmatrix m1(3,3), m2(3,3);
srsmatrix ms(3), ms2(3);
m1.randomize(-2.,2.);
m2.randomize(-2.,2.);
ms.randomize(-1.,2.);
ms2 = ms;
ms.syr2k (false, alpha, m1, m2, beta);
ms2 = alpha * srsmatrix (m1 * ~m2 + m2 * ~m1) + beta * ms2;
std::cout << ms - ms2;
\end{Verbatim}
prints
\begin{Verbatim}
+1.11e-015 +0.00e+000 +0.00e+000
+0.00e+000 +8.88e-016 +1.78e-015
+0.00e+000 +1.78e-015 -1.78e-015
\end{Verbatim}
Example:
\begin{Verbatim}
const treal alpha = 2.12;
const treal beta = -3.07;
rmatrix m1(3,3), m2(3,3);
srsmatrix ms(3), ms2(3);
m1.randomize(-2.,2.);
m2.randomize(-2.,2.);
ms.randomize(-1.,2.);
ms2 = ms;
ms.syr2k (true, alpha, m1, m2, beta);
ms2 = alpha * srsmatrix (~m1 * m2 + ~m2 * m1) + beta * ms2;
std::cout << ms - ms2;
\end{Verbatim}
prints
\begin{Verbatim}
+0.00e+000 +1.78e-015 +0.00e+000
+1.78e-015 +0.00e+000 +1.33e-015
+0.00e+000 +1.33e-015 +0.00e+000
\end{Verbatim}
\newpage









\subsubsection{inv}
Functions%
\pdfdest name {srsmatrix.inv} fit
\begin{verbatim}
srsmatrix& srsmatrix::inv (const srsmatrix& m) throw (cvmexception);
srsmatrix srsmatrix::inv () const throw (cvmexception);
\end{verbatim}
implement symmetric matrix inversion.
The first version sets  calling symmetric matrix to be equal to 
 symmetric matrix \verb"m" inverted and the second one
creates an object of type \verb"srsmatrix" as inverted calling matrix.
Functions throw  \GOT{cvmexception}{cvmexception} 
in case of inappropriate
sizes of the operands or when  matrix to be inverted is close to
singular.
See also
\GOT{srsmatrix}{srsmatrix}.
Example:
\begin{Verbatim}
using namespace cvm;

std::cout.setf (std::ios::scientific | std::ios::left); 
std::cout.precision (5);

double a[] = {1., 2., 3., 2., 5., 6., 3., 6., 9.05};
const srsmatrix m(a, 3);
const srsmatrix mi = m.inv();

std::cout << mi << std::endl;
std::cout << mi * m - eye_real(3);
\end{Verbatim}
prints
\begin{Verbatim}
1.85000e+002 -2.00000e+000 -6.00000e+001
-2.00000e+000 1.00000e+000 0.00000e+000
-6.00000e+001 0.00000e+000 2.00000e+001

0.00000e+000 0.00000e+000 0.00000e+000
0.00000e+000 0.00000e+000 0.00000e+000
0.00000e+000 0.00000e+000 0.00000e+000
\end{Verbatim}
\newpage




\subsubsection{exp}
Functions%
\pdfdest name {srsmatrix.exp} fit
\begin{verbatim}
srsmatrix& srsmatrix::exp (const srsmatrix& m, TR tol = cvmMachSp())
throw (cvmexception);

srsmatrix srsmatrix::exp (TR tol = cvmMachSp()) const
throw (cvmexception);
\end{verbatim}
compute  exponent of a calling symmetric matrix using Pad\'e approximation
defined as
\begin{equation*}
R_{pq}(z)=D_{pq}(z)^{-1}N_{pq}(z)=1+z+\dots+z^p/p!\,,
\end{equation*}
where
\begin{align*}
N_{pq}(z)&=\sum_{k=0}^p\frac{(p+q-k)!p!}{(p+q)!k!(q-k)!}z^k,\\
D_{pq}(z)&=\sum_{k=0}^q\frac{(p+q-k)!p!}{(p+q)!k!(q-k)!}(-z)^k
\end{align*}
along with a matrix normalizing as described in
\GO{\cite{Golub}}{biblio}, p.~572.
Functions use \verb"DMEXP" (or \verb"SMEXP" for \verb"float" version)
\FORTRAN subroutine implementing the algorithm.
The first version sets  calling symmetric matrix to be equal to
 exponent of symmetric matrix \verb"m" and returns 
 reference to the matrix
changed. The second version
creates an object of type \verb"srsmatrix" as 
exponent of a calling matrix.
The algorithm uses parameter \verb"tol"
as $\varepsilon(p,q)$ in order to choose constants $p$ and $q$
so that
\begin{equation*}
\varepsilon(p,q)\ge 2^{3-(p+q)}\frac{p!q!}{(p+q)!(p+q+1)!}.
\end{equation*}
This parameter is equal to the
\GO{largest relative spacing}{Utilities.cvmMachSp} by default.
Functions throw  \GOT{cvmexception}{cvmexception} 
in case of
inappropriate sizes of the operands or when LAPACK subroutine fails.
See also
\GOT{srsmatrix}{srsmatrix}.
Example:
\begin{Verbatim}
using namespace cvm;

std::cout.setf (std::ios::scientific | std::ios::left); 
std::cout.precision (15);

double a[] = {1., 2., 1., 2., 0., -1., 1., -1., 2.};
const srsmatrix m(a, 3);
std::cout << m.exp();
\end{Verbatim}
prints
\begin{Verbatim}
9.198262499129184e+000 5.558586002658855e+000 3.852443363622591e+000
5.558586002658857e+000 5.345819135506593e+000 -1.706142639036265e+000
3.852443363622590e+000 -1.706142639036266e+000 1.090440513816545e+001
\end{Verbatim}
Matlab output:
\begin{Verbatim}
  Columns 1 through 2 

    9.198262499129212e+000    5.558586002658862e+000
    5.558586002658865e+000    5.345819135506588e+000
    3.852443363622600e+000   -1.706142639036258e+000

  Column 3 

    3.852443363622601e+000
   -1.706142639036260e+000
    1.090440513816545e+001
\end{Verbatim}
\newpage



\subsubsection{polynomial}
Functions%
\pdfdest name {srsmatrix.polynom} fit
\begin{verbatim}
srsmatrix& srsmatrix::polynom (const srsmatrix& m, const rvector& v)
throw (cvmexception);

srsmatrix srsmatrix::polynom (const rvector& v) const
throw (cvmexception);
\end{verbatim}
compute  symmetric matrix polynomial defined as
\begin{equation*}
p(A)=b_0I+b_1A+\dots+b_qA^q
\end{equation*}
using the Horner's rule:
\begin{equation*}
p(A)=\sum_{k=0}^r B_k(A^s)^k,\quad s=\floor(\!\sqrt{q}\,),\ r=\floor(q/s)
\end{equation*}
where
\begin{equation*}
B_k=\begin{cases}
\sum\limits_{i=0}^{s-1}b_{sk+i} A^i, & k=0,1,\dots,r-1\\
\sum\limits_{i=0}^{q-sr}b_{sr+i} A^i, & k=r.
\end{cases}
\end{equation*}
See also \GO{\cite{Golub}}{biblio}, p.~568.
Coefficients $b_0,b_1,\dots,b_q$ are passed in  parameter 
\verb"v",
where $q$ is equal to \verb"v.size()-1", so  functions
compute matrix polynomial equal to
\begin{equation*}
\mathtt{v[1]*I + v[2]*m +\cdots + v[v.size()]*m^{v.size()-1}}
\end{equation*}
The first version sets  calling symmetric matrix to be equal 
to the polynomial of  symmetric matrix \verb"m"
and the second one
creates an object of type \verb"srsmatrix" as the 
polynomial of a calling symmetric matrix.
Functions use \verb"DPOLY" (or \verb"SPOLY" for \verb"float" version)
\FORTRAN subroutine implementing the Horner's algorithm.
Functions throw  \GOT{cvmexception}{cvmexception} 
in case of inappropriate sizes of the operands.
See also
\GOT{srsmatrix}{srsmatrix}.
Example:
\begin{Verbatim}
using namespace cvm;

std::cout.setf (std::ios::scientific | std::ios::left); 
std::cout.precision (7);
double a[] = {1., 2., 1., 2., 0., -1., 1., -1., 2.};
double av[] = {2.2, 1.3, 1.1, -0.9, 0.2,
              -0.45, 45, -30, 10, 3, 3.2};
const rvector v(av, 11);
const srsmatrix m(a, 3);

std::cout << m.polynom (v);
\end{Verbatim}
prints
\begin{Verbatim}
6.2127400e+004 2.3998000e+004 3.4100550e+004
2.3998000e+004 2.8026850e+004 1.0102550e+004
3.4100550e+004 1.0102550e+004 5.2024850e+004
\end{Verbatim}
Matlab output:
\begin{Verbatim}
  Columns 1 through 2 

    6.212740000000001e+004    2.399800000000000e+004
    2.399800000000000e+004    2.802685000000000e+004
    3.410055000000000e+004    1.010255000000000e+004

  Column 3 

    3.410055000000000e+004
    1.010255000000000e+004
    5.202485000000000e+004
\end{Verbatim}
\newpage





\subsubsection{eig}
Functions%
\pdfdest name {srsmatrix.eig} fit
\begin{verbatim}
rvector srsmatrix::eig (srmatrix& mEigVect) const throw (cvmexception);
rvector srsmatrix::eig () const throw (cvmexception);
\end{verbatim}
solve the 
\GO{symmetric eigenvalue problem}{rvector.eig}
and return  real vector with eigenvalues
of a calling symmetric matrix.
The first version
sets  output parameter \verb"mEigVect" to be equal
to  square matrix containing orthogonal
eigenvectors as columns.
All  functions
throw 
 \GOT{cvmexception}{cvmexception} 
in case of inappropriate sizes of the operands
or in case of convergence error.
See also
\GOT{rvector}{rvector},
\GOT{srmatrix}{srmatrix} and
\GOT{srsmatrix}{srsmatrix}.
Example:
\begin{Verbatim}
using namespace cvm;

std::cout.setf (std::ios::scientific | std::ios::left); 
std::cout.precision (10);
double a[] = {1., 2., 1., 2., 0., -1., 1., -1., 2.};
const srsmatrix m(a, 3);
srmatrix me(3);
rvector v(3);

v = m.eig(me);
std::cout << v << std::endl;

std::cout << m * me(1) - me(1) * v(1);
std::cout << m * me(2) - me(2) * v(2);
std::cout << m * me(3) - me(3) * v(3);
\end{Verbatim}
prints
\begin{Verbatim}
-2.0489173395e+000 2.3568958679e+000 2.6920214716e+000

4.4408920985e-016 0.0000000000e+000 5.5511151231e-016
-1.1102230246e-016 2.2204460493e-016 2.2204460493e-016
0.0000000000e+000 -1.1102230246e-016 -4.4408920985e-016
\end{Verbatim}
\newpage



\subsubsection{Cholesky}
Function%
\pdfdest name {srsmatrix.cholesky} fit
\begin{verbatim}
srmatrix srsmatrix::cholesky () const throw (cvmexception);
\end{verbatim}
forms the Cholesky factorization of  symmetric positive-definite 
matrix $A$ defined as
\begin{equation*}
A=U^T U,
\end{equation*}
where $U$ is upper triangular matrix.
It utilizes one of \verb"?POTRF" routines of the
\URL{LAPACK library}{http://www.netlib.org/lapack}.
Function
creates an object of type \verb"srmatrix" as 
 factorization of a calling matrix.
Function
throws 
 \GOT{cvmexception}{cvmexception} 
in case of convergence error.
See also
\GOT{srmatrix}{srmatrix} and
\GOT{srsmatrix}{srsmatrix}.
Example:
\begin{Verbatim}
using namespace cvm;

try {
    double a[] = {1., 2., 1., 2., 5., -1., 1., -1., 20.};
    const srsmatrix m(a, 3);
    
    srmatrix h = m.cholesky();
    std::cout << h << std::endl;
    std::cout << ~h * h - m;
}
catch (std::exception& e) {
    std::cout << "Exception: " << e.what () << std::endl;
}
\end{Verbatim}
prints
\begin{Verbatim}
1 2 1
0 1 -3
0 0 3.16228

0 0 0
0 0 0
0 0 0
\end{Verbatim}
\newpage



\subsubsection{Bunch-Kaufman}
Function%
\pdfdest name {srsmatrix.bunch_kaufman} fit
\begin{verbatim}
srmatrix srsmatrix::bunch_kaufman () throw (cvmexception);
\end{verbatim}
forms the Bunch-Kaufman factorization of a calling symmetric matrix 
(cited from the MKL library documentation):
\begin{equation*}
A=PUDU^TP^T,
\end{equation*}
where $A$ is calling symmetric matrix,
$P$ is  permutation matrix, $U$ and $L$ are upper and lower triangular
matrices with unit diagonal, and $D$ is  symmetric 
block-diagonal matrix with $1$-by-$1$ and $2$-by-$2$
diagonal blocks. $U$ and $L$ have $2$-by-$2$ unit diagonal 
blocks corresponding to the $2$-by-$2$ blocks of $D$.
It utilizes one of \verb"?SYTRF" routines of the
\URL{LAPACK library}{http://www.netlib.org/lapack}.
Function
creates an object of type \verb"srmatrix" as 
 factorization of a calling matrix.
Function throws 
 \GOT{cvmexception}{cvmexception} 
in case of convergence error.
See also
\GOT{srmatrix}{srmatrix} and
\GOT{srsmatrix}{srsmatrix}.
Function is mostly designed to be used for subsequent calls
of \verb"?SYTRS", \verb"?SYCON" and \verb"?SYTRI" routines of the
\URL{LAPACK library}{http://www.netlib.org/lapack}.
Currently it's used internally in \GOT{srmatrix::det}{srmatrix.det} flow
when argument is symmetric but not positive-definite.
\newpage





\subsubsection{identity}
Function%
\pdfdest name {srsmatrix.identity} fit
\begin{verbatim}
srsmatrix& srsmatrix::identity();
\end{verbatim}
sets  calling symmetric matrix to be equal to identity matrix
and returns a reference to
the matrix changed. 
See also \GOT{srsmatrix}{srsmatrix}.
Example:
\begin{Verbatim}
using namespace cvm;

std::cout.setf (std::ios::scientific | std::ios::left); 
std::cout.precision (3);
srsmatrix m(3);
m.randomize(0.,1.);

std::cout << m << std::endl;
std::cout << m.identity();
\end{Verbatim}
prints
\begin{Verbatim}
1.329e-001 8.527e-001 3.110e-001
8.527e-001 6.152e-001 3.247e-001
3.110e-001 3.247e-001 9.145e-001

1.000e+000 0.000e+000 0.000e+000
0.000e+000 1.000e+000 0.000e+000
0.000e+000 0.000e+000 1.000e+000
\end{Verbatim}
\newpage



\subsubsection{vanish}
Function%
\pdfdest name {srsmatrix.vanish} fit
\begin{verbatim}
srsmatrix& srsmatrix::vanish();
\end{verbatim}
sets every element of a calling symmetric 
matrix to be equal to zero
and returns a reference to
the matrix changed.
See also \GOT{srsmatrix}{srsmatrix}.
Example:
\begin{Verbatim}
using namespace cvm;

std::cout.setf (std::ios::scientific | std::ios::left); 
std::cout.precision (3);
srsmatrix m(3);
m.randomize(0.,1.);

std::cout << m << std::endl;
std::cout << m.vanish();
\end{Verbatim}
prints
\begin{Verbatim}
1.422e-001 1.477e-001 1.445e-001
1.477e-001 8.893e-001 1.669e-002
1.445e-001 1.669e-002 7.766e-001

0.000e+000 0.000e+000 0.000e+000
0.000e+000 0.000e+000 0.000e+000
0.000e+000 0.000e+000 0.000e+000
\end{Verbatim}
\newpage




\subsubsection{randomize}
Function%
\pdfdest name {srsmatrix.randomize} fit
\begin{verbatim}
srsmatrix& srsmatrix::randomize (TR dFrom, TR dTo);
\end{verbatim}
fills  calling symmetric matrix with pseudo-random 
numbers distributed between
\verb"dFrom" and \verb"dTo" keeping it to be symmetric.
Function
returns a reference to the matrix changed.
See also
\GOT{srsmatrix}{srsmatrix}.
Example:
\begin{Verbatim}
using namespace cvm;

std::cout.setf (std::ios::scientific | std::ios::left); 
std::cout.precision (7);

srsmatrix m(3);
m.randomize(-2.,3.);
std::cout << m;
\end{Verbatim}
prints
\begin{Verbatim}
-1.2277291e+000 3.6610004e-001 2.1380047e+000
3.6610004e-001 1.0336924e+000 -1.8565630e+000
2.1380047e+000 -1.8565630e+000 1.7774285e+000
\end{Verbatim}
\newpage

