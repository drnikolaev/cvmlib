\bigskip
\noindent
\verb"template <typename TR, typename TC>"\\
\verb"class cmatrix : public Matrix <TR,TC> {"\\
\verb"public:"\\
\verb"    "\GOT{cmatrix}{cmatrix.cmatrix ()}\verb" ();"\\
\verb"    "\GOT{cmatrix}{cmatrix.cmatrix (int,int)}\verb" (int nM, int nN);"\\
\verb"    "\GOT{cmatrix}{cmatrix.cmatrix (TC*,int,int)}\verb" (TC* pD, int nM, int nN);"\\
\verb"    "\GOT{cmatrix}{cmatrix.cmatrix (const TC*,int,int)}\verb" (const TC* pD, int nM, int nN);"\\
\verb"    "\GOT{cmatrix}{cmatrix.cmatrix (const cmatrix&)}\verb" (const cmatrix& m);"\\
\verb"    explicit "\GOT{cmatrix}{cmatrix.cmatrix (const cvector&,bool)}\verb" (const cvector& v, bool bBeColumn = true);"\\
\verb"    explicit "\GOT{cmatrix}{cmatrix.cmatrix (const rmatrix&,const bool)}\verb" (const rmatrix& m, bool bRealPart = true);"\\
\verb"    "\GOT{cmatrix}{cmatrix.cmatrix (TR*,TR*,int,int)}\verb" (const TR* pRe, const TR* pIm, int nM, int nN);"\\
\verb"    "\GOT{cmatrix}{cmatrix.cmatrix (const rmatrix&, const rmatrix&)}\verb" (const rmatrix& mRe, const rmatrix& mIm);"\\
\verb"    "\GOT{cmatrix}{cmatrix.submatrixctr}\verb" (cmatrix& m, int nRow, int nCol, int nHeight, int nWidth);"\\
\verb"    TC& "\GOT{operator ()}{cmatrix.operator (,)}\verb" (int im, int in) throw (cvmexception);"\\
\verb"    TC "\GOT{operator ()}{cmatrix.operator (,)}\verb" (int im, int in) const throw (cvmexception);"\\
\verb"    cvector "\GOT{operator ()}{cmatrix.operator ()}\verb" (int i) throw (cvmexception);"\\
\verb"    const cvector "\GOT{operator ()}{cmatrix.operator ()}\verb" (int i) const throw (cvmexception);"\\
\verb"    cvector "\GOT{operator []}{cmatrix.operator []}\verb" (int i) throw (cvmexception);"\\
\verb"    const cvector "\GOT{operator []}{cmatrix.operator []}\verb" (int i) const throw (cvmexception);"\\
\verb"    cvector "\GOT{diag}{cmatrix.diag}\verb" (int i) throw (cvmexception);"\\
\verb"    const cvector "\GOT{diag}{cmatrix.diag}\verb" (int i) const throw (cvmexception);"\\
\verb"    const rmatrix "\GOT{real}{cmatrix.real}\verb" () const;"\\
\verb"    const rmatrix "\GOT{imag}{cmatrix.imag}\verb" () const;"\\
\verb"    cmatrix& "\GOT{operator =}{cmatrix.operator = (const cmatrix&)}\verb" (const cmatrix& m) throw (cvmexception);"\\
\verb"    cmatrix& "\GOT{assign}{cmatrix.assign}\verb" (const cvector& v) throw (cvmexception);"\\
\verb"    cmatrix& "\GOT{assign}{cmatrix.assign}\verb" (const TC* pD);"\\
\verb"    rmatrix& "\GOT{assign}{cmatrix.assign (int, int, const cmatrix&)}\verb" (int nRow, int nCol, const cmatrix& m)"\\
\verb"                     throw (cvmexception);"\\
\verb"    cmatrix& "\GOT{set}{cmatrix.set}\verb" (TC z);"\\
\verb"    cmatrix& "\GOT{assign\_real}{cmatrix.assignreal}\verb" (const rmatrix& mRe) throw (cvmexception);"\\
\verb"    cmatrix& "\GOT{assign\_imag}{cmatrix.assignimag}\verb" (const rmatrix& mIm) throw (cvmexception);"\\
\verb"    cmatrix& "\GOT{set\_real}{cmatrix.setreal}\verb" (TR d);"\\
\verb"    cmatrix& "\GOT{set\_imag}{cmatrix.setimag}\verb" (TR d);"\\
\verb"    cmatrix& "\GOT{resize}{cmatrix.resize}\verb" (int nNewM, int nNewN) throw (cvmexception);"\\
\verb"    bool "\GOT{operator ==}{cmatrix.operator ==}\verb" (const cmatrix& v) const;"\\
\verb"    bool "\GOT{operator !=}{cmatrix.operator !=}\verb" (const cmatrix& v) const;"\\
\verb"    cmatrix& "\GOT{operator <{}<}{cmatrix.operator <<}\verb" (const cmatrix& m) throw (cvmexception);"\\
\verb"    cmatrix "\GOT{operator +}{cmatrix.operator +}\verb" (const cmatrix& m) const throw (cvmexception);"\\
\verb"    cmatrix "\GOT{operator -}{cmatrix.operator -}\verb" (const cmatrix& m) const throw (cvmexception);"\\
\verb"    cmatrix& "\GOT{sum}{cmatrix.sum}\verb" (const cmatrix& m1,"\\
\verb"                  const cmatrix& m2) throw (cvmexception);"\\
\verb"    cmatrix& "\GOT{diff}{cmatrix.diff}\verb" (const cmatrix& m1,"\\
\verb"                   const cmatrix& m2) throw (cvmexception);"\\
\verb"    cmatrix& "\GOT{operator +=}{cmatrix.operator +=}\verb" (const cmatrix& m) throw (cvmexception);"\\
\verb"    cmatrix& "\GOT{operator -=}{cmatrix.operator -=}\verb" (const cmatrix& m) throw (cvmexception);"\\
\verb"    cmatrix "\GOT{operator -}{cmatrix.operator - ()}\verb" () const;"\\
\verb"    cmatrix "\GOT{operator *}{cmatrix.operator * (TR)}\verb" (TR d) const;"\\
\verb"    cmatrix "\GOT{operator /}{cmatrix.operator / (TR)}\verb" (TR d) const throw (cvmexception);"\\
\verb"    cmatrix "\GOT{operator *}{cmatrix.operator * (TC)}\verb" (TC z) const;"\\
\verb"    cmatrix "\GOT{operator /}{cmatrix.operator / (TC)}\verb" (TC z) const throw (cvmexception);"\\
\verb"    cmatrix& "\GOT{operator *=}{cmatrix.operator *= (TR)}\verb" (TR d);"\\
\verb"    cmatrix& "\GOT{operator /=}{cmatrix.operator /= (TR)}\verb" (TR d) throw (cvmexception);"\\
\verb"    cmatrix& "\GOT{operator *=}{cmatrix.operator *= (TC)}\verb" (TC z);"\\
\verb"    cmatrix& "\GOT{operator /=}{cmatrix.operator /= (TC)}\verb" (TC z) throw (cvmexception);"\\
\verb"    cmatrix& "\GOT{normalize}{cmatrix.normalize}\verb" ();"\\
\verb"    cmatrix "\GOT{operator \TildaT}{cmatrix.conj}\verb"() const throw (cvmexception);"\\
\verb"    cmatrix "\GOT{operator !}{cmatrix.transpose}\verb"() const throw (cvmexception);"\\
\verb"    cmatrix& "\GOT{conj}{cmatrix.conj}\verb" (const cmatrix& m) throw (cvmexception);"\\
\verb"    cmatrix& "\GOT{conj}{cmatrix.conj}\verb" () throw (cvmexception);"\\
\verb"    cmatrix& "\GOT{transpose}{cmatrix.transpose}\verb" (const cmatrix& m) throw (cvmexception);"\\
\verb"    cmatrix& "\GOT{transpose}{cmatrix.transpose}\verb" () throw (cvmexception);"\\
\verb"    cvector "\GOT{operator *}{cmatrix.operator * (const cvector&)}\verb" (const cvector& v) const"\\
\verb"                        throw (cvmexception);"\\
\verb"    cmatrix "\GOT{operator *}{cmatrix.operator * (const cmatrix&)}\verb" (const cmatrix& m) const"\\
\verb"                        throw (cvmexception);"\\
\verb"    cmatrix& "\GOT{mult}{cmatrix.mult}\verb" (const cmatrix& m1, const cmatrix& m2)"\\
\verb"                   throw (cvmexception);"\\
\verb"    cmatrix& "\GOT{rank1update\_u}{cmatrix.rank1update_u}\verb" (const rvector& vCol,"\\
\verb"                            const rvector& vRow) throw (cvmexception);"\\
\verb"    cmatrix& "\GOT{rank1update\_c}{cmatrix.rank1update_c}\verb" (const rvector& vCol,"\\
\verb"                            const rvector& vRow) throw (cvmexception);"\\
\verb"    cmatrix& "\GOT{swap\_rows}{cmatrix.swap_rows}\verb" (int n1, int n2) throw (cvmexception);"\\
\verb"    cmatrix& "\GOT{swap\_cols}{cmatrix.swap_cols}\verb" (int n1, int n2) throw (cvmexception);"\\
\verb"    cmatrix& "\GOT{solve}{cmatrix.solve}\verb" (const scmatrix& mA,"\\
\verb"                    const cmatrix& mB, TR& dErr)"\\
\verb"                    throw (cvmexception);"\\
\verb"    cmatrix& "\GOT{solve}{cmatrix.solve}\verb" (const scmatrix& mA,"\\
\verb"                    const cmatrix& mB) throw (cvmexception);"\\
\verb"    cmatrix& "\GOT{solve\_tran}{cmatrix.solvetran}\verb" (const scmatrix& mA,"\\
\verb"                         const cmatrix& mB, TR& dErr)"\\
\verb"                         throw (cvmexception);"\\
\verb"    cmatrix& "\GOT{solve\_tran}{cmatrix.solvetran}\verb" (const scmatrix& mA,"\\
\verb"                         const cmatrix& mB) throw (cvmexception);"\\
\verb"    cmatrix& "\GOT{solve\_conj}{cmatrix.solveconj}\verb" (const scmatrix& mA,"\\
\verb"                         const cmatrix& mB, TR& dErr)"\\
\verb"                         throw (cvmexception);"\\
\verb"    cmatrix& "\GOT{solve\_conj}{cmatrix.solveconj}\verb" (const scmatrix& mA,"\\
\verb"                         const cmatrix& mB) throw (cvmexception);"\\
\verb"    cmatrix& "\GOT{solve\_lu}{cmatrix.solvelu}\verb" (const scmatrix& mA, const scmatrix& mLU,"\\
\verb"                       const int* pPivots, const cmatrix& mB, TR& dErr)"\\
\verb"                       throw (cvmexception);"\\
\verb"    cmatrix& "\GOT{solve\_lu}{cmatrix.solvelu}\verb" (const scmatrix& mA, const scmatrix& mLU,"\\
\verb"                       const int* pPivots, const cmatrix& mB)"\\
\verb"                       throw (cvmexception);"\\
\verb"    rvector "\GOT{svd}{cmatrix.svd}\verb" () throw (cvmexception);"\\
\verb"    rvector "\GOT{svd}{cmatrix.svd}\verb" (cmatrix& mLSingVect, cmatrix& mRSingVect)"\\
\verb"                 throw (cvmexception);"\\
\verb"    cmatrix "\GOT{pinv}{cmatrix.pinv}\verb" (TR threshold = cvmMachSp()) const"\\
\verb"                  throw (cvmexception);"\\
\verb"    cmatrix& "\GOT{pinv}{cmatrix.pinv}\verb" (const cmatrix& mA,"\\
\verb"                   TR threshold = cvmMachSp())"\\
\verb"                   throw (cvmexception);"\\
\verb"    int "\GOT{rank}{cmatrix.rank}\verb" (TR eps = cvmMachSp()) const throw (cvmexception);"\\
\verb"    cmatrix "\GOT{gels}{cmatrix.gels}\verb" (bool conjugate, const cmatrix& mB,"\\
\verb"                  cvector& vErr) const throw (cvmexception);"\\
\verb"    cmatrix& "\GOT{gels}{cmatrix.gels}\verb" (bool conjugate, const cmatrix& mA, const cmatrix& mB,"\\
\verb"                   cvector& vErr) throw (cvmexception);"\\
\verb"    cvector "\GOT{gels}{cmatrix.gels}\verb" (bool conjugate, const cvector& mB,"\\
\verb"                   TC& dErr) const throw (cvmexception);"\\
\verb"    cmatrix "\GOT{gelsy}{cmatrix.gelsy}\verb" (const cmatrix& mB, int& rank,"\\
\verb"                   TR tol = cvmMachSp()) const throw (cvmexception);"\\
\verb"    cmatrix& "\GOT{gelsy}{cmatrix.gelsy}\verb" (const cmatrix& mA, const cmatrix& mB, int& rank,"\\
\verb"                   TR tol = cvmMachSp()) throw (cvmexception);"\\
\verb"    cvector "\GOT{gelsy}{cmatrix.gelsy}\verb" (const cvector& mB, int& rank,"\\
\verb"                   TR tol = cvmMachSp()) const throw (cvmexception);"\\
\verb"    rmatrix "\GOT{gelss}{cmatrix.gelss}\verb" (const cmatrix& mB, rvector& sv,"\\
\verb"                   int& rank, TR tol = cvmMachSp()) const"\\
\verb"                   throw (cvmexception);"\\
\verb"    rmatrix& "\GOT{gelss}{cmatrix.gelss}\verb" (const cmatrix& mA, const cmatrix& mB, rvector& sv,"\\
\verb"                   int& rank, TR tol = cvmMachSp())"\\
\verb"                   throw (cvmexception);"\\
\verb"    rvector "\GOT{gelss}{cmatrix.gelss}\verb" (const cvector& mB, rvector& sv,"\\
\verb"                   int& rank, TR tol = cvmMachSp()) const"\\
\verb"                   throw (cvmexception);"\\
\verb"    rmatrix "\GOT{gelsd}{cmatrix.gelsd}\verb" (const cmatrix& mB, rvector& sv,"\\
\verb"                   int& rank, TR tol = cvmMachSp()) const"\\
\verb"                   throw (cvmexception);"\\
\verb"    rmatrix& "\GOT{gelsd}{cmatrix.gelsd}\verb" (const cmatrix& mA, const cmatrix& mB, rvector& sv,"\\
\verb"                   int& rank, TR tol = cvmMachSp())"\\
\verb"                   throw (cvmexception);"\\
\verb"    rvector "\GOT{gelsd}{cmatrix.gelsd}\verb" (const cvector& mB, rvector& sv,"\\
\verb"                   int& rank, TR tol = cvmMachSp()) const"\\
\verb"                   throw (cvmexception);"\\
\verb"    void "\GOT{qr}{cmatrix.qr}\verb"(cmatrix& mQ, scmatrix& mR) const throw (cvmexception);"\\
\verb"    void "\GOT{qr}{cmatrix.qr}\verb"(scmatrix& mQ, cmatrix& mR) const throw (cvmexception);"\\
\verb"    void "\GOT{lq}{cmatrix.lq}\verb"(scmatrix& mL, cmatrix& mQ) const throw (cvmexception);"\\
\verb"    void "\GOT{lq}{cmatrix.lq}\verb"(cmatrix& mL, scmatrix& mQ) const throw (cvmexception);"\\
\verb"    void "\GOT{rq}{cmatrix.rq}\verb"(scmatrix& mR, cmatrix& mQ) const throw (cvmexception);"\\
\verb"    void "\GOT{rq}{cmatrix.rq}\verb"(cmatrix& mR, scmatrix& mQ) const throw (cvmexception);"\\
\verb"    void "\GOT{ql}{cmatrix.ql}\verb"(cmatrix& mQ, scmatrix& mL) const throw (cvmexception);"\\
\verb"    void "\GOT{ql}{cmatrix.ql}\verb"(scmatrix& mQ, cmatrix& mL) const throw (cvmexception);"\\
\verb"    cmatrix& "\GOT{vanish}{cmatrix.vanish}\verb" ();"\\
\verb"    cmatrix& "\GOT{geru}{cmatrix.geru}\verb" (TC alpha, const cvector& vCol,"\\
\verb"                   const cvector& vRow) throw (cvmexception);"\\
\verb"    cmatrix& "\GOT{gerc}{cmatrix.gerc}\verb" (TC alpha, const cvector& vCol,"\\
\verb"                   const cvector& vRow) throw (cvmexception);"\\
\verb"    cmatrix& "\GOT{gemm}{cmatrix.gemm}\verb" (const cmatrix& m1, bool bTrans1,"\\
\verb"                   const cmatrix& m2, bool bTrans2,"\\
\verb"                   TC dAlpha, TC dBeta) throw (cvmexception);"\\
\verb"    cmatrix& "\GOT{hemm}{cmatrix.hemm}\verb" (bool bLeft, const schmatrix& ms, const cmatrix& m,"\\
\verb"                   TC dAlpha, TC dBeta) throw (cvmexception);"\\
\verb"    cmatrix& "\GOT{randomize\_real}{cmatrix.randomizereal}\verb" (TR dFrom, TR dTo);"\\
\verb"    cmatrix& "\GOT{randomize\_imag}{cmatrix.randomizeimag}\verb" (TR dFrom, TR dTo);"\\
\verb"};"
\newpage




\subsubsection{cmatrix ()}
Constructor%
\pdfdest name {cmatrix.cmatrix ()} fit
\begin{verbatim}
cmatrix::cmatrix ();
\end{verbatim}
creates  empty \verb"cmatrix" object.
See also \GOT{cmatrix}{cmatrix}.
Example:
\begin{Verbatim}
using namespace cvm;

cmatrix m;
std::cout << m.msize() << std::endl 
          << m.nsize() << std::endl;

m << eye_complex(3);
std::cout << m;
\end{Verbatim}
prints
\begin{Verbatim}
0
0
(1,0) (0,0) (0,0)
(0,0) (1,0) (0,0)
(0,0) (0,0) (1,0)
\end{Verbatim}
\newpage



\subsubsection{cmatrix (int,int)}
Constructor%
\pdfdest name {cmatrix.cmatrix (int,int)} fit
\begin{verbatim}
cmatrix::cmatrix (int nM, int nN);
\end{verbatim}
creates  $m\times n$ \verb"cmatrix" object where $m$ is passed in
\verb"nM" parameter (number of rows) and $n$ is passed in
\verb"nN" (number of columns).
Constructor throws  \GOT{cvmexception}{cvmexception}
in case of non-positive sizes passed or memory allocation failure.
See also \GOT{cmatrix}{cmatrix}.
Example:
\begin{Verbatim}
using namespace cvm;

cmatrix m (3, 4);
std::cout << m.msize() << std::endl 
          << m.nsize() << std::endl 
          << m.size()  << std::endl << m;
\end{Verbatim}
prints
\begin{Verbatim}
3
4
12
(0,0) (0,0) (0,0) (0,0)
(0,0) (0,0) (0,0) (0,0)
(0,0) (0,0) (0,0) (0,0)
\end{Verbatim}
\newpage




\subsubsection{cmatrix (TC*,int,int)}
Constructor%
\pdfdest name {cmatrix.cmatrix (TC*,int,int)} fit
\begin{verbatim}
cmatrix::cmatrix (TC* pD, int nM, int nN);
\end{verbatim}
creates  $m\times n$ \verb"cmatrix" object where $m$ is passed in
\verb"nM" parameter (number of rows) and $n$ is passed in
\verb"nN" (number of columns).
Unlike others, this constructor \textit{does not allocate memory}.
It just shares  memory with  array pointed to by \verb"pD".
Constructor throws  \GOT{cvmexception}{cvmexception}
in case of non-positive size passed.
See also \GOT{cmatrix}{cmatrix}, \GOT{cmatrix (const TC*,int,int)}{cmatrix.cmatrix (const TC*,int,int)}.
Example:
\begin{Verbatim}
using namespace cvm;
double a[] = {5., 5., 5., 5., 5., 5.,
              5., 5., 5., 5., 5., 5.};
cmatrix m ((std::complex<double>*) a, 2, 3);

m(1,1) = std::complex<double>(1.,2.);
std::cout << m << std::endl;
std::cout << a[0] << " " << a[1] << " "
          << a[2] << " " << a[3] << std::endl;
\end{Verbatim}
prints
\begin{Verbatim}
(1,2) (5,5) (5,5)
(5,5) (5,5) (5,5)

1 2 5 5
\end{Verbatim}
\newpage



\subsubsection{cmatrix (const TC*,int,int)}
Constructor%
\pdfdest name {cmatrix.cmatrix (const TC*,int,int)} fit
\begin{verbatim}
cmatrix::cmatrix (const TC* pD, int nM, int nN);
\end{verbatim}
creates  $m\times n$ \verb"cmatrix" object where $m$ is passed in
\verb"nM" parameter (number of rows) and $n$ is passed in
\verb"nN" (number of columns)
and copies $m*n$ elements of  array \verb"pD" to it \GOT{by colums}{SubSubSectionStorage}.
Constructor throws  \GOT{cvmexception}{cvmexception}
in case of non-positive size passed.
See also \GOT{cmatrix}{cmatrix}, \GOT{cmatrix (TC*,int,int)}{cmatrix.cmatrix (TC*,int,int)}.
Example:
\begin{Verbatim}
using namespace cvm;
const double a[] = {5., 5., 5., 5., 5., 5.,
                    5., 5., 5., 5., 5., 5.};
cmatrix m ((const std::complex<double>*) a, 2, 3);

m(1,1) = std::complex<double>(1.,2.);
std::cout << m << std::endl;
std::cout << a[0] << " " << a[1] << " "
          << a[2] << " " << a[3] << std::endl;
\end{Verbatim}
prints
\begin{Verbatim}
(1,2) (5,5) (5,5)
(5,5) (5,5) (5,5)

5 5 5 5
\end{Verbatim}
\newpage



\subsubsection{cmatrix (const cmatrix\&)}
Copy constructor%
\pdfdest name {cmatrix.cmatrix (const cmatrix&)} fit
\begin{verbatim}
cmatrix::cmatrix (const cmatrix& m);
\end{verbatim}
creates  \verb"cmatrix" object as a copy of \verb"m".
Constructor throws  \GOT{cvmexception}{cvmexception}
in case of memory allocation failure.
See also \GOT{cmatrix}{cmatrix}.
Example:
\begin{Verbatim}
using namespace cvm;

double a[] = {1., 2., 3., 4., 5., 6.,
              7., 8., 9., 10., 11., 12.};
cmatrix m ((std::complex<double>*) a, 2, 3);
cmatrix mc(m);

m(1,1) = std::complex<double>(7.77,7.77);
std::cout << m << std::endl << mc;
\end{Verbatim}
prints
\begin{Verbatim}
(7.77,7.77) (5,6) (9,10)
(3,4) (7,8) (11,12)

(1,2) (5,6) (9,10)
(3,4) (7,8) (11,12)
\end{Verbatim}
\newpage




\subsubsection{cmatrix (const cvector\&,bool)}
Constructor%
\pdfdest name {cmatrix.cmatrix (const cvector&,bool)} fit
\begin{verbatim}
explicit cmatrix::cmatrix (const cvector& v, bool bBeColumn = true);
\end{verbatim}
creates  \verb"cmatrix" object containing \verb"v.size()" rows and
1 column if \verb"bBeColumn" is \verb"true" or 1 row and
\verb"v.size()" columns otherwise.
After that it copies vector \verb"v"'s elements to
the matrix created.
Constructor throws  \GOT{cvmexception}{cvmexception}
in case of memory allocation failure.
See also \GOT{cmatrix}{cmatrix}, \GOT{cvector}{cvector}.
Example:
\begin{Verbatim}
using namespace cvm;

cvector v(3);
v(1) = std::complex<double>(1.,2.);
v(2) = std::complex<double>(2.,3.);
v(3) = std::complex<double>(3.,4.);

cmatrix mc (v);
cmatrix mr (v, false);

std::cout << mc << std::endl << mr;
\end{Verbatim}
prints
\begin{Verbatim}
(1,2)
(2,3)
(3,4)

(1,2) (2,3) (3,4)
\end{Verbatim}
\newpage




\subsubsection{cmatrix (const rmatrix\&,bool)}
Constructor%
\pdfdest name {cmatrix.cmatrix (const rmatrix&,const bool)} fit
\begin{verbatim}
explicit cmatrix::cmatrix (const rmatrix& m, bool bRealPart = true);
\end{verbatim}
creates  \verb"cmatrix" object containing \verb"m.msize()" rows and
\verb"m.nsize()" columns
and copies matrix \verb"m" to its real part if
\verb"bRealPart" is \verb"true" or
to its imaginary part otherwise.
Constructor throws  \GOT{cvmexception}{cvmexception}
in case of memory allocation failure.
See also \GOT{cmatrix}{cmatrix}, \GOT{rmatrix}{rmatrix}.
Example:
\begin{Verbatim}
using namespace cvm;

double a[] = {1., 2., 3., 4., 5., 6.};
const rmatrix m (a, 2, 3);
cmatrix mr(m), mi(m, false);

std::cout << mr << std::endl << mi;
\end{Verbatim}
prints
\begin{Verbatim}
(1,0) (3,0) (5,0)
(2,0) (4,0) (6,0)

(0,1) (0,3) (0,5)
(0,2) (0,4) (0,6)
\end{Verbatim}
\newpage




\subsubsection{cmatrix (const TR*,const TR*,int,int)}
Constructor%
\pdfdest name {cmatrix.cmatrix (TR*,TR*,int,int)} fit
\begin{verbatim}
cmatrix::cmatrix (const TR* pRe, const TR* pIm,
                  const int nM, const int nN);
\end{verbatim}
creates  \verb"cmatrix" object
of size \verb"nM" by \verb"nN" and copies every
element of arrays pointed to by \verb"pRe" and \verb"pIm"
to  real and imaginary part of the matrix created respectively.
Use \verb"NULL" pointer to fill up appropriate
part with zero values.
Constructor throws  \GOT{cvmexception}{cvmexception}
in case of memory allocation failure.
See also \GOT{cmatrix}{cmatrix}.
Example:
\begin{Verbatim}
using namespace cvm;

double re[] = {1., 2., 3., 4., 5., 6.};
double im[] = {6., 5., 4., 3., 2., 1.};
cmatrix m (re, im, 3, 2);
std::cout << m << std::endl;
re[1] = 7.77;
std::cout << m << std::endl;

const double re2[] = {1., 2., 3., 4.};
const cmatrix m2 (re2, NULL, 2, 2);
std::cout << m2;
\end{Verbatim}
prints
\begin{Verbatim}
(1,6) (4,3)
(2,5) (5,2)
(3,4) (6,1)

(1,6) (4,3)
(2,5) (5,2)
(3,4) (6,1)

(1,0) (3,0)
(2,0) (4,0)
\end{Verbatim}
\newpage




\subsubsection{cmatrix (const rmatrix\&, const rmatrix\&)}
Constructor%
\pdfdest name {cmatrix.cmatrix (const rmatrix&, const rmatrix&)} fit
\begin{verbatim}
cmatrix::cmatrix (const rmatrix& mRe, const rmatrix& mIm);
\end{verbatim}
creates  \verb"cmatrix" object
of size \verb"mRe.msize()" by \verb"mRe.nsize()"
(if one of these sizes differs from appropriate size of matrix
\verb"mIm" then  constructor throws \GOT{cvmexception}{cvmexception})
and copies matrices \verb"mRe" and \verb"mIm"
to  real and imaginary part of the matrix created respectively.
Constructor throws  \GOT{cvmexception}{cvmexception}
in case of memory allocation failure.
See also \GOT{cmatrix}{cmatrix}, \GOT{rmatrix}{rmatrix}.
Example:
\begin{Verbatim}
using namespace cvm;

rmatrix mr(3,3), mi(3,3);
mr(1,1) = 1.;
mr(2,2) = 2.;
mr(3,3) = 3.;
mi(1,3) = 6.;
mi(2,2) = 5.;
mi(3,1) = 4.;

const cmatrix mc(mr, mi);
std::cout << mc;
\end{Verbatim}
prints
\begin{Verbatim}
(1,0) (0,0) (0,6)
(0,0) (2,5) (0,0)
(0,4) (0,0) (3,0)
\end{Verbatim}
\newpage




\subsubsection{submatrix}
Submatrix constructor%
\pdfdest name {cmatrix.submatrixctr} fit
\begin{verbatim}
cmatrix::cmatrix (cmatrix& m, int nRow, int nCol, 
                              int nHeight, int nWidth);
\end{verbatim}
creates  \verb"cmatrix" object as  \emph{submatrix} of \verb"m".
It means that the matrix object created shares memory with some part
of \verb"m". This part is defined by its upper left corner (parameters 
\verb"nRow" and \verb"nCol", both are \Based)
and its height and width (parameters 
\verb"nHeight" and \verb"nWidth").
See also \GOT{cmatrix}{cmatrix}.
Example:
\begin{Verbatim}
using namespace cvm;

cmatrix m(4,5);
cmatrix subm(m, 2, 2, 2, 2);
subm.set(std::complex<double>(1.,1.));

std::cout << m;
\end{Verbatim}
prints
\begin{Verbatim}
(0,0) (0,0) (0,0) (0,0) (0,0)
(0,0) (1,1) (1,1) (0,0) (0,0)
(0,0) (1,1) (1,1) (0,0) (0,0)
(0,0) (0,0) (0,0) (0,0) (0,0)
\end{Verbatim}
\newpage



\subsubsection{operator (,)}
Indexing operators%
\pdfdest name {cmatrix.operator (,)} fit
\begin{verbatim}
TC& cmatrix::operator () (int im, int in) throw (cvmexception);
TC cmatrix::operator () (int im, int in) const throw (cvmexception);
\end{verbatim}
provide access to a particular element of calling matrix. The first version
of operator is applicable to  non-constant object.
This version returns  \emph{l-value}
in order to make possible write access to an element.
Both operators are \Based.
Operators throw \GOT{cvmexception}{cvmexception}
if \verb"im" is outside of \verb"[1,msize()]" range or
\verb"in" is outside of \verb"[1,nsize()]" range.
Operators are \emph{inherited}
in the classes
\GOT{scmatrix}{scmatrix}, \GOT{scbmatrix}{scbmatrix} 
and \GOT{schmatrix}{schmatrix}.
See also \GOT{cmatrix}{cmatrix}.
Example:
\begin{Verbatim}
using namespace cvm;

try {
    double a[] = {1., 2., 3., 4., 5., 6.,
                  7., 8., 9., 10., 11., 12.};
    const cmatrix m ((std::complex<double>*) a, 2, 3);
    scmatrix ms(2);
    std::cout << m(1,1) << " " 
              << m(2,3) << std::endl << std::endl;
    ms(2,2) = std::complex<double>(7.77,7.77);
    std::cout << ms;
}
catch (std::exception& e) {
    std::cout << "Exception " << e.what () << std::endl;
}
\end{Verbatim}
prints
\begin{Verbatim}
(1,2) (11,12)

(0,0) (0,0)
(0,0) (7.77,7.77)
\end{Verbatim}
\newpage




\subsubsection{operator ()}
Indexing operators%
\pdfdest name {cmatrix.operator ()} fit
\begin{verbatim}
cvector cmatrix::operator () (int i) throw (cvmexception);
const cvector cmatrix::operator () (int i) const throw (cvmexception);
\end{verbatim}
provide access to \hbox{$i$-th} column of calling matrix. The first version
of operator is applicable to  non-constant object and
\emph{returns  l-value}, i.e. vector returned shares memory
with \hbox{$i$-th} column of calling matrix
in order to make possible write access to it.
The second version creates  \emph{copy} of a column and therefore is
\emph{not  l-value}.
Both operators are \Based.
The operators throw \GOT{cvmexception}{cvmexception}
if parameter \verb"i" is outside of \verb"[1,nsize()]" range.
Operators are \emph{inherited}
in the classes
\GOT{scmatrix}{scmatrix}, \GOT{scbmatrix}{scbmatrix} 
and \GOT{schmatrix}{schmatrix}.
See also \GOT{cmatrix}{cmatrix},
\GOT{Matrix::msize()}{Matrix.msize}.
Example:
\begin{Verbatim}
using namespace cvm;

try {
    double a[] = {1., 2., 3., 4., 5., 6.,
                  7., 8., 9., 10., 11., 12.};
    const cmatrix m ((std::complex<double>*) a, 2, 3);
    scmatrix ms(2);

    std::cout << m(2) << std::endl;

    ms(2) = m(3);
    std::cout << ms;
}
catch (std::exception& e) {
    std::cout << "Exception " << e.what () << std::endl;
}
\end{Verbatim}
prints
\begin{Verbatim}
(5,6) (7,8)

(0,0) (9,10)
(0,0) (11,12)
\end{Verbatim}
\newpage




\subsubsection{operator []}
Indexing operators%
\pdfdest name {cmatrix.operator []} fit
\begin{verbatim}
cvector cmatrix::operator [] (int i) throw (cvmexception);
const cvector cmatrix::operator [] (int i) const throw (cvmexception);
\end{verbatim}
provide access to  \hbox{$i$-th} row of calling matrix. The first version
of operator is applicable to  non-constant object and
\emph{returns  l-value}, i.e. the vector returned shares memory
with \hbox{$i$-th} row of calling matrix
in order to make possible write access to it.
The second version creates  \emph{copy} of a row and therefore is
\emph{not  l-value}.
Both operators are \Based.
The operators throw \GOT{cvmexception}{cvmexception}
if \verb"i" is outside of \verb"[1,msize()]" range.
Operators are \emph{inherited}
in the classes
\GOT{scmatrix}{scmatrix}, \GOT{scbmatrix}{scbmatrix} 
and \GOT{schmatrix}{schmatrix}.
See also \GOT{cmatrix}{cmatrix},
\GOT{Matrix::msize()}{Matrix.msize}.
Example:
\begin{Verbatim}
using namespace cvm;

try {
    double a[] = {1., 2., 3., 4., 5., 6.,
                  7., 8., 9., 10., 11., 12.};
    const cmatrix m ((std::complex<double>*) a, 2, 3);
    scmatrix ms(3);
    std::cout << m[1] << std::endl;
    ms[1] = m[2];
    std::cout << ms;
}
catch (std::exception& e) {
    std::cout << "Exception " << e.what () << std::endl;
}
\end{Verbatim}
prints
\begin{Verbatim}
(1,2) (5,6) (9,10)

(3,4) (7,8) (11,12)
(0,0) (0,0) (0,0)
(0,0) (0,0) (0,0)
\end{Verbatim}
\newpage




\subsubsection{diag}
Functions%
\pdfdest name {cmatrix.diag} fit
\begin{verbatim}
cvector cmatrix::diag (int i) throw (cvmexception);
const cvector cmatrix::diag (int i) const throw (cvmexception);
\end{verbatim}
provide access to  \hbox{$i$-th} diagonal of calling matrix,
where $i=0$ for main diagonal, $i<0$ for lower diagonals 
and $i>0$ for upper ones.
The first version
of  function is applicable to  non-constant object and
\emph{returns  l-value}, i.e. the vector returned shares memory
with  \hbox{$i$-th} diagonal of calling matrix
in order to make possible write access to it.
The second version creates  \emph{copy} of a diagonal and therefore is
\emph{not  l-value}.
Functions throw \GOT{cvmexception}{cvmexception}
if  parameter \verb"i" is outside of 
\verb"[-msize()+1,nsize()-1]" range.
Functions are \emph{inherited}
in the classes
\GOT{scmatrix}{scmatrix} and \GOT{scbmatrix}{scbmatrix}.
Functions are \emph{redefined} in the class \GOT{schmatrix}{schmatrix}.
See also \GOT{cmatrix}{cmatrix}.
Example:
\begin{Verbatim}
using namespace cvm;

try {
    double a[] = {1., 2., 3., 4., 5., 6., 7., 8., 9., 
                  10., 11., 12., 13., 14., 15., 16., 17., 18.};
    cmatrix m (2, 3);
    const scmatrix ms((std::complex<double>*)a, 3);
    m.diag(-1).set(std::complex<double>(1.,1.));
    m.diag(0).set(std::complex<double>(2.,2.));
    m.diag(1).set(std::complex<double>(3.,3.));
    m.diag(2).set(std::complex<double>(4.,4.));
    std::cout << m << std::endl;
    std::cout << ms << std::endl;
    std::cout << ms.diag(0) << ms.diag(1);
}
catch (std::exception& e) {
    std::cout << "Exception " << e.what () << std::endl;
}
\end{Verbatim}
prints
\begin{Verbatim}
(2,2) (3,3) (4,4)
(1,1) (2,2) (3,3)

(1,2) (7,8) (13,14)
(3,4) (9,10) (15,16)
(5,6) (11,12) (17,18)

(1,2) (9,10) (17,18)
(7,8) (15,16)
\end{Verbatim}
\newpage



\subsubsection{real}
Function%
\pdfdest name {cmatrix.real} fit
\begin{verbatim}
const rmatrix cmatrix::real () const;
\end{verbatim}
creates  object of type \verb"const rmatrix"
as  real part
of  calling matrix.
Please note that, unlike
\GO{cvector::real}{cvector.real}, this
function creates new object \emph{not sharing}  memory
with real part of calling matrix, i.e.
the matrix returned is \emph{not  l-value}.
Function is \emph{redefined} in the classes
\GOT{scmatrix}{scmatrix}, \GOT{scbmatrix}{scbmatrix} and
\GOT{schmatrix}{schmatrix}.
See also
\GOT{rmatrix}{rmatrix},
\GOT{cmatrix}{cmatrix}.
Example:
\begin{Verbatim}
using namespace cvm;

double a[] = {1., 2., 3., 4., 5., 6.,
              7., 8., 9., 10., 11., 12.};
cmatrix m((std::complex<double>*) a, 2, 3);

std::cout << m << std::endl << m.real();
\end{Verbatim}
prints
\begin{Verbatim}
(1,2) (5,6) (9,10)
(3,4) (7,8) (11,12)

1 5 9
3 7 11
\end{Verbatim}
\newpage




\subsubsection{imag}
Function%
\pdfdest name {cmatrix.imag} fit
\begin{verbatim}
const rmatrix cmatrix::imag () const;
\end{verbatim}
creates  object of type \verb"const rmatrix"
as  imaginary part
of  calling matrix.
Please note that, unlike
\GO{cvector::imag}{cvector.imag}, this
function creates new object \emph{not sharing}  memory
with  imaginary part of  calling matrix, i.e.
the matrix returned is \emph{not  l-value}.
Function is \emph{redefined} in the classes
\GOT{scmatrix}{scmatrix}, \GOT{scbmatrix}{scbmatrix} and
\GOT{schmatrix}{schmatrix}.
See also
\GOT{rmatrix}{rmatrix},
\GOT{cmatrix}{cmatrix}.
Example:
\begin{Verbatim}
using namespace cvm;

double a[] = {1., 2., 3., 4., 5., 6.,
              7., 8., 9., 10., 11., 12.};
cmatrix m((std::complex<double>*) a, 2, 3);

std::cout << m << std::endl << m.imag();
\end{Verbatim}
prints
\begin{Verbatim}
(1,2) (5,6) (9,10)
(3,4) (7,8) (11,12)

2 6 10
4 8 12
\end{Verbatim}
\newpage





\subsubsection{operator = (const cmatrix\&)}
Operator%
\pdfdest name {cmatrix.operator = (const cmatrix&)} fit
\begin{verbatim}
cmatrix& cmatrix::operator = (const cmatrix& m) throw (cvmexception);
\end{verbatim}
sets  every element of  calling rmatrix to be equal to
appropriate element of  matrix \verb"m"
and returns  reference to
the matrix changed.
Operator throws  \GOT{cvmexception}{cvmexception}
in case of different matrix sizes.
Operator is \emph{redefined} in the classes
\GOT{scmatrix}{scmatrix}, \GOT{scbmatrix}{scbmatrix} 
and \GOT{schmatrix}{schmatrix}.
See also \GOT{cmatrix}{cmatrix}.
Example:
\begin{Verbatim}
using namespace cvm;

try {
    double a[] = {1., 2., 3., 4., 5., 6.,
                  7., 8., 9., 10., 11., 12.};
    const cmatrix m1((std::complex<double>*) a, 2, 3);
    cmatrix m2(2, 3);

    m2 = m1;
    std::cout << m2;
}
catch (std::exception& e) {
    std::cout << "Exception " << e.what () << std::endl;
}
\end{Verbatim}
prints
\begin{Verbatim}
(1,2) (5,6) (9,10)
(3,4) (7,8) (11,12)
\end{Verbatim}
\newpage




\subsubsection{assign (const TC*)}
Function%
\pdfdest name {cmatrix.assign} fit
\begin{verbatim}
cmatrix& cmatrix::assign (const cvector& v) throw (cvmexception);
cmatrix& cmatrix::assign (const TC* pD);
\end{verbatim}
sets every element of  calling matrix to  value of
appropriate element of a vector~\verb'v'
or  array pointed to by~\verb"pD"
and returns  reference to
the matrix changed.
In first version it's assumed that vector passed is long
enough to fill calling matrix. Function throws \GOT{cvmexception}{cvmexception} 
otherwise.
Function is \emph{redefined} in the classes
\GOT{scmatrix}{scmatrix}, \GOT{scbmatrix}{scbmatrix} 
and \GOT{schmatrix}{schmatrix}.
See also \GOT{cmatrix}{cmatrix}.
Example:
\begin{Verbatim}
using namespace cvm;

const double a[] = {1., 2., 3., 4., 5., 6.,
                    7., 8., 9., 10., 11., 12.};
cmatrix m(2, 3);

m.assign ((const std::complex<double>*) a);
std::cout << m;
\end{Verbatim}
prints
\begin{Verbatim}
(1,2) (5,6) (9,10)
(3,4) (7,8) (11,12)
\end{Verbatim}
\newpage




\subsubsection{assign (int, int, const cmatrix\&)}
Function%
\pdfdest name {cmatrix.assign (int, int, const cmatrix&)} fit
\begin{verbatim}
cmatrix& cmatrix::assign (int nRow, int nCol, const cmatrix& m)
throw (cvmexception);
\end{verbatim}
sets sub-matrix of  calling matrix beginning with \Based row
\verb"nRow" and column \verb"nCol" to  matrix \verb"m" and
returns  reference to the matrix changed. Function throws 
 \GOT{cvmexception}{cvmexception} if \verb"nRow"
or \verb"nCol" are not positive or matrix \verb"m" doesn't fit.
Function is \emph{redefined} in the classes
\GOT{scmatrix}{scmatrix}, \GOT{scbmatrix}{scbmatrix} 
and \GOT{schmatrix}{schmatrix}.
See also \GOT{cmatrix}{cmatrix}.
Example:
\begin{Verbatim}
using namespace cvm;

rmatrix m1(4,5);
rmatrix m2(2,2);
m1.set(1.);
m2.set(2.);
m1.assign(2,3,m2);
std::cout << m1;
\end{Verbatim}
prints
\begin{Verbatim}
(1,1) (1,1) (1,1) (1,1) (1,1)
(1,1) (1,1) (2,2) (2,2) (1,1)
(1,1) (1,1) (2,2) (2,2) (1,1)
(1,1) (1,1) (1,1) (1,1) (1,1)
\end{Verbatim}
\newpage



\subsubsection{set (TC)}
Function%
\pdfdest name {cmatrix.set} fit
\begin{verbatim}
cmatrix& cmatrix::set (TC z);
\end{verbatim}
sets every element of  calling matrix to  value of
parameter \verb"z" and returns  reference to
the matrix changed.
Use \GOT{vanish}{cmatrix.vanish} to set every element
of  calling matrix to be equal to zero.
Function is \emph{redefined} in the classes
\GOT{scmatrix}{scmatrix}, \GOT{scbmatrix}{scbmatrix} 
and \GOT{schmatrix}{schmatrix}.
See also \GOT{cmatrix}{cmatrix}.
Example:
\begin{Verbatim}
using namespace cvm;

cmatrix m(3, 4);
m.set(std::complex<double>(3.,4.));
std::cout << m;
\end{Verbatim}
prints
\begin{Verbatim}
(3,4) (3,4) (3,4) (3,4)
(3,4) (3,4) (3,4) (3,4)
(3,4) (3,4) (3,4) (3,4)
\end{Verbatim}
\newpage




\subsubsection{assign\_real}
Function%
\pdfdest name {cmatrix.assignreal} fit
\begin{verbatim}
cmatrix& cmatrix::assign_real (const rmatrix& mRe) throw (cvmexception);
\end{verbatim}
sets real part of every element of  calling matrix to  value of
appropriate element of  matrix \verb"mRe"
and returns  reference to
the matrix changed.
Function throws  \GOT{cvmexception}{cvmexception}
in case of different sizes of the operands.
See also \GOT{cmatrix}{cmatrix} and \GOT{rmatrix}{rmatrix}.
Function is \emph{redefined} in the classes
\GOT{scmatrix}{scmatrix}, \GOT{scbmatrix}{scbmatrix} 
and \GOT{schmatrix}{schmatrix}.
Example:
\begin{Verbatim}
using namespace cvm;

rmatrix m (2,3);
cmatrix mc(2,3);
m.randomize (0., 1.);

mc.assign_real(m);
std::cout << mc;
\end{Verbatim}
prints
\begin{Verbatim}
(0.126835,0) (0.57271,0) (0.28312,0)
(0.784417,0) (0.541673,0) (0.663869,0)
\end{Verbatim}
\newpage




\subsubsection{assign\_imag}
Function%
\pdfdest name {cmatrix.assignimag} fit
\begin{verbatim}
cmatrix& cmatrix::assign_imag (const rmatrix& mIm) throw (cvmexception);
\end{verbatim}
sets imaginary part of every element of  calling matrix to  value of
appropriate element of  matrix \verb"mIm"
and returns  reference to
the matrix changed.
Function throws  \GOT{cvmexception}{cvmexception}
in case of different sizes of the operands.
Function is \emph{redefined} in the classes
\GOT{scmatrix}{scmatrix}, \GOT{scbmatrix}{scbmatrix} 
and \GOT{schmatrix}{schmatrix}.
See also \GOT{cmatrix}{cmatrix} and \GOT{rmatrix}{rmatrix}.
Example:
\begin{Verbatim}
using namespace cvm;

rmatrix m (2,3);
cmatrix mc(2,3);
m.randomize (0., 1.);

mc.assign_imag(m);
std::cout << mc;
\end{Verbatim}
prints
\begin{Verbatim}
(0,0.13831) (0,0.267373) (0,0.482345)
(0,0.50618) (0,0.992401) (0,0.444777)
\end{Verbatim}
\newpage




\subsubsection{set\_real}
Function%
\pdfdest name {cmatrix.setreal} fit
\begin{verbatim}
cmatrix& cmatrix::set_real (TR d);
\end{verbatim}
sets real part of every element of  calling matrix to  value of
parameter \verb"d"
and returns  reference to the matrix changed.
See also \GOT{cmatrix}{cmatrix}.
Function is \emph{redefined} in the classes
\GOT{scmatrix}{scmatrix}, \GOT{scbmatrix}{scbmatrix} 
and \GOT{schmatrix}{schmatrix}.
Example:
\begin{Verbatim}
using namespace cvm;

cmatrix m(2,3);
m.set_real(1.);
std::cout << m;
\end{Verbatim}
prints
\begin{Verbatim}
(1,0) (1,0) (1,0)
(1,0) (1,0) (1,0)
\end{Verbatim}
\newpage



\subsubsection{set\_imag}
Function%
\pdfdest name {cmatrix.setimag} fit
\begin{verbatim}
cmatrix& cmatrix::set_imag (TR d);
\end{verbatim}
sets imaginary part of every element of  calling matrix to value of
parameter \verb"d"
and returns  reference to the matrix changed.
See also \GOT{cmatrix}{cmatrix}.
Function is \emph{redefined} in the classes
\GOT{scmatrix}{scmatrix}, \GOT{scbmatrix}{scbmatrix} 
and \GOT{schmatrix}{schmatrix}.
Example:
\begin{Verbatim}
using namespace cvm;

cmatrix m(2,3);
m.set_imag(1.);
std::cout << m;
\end{Verbatim}
prints
\begin{Verbatim}
(0,1) (0,1) (0,1)
(0,1) (0,1) (0,1)
\end{Verbatim}
\newpage



\subsubsection{resize}
Function%
\pdfdest name {cmatrix.resize} fit
\begin{verbatim}
cmatrix& cmatrix::resize (int nNewM, int nNewN) throw (cvmexception);
\end{verbatim}
changes  size of  calling matrix to \verb"nNewM" by \verb"nNewN"
and returns  reference to
the matrix changed. In case of increasing of its size, the matrix
is filled up with zeroes.
Function throws  \GOT{cvmexception}{cvmexception}
in case of negative size passed or memory allocation failure.
Function is \emph{redefined} in the classes
\GOT{scmatrix}{scmatrix}, \GOT{scbmatrix}{scbmatrix} 
and \GOT{schmatrix}{schmatrix}.
See also \GOT{cmatrix}{cmatrix}.
Example:
\begin{Verbatim}
using namespace cvm;

try {
    double a[] = {1., 2., 3., 4., 5., 6.,
                  7., 8., 9., 10., 11., 12.};
    cmatrix m((std::complex<double>*) a, 2, 3);
    std::cout << m << std::endl;
    m.resize (2, 2);
    std::cout << m << std::endl;
    m.resize (3, 3);
    std::cout << m;
}
catch (std::exception& e) {
    std::cout << "Exception " << e.what () << std::endl;
}
\end{Verbatim}
prints
\begin{Verbatim}
(1,2) (5,6) (9,10)
(3,4) (7,8) (11,12)

(1,2) (5,6)
(3,4) (7,8)

(1,2) (5,6) (0,0)
(3,4) (7,8) (0,0)
(0,0) (0,0) (0,0)
\end{Verbatim}
\newpage



\subsubsection{operator ==}
Operator%
\pdfdest name {cmatrix.operator ==} fit
\begin{verbatim}
bool cmatrix::operator == (const cmatrix& m) const;
\end{verbatim}
compares  calling matrix with  matrix \verb"m"
and returns \verb"true" if they have the same sizes
and their appropriate elements differ by not more than the
\GO{smallest normalized positive number}{Utilities.cvmMachMin}.
Returns \verb"false" otherwise.
Operator is \emph{redefined} in the classes
\GOT{scmatrix}{scmatrix}, \GOT{scbmatrix}{scbmatrix} 
and \GOT{schmatrix}{schmatrix}.
See also \GOT{cmatrix}{cmatrix}.
Example:
\begin{Verbatim}
using namespace cvm;

cmatrix m1(2, 3);
cmatrix m2(2, 3);
m1.set_real(1.);
m2.set_real(1.);
std::cout << (m1 == m2) << std::endl;
\end{Verbatim}
prints
\begin{Verbatim}
1
\end{Verbatim}
\newpage



\subsubsection{operator !=}
Operator%
\pdfdest name {cmatrix.operator !=} fit
\begin{verbatim}
bool cmatrix::operator != (const cmatrix& m) const;
\end{verbatim}
compares  calling matrix with  matrix \verb"m"
and returns \verb"true" if they have different sizes
or at least of their appropriate elements
differs by more than the
\GO{smallest normalized positive number}{Utilities.cvmMachMin}.
Returns \verb"false" otherwise.
Operator is \emph{redefined} in the classes
\GOT{scmatrix}{scmatrix}, \GOT{scbmatrix}{scbmatrix} 
and \GOT{schmatrix}{schmatrix}.
See also \GOT{cmatrix}{cmatrix}.
Example:
\begin{Verbatim}
using namespace cvm;

cmatrix m1(2, 3);
cmatrix m2(2, 3);
m1.set_real(1.);
m2.set_real(1.);
std::cout << (m1 != m2) << std::endl;
\end{Verbatim}
prints
\begin{Verbatim}
0
\end{Verbatim}
\newpage



\subsubsection{operator <{}<}
Operator%
\pdfdest name {cmatrix.operator <<} fit
\begin{verbatim}
cmatrix& cmatrix::operator << (const cmatrix& m) throw (cvmexception);
\end{verbatim}
destroys  calling matrix, creates  new one as a copy of \verb"m"
and returns  reference to
the matrix changed.
Operator throws  \GOT{cvmexception}{cvmexception}
in case of memory allocation failure.
Oerator is \emph{redefined} in the classes
\GOT{scmatrix}{scmatrix}, \GOT{scbmatrix}{scbmatrix} 
and \GOT{schmatrix}{schmatrix}.
See also \GOT{cmatrix}{cmatrix}.
Example:
\begin{Verbatim}
using namespace cvm;

try {
    cmatrix m(2,3);
    cmatrix mc(1,1);
    m(1,2) = std::complex<double>(1.,2.);
    m(2,3) = std::complex<double>(2.,4.);
    std::cout << m << mc << std::endl;

    mc << m;
    std::cout << mc;
}
catch (std::exception& e) {
    std::cout << "Exception " << e.what () << std::endl;
}
\end{Verbatim}
prints
\begin{Verbatim}
(0,0) (1,2) (0,0)
(0,0) (0,0) (2,4)
(0,0)

(0,0) (1,2) (0,0)
(0,0) (0,0) (2,4)
\end{Verbatim}
\newpage




\subsubsection{operator +}
Operator%
\pdfdest name {cmatrix.operator +} fit
\begin{verbatim}
cmatrix cmatrix::operator + (const cmatrix& m) const
throw (cvmexception);
\end{verbatim}
creates  object of type \verb"cmatrix" as  sum of
 calling matrix and  matrix \verb"m".
Operator throws  \GOT{cvmexception}{cvmexception}
in case of different sizes of the operands.
Operator is \emph{redefined} in the classes
\GOT{scmatrix}{scmatrix}, \GOT{scbmatrix}{scbmatrix} 
and \GOT{schmatrix}{schmatrix}.
See also \GOT{cvector::sum}{cvector.sum}, \GOT{cmatrix}{cmatrix}.
Example:
\begin{Verbatim}
using namespace cvm;

try {
    double a[] = {1., 2., 3., 4., 5., 6.,
                  7., 8., 9., 10., 11., 12.};
    const cmatrix ma ((std::complex<double>*) a, 2, 3);
    cmatrix mb (2, 3);
    mb.set (std::complex<double>(1.,1.));

    std::cout << ma + mb << std::endl;
    std::cout << ma + ma;
}
catch (std::exception& e) {
    std::cout << "Exception " << e.what () << std::endl;
}
\end{Verbatim}
prints
\begin{Verbatim}
(2,3) (6,7) (10,11)
(4,5) (8,9) (12,13)

(2,4) (10,12) (18,20)
(6,8) (14,16) (22,24)
\end{Verbatim}
\newpage



\subsubsection{operator -}
Operator%
\pdfdest name {cmatrix.operator -} fit
\begin{verbatim}
cmatrix cmatrix::operator - (const cmatrix& m) const
throw (cvmexception);
\end{verbatim}
creates  object of type \verb"cmatrix" as  difference of
 calling matrix and  matrix \verb"m".
It throws  \GOT{cvmexception}{cvmexception}
in case of different sizes of the operands.
Operator is \emph{redefined} in the classes
\GOT{scmatrix}{scmatrix}, \GOT{scbmatrix}{scbmatrix} 
and \GOT{schmatrix}{schmatrix}.
See also \GOT{cvector::diff}{cvector.diff}, \GOT{cmatrix}{cmatrix}.
Example:
\begin{Verbatim}
using namespace cvm;

try {
    double a[] = {1., 2., 3., 4., 5., 6.,
                  7., 8., 9., 10., 11., 12.};
    const cmatrix ma ((std::complex<double>*) a, 2, 3);
    cmatrix mb (2, 3);
    mb.set (std::complex<double>(1.,1.));

    std::cout << ma - mb << std::endl;
    std::cout << ma - ma;
}
catch (std::exception& e) {
    std::cout << "Exception " << e.what () << std::endl;
}
\end{Verbatim}
prints
\begin{Verbatim}
(0,1) (4,5) (8,9)
(2,3) (6,7) (10,11)

(0,0) (0,0) (0,0)
(0,0) (0,0) (0,0)
\end{Verbatim}
\newpage



\subsubsection{sum}
Function%
\pdfdest name {cmatrix.sum} fit
\begin{verbatim}
cmatrix& cmatrix::sum (const cmatrix& m1, const cmatrix& m2)
throw (cvmexception);
\end{verbatim}
assigns  result of addition of
matrices \verb"m1" and \verb"m2"  to  calling matrix
and returns  reference to
the matrix changed.
It throws  \GOT{cvmexception}{cvmexception}
in case of different sizes of the operands.
Function is \emph{redefined} in the classes
\GOT{scmatrix}{scmatrix}, \GOT{scbmatrix}{scbmatrix} 
and \GOT{schmatrix}{schmatrix}.
See also \GOT{cmatrix::operator +~}{cmatrix.operator +},
\GOT{cmatrix}{cmatrix}.
Example:
\begin{Verbatim}
using namespace cvm;

try {
    double a[] = {1., 2., 3., 4., 5., 6.,
                  7., 8., 9., 10., 11., 12.};
    const cmatrix ma ((std::complex<double>*) a, 2, 3);
    cmatrix mb (2, 3);
    cmatrix m (2, 3);
    mb.set (std::complex<double>(1.,1.));

    std::cout << m.sum(ma, mb) << std::endl;
    std::cout << m.sum(m, mb);
}
catch (std::exception& e) {
    std::cout << "Exception " << e.what () << std::endl;
}
\end{Verbatim}
prints
\begin{Verbatim}
(2,3) (6,7) (10,11)
(4,5) (8,9) (12,13)

(3,4) (7,8) (11,12)
(5,6) (9,10) (13,14)
\end{Verbatim}
\newpage



\subsubsection{diff}
Function%
\pdfdest name {cmatrix.diff} fit
\begin{verbatim}
cmatrix& cmatrix::diff (const cmatrix& m1, const cmatrix& m2)
throw (cvmexception);
\end{verbatim}
assigns  result of subtraction of
matrices \verb"m1" and \verb"m2" to  calling matrix
and returns  reference to
the matrix changed.
It throws  \GOT{cvmexception}{cvmexception}
in case of different sizes of the operands.
Function is \emph{redefined} in the classes
\GOT{scmatrix}{scmatrix}, \GOT{scbmatrix}{scbmatrix} 
and \GOT{schmatrix}{schmatrix}.
See also \GOT{cmatrix::operator -~}{cmatrix.operator -},
\GOT{cmatrix}{cmatrix}.
Example:
\begin{Verbatim}
using namespace cvm;

try {
    double a[] = {1., 2., 3., 4., 5., 6.,
                  7., 8., 9., 10., 11., 12.};
    const cmatrix ma ((std::complex<double>*) a, 2, 3);
    cmatrix mb (2, 3);
    cmatrix m (2, 3);
    mb.set (std::complex<double>(1.,1.));

    std::cout << m.diff(ma, mb) << std::endl;
    std::cout << m.diff(m, mb);
}
catch (std::exception& e) {
    std::cout << "Exception " << e.what () << std::endl;
}
\end{Verbatim}
prints
\begin{Verbatim}
(0,1) (4,5) (8,9)
(2,3) (6,7) (10,11)

(-1,0) (3,4) (7,8)
(1,2) (5,6) (9,10)
\end{Verbatim}
\newpage




\subsubsection{operator +=}
Operator%
\pdfdest name {cmatrix.operator +=} fit
\begin{verbatim}
cmatrix& cmatrix::operator += (const cmatrix& m) throw (cvmexception);
\end{verbatim}
adds  matrix \verb"m" to  calling matrix 
and returns  reference to
the matrix changed.
It throws  \GOT{cvmexception}{cvmexception}
in case of different sizes of the operands.
Operator is \emph{redefined} in the classes
\GOT{scmatrix}{scmatrix}, \GOT{scbmatrix}{scbmatrix} 
and \GOT{schmatrix}{schmatrix}.
See also \GOT{cmatrix::operator +~}{cmatrix.operator +},
\GOT{cmatrix::sum}{cmatrix.sum},
\GOT{cmatrix}{cmatrix}.
Example:
\begin{Verbatim}
using namespace cvm;

try {
    cmatrix m1(2, 3);
    cmatrix m2(2, 3);
    m1.set(std::complex<double>(1.,1.));
    m2.set(std::complex<double>(2.,2.));

    m1 += m2;
    std::cout << m1 << std::endl;

    // well, you can do this too, but temporary object would be created
    m2 += m2; 
    std::cout << m2;
}
catch (std::exception& e) {
    std::cout << "Exception " << e.what () << std::endl;
}
\end{Verbatim}
prints
\begin{Verbatim}
(3,3) (3,3) (3,3)
(3,3) (3,3) (3,3)

(4,4) (4,4) (4,4)
(4,4) (4,4) (4,4)
\end{Verbatim}
\newpage



\subsubsection{operator -=}
Operator%
\pdfdest name {cmatrix.operator -=} fit
\begin{verbatim}
cmatrix& cmatrix::operator -= (const cmatrix& m) throw (cvmexception);
\end{verbatim}
subtracts  matrix \verb"m" from  calling matrix
and returns  reference to
the matrix changed.
It throws  \GOT{cvmexception}{cvmexception}
in case of different sizes of the operands.
Operator is \emph{redefined} in the classes
\GOT{scmatrix}{scmatrix}, \GOT{scbmatrix}{scbmatrix} 
and \GOT{schmatrix}{schmatrix}.
See also \GOT{cmatrix::operator -~}{cmatrix.operator -},
\GOT{cmatrix::diff}{cmatrix.diff},
\GOT{cmatrix}{cmatrix}.
Example:
\begin{Verbatim}
using namespace cvm;

try {
    cmatrix m1(2, 3);
    cmatrix m2(2, 3);
    m1.set(std::complex<double>(1.,1.));
    m2.set(std::complex<double>(2.,2.));

    m1 -= m2;
    std::cout << m1 << std::endl;

    // well, you can do this too, but temporary object would be created
    m2 -= m2; 
    std::cout << m2;
}
catch (std::exception& e) {
    std::cout << "Exception " << e.what () << std::endl;
}
\end{Verbatim}
prints
\begin{Verbatim}
(-1,-1) (-1,-1) (-1,-1)
(-1,-1) (-1,-1) (-1,-1)

(0,0) (0,0) (0,0)
(0,0) (0,0) (0,0)
\end{Verbatim}
\newpage



\subsubsection{operator - ()}
Operator%
\pdfdest name {cmatrix.operator - ()} fit
\begin{verbatim}
cmatrix cmatrix::operator - () const throw (cvmexception);
\end{verbatim}
creates  object of type \verb"cmatrix" as
 calling matrix multiplied by $-1$.
Operator is \emph{redefined} in the classes
\GOT{scmatrix}{scmatrix}, \GOT{scbmatrix}{scbmatrix} 
and \GOT{schmatrix}{schmatrix}.
See also \GOT{cmatrix}{cmatrix}.
Example:
\begin{Verbatim}
using namespace cvm;

double a[] = {1., 2., 3., 4., 5., 6.,
              7., 8., 9., 10., 11., 12.};
const cmatrix ma ((std::complex<double>*) a, 2, 3);

std::cout << - ma;
\end{Verbatim}
prints
\begin{Verbatim}
(-1,-2) (-5,-6) (-9,-10)
(-3,-4) (-7,-8) (-11,-12)
\end{Verbatim}
\newpage




\subsubsection{operator * (TR)}
Operator%
\pdfdest name {cmatrix.operator * (TR)} fit
\begin{verbatim}
cmatrix cmatrix::operator * (TR d) const;
\end{verbatim}
creates  object of type \verb"cmatrix" as  product of
 calling matrix and  real number~\verb"d".
Operator is \emph{redefined} in the classes
\GOT{scmatrix}{scmatrix}, \GOT{scbmatrix}{scbmatrix} 
and \GOT{schmatrix}{schmatrix}.
See also \GOT{cmatrix::operator *=~}
{cmatrix.operator *= (TR)},
\GOT{cmatrix}{cmatrix}.
Example:
\begin{Verbatim}
using namespace cvm;

double a[] = {1., 2., 3., 4., 5., 6.,
              7., 8., 9., 10., 11., 12.};
const cmatrix ma ((std::complex<double>*) a, 2, 3);

std::cout << ma * 5.;
\end{Verbatim}
prints
\begin{Verbatim}
(5,10) (25,30) (45,50)
(15,20) (35,40) (55,60)
\end{Verbatim}
\newpage



\subsubsection{operator / (TR)}
Operator%
\pdfdest name {cmatrix.operator / (TR)} fit
\begin{verbatim}
cmatrix cmatrix::operator / (TR d) const throw (cvmexception);
\end{verbatim}
creates  object of type \verb"cmatrix" as  quotient of
 calling matrix and  real number~\verb"d". It throws
 \GOT{cvmexception}{cvmexception}
if \verb"d" has  absolute value equal or less than the
\GO{smallest normalized positive number}{Utilities.cvmMachMin}.
Operator is \emph{redefined} in the classes
\GOT{scmatrix}{scmatrix}, \GOT{scbmatrix}{scbmatrix} 
and \GOT{schmatrix}{schmatrix}.
See also \GOT{cmatrix::operator /=~}
{cmatrix.operator /= (TR)},
\GOT{cmatrix}{cmatrix}.
Example:
\begin{Verbatim}
using namespace cvm;

try {
    double a[] = {1., 2., 3., 4., 5., 6.,
                  7., 8., 9., 10., 11., 12.};
    const cmatrix ma ((std::complex<double>*) a, 2, 3);

    std::cout << ma / 4.;
}
catch (std::exception& e) {
    std::cout << "Exception " << e.what () << std::endl;
}
\end{Verbatim}
prints
\begin{Verbatim}
(0.25,0.5) (1.25,1.5) (2.25,2.5)
(0.75,1) (1.75,2) (2.75,3)
\end{Verbatim}
\newpage



\subsubsection{operator * (TC)}
Operator%
\pdfdest name {cmatrix.operator * (TC)} fit
\begin{verbatim}
cmatrix cmatrix::operator * (TC z) const;
\end{verbatim}
creates  object of type \verb"cmatrix" as  product of
 calling matrix and  complex number~\verb"z".
Operator is \emph{redefined} in the classes
\GOT{scmatrix}{scmatrix}, \GOT{scbmatrix}{scbmatrix} 
and \GOT{schmatrix}{schmatrix}.
See also \GOT{cmatrix}{cmatrix},
\GOT{cmatrix::operator *=~}{cmatrix.operator *= (TC)}.
Example:
\begin{Verbatim}
using namespace cvm;

double a[] = {1., 2., 3., 4., 5., 6.,
              7., 8., 9., 10., 11., 12.};
const cmatrix ma ((std::complex<double>*) a, 2, 3);

std::cout << ma * std::complex<double>(5.,2.);
\end{Verbatim}
prints
\begin{Verbatim}
(1,12) (13,40) (25,68)
(7,26) (19,54) (31,82)
\end{Verbatim}
\newpage



\subsubsection{operator / (TC)}
Operator%
\pdfdest name {cmatrix.operator / (TC)} fit
\begin{verbatim}
cmatrix cmatrix::operator / (TC z) const throw (cvmexception);
\end{verbatim}
creates  object of type \verb"cmatrix" as  quotient of
 calling matrix and  complex number~\verb"z". 
It throws
 \GOT{cvmexception}{cvmexception}
if \verb"z" has  absolute value equal or less than the
\GO{smallest normalized positive number}{Utilities.cvmMachMin}.
Operator is \emph{redefined} in the classes
\GOT{scmatrix}{scmatrix}, \GOT{scbmatrix}{scbmatrix} 
and \GOT{schmatrix}{schmatrix}.
See also \GOT{cmatrix::operator /=~}
{cmatrix.operator /= (TC)},
\GOT{cmatrix}{cmatrix}.
Example:
\begin{Verbatim}
using namespace cvm;

try {
    double a[] = {1., 2., 3., 4., 5., 6.,
                  7., 8., 9., 10., 11., 12.};
    const cmatrix ma ((std::complex<double>*) a, 2, 3);

    std::cout << ma / std::complex<double>(4.,2.);
}
catch (std::exception& e) {
    std::cout << "Exception " << e.what () << std::endl;
}
\end{Verbatim}
prints
\begin{Verbatim}
(0.4,0.3) (1.6,0.7) (2.8,1.1)
(1,0.5) (2.2,0.9) (3.4,1.3)
\end{Verbatim}
\newpage



\subsubsection{operator *= (TR)}
Operator%
\pdfdest name {cmatrix.operator *= (TR)} fit
\begin{verbatim}
cmatrix& cmatrix::operator *= (TR d);
\end{verbatim}
multiplies  calling matrix by  real number~\verb"d"
and returns  reference to
the matrix changed.
Operator is \emph{redefined} in the classes
\GOT{scmatrix}{scmatrix}, \GOT{scbmatrix}{scbmatrix} 
and \GOT{schmatrix}{schmatrix}.
See also \GOT{cmatrix}{cmatrix},
\GOT{cmatrix::operator *}{cmatrix.operator * (TR)}.
Example:
\begin{Verbatim}
using namespace cvm;

double a[] = {1., 2., 3., 4., 5., 6.,
              7., 8., 9., 10., 11., 12.};
cmatrix ma ((std::complex<double>*) a, 2, 3);

ma *= 2.;
std::cout << ma;
\end{Verbatim}
prints
\begin{Verbatim}
(2,4) (10,12) (18,20)
(6,8) (14,16) (22,24)
\end{Verbatim}
\newpage



\subsubsection{operator /= (TR)}
Operator%
\pdfdest name {cmatrix.operator /= (TR)} fit
\begin{verbatim}
cmatrix& cmatrix::operator /= (TR d) throw (cvmexception);
\end{verbatim}
divides  calling matrix by  real number~\verb"d"
and returns  reference to
the matrix changed.
It throws  \GOT{cvmexception}{cvmexception}
if \verb"d" has  absolute value equal or less
than the 
\GO{smallest normalized positive number}{Utilities.cvmMachMin}.
Operator is \emph{redefined} in the classes
\GOT{scmatrix}{scmatrix}, \GOT{scbmatrix}{scbmatrix} 
and \GOT{schmatrix}{schmatrix}.
See also \GOT{cmatrix::operator /~}{cmatrix.operator / (TR)},
\GOT{cmatrix}{cmatrix}.
Example:
\begin{Verbatim}
using namespace cvm;

try {
    double a[] = {1., 2., 3., 4., 5., 6.,
                  7., 8., 9., 10., 11., 12.};
    cmatrix ma ((std::complex<double>*) a, 2, 3);

    ma /= 2.;
    std::cout << ma;
}
catch (std::exception& e) {
    std::cout << "Exception " << e.what () << std::endl;
}
\end{Verbatim}
prints
\begin{Verbatim}
(0.5,1) (2.5,3) (4.5,5)
(1.5,2) (3.5,4) (5.5,6)
\end{Verbatim}
\newpage




\subsubsection{operator *= (TC)}
Operator%
\pdfdest name {cmatrix.operator *= (TC)} fit
\begin{verbatim}
cmatrix& cmatrix::operator *= (TC z);
\end{verbatim}
multiplies  calling matrix by  complex number~\verb"z"
and returns  reference to
the matrix changed.
Operator is \emph{redefined} in the classes
\GOT{scmatrix}{scmatrix}, \GOT{scbmatrix}{scbmatrix} 
and \GOT{schmatrix}{schmatrix}.
See also \GOT{cmatrix::operator *~}{cmatrix.operator * (TC)},
\GOT{cmatrix}{cmatrix}.
Example:
\begin{Verbatim}
using namespace cvm;

double a[] = {1., 2., 3., 4., 5., 6.,
              7., 8., 9., 10., 11., 12.};
cmatrix ma ((std::complex<double>*) a, 2, 3);

ma *= std::complex<double>(2.,1.);
std::cout << ma;
\end{Verbatim}
prints
\begin{Verbatim}
(0,5) (4,17) (8,29)
(2,11) (6,23) (10,35)
\end{Verbatim}
\newpage



\subsubsection{operator /= (TC)}
Operator%
\pdfdest name {cmatrix.operator /= (TC)} fit
\begin{verbatim}
cmatrix& cmatrix::operator /= (TC z) throw (cvmexception);
\end{verbatim}
divides  calling matrix by  complex number~\verb"z"
and returns  reference to
the matrix changed.
It throws  \GOT{cvmexception}{cvmexception}
if \verb"z" has  absolute value equal or less
than the 
\GO{smallest normalized positive number}{Utilities.cvmMachMin}.
Operator is \emph{redefined} in the classes
\GOT{scmatrix}{scmatrix}, \GOT{scbmatrix}{scbmatrix} 
and \GOT{schmatrix}{schmatrix}.
See also \GOT{cmatrix::operator /~}{cmatrix.operator / (TC)},
\GOT{cmatrix}{cmatrix}.
Example:
\begin{Verbatim}
using namespace cvm;

try {
    double a[] = {1., 2., 3., 4., 5., 6.,
                  7., 8., 9., 10., 11., 12.};
    cmatrix ma ((std::complex<double>*) a, 2, 3);

    ma /= std::complex<double>(2.,1.);
    std::cout << ma;
}
catch (std::exception& e) {
    std::cout << "Exception " << e.what () << std::endl;
}
\end{Verbatim}
prints
\begin{Verbatim}
(0.8,0.6) (3.2,1.4) (5.6,2.2)
(2,1) (4.4,1.8) (6.8,2.6)
\end{Verbatim}
\newpage




\subsubsection{normalize}
Function%
\pdfdest name {cmatrix.normalize} fit
\begin{verbatim}
cmatrix& cmatrix::normalize ();
\end{verbatim}
normalizes  calling matrix so its \GO{Euclidean norm}{Array.norm}
becomes equal to $1$ if it was greater than the 
\GO{smallest normalized positive number}{Utilities.cvmMachMin}
before the call (otherwise  function does nothing).
Function is \emph{redefined} in the classes
\GOT{scmatrix}{scmatrix}, \GOT{scbmatrix}{scbmatrix} 
and \GOT{schmatrix}{schmatrix}.
See also \GOT{cmatrix}{cmatrix}.
Example:
\begin{Verbatim}
using namespace cvm;

double a[] = {1., 2., 3., 4., 5., 6.,
              7., 8., 9., 10., 11., 12.};
cmatrix ma ((std::complex<double>*) a, 2, 3);

ma.normalize();
std::cout << ma << ma.norm() << std::endl;
\end{Verbatim}
prints
\begin{Verbatim}
(0.0392232,0.0784465) (0.196116,0.235339) (0.353009,0.392232)
(0.11767,0.156893) (0.274563,0.313786) (0.431455,0.470679)
1
\end{Verbatim}
\newpage



\subsubsection{conjugation}
Operator and functions%
\pdfdest name {cmatrix.conj} fit
\begin{verbatim}
cmatrix cmatrix::operator ~ () const throw (cvmexception);
cmatrix& cmatrix::conj (const cmatrix& m) throw (cvmexception);
cmatrix& cmatrix::conj () throw (cvmexception);
\end{verbatim}
implement complex matrix conjugation.
First operator creates  object of type \verb"cmatrix" as
 conjugated calling matrix
(it throws  
\GOT{cvmexception}{cvmexception}
in case of memory allocation failure). 
Second function sets  calling matrix to be equal to  matrix
\verb"m" conjugated
(it throws  
\GOT{cvmexception}{cvmexception}
in case of not appropriate sizes of the operands), 
third one makes it to be equal to
conjugated itself (it also throws  
\GOT{cvmexception}{cvmexception}
in case of memory allocation failure). 
Functions are \emph{redefined} in the classes
\GOT{scmatrix}{scmatrix}, \GOT{scbmatrix}{scbmatrix} 
and \GOT{schmatrix}{schmatrix}.
See also \GOT{cmatrix::tramspose}{cmatrix.transpose}, \GOT{cmatrix}{cmatrix}.
Example:
\begin{Verbatim}
using namespace cvm;

double a[] = {1., 2., 3., 4., 5., 6.,
              7., 8., 9., 10., 11., 12.};
cmatrix m((std::complex<double>*) a, 2, 3);
cmatrix mc(3,2);
std::cout << m << std::endl << ~m << std::endl ;
mc.conj(m);
std::cout << mc << std::endl;
mc.conj();
std::cout << mc;
\end{Verbatim}
prints
\begin{Verbatim}
(1,2) (5,6) (9,10)
(3,4) (7,8) (11,12)

(1,-2) (3,-4)
(5,-6) (7,-8)
(9,-10) (11,-12)

(1,-2) (3,-4)
(5,-6) (7,-8)
(9,-10) (11,-12)

(1,2) (5,6) (9,10)
(3,4) (7,8) (11,12)
\end{Verbatim}
\newpage




\subsubsection{transposition}
Operator and functions%
\pdfdest name {cmatrix.transpose} fit
\begin{verbatim}
cmatrix cmatrix::operator ! () const throw (cvmexception);
cmatrix& cmatrix::transpose (const cmatrix& m) throw (cvmexception);
cmatrix& cmatrix::transpose () throw (cvmexception);
\end{verbatim}
implement complex matrix transposition (\emph{not} conjugation).
First operator creates  object of type \verb"cmatrix" as
transposed calling matrix
(it throws  
\GOT{cvmexception}{cvmexception}
in case of memory allocation failure). 
Second function sets  calling matrix to be equal to  matrix
\verb"m" transposed
(it throws  
\GOT{cvmexception}{cvmexception}
in case of not appropriate sizes of the operands), 
third one makes it to be equal to
transposed itself (it also throws  
\GOT{cvmexception}{cvmexception}
in case of memory allocation failure). 
Functions are \emph{redefined} in the classes
\GOT{scmatrix}{scmatrix}, \GOT{scbmatrix}{scbmatrix} 
and \GOT{schmatrix}{schmatrix}.
See also \GOT{cmatrix::conj}{cmatrix.conj}, \GOT{cmatrix}{cmatrix}.
Example:
\begin{Verbatim}
using namespace cvm;

double a[] = {1., 2., 3., 4., 5., 6.,
              7., 8., 9., 10., 11., 12.};
cmatrix m((std::complex<double>*) a, 2, 3);
cmatrix mc(3,2);
std::cout << m << std::endl << !m << std::endl ;
mc.transpose(m);
std::cout << mc << std::endl;
mc.transpose();
std::cout << mc;
\end{Verbatim}
prints
\begin{Verbatim}
(1,2) (5,6) (9,10)
(3,4) (7,8) (11,12)

(1,2) (3,4)
(5,6) (7,8)
(9,10) (11,12)

(1,2) (3,4)
(5,6) (7,8)
(9,10) (11,12)

(1,2) (5,6) (9,10)
(3,4) (7,8) (11,12)
\end{Verbatim}
\newpage













\subsubsection{operator * (const cvector\&)}
Operator%
\pdfdest name {cmatrix.operator * (const cvector&)} fit
\begin{verbatim}
cvector cmatrix::operator * (const cvector& v) const
throw (cvmexception);
\end{verbatim}
creates  object of type \verb"cvector"
as  product of  calling matrix and  vector \verb"v".
It throws  \GOT{cvmexception}{cvmexception}
if number of columns of calling matrix
differs from size of  vector \verb"v".
Use \GOT{cvector::mult}{cvector.mult (const cmatrix&, const cvector&)}
in order to avoid of  new object creation.
Function is \emph{redefined} in the classes
\GOT{scmatrix}{scmatrix}, \GOT{scbmatrix}{scbmatrix} 
and \GOT{schmatrix}{schmatrix}.
See also
\GOT{cmatrix}{cmatrix}, \GOT{cvector}{cvector}.
Example:
\begin{Verbatim}
using namespace cvm;

cmatrix m(2, 3);
cvector v(3);
m.set(std::complex<double>(1.,1.));
v.set(std::complex<double>(1.,1.));

std::cout << m * v;
\end{Verbatim}
prints
\begin{Verbatim}
(0,6) (0,6)
\end{Verbatim}
\newpage



\subsubsection{operator * (const cmatrix\&)}
Operator%
\pdfdest name {cmatrix.operator * (const cmatrix&)} fit
\begin{verbatim}
cmatrix cmatrix::operator * (const cmatrix& m) const
throw (cvmexception);
\end{verbatim}
creates  object of type \verb"cmatrix"
as  product of  calling matrix and  matrix \verb"m".
It throws  \GOT{cvmexception}{cvmexception}
if number of columns of calling matrix
differs from number of rows of  matrix \verb"m".
Use \GOT{cmatrix::mult}{cmatrix.mult} in order to avoid of
 new object creation.
Operator is \emph{redefined} in the classes
\GOT{scmatrix}{scmatrix}, \GOT{scbmatrix}{scbmatrix} 
and \GOT{schmatrix}{schmatrix}.
See also
\GOT{cmatrix}{cmatrix}.
Example:
\begin{Verbatim}
using namespace cvm;

cmatrix m1(2, 3);
cmatrix m2(3, 2);
m1.set(std::complex<double>(1.,1.));
m2.set(std::complex<double>(1.,1.));

std::cout << m1 * m2;
\end{Verbatim}
prints
\begin{Verbatim}
(0,6) (0,6)
(0,6) (0,6)
\end{Verbatim}
\newpage



\subsubsection{mult}
Function%
\pdfdest name {cmatrix.mult} fit
\begin{verbatim}
cmatrix& cmatrix::mult (const cmatrix& m1, const cmatrix& m2)
throw (cvmexception);
\end{verbatim}
sets  calling matrix to be equal to  product of  matrix
\verb"m1" by  matrix \verb"m2"
and returns  reference to
the matrix changed.
Function throws  \GOT{cvmexception}{cvmexception}
in case of inappropriate sizes of the operands.
Function is is \emph{inherited} in the class \GOT{scmatrix}{scmatrix}
and \emph{redefined} in the classes
\GOT{scbmatrix}{scbmatrix} 
and \GOT{schmatrix}{schmatrix}.
See also
\GOT{cmatrix}{cmatrix}.
Example:
\begin{Verbatim}
using namespace cvm;

cmatrix m1(2, 3);
cmatrix m2(3, 2);
scmatrix m(2);
m1.set(std::complex<double>(1.,1.));
m2.set(std::complex<double>(1.,1.));

m.mult(m1, m2);
std::cout << m;
\end{Verbatim}
prints
\begin{Verbatim}
(0,6) (0,6)
(0,6) (0,6)
\end{Verbatim}
\newpage



\subsubsection{rank1update\_u}
Function%
\pdfdest name {cmatrix.rank1update_u} fit
\begin{verbatim}
cmatrix&
cmatrix::rank1update_u (const cvector& vCol, const cvector& vRow)
throw (cvmexception);
\end{verbatim}
sets calling matrix to be equal to 
\GO{rank-1 update}{cvector.rank1update_u} (uncojugated) of vectors
\verb"vCol" and \verb"vRow"
and returns  reference to
the matrix changed.
Function throws  \GOT{cvmexception}{cvmexception}
if number of rows of  calling matrix
is not equal to \verb"vCol.size()"
or number of columns is not equal to \verb"vRow.size()".
Function is \emph{inherited} in the class
\GOT{scmatrix}{scmatrix} and
\emph{not applicable} to objects of the classes
\GOT{scbmatrix}{scbmatrix} and
\GOT{schmatrix}{schmatrix} (i.e. 
\GOT{cvmexception}{cvmexception} would be thrown
in case of using it for objects of those classes).
See also \GO{cvector::rank1update\_u}{cvector.rank1update_u},
\GOT{cmatrix}{cmatrix}.
Example:
\begin{Verbatim}
using namespace cvm;

cvector vc(3), vr(2);
cmatrix m(3, 2);
vc.set(std::complex<double>(1.,1.));
vr.set(std::complex<double>(1.,1.));

std::cout << m.rank1update_u (vc, vr);
\end{Verbatim}
prints
\begin{Verbatim}
(0,2) (0,2)
(0,2) (0,2)
(0,2) (0,2)
\end{Verbatim}
\newpage



\subsubsection{rank1update\_c}
Function%
\pdfdest name {cmatrix.rank1update_c} fit
\begin{verbatim}
cmatrix&
cmatrix::rank1update_c (const cvector& vCol, const cvector& vRow)
throw (cvmexception);
\end{verbatim}
sets calling matrix to be equal to 
\GO{rank-1 update}{cvector.rank1update_c} (conjugated) of vectors
\verb"vCol" and \verb"vRow"
and returns  reference to
the matrix changed.
Function throws  \GOT{cvmexception}{cvmexception}
if number of rows of  calling matrix
is not equal to \verb"vCol.size()"
or number of columns is not equal to \verb"vRow.size()".
Function is \emph{inherited} in the class
\GOT{scmatrix}{scmatrix} and
\emph{not applicable} to objects of the classes
\GOT{scbmatrix}{scbmatrix} and
\GOT{schmatrix}{schmatrix} (i.e. 
\GOT{cvmexception}{cvmexception} would be thrown
in case of using it for objects of those classes).
See also \GO{cvector::rank1update\_c}{cvector.rank1update_c},
\GOT{cmatrix}{cmatrix}.
Example:
\begin{Verbatim}
using namespace cvm;

cvector vc(3), vr(2);
cmatrix m(3, 2);
vc.set(std::complex<double>(1.,1.));
vr.set(std::complex<double>(1.,1.));

std::cout << m.rank1update_c (vc, vr);
\end{Verbatim}
prints
\begin{Verbatim}
(2,0) (2,0)
(2,0) (2,0)
(2,0) (2,0)
\end{Verbatim}
\newpage



\subsubsection{swap\_rows}
Function%
\pdfdest name {cmatrix.swap_rows} fit
\begin{verbatim}
cmatrix& cmatrix::swap_rows (int n1, int n2) throw (cvmexception);
\end{verbatim}
swaps two rows of  calling matrix and returns  reference to
the matrix changed. \verb"n1" and \verb"n2" are
numbers of rows to be swapped, 
both are \Based).
Function throws  \GOT{cvmexception}{cvmexception}
if one of  parameters is outside of the range
\verb"[1,msize()]".
Function is \emph{redefined} in the class
\GOT{scmatrix}{scmatrix} and
\emph{not applicable} to objects of the classes
\GOT{scbmatrix}{scbmatrix} and
\GOT{schmatrix}{schmatrix} (i.e. 
\GOT{cvmexception}{cvmexception} would be thrown
in case of using it for objects of those classes).
See also \GOT{cmatrix}{cmatrix}.
Example:
\begin{Verbatim}
using namespace cvm;

double a[] = {1., 2., 3., 4., 5., 6.,
              7., 8., 9., 10., 11., 12.};
cmatrix m ((std::complex<double>*)a, 3, 2);

std::cout << m << std::endl;
std::cout << m.swap_rows(2,3);
\end{Verbatim}
prints
\begin{Verbatim}
(1,2) (7,8)
(3,4) (9,10)
(5,6) (11,12)

(1,2) (7,8)
(5,6) (11,12)
(3,4) (9,10)
\end{Verbatim}
\newpage



\subsubsection{swap\_cols}
Function%
\pdfdest name {cmatrix.swap_cols} fit
\begin{verbatim}
cmatrix& cmatrix::swap_cols (int n1, int n2) throw (cvmexception);
\end{verbatim}
swaps two columns of  calling matrix and returns  reference to
the matrix changed. \verb"n1" and \verb"n2" are
numbers of columns to be swapped, 
both are \Based).
Function throws  \GOT{cvmexception}{cvmexception}
if one of  parameters is outside of the range
\verb"[1,nsize()]".
Function is \emph{redefined} in the class
\GOT{scmatrix}{scmatrix} and
\emph{not applicable} to objects of the classes
\GOT{scbmatrix}{scbmatrix} and
\GOT{schmatrix}{schmatrix} (i.e. 
\GOT{cvmexception}{cvmexception} would be thrown
in case of using it for objects of those classes).
See also \GOT{cmatrix}{cmatrix}.
Example:
\begin{Verbatim}
using namespace cvm;

double a[] = {1., 2., 3., 4., 5., 6.,
              7., 8., 9., 10., 11., 12.};
cmatrix m ((std::complex<double>*)a, 2, 3);

std::cout << m << std::endl;
std::cout << m.swap_cols(2,3);
\end{Verbatim}
prints
\begin{Verbatim}
(1,2) (5,6) (9,10)
(3,4) (7,8) (11,12)

(1,2) (9,10) (5,6)
(3,4) (11,12) (7,8)
\end{Verbatim}
\newpage




\subsubsection{solve}
Functions%
\pdfdest name {cmatrix.solve} fit
\begin{verbatim}
cmatrix&
cmatrix::solve (const scmatrix& mA,
                const cmatrix& mB, TR& dErr) throw (cvmexception);
cmatrix&
cmatrix::solve (const scmatrix& mA,
                const cmatrix& vB) throw (cvmexception);
\end{verbatim}
set  calling matrix to be equal to  solution $X$ of the
matrix linear equation
$A*X=B$ where parameter \verb"mA" is square matrix $A$
and parameter \verb"vB" is matrix $B$.
Every function returns  reference to the matrix changed.
The first version also sets output parameter \verb"dErr" to be equal
to the norm of computation error.
Functions throw \GOT{cvmexception}{cvmexception}
in case of inappropriate sizes
of operands or when matrix $A$ is close to singular.
Functions are \emph{redefined} in the class
\GOT{scmatrix}{scmatrix} and
\emph{inherited} thereafter in the classes
\GOT{scbmatrix}{scbmatrix} and
\GOT{schmatrix}{schmatrix}.
See also
\GOT{cvector::solve}{cvector.solve},
\GOT{cmatrix}{cmatrix},
\GOT{scmatrix}{scmatrix}.
Example:
\begin{Verbatim}
using namespace cvm;
std::cout.setf (std::ios::scientific | std::ios::showpos);
std::cout.precision (4);

scmatrix ma(3);
cmatrix  mb(3,2);
cmatrix  mx(3,2);
double dErr;
ma.randomize_real(0.,10.); ma.randomize_imag(0.,10.);
mb.randomize_real(0.,10.); mb.randomize_imag(0.,10.);

mx.solve (ma, mb, dErr);
std::cout << mx << std::endl << ma * mx - mb
          << dErr << std::endl;
\end{Verbatim}
prints
\begin{Verbatim}
(+3.2103e-001,+2.8598e-002) (+1.1907e+000,-1.1101e+000)
(+3.7451e-001,+4.2471e-001) (-2.3858e-001,+3.3028e-001)
(+3.0085e-001,-3.0427e-001) (-7.6874e-001,+4.1100e-001)

(+4.4409e-016,+0.0000e+000) (-8.8818e-016,+1.3323e-015)
(+7.7716e-016,-8.8818e-016) (+0.0000e+000,+1.7764e-015)
(+0.0000e+000,-8.8818e-016) (-1.7764e-015,+8.8818e-016)
+5.4774e-015
\end{Verbatim}
\newpage




\subsubsection{solve\_tran}
Functions%
\pdfdest name {cmatrix.solvetran} fit
\begin{verbatim}
cmatrix&
cmatrix::solve_tran (const scmatrix& mA,
                     const cmatrix& mB, TR& dErr) throw (cvmexception);
cmatrix&
cmatrix::solve_tran (const scmatrix& mA,
                     const cmatrix& vB) throw (cvmexception);
\end{verbatim}
set calling matrix to be equal to  solution $X$ of 
matrix linear equation
$A^T*X=B$ (which is equivalent to $X^T*A=B^T$)
where parameter \verb"mA" is square matrix $A$
and parameter \verb"vB" is matrix $B$.
Every function returns  reference to the matrix changed.
The first version also sets  output parameter \verb"dErr" to be equal
to the norm of computation error.
Functions throw \GOT{cvmexception}{cvmexception}
in case of inappropriate sizes
of the operands or when the matrix $A$ is close to singular.
Functions are \emph{redefined} in  the class
\GOT{scmatrix}{scmatrix} and
\emph{inherited} thereafter in the classes
\GOT{scbmatrix}{scbmatrix} and
\GOT{schmatrix}{schmatrix}.
See also
\GOT{cmatrix::solve}{cmatrix.solve},
\GOT{cvector::solve\_tran}{cvector.solvetran},
\GOT{cmatrix}{cmatrix},
\GOT{scmatrix}{scmatrix}.
Example:
\begin{Verbatim}
using namespace cvm;
std::cout.setf (std::ios::scientific | std::ios::showpos); 
std::cout.precision (3);

scmatrix ma(3);
cmatrix  mb(3,2);
cmatrix  mx(3,2);
double dErr;
ma.randomize_real(0.,10.); ma.randomize_imag(0.,10.);
mb.randomize_real(0.,10.); mb.randomize_imag(0.,10.);

mx.solve_tran (ma, mb, dErr);

std::cout << !ma * mx - mb << dErr << std::endl;
std::cout << !mx * ma - !mb << std::endl;
\end{Verbatim}
prints
\begin{Verbatim}
(+6.661e-016,+1.776e-015) (+0.000e+000,+5.690e-016)
(-1.776e-015,-1.776e-015) (-8.882e-016,-8.882e-016)
(-2.665e-015,+1.776e-015) (+0.000e+000,-1.110e-015)
+7.079e-015
(+6.661e-016,+1.776e-015) (-1.776e-015,-1.776e-015) (-1.776e-015,+1.776e-015)
(+0.000e+000,+5.690e-016) (-8.882e-016,-8.882e-016) (+0.000e+000,-1.554e-015)
\end{Verbatim}
\newpage





\subsubsection{solve\_conj}
Functions%
\pdfdest name {cmatrix.solveconj} fit
\begin{verbatim}
cmatrix&
cmatrix::solve_conj (const scmatrix& mA,
                     const cmatrix& mB, TR& dErr) throw (cvmexception);
cmatrix&
cmatrix::solve_conj (const scmatrix& mA,
                     const cmatrix& vB) throw (cvmexception);
\end{verbatim}
set calling matrix to be equal to  solution $X$ of 
matrix linear equation
$A^H*X=B$ (which is equivalent to $X^H*A=B^H$), where $A^H$ is conjugated $A$.
Here parameter \verb"mA" is square matrix $A$
and parameter \verb"vB" is matrix $B$.
Every function returns  reference to the matrix changed.
The first version also sets  output parameter \verb"dErr" to be equal
to the norm of computation error.
Functions throw \GOT{cvmexception}{cvmexception}
in case of inappropriate sizes
of the operands or when the matrix $A$ is close to singular.
Functions are \emph{redefined} in  the class
\GOT{scmatrix}{scmatrix} and
\emph{inherited} thereafter in the classes
\GOT{scbmatrix}{scbmatrix} and
\GOT{schmatrix}{schmatrix}.
See also
\GOT{cmatrix::solve}{cmatrix.solve},
\GOT{cvector::solve\_conj}{cvector.solveconj},
\GOT{cmatrix}{cmatrix},
\GOT{scmatrix}{scmatrix}.
Example:
\begin{Verbatim}
using namespace cvm;
std::cout.setf (std::ios::scientific | std::ios::showpos); 
std::cout.precision (3);

scmatrix ma(3);
cmatrix  mb(3,2);
cmatrix  mx(3,2);
double dErr;
ma.randomize_real(0.,10.); ma.randomize_imag(0.,10.);
mb.randomize_real(0.,10.); mb.randomize_imag(0.,10.);

mx.solve_conj (ma, mb, dErr);

std::cout << ~ma * mx - mb << dErr << std::endl;
std::cout << ~mx * ma - ~mb << std::endl;
\end{Verbatim}
prints
\begin{Verbatim}
(+7.105e-015,-1.776e-015) (+1.332e-015,+8.882e-016)
(+1.776e-015,-8.882e-016) (+8.882e-016,-8.882e-016)
(+0.000e+000,-6.661e-016) (-1.776e-015,-2.665e-015)
+1.169e-014
(+7.105e-015,+1.776e-015) (+1.776e-015,+8.882e-016) (-2.665e-015,+6.661e-016)
(+1.332e-015,-8.882e-016) (+8.882e-016,+8.882e-016) (-1.776e-015,+2.665e-015)
\end{Verbatim}
\newpage







\subsubsection{solve\_lu}
Functions%
\pdfdest name {cmatrix.solvelu} fit
\begin{verbatim}
cmatrix&
cmatrix::solve_lu (const scmatrix& mA, const scmatrix& mLU,
                   const int* pPivots, const cmatrix& mB, TR& dErr)
                   throw (cvmexception);
cmatrix&
cmatrix::solve_lu (const scmatrix& mA, const scmatrix& mLU,
                   const int* pPivots, const cmatrix& vB)
                   throw (cvmexception);
\end{verbatim}
set  calling matrix to be equal to solution $X$ of the
matrix linear equation
$A*X=B$ where  parameter \verb"mA" is  square complex matrix $A$,
parameter \verb"mLU" is \GO{$LU$ factorization}{scmatrix.low_up}
of  matrix $A$, parameter \verb"pPivots" is  array of pivot numbers
created while factorizing  matrix $A$
and  parameter \verb"vB" is  matrix $B$.
Every function returns  reference to the matrix changed.
The first version also sets output parameter \verb"dErr" to be equal
to the norm of computation error.
These functions are useful when you need to solve few linear equations
of kind $AX=B$ with the same matrix $A$ and different matrices $B$.
In such case you save on matrix $A$ factorization since it's needed to be
performed just one time.
Functions throw \GOT{cvmexception}{cvmexception}
in case of inappropriate sizes
of operands or when  matrix $A$ is close to singular.
See also
\GOT{cvector::solve}{cvector.solve},
\GOT{cmatrix}{cmatrix},
\GOT{scmatrix}{scmatrix}.
Example:
\begin{Verbatim}
using namespace cvm;

std::cout.setf (std::ios::scientific | std::ios::left); 
std::cout.precision (4);
try {
    scmatrix ma(3);
    scmatrix mLU(3);
    cmatrix  mb1(3,2);
    cmatrix  mb2(3,2);
    cmatrix  mx1(3,2);
    cmatrix  mx2(3,2);
    iarray   nPivots(3);
    double   dErr = 0.;
    ma.randomize_real(0.,10.); ma.randomize_imag(0.,10.);
    mb1.randomize_real(0.,10.); mb1.randomize_imag(0.,10.);
    mb2.randomize_real(0.,10.); mb2.randomize_imag(0.,10.);

    mLU.low_up(ma, nPivots);
    std::cout << mx1.solve_lu (ma, mLU, nPivots, mb1, dErr);
    std::cout << dErr << std::endl;
    std::cout << mx2.solve_lu (ma, mLU, nPivots, mb2) << std::endl;
    std::cout << ma * mx1 - mb1 << std::endl << ma * mx2 - mb2;
}
catch (std::exception& e) {
    std::cout << "Exception " << e.what () << std::endl;
}
\end{Verbatim}
prints
\begin{Verbatim}
(4.2888e-01,8.2409e-02) (-1.1261e-01,-5.7778e-01)
(5.8052e-01,3.2179e-01) (2.5811e-01,-3.8609e-02)
(-3.1499e-02,-7.0014e-01) (1.2652e+00,4.5309e-01)
5.2931e-15
(3.0153e-01,-5.6606e-01) (-1.6308e-01,1.8217e-01)
(7.4971e-01,-1.1305e-01) (5.2187e-01,2.3441e-01)
(-1.9916e-01,1.4493e+00) (9.1046e-02,3.5242e-01)

(0.0000e+00,-8.8818e-16) (0.0000e+00,-8.8818e-16)
(-4.4409e-16,0.0000e+00) (0.0000e+00,-8.8818e-16)
(0.0000e+00,0.0000e+00) (0.0000e+00,0.0000e+00)

(-8.8818e-16,8.8818e-16) (-8.8818e-16,-1.7764e-15)
(0.0000e+00,0.0000e+00) (2.2204e-16,-8.8818e-16)
(4.4409e-16,-8.8818e-16) (1.3878e-17,-4.4409e-16)
\end{Verbatim}
\newpage


\subsubsection{svd}
Functions%
\pdfdest name {cmatrix.svd} fit
\begin{verbatim}
rvector
cmatrix::svd () throw (cvmexception);
rvector
cmatrix::svd (scmatrix& mU, scmatrix& mVH) throw (cvmexception);
\end{verbatim}
create  object of type \verb"rvector"
as  vector of
\GO{singular values}{rvector.svd} of  calling matrix.
The second version 
sets output parameter \verb"mU" to be equal
to  matrix
$U$ of size $m\times m$ (and change
 size of the object if it's needed) and
\verb"mVH" to be equal
to  matrix $V^H$ of size $n\times n$.
All  functions throw
 \GOT{cvmexception}{cvmexception}
in case of inappropriate sizes of the operands
or in case of convergence error.
Use \GOT{rvector::svd}{rvector.svd} in order to avoid of
 new vector creation.
Function is \emph{redefined} in the  classes
\GOT{scmatrix}{scmatrix}, \GOT{scbmatrix}{scbmatrix},
\GOT{schmatrix}{schmatrix}.
See also
\GOT{rvector}{rvector},
\GOT{cmatrix}{cmatrix}.
Example:
\begin{Verbatim}
using namespace cvm;

std::cout.setf (std::ios::scientific | std::ios::left); 
std::cout.precision (3);
try {
    double m[] = {1., -1., 1., 2., -2., 1.,
                  3., -2., 1., 0., -2., 1.};
    cmatrix mA((std::complex<double>*) m, 2, 3);
    cmatrix mSigma(2,3);
    rvector v(2);
    scmatrix mU(2), mVH(3);

    v = mA.svd(mU, mVH);
    mSigma.diag(0) = cvector(v);

    std::cout << mU << std::endl;
    std::cout << mVH << std::endl;
    std::cout << mSigma << std::endl;

    std::cout << (mA * ~mVH - mU * mSigma).norm() << std::endl;
    std::cout << (~mA * mU - ~(mSigma * mVH)).norm() << std::endl;
}
catch (std::exception& e) {
    std::cout << "Exception " << e.what () << std::endl;
}
\end{Verbatim}
prints
\begin{Verbatim}
(-4.861e-01,0.000e+00) (8.739e-01,0.000e+00)
(7.956e-01,-3.616e-01) (4.425e-01,-2.012e-01)

(-7.590e-02,4.474e-01) (7.488e-01,-1.820e-01) (-4.474e-01,1.327e-02)
(8.084e-01,1.878e-01) (-1.576e-02,5.238e-01) (-1.878e-01,3.558e-02)
(1.065e-01,3.065e-01) (3.597e-01,4.669e-02) (8.727e-01,4.012e-02)

(5.452e+00,0.000e+00) (0.000e+00,0.000e+00) (0.000e+00,0.000e+00)
(0.000e+00,0.000e+00) (1.131e+00,0.000e+00) (0.000e+00,0.000e+00)

1.357e-15
1.267e-15
\end{Verbatim}
\newpage






\subsubsection{pinv}
Functions%
\pdfdest name {cmatrix.pinv} fit
\begin{verbatim}
cmatrix  cmatrix::pinv (TR threshold = cvmMachSp()) const 
                        throw (cvmexception);
cmatrix& cmatrix::pinv (const basic_rmatrix& mA, 
                        TR threshold = cvmMachSp()) 
                        throw (cvmexception);
\end{verbatim}
implement complex matrix pseudo inversion~\GO{\cite{Gantmaher},~p.~33}{biblio}
(or Moore-Penrose generalized inversion~\GO{\cite{Horn},~p.~421}{biblio}).
Strictly defined,  $n\times m$ matrix $A^{+}$ is 
\textit{pseudo inversion} of $m\times n$ matrix $A$ if the following two equations
are satisfied:
\begin{gather*}
AA^{+}A=A,\\
A^{+}=QA^H=A^HP
\end{gather*}
where $Q$ and $P$ are some matrices. To compute  pseudo inversion, we use
\GO{Singular Value Decomposition (SVD)}{rvector.svd} 
\begin{equation*}
A = U\Sigma V^H
\end{equation*}
of matrix $A$, thus
\begin{equation*}
A^{+} = V\Sigma^{-1}U^{H},
\end{equation*}
where $\Sigma^{-1}$ is  diagonal $n\times m$ matrix having inverted
diagonal values of  matrix $\Sigma$ if they are greater than some threshold,
and zeros otherwise.

First version creates  object of type \verb"cmatrix" as
 pseudo inverted calling matrix
(it throws 
\GOT{cvmexception}{cvmexception}
in case of memory allocation failure).
Second function sets  calling matrix to be equal to  matrix
\verb"mA" pseudo inverted
(it throws 
\GOT{cvmexception}{cvmexception}
in case of not appropriate sizes of the operands).
The threshold parameter sets  minimum distinguishable
from zero singular value to be used to compute the pseudo inversion.
All values equal or less than the threshold are treated as zeros.
Functions are \emph{inherited} in the classes
\GOT{scmatrix}{scmatrix} and \GOT{schmatrix}{schmatrix}
and \emph{redefined} in \GOT{scbmatrix}{scbmatrix}.
See also \GOT{cmatrix}{cmatrix}.
Example:
\begin{Verbatim}
using namespace cvm;
try {
    cmatrix mA(2,3);
    mA(1,1) = tcomplex(1.,2.);  mA(1,2) = tcomplex(-1.,3.); 
    mA(1,3) = tcomplex(0.,-1.);
    mA(2,1) = tcomplex(1.,-2.); mA(2,2) = tcomplex(1.,-4.); 
    mA(2,3) = tcomplex(1.,0.);

    cmatrix mX = mA.pinv();
    std::cout << mX << (mA * mX * mA - mA).norm2() << std::endl;
}
catch (std::exception& e) {
    std::cout << "Exception " << e.what () << std::endl;
}
\end{Verbatim}
prints
\begin{Verbatim}
(+3.2407407e-01,-1.1111111e-01) (+2.5925926e-01,-1.8518519e-02)
(-1.3888889e-01,-2.7777778e-02) (-5.5555556e-02,+1.6666667e-01)
(+1.4814815e-01,+2.3148148e-01) (+1.6666667e-01,+1.4814815e-01)
+3.3217718e-15
\end{Verbatim}
Band matrix example:
\begin{Verbatim}
using namespace cvm;

try {
    scbmatrix mA (40, 1, 0);
    mA.diag(0).randomize_real(-1.,1.);
    mA.diag(0).randomize_imag(-3.,2.);
    mA.diag(-1).randomize_real(5.,10.);
    mA.diag(-1).randomize_imag(-3.,7.);

    scmatrix mX (40);
    mX.pinv(mA);
    std::cout << (mA * mX * mA - mA).norm2() << std::endl;
}
catch (std::exception& e) {
    std::cout << "Exception " << e.what () << std::endl;
}
\end{Verbatim}
prints
\begin{Verbatim}
+6.3113999e-14
\end{Verbatim}
\newpage




\subsubsection{gels}
Functions%
\pdfdest name {cmatrix.gels} fit
\begin{verbatim}
cmatrix  cmatrix::gels (bool conjugate, const cmatrix& mB,
                        cvector& vErr) const throw (cvmexception);
cmatrix& cmatrix::gels (bool conjugate, 
                        const cmatrix& mA, const cmatrix& mB,
                        cvector& vErr) throw (cvmexception);
cvector  cmatrix::gels (bool conjugate, const cvector& mB,
                        TC& cErr) const throw (cvmexception);
\end{verbatim}
solve overdetermined or underdetermined linear systems 
\begin{equation*}
A*x=b
\end{equation*}
for $m\times n$ matrix $A$ (or its conjugated) where 
$b$ is a vector of length $k$ 
or systems 
\begin{equation*}
A*X=B
\end{equation*}
for multiple vectors $b$ stored as columns of $k\times l$ matrix $B$ where
$k=m$ in non-transposed case and $k=n$ otherwise.
The algorithm uses  QR or LQ factorization of $A$.
It is assumed that $A$ has full rank, infinity returned otherwise.
Internally  functions use \verb"?GELS" LAPACK routines. 
If $m>n$ and \verb"conjugate=false" or $m<n$ and \verb"conjugate=true", then 
the system is overdetermined, thus the algorithm tries to find the least squares solution $x$
of the problem
\begin{equation*}
\|A*x-b\|_2\to\min\quad\text{or}\quad\|A^H*x-b\|_2\to\min\
\end{equation*}
respectively. Output vector \verb"vErr" of length $l$ 
(or complex number \verb"cErr" for single vector $b$) returns residual sum of squares.
The system is underdetermined otherwise, and the algorithm finds its minimum norm solution.
In this case \verb"vErr" (or \verb"cErr") is set to zero. In both cases the solution computed
satisfies $x=\pinv(A)*b$, but this algorithm is faster than pseudo inversion.
Matrix $A$ is passed as  argument \verb"mA" or (in second case) it's  calling object.
First and third versions return the solution as  new matrix or vector object.
Second version sets calling object to be the solution.
Functions throw
 \GOT{cvmexception}{cvmexception}
in case of inappropriate sizes of the operands.
Functions are \emph{inherited} in all complex matrix classes.
See also \GOT{cmatrix}{cmatrix}, \GOT{cvector::gels}{cvector.gels},
\GOT{cmatrix::pinv}{cmatrix.pinv}.
Example:
\begin{Verbatim}
using namespace cvm;

cmatrix a(7, 5);
cmatrix b(7, 2);
cmatrix bt(5, 2);
cvector bv(5);
cvector vErr(2);
tcomplex dErr;
a.randomize_real(-10., 10.);
a.randomize_imag(-10., 10.);
b.randomize_real(-10., 10.);
b.randomize_imag(-10., 10.);
bt.randomize_real(-10., 10.);
bt.randomize_imag(-10., 10.);
bv.randomize_real(-10., 10.);
bv.randomize_imag(-10., 10.);

cmatrix x = a.gels(false, b, vErr);
cvector vt = a.gels(true, bv, dErr);
cmatrix xt(7,2);
xt.gels(true, a, bt, vErr);

std::cout << (a.pinv()*b - x).norm() << std::endl;
std::cout << (~a.pinv()*bv - vt).norm() << std::endl;
std::cout << (~a.pinv()*bt - xt).norm() << std::endl;
\end{Verbatim}
prints
\begin{Verbatim}
+1.3258216e-015
+1.0696705e-015
+1.1685394e-015
\end{Verbatim}
\newpage





\subsubsection{gelsy}
Functions%
\pdfdest name {cmatrix.gelsy} fit
\begin{verbatim}
cmatrix  cmatrix::gelsy (const cmatrix& mB, int& rank,
                         TR tol = cvmMachSp()) const throw (cvmexception);
cmatrix& cmatrix::gelsy (const cmatrix& mA, const cmatrix& mB, int& rank,
                         TR tol = cvmMachSp()) throw (cvmexception);
cvector  cmatrix::gelsy (const cvector& mB, int& rank,
                         TR tol = cvmMachSp()) const throw (cvmexception);
\end{verbatim}
compute the minimum-norm solution to the linear
least squares problem 
\begin{equation*}
\|A*x-b\|_2\to\min
\end{equation*}
using  complete orthogonal factorization of $m\times n$ matrix $A$.
Here $b$ is a vector of length $m$.
Multiple vectors $b$ can be stored as columns of $m\times l$ matrix $B$.
Matrix $A$ may be rank-deficient, functions return its effective rank in \verb"rank"
output parameter using \verb"tol" tolerance.
Internally functions use \verb"?GELSY" LAPACK routines, see more details
about the algorithm in those routines' documentation.
Matrix $A$ is passed as  argument \verb"mA" or (in second case) it's  calling object.
First and third versions return the solution as  new matrix or vector object.
Second version sets calling object to be the solution.
Functions throw
 \GOT{cvmexception}{cvmexception}
in case of inappropriate sizes of the operands.
Functions are \emph{inherited} in all complex matrix classes.
See also \GOT{cmatrix}{cmatrix}, \GOT{cvector::gelsy}{cvector.gelsy}, 
\GOT{cmatrix::gelss}{cmatrix.gelss}, \GOT{cmatrix::gelsd}{cmatrix.gelsd}.
Example:
\begin{Verbatim}
using namespace cvm;

cmatrix a(4, 5);
cmatrix b(4, 2);
cvector bv(4);
a.randomize_real(-10., 10.);
a.randomize_imag(-10., 10.);
b.randomize_real(-10., 10.);
b.randomize_imag(-10., 10.);
bv.randomize_real(-10., 10.);
bv.randomize_imag(-10., 10.);
tint rank;

cmatrix x = a.gelsy(b, rank);
cvector xv = a.gelsy(bv, rank);
cmatrix x2(5,2);
x2.gelsy(a, b, rank);

std::cout << (a*x - b).norm() << std::endl;
std::cout << (a*xv - bv).norm() << std::endl;
std::cout << (a*x2 - b).norm() << std::endl;
std::cout << rank << " " << a.rank() << std::endl;
\end{Verbatim}
prints
\begin{Verbatim}
+8.9701623e-015
+6.9342294e-015
+8.9701623e-015
+4 +4
\end{Verbatim}
\newpage



\subsubsection{gelss}
Functions%
\pdfdest name {cmatrix.gelss} fit
\begin{verbatim}
cmatrix  cmatrix::gelss (const cmatrix& mB, int& rank, rvector& sv,
                         TR tol = cvmMachSp()) const throw (cvmexception);
cmatrix& cmatrix::gelss (const cmatrix& mA, const cmatrix& mB, 
                         rvector& sv, int& rank, TR tol = cvmMachSp()) 
                         throw (cvmexception);
cvector  cmatrix::gelss (const cvector& mB, rvector& sv, int& rank,
                         TR tol = cvmMachSp()) const throw (cvmexception);
\end{verbatim}
compute the minimum-norm solution to the linear
least squares problem 
\begin{equation*}
\|A*x-b\|_2\to\min
\end{equation*}
using singular value decomposition  of $m\times n$ matrix $A$.
Here $b$ is a vector of length $m$.
Multiple vectors $b$ can be stored as columns of $m\times l$ matrix $B$.
Matrix $A$ may be rank-deficient, functions return its effective rank in \verb"rank"
output parameter using \verb"tol" tolerance. These functions also compute 
singular values of $A$ in decreasing order and return them in \verb"sv" output parameter
having $\min(m,n)$ size.
Internally functions use \verb"?GELSS" LAPACK routines, see more details
about the algorithm in those routines' documentation.
Matrix $A$ is passed as  argument \verb"mA" or (in second case) it's  calling object.
First and third versions return the solution as  new matrix or vector object.
Second version sets calling object to be the solution.
Functions throw
 \GOT{cvmexception}{cvmexception}
in case of inappropriate sizes of the operands.
Functions are \emph{inherited} in all complex matrix classes.
See also \GOT{cmatrix}{cmatrix}, \GOT{cvector::gelss}{cvector.gelss},
\GOT{cmatrix::gelsy}{cmatrix.gelsy}, \GOT{cmatrix::gelsd}{cmatrix.gelsd}.
Example:

\begin{Verbatim}
using namespace cvm;

cmatrix a(4, 5);
cmatrix b(4, 2);
cvector bv(4);
rvector sv(4);
tint rank;
a.randomize_real(-10., 10.);
a.randomize_imag(-10., 10.);
b.randomize_real(-10., 10.);
b.randomize_imag(-10., 10.);
bv.randomize_real(-10., 10.);
bv.randomize_imag(-10., 10.);

cmatrix x = a.gelss(b, sv, rank);
cvector xv = a.gelss(bv, sv, rank);
cmatrix xt(5,2);
xt.gelss(a, b, sv, rank);

std::cout << (a*x - b).norm() << std::endl;
std::cout << (a*xv - bv).norm() << std::endl;
std::cout << (a*xt - b).norm() << std::endl;
std::cout << (sv - a.svd()).norm() << std::endl;
std::cout << rank << " " << a.rank() << std::endl;
\end{Verbatim}
prints
\begin{Verbatim}
+2.3444861e-013
+4.6335022e-014
+2.3444861e-013
+1.1234667e-014
+4 +4
\end{Verbatim}
\newpage






\subsubsection{gelsd}
Functions%
\pdfdest name {cmatrix.gelsd} fit
\begin{verbatim}
cmatrix  cmatrix::gelsd (const cmatrix& mB, int& rank, rvector& sv,
                         TR tol = cvmMachSp()) const throw (cvmexception);
cmatrix& cmatrix::gelsd (const cmatrix& mA, const cmatrix& mB, 
                         rvector& sv, int& rank, TR tol = cvmMachSp()) 
                         throw (cvmexception);
cvector  cmatrix::gelsd (const cvector& mB, rvector& sv, int& rank,
                         TR tol = cvmMachSp()) const throw (cvmexception);
\end{verbatim}
compute the minimum-norm solution to the linear
least squares problem 
\begin{equation*}
\|A*x-b\|_2\to\min
\end{equation*}
using singular value decomposition of $m\times n$ matrix $A$
and divide and conquer method.
Here $b$ is a vector of length $m$.
Multiple vectors $b$ can be stored as columns of $m\times l$ matrix $B$.
Matrix $A$ may be rank-deficient, functions return its effective rank in \verb"rank"
output parameter using \verb"tol" tolerance. These functions also compute 
singular values of $A$ in decreasing order and return them in \verb"sv" output parameter
having $\min(m,n)$ size.
Internally functions use \verb"?GELSD" LAPACK routines, see more details
about the algorithm in those routines' documentation.
Matrix $A$ is passed as  argument \verb"mA" or (in second case) it's  calling object.
First and third versions return the solution as  new matrix or vector object.
Second version sets calling object to be the solution.
Functions throw
 \GOT{cvmexception}{cvmexception}
in case of inappropriate sizes of the operands.
Functions are \emph{inherited} in all complex matrix classes.
See also \GOT{cmatrix}{cmatrix}, \GOT{cvector::gelsd}{cvector.gelsd},
\GOT{cmatrix::gelsy}{cmatrix.gelsy}, \GOT{cmatrix::gelss}{cmatrix.gelss}.
Example:

\begin{Verbatim}
using namespace cvm;

cmatrix a(4, 5);
cmatrix b(4, 2);
cvector bv(4);
rvector sv(4);
tint rank;
a.randomize_real(-10., 10.);
a.randomize_imag(-10., 10.);
b.randomize_real(-10., 10.);
b.randomize_imag(-10., 10.);
bv.randomize_real(-10., 10.);
bv.randomize_imag(-10., 10.);

cmatrix x = a.gelsd(b, sv, rank);
cvector xv = a.gelsd(bv, sv, rank);
cmatrix xt(5,2);
xt.gelsd(a, b, sv, rank);

std::cout << (a*x - b).norm() << std::endl;
std::cout << (a*xv - bv).norm() << std::endl;
std::cout << (a*xt - b).norm() << std::endl;
std::cout << (sv - a.svd()).norm() << std::endl;
std::cout << rank << " " << a.rank() << std::endl;
\end{Verbatim}
prints
\begin{Verbatim}
+1.2865301e-014
+1.2824878e-014
+1.2865301e-014
+1.2809491e-014
+4 +4
\end{Verbatim}
\newpage





\subsubsection{rank}
Function%
\pdfdest name {cmatrix.rank} fit
\begin{verbatim}
int cmatrix::rank (TR eps = cvmMachSp()) const throw (cvmexception);
\end{verbatim}
returns  rank of  calling matrix as  number of
\GO{singular values}{rvector.svd}
with \GO{normalized}{cvector.normalize}
absolute value greater than or
equal to  parameter \verb"eps" (this is the
\GO{largest relative spacing}{Utilities.cvmMachSp} by default).
Function throws \GOT{cvmexception}{cvmexception}
in case of convergence error.
Function is \emph{inherited} in the classes
\GOT{scmatrix}{scmatrix}, \GOT{scbmatrix}{scbmatrix},
\GOT{schmatrix}{schmatrix}.
See also
\GOT{cmatrix}{cmatrix}.
Example:
\begin{Verbatim}
using namespace cvm;

double a[] = {1., 2., 3., 4., 5., 6.,
              7., 8., 9., 10., 11., 12.};
cmatrix m(a, NULL, 3, 4);

std::cout << m << m.rank() << std::endl;
m(3,4) = std::complex<double>(0.,1.);
std::cout << m.rank() << std::endl;
\end{Verbatim}
prints
\begin{Verbatim}
(1,0) (4,0) (7,0) (10,0)
(2,0) (5,0) (8,0) (11,0)
(3,0) (6,0) (9,0) (12,0)
2
3
\end{Verbatim}
\newpage



\subsubsection{qr}
Functions%
\pdfdest name {cmatrix.qr} fit
\begin{verbatim}
void cmatrix::qr (cmatrix& mQ, scmatrix& mR) const throw (cvmexception);
void cmatrix::qr (scmatrix& mQ, cmatrix& mR) const throw (cvmexception);
\end{verbatim}
compute QR factorization as
\begin{equation*}
M=QR
\end{equation*}
where 
$M$ is  calling matrix, unitary matrix $Q$ 
and upper triangular matrix $R$ are \verb"mQ" and \verb"mR"
respectively. First version implements so-called "economy" algorithm
which for  given $m\times n$ matrix $M$ computes $m\times n$ matrix $Q$
and $n\times n$ matrix $R$. Second version is  "full" mode one 
computing $m\times m$ matrix $Q$ and $m\times n$ matrix $R$.
Functions throw
 \GOT{cvmexception}{cvmexception}
in case of inappropriate sizes of the operands.
Functions are \emph{redefined} in the class
\GOT{scmatrix}{scmatrix}.
See also \GOT{cmatrix}{cmatrix},
\GOT{scmatrix}{scmatrix}.
Example:
\begin{Verbatim}
using namespace cvm;

treal ar[] = {1., 2., 3., 4., 5., 6.};
treal ai[] = {1., -1., 2., -2., 3., -3.};
const cvm::cmatrix mh(ar, ai, 2, 3);
const cvm::cmatrix mv(ar, ai, 3, 2);
cvm::scmatrix s2(2), s3(3);
cvm::cmatrix  h(2,3), v(3,2);

mh.qr(h,s3);
std::cout << (eye_complex(2)-~cmatrix(h,1,1,2,2)*cmatrix(h,1,1,2,2)).norm()
          << " " << (mh - h * s3).norm() << std::endl;
mh.qr(s2,h);
std::cout << (eye_complex(2) - ~s2 * s2).norm()
          << " " << (mh - s2 * h).norm() << std::endl;
mv.qr(v,s2);
std::cout << (eye_complex(2) - ~v * v).norm()
          << " " << (mv - v * s2).norm() << std::endl;
mv.qr(s3,v);
std::cout << (eye_complex(3) - ~s3 * s3).norm()
          << " " << (mv - s3 * v).norm() << std::endl;
\end{Verbatim}
prints
\begin{Verbatim}
+2.5145832e-16 +1.3506446e-15
+2.5145832e-16 +1.3506446e-15
+2.5367068e-16 +1.4432899e-15
+3.8435519e-16 +1.4432899e-15
\end{Verbatim}
\newpage




\subsubsection{lq}
Functions%
\pdfdest name {cmatrix.lq} fit
\begin{verbatim}
void cmatrix::lq (scmatrix& mL, cmatrix& mQ) const throw (cvmexception);
void cmatrix::lq (cmatrix& mL, scmatrix& mQ) const throw (cvmexception);
\end{verbatim}
compute LQ factorization as
\begin{equation*}
M=LQ
\end{equation*}
where 
$M$ is  calling matrix, upper triangular (trapezoidal) matrix $L$ 
and unitary matrix $Q$ are \verb"mL" and \verb"mQ"
respectively. First version implements so-called "economy" algorithm
which for  given $m\times n$ matrix $M$ computes $m\times m$ matrix $L$
and $m\times n$ matrix $Q$. Second version is  "full" mode one 
computing $m\times n$ matrix $L$ and $n\times n$ matrix $Q$.
Functions throw
 \GOT{cvmexception}{cvmexception}
in case of inappropriate sizes of the operands.
Functions are \emph{redefined} in the class
\GOT{scmatrix}{scmatrix}.
See also \GOT{cmatrix}{cmatrix},
\GOT{scmatrix}{scmatrix}.
Example:
\begin{Verbatim}
using namespace cvm;

treal ar[] = {1., 2., 3., 4., 5., 6.};
treal ai[] = {1., -1., 2., -2., 3., -3.};
const cvm::cmatrix mh(ar, ai, 2, 3);
const cvm::cmatrix mv(ar, ai, 3, 2);
cvm::scmatrix s2(2), s3(3);
cvm::cmatrix  h(2,3), v(3,2);

mh.lq(s2,h);
std::cout << (eye_complex(2) - h * ~h).norm() << " " << (mh - s2 * h)
             .norm() << std::endl;
mv.lq(s3,v);
std::cout << (eye_complex(2) - ~cmatrix(v,1,1,2,2) * cmatrix(v,1,1,2,2))
             .norm() << " " << (mv - s3 * v).norm() << std::endl;
mh.lq(h,s3);
std::cout << (eye_complex(3) - s3 * ~s3).norm() << " " << (mh - h * s3)
             .norm() << std::endl;
mv.lq(v,s2);
std::cout << (eye_complex(2) - s2 * ~s2).norm() << " " << (mv - v * s2)
             .norm() << std::endl;
\end{Verbatim}
prints
\begin{Verbatim}
+1.3374617e-015 +7.4806297e-015
+6.6728929e-016 +2.9205408e-015
+1.4349634e-015 +7.4806297e-015
+6.7217793e-016 +2.9205408e-015
\end{Verbatim}
\newpage



\subsubsection{rq}
Functions%
\pdfdest name {cmatrix.rq} fit
\begin{verbatim}
void cmatrix::rq (scmatrix& mR, cmatrix& mQ) const throw (cvmexception);
void cmatrix::rq (cmatrix& mR, scmatrix& mQ) const throw (cvmexception);
\end{verbatim}
compute RQ factorization as
\begin{equation*}
M=RQ
\end{equation*}
where 
$M$ is  calling matrix, upper triangular matrix $R$
and unitary matrix $Q$ are \verb"mR" and \verb"mQ"
respectively. First version implements so-called "economy" algorithm
which for  given $m\times n$ matrix $M$ computes $m\times m$ matrix $R$
and $m\times n$ matrix $Q$. Second version is  "full" mode one
computing $m\times n$ matrix $R$ and $n\times n$ matrix $Q$.
Following \URL{this definition}{http://www.netlib.org/scalapack/slug/node57.html}
the implementation assumes that $m\le n$ and throws \GOT{cvmexception}{cvmexception}
otherwise.
Functions also throw
 \GOT{cvmexception}{cvmexception}
in case of inappropriate sizes of the operands passed.
Functions are \emph{redefined} in the class
\GOT{srmatrix}{srmatrix}.
See also \GOT{rmatrix}{rmatrix},
\GOT{srmatrix}{srmatrix}.
Example:
\begin{Verbatim}
using namespace cvm;

treal ar[] = {1., 2., 3., 4., 5., 6.};
treal ai[] = {1., -1., 2., -2., 3., -3.};
const cvm::cmatrix mh(ar, ai, 2, 3);
const cvm::cmatrix mv(ar, ai, 3, 2);
cvm::scmatrix s2(2), s3(3);
cvm::cmatrix  h(2,3), v(3,2);

mh.rq(h,s3);
std::cout << (eye_complex(2) - cmatrix(s3,2,1,2,3) * ~cmatrix(s3,2,1,2,3))
             .norm() << " " << (mh - h * s3).norm() << std::endl;

mh.rq(s2,h);
std::cout << (eye_complex(2) - h * ~h).norm()
          << " " << (mh - s2 * h).norm() << std::endl;
\end{Verbatim}
prints
\begin{Verbatim}
+3.1898416e-016 +2.2342807e-015
+3.1898416e-016 +2.2342807e-015
\end{Verbatim}
\newpage



\subsubsection{ql}
Functions%
\pdfdest name {cmatrix.ql} fit
\begin{verbatim}
void cmatrix::ql (cmatrix& mQ, scmatrix& mL) const throw (cvmexception);
void cmatrix::ql (scmatrix& mQ, cmatrix& mL) const throw (cvmexception);
\end{verbatim}
compute QL factorization as
\begin{equation*}
M=QL
\end{equation*}
where 
$M$ is  calling matrix, unitary matrix $Q$ 
and lower triangular matrix $L$ are \verb"mQ" and \verb"mL"
respectively. First version implements so-called "economy" algorithm
which for  given $m\times n$ matrix $M$ computes $m\times n$ matrix $Q$
and $n\times n$ matrix $L$. Second version is  "full" mode one
computing $m\times m$ matrix $Q$ and $m\times n$ matrix $L$.
Following \URL{this definition}{http://www.netlib.org/scalapack/slug/node57.html}
the implementation assumes that $m\ge n$ and throws \GOT{cvmexception}{cvmexception}
otherwise.
Functions also throw
 \GOT{cvmexception}{cvmexception}
in case of inappropriate sizes of the operands passed.
Functions are \emph{redefined} in the class
\GOT{scmatrix}{scmatrix}.
See also \GOT{cmatrix}{cmatrix},
\GOT{scmatrix}{scmatrix}.
Example:
\begin{Verbatim}
using namespace cvm;

treal ar[] = {1., 2., 3., 4., 5., 6.};
treal ai[] = {1., -1., 2., -2., 3., -3.};
const cvm::cmatrix mv(ar, ai, 3, 2);
cvm::scmatrix s2(2), s3(3);
cvm::cmatrix v(3,2);

mv.ql(v,s2);
std::cout << (eye_complex(2) - ~v * v).norm()
          << " " << (mv - v * s2).norm() << std::endl;

mv.ql(s3,v);
std::cout << (eye_complex(2) - ~cmatrix(s3,1,2,3,2) * cmatrix(s3,1,2,3,2))
             .norm() << " " << (mv - s3 * v).norm() << std::endl;
\end{Verbatim}
prints
\begin{Verbatim}
+1.1857187e-016 +9.4857497e-016
+1.1857187e-016 +9.4857497e-016
\end{Verbatim}
\newpage




\subsubsection{vanish}
Function%
\pdfdest name {cmatrix.vanish} fit
\begin{verbatim}
cmatrix& cmatrix::vanish();
\end{verbatim}
sets every element of  calling matrix to be equal to zero
and returns  reference to
the matrix changed. This function is faster
than, for example,
\GOT{cmatrix::set(TC)}{cmatrix.set}
with zero operand passed.
Function is \emph{redefined} in the classes
\GOT{scmatrix}{scmatrix}, \GOT{scbmatrix}{scbmatrix},
\GOT{schmatrix}{schmatrix}.
See also \GOT{cmatrix}{cmatrix}.
Example:
\begin{Verbatim}
using namespace cvm;

cmatrix m(4, 3);
m.randomize_real(0.,1.);
m.randomize_imag(1.,2.);

std::cout << m << std::endl;
std::cout << m.vanish ();
\end{Verbatim}
prints
\begin{Verbatim}
(0.851527,1.16376) (0.557512,1.90188) (0.0343638,1.52068)
(0.478042,1.29106) (0.561724,1.19764) (0.320994,1.35804)
(0.264534,1.40986) (0.113468,1.75137) (0.37727,1.54994)
(0.521409,1.83035) (0.559465,1.35072) (0.809198,1.12537)

(0,0) (0,0) (0,0)
(0,0) (0,0) (0,0)
(0,0) (0,0) (0,0)
(0,0) (0,0) (0,0)
\end{Verbatim}
\newpage



\subsubsection{geru}
Function%
\pdfdest name {cmatrix.geru} fit
\begin{verbatim}
cmatrix&
cmatrix::geru (TC dAlpha, const cvector& vCol, const cvector& vRow)
throw (cvmexception);
\end{verbatim}
calls one of \verb"?GERU" routines of the
\URL{BLAS library}{http://www.netlib.org/blas}
performing  \GO{rank-1 update}{cvector.rank1update_u}
(unconjugated)
matrix-vector operation defined as
\begin{equation*}
M=\alpha\begin{pmatrix}
x_1 \\
x_2 \\
\vdots \\
x_m
\end{pmatrix}
\begin{pmatrix}
y_1 & y_2 & \cdots & y_n
\end{pmatrix} + M,
\end{equation*}
where $\alpha$ is  complex number
(parameter \verb"dAlpha"),
$M$ is  calling matrix
and $x$ and $y$ are complex vectors (parameters \verb"vCol"
and \verb"vRow" respectively).
Function
returns  reference to the matrix changed and throws
 \GOT{cvmexception}{cvmexception}
in case of inappropriate sizes of the operands.
Function is \emph{inherited} in  the class
\GOT{scmatrix}{scmatrix} and
\emph{not applicable} to objects of the classes
\GOT{scbmatrix}{scbmatrix} and
\GOT{schmatrix}{schmatrix} (i.e. \GOT{cvmexception}{cvmexception} would be thrown
in case of using it for objects of those classes).
See also
\GOT{cvector}{cvector},
\GOT{cmatrix}{cmatrix}.
Example:
\begin{Verbatim}
using namespace cvm;

std::cout.setf (std::ios::scientific | std::ios::left); 
std::cout.precision (5);

std::complex<double> alpha = std::complex<double>(1.2,4.11);
cmatrix m(3,2);
cvector vc(3);
cvector vr(2);
m.randomize_real(-1., 2.); m.randomize_imag(-3., 2.);
vc.randomize_real(-1., 3.); vc.randomize_imag(1., 3.); 
vr.randomize_real(0., 2.); vr.randomize_imag(-1., 2.); 

std::cout << m + vc.rank1update_u (vr) * alpha << std::endl;
std::cout << m.geru(alpha, vc, vr);
\end{Verbatim}
prints
\begin{Verbatim}
(2.88144e+00,3.54299e+00) (-8.14760e+00,-1.03789e+00)
(6.33361e-01,3.35209e+00) (-4.81787e+00,-8.53964e+00)
(5.44811e-01,1.37156e+00) (-5.97006e+00,-5.00794e+00)

(2.88144e+00,3.54299e+00) (-8.14760e+00,-1.03789e+00)
(6.33361e-01,3.35209e+00) (-4.81787e+00,-8.53964e+00)
(5.44811e-01,1.37156e+00) (-5.97006e+00,-5.00794e+00)
\end{Verbatim}
\newpage



\subsubsection{gerc}
Function%
\pdfdest name {cmatrix.gerc} fit
\begin{verbatim}
cmatrix&
cmatrix::gerc (TC dAlpha, const cvector& vCol, const cvector& vRow)
throw (cvmexception);
\end{verbatim}
calls one of \verb"?GERC" routines of the
\URL{BLAS library}{http://www.netlib.org/blas}
performing  \GO{rank-1 update}{cvector.rank1update_c}
(conjugated)
matrix-vector operation defined as
\begin{equation*}
M=\alpha\begin{pmatrix}
x_1 \\
x_2 \\
\vdots \\
x_m
\end{pmatrix}
\begin{pmatrix}
y_1^* & y_2^* & \hdots & y_n^*
\end{pmatrix} + M,
\end{equation*}
where $\alpha$ is  complex number
(parameter \verb"dAlpha"),
$M$ is  calling matrix
and $x$ and $y$ are complex vectors (parameters \verb"vCol"
and \verb"vRow" respectively).
Function
returns  reference to the matrix changed and throws
 \GOT{cvmexception}{cvmexception}
in case of inappropriate sizes of the operands.
Function is \emph{inherited} in the class
\GOT{scmatrix}{scmatrix} and
\emph{not applicable} to objects of the classes
\GOT{scbmatrix}{scbmatrix} and
\GOT{schmatrix}{schmatrix} (i.e. \GOT{cvmexception}{cvmexception} would be thrown
in case of using it for objects of those classes).
See also
\GOT{cvector}{cvector},
\GOT{cmatrix}{cmatrix}.
Example:
\begin{Verbatim}
using namespace cvm;

std::cout.setf (std::ios::scientific | std::ios::left); 
std::cout.precision (5);

std::complex<double> alpha = std::complex<double>(1.2,4.11);
cmatrix m(3,2);
cvector vc(3);
cvector vr(2);
m.randomize_real(-1., 2.); m.randomize_imag(-3., 2.);
vc.randomize_real(-1., 3.); vc.randomize_imag(1., 3.); 
vr.randomize_real(0., 2.); vr.randomize_imag(-1., 2.); 

std::cout << m + vc.rank1update_c (vr) * alpha << std::endl;
std::cout << m.gerc(alpha, vc, vr);
\end{Verbatim}
prints
\begin{Verbatim}
(1.27138e+01,1.58049e+01) (1.00616e+01,2.21197e+01)
(1.93326e+01,1.41763e+01) (1.74769e+01,2.49013e+01)
(8.09961e+00,1.36259e+01) (5.86738e+00,1.97800e+01)

(1.27138e+01,1.58049e+01) (1.00616e+01,2.21197e+01)
(1.93326e+01,1.41763e+01) (1.74769e+01,2.49013e+01)
(8.09961e+00,1.36259e+01) (5.86738e+00,1.97800e+01)
\end{Verbatim}
\newpage



\subsubsection{gemm}
Function%
\pdfdest name {cmatrix.gemm} fit
\begin{verbatim}
cmatrix& cmatrix::gemm (const cmatrix& m1, bool bConj1, 
                        const cmatrix& m2, bool bConj2, 
                        TC dAlpha, TC dBeta) throw (cvmexception);
\end{verbatim}
calls one of \verb"?GEMM" routines of the
\URL{BLAS library}{http://www.netlib.org/blas}
performing  
matrix-matrix operation defined as
\begin{equation*}
M=\alpha\,\mathcal{C}(M_1)\cdot\mathcal{C}(M_2) + \beta M,
\end{equation*}
where $\alpha$ and $\beta$ are complex numbers
(parameters \verb"dAlpha" and \verb"dBeta"),
$M$ is  calling matrix
and $M_1$ and $M_2$ are matrices (parameters \verb"m1"
and \verb"m2" respectively). Function $\mathcal{C}(M_i)$
congugates matrix $M_i$ if appropriate boolean
parameter \verb"bConj*" is equal to \verb"true"
and does nothing otherwise.
Function
returns  reference to the matrix changed and throws
 \GOT{cvmexception}{cvmexception}
in case of inappropriate sizes of the operands.
Function is \emph{inherited} in  the class
\GOT{scmatrix}{scmatrix} and
\emph{not applicable} to objects of the classes
\GOT{scbmatrix}{scbmatrix} and
\GOT{schmatrix}{schmatrix} (i.e. \GOT{cvmexception}{cvmexception} would be thrown
in case of using it for objects of those classes).
See also
\GOT{cmatrix}{cmatrix}.
Example:
\begin{Verbatim}
using namespace cvm;
std::cout.setf (std::ios::scientific | std::ios::left); 
std::cout.precision (3);

try {
    std::complex<double> alpha = std::complex<double>(1.1,2.1);
    std::complex<double> beta = std::complex<double>(0.71,0.12);
    cmatrix m1(4,3); cmatrix m2(4,3);
    cmatrix m(3,3);
    m.randomize_real(-1., 2.); m.randomize_imag(1., 3.); 
    m1.randomize_real(-1., 3.); m1.randomize_imag(-2., 4.); 
    m2.randomize_real(0., 2.); m2.randomize_imag(-3., 2.);

    std::cout << ~m1 * m2 * alpha + m * beta << std::endl;
    std::cout << m.gemm(m1, true, m2, false, alpha, beta);
}
catch (std::exception& e) {
    std::cout << "Exception " << e.what () << std::endl;
}
\end{Verbatim}
prints
\begin{Verbatim}
(3.225e+01,3.611e+01) (2.042e+01,1.206e+01) (5.065e+01,-2.261e+01)
(3.009e+01,3.665e+00) (2.167e+01,-3.327e+00) (4.305e+01,-1.960e+01)
(1.156e+01,-4.966e+00) (4.067e+00,-1.181e+01) (1.121e+01,-2.684e+01)

(3.225e+01,3.611e+01) (2.042e+01,1.206e+01) (5.065e+01,-2.261e+01)
(3.009e+01,3.665e+00) (2.167e+01,-3.327e+00) (4.305e+01,-1.960e+01)
(1.156e+01,-4.966e+00) (4.067e+00,-1.181e+01) (1.121e+01,-2.684e+01)
\end{Verbatim}
\newpage


\subsubsection{hemm}
Function%
\pdfdest name {cmatrix.hemm} fit
\begin{verbatim}
cmatrix& cmatrix::hemm (bool bLeft, const schmatrix& ms,
                        const cmatrix& m, TC dAlpha, TC dBeta)
                        throw (cvmexception);
\end{verbatim}
calls one of \verb"?HEMM" routines of the
\URL{BLAS library}{http://www.netlib.org/blas}
performing one of 
matrix-matrix operations defined as
\begin{equation*}
M=\alpha\,M_h\cdot M_1 + \beta M\quad\text{or}\quad M=\alpha\,M_1\cdot M_h + \beta M,
\end{equation*}
where $\alpha$ and $\beta$ are complex numbers
(parameters \verb"dAlpha" and \verb"dBeta"),
$M$ is  calling matrix,
$M_h$ is  hermitian matrix and $M_1$ 
is  complex matrix (parameters \verb"ms"
and \verb"m" respectively). 
First operation is performed if \verb"bLeft" passed
is \verb"true" and second one otherwise.
Function
returns  reference to the matrix changed and throws
 \GOT{cvmexception}{cvmexception}
in case of inappropriate sizes of the operands.
Function is \emph{inherited} in  the classes
\GOT{scmatrix}{scmatrix} and
\GOT{schmatrix}{schmatrix} and
\emph{not applicable} to objects of the class
\GOT{scbmatrix}{scbmatrix} (i.e. \GOT{cvmexception}{cvmexception} would be thrown
in case of using it for objects of that class).
See also \GOT{schmatrix}{schmatrix},
\GOT{cmatrix}{cmatrix}.
Example:
\begin{Verbatim}
using namespace cvm;
std::cout.setf (std::ios::scientific | std::ios::left); 
std::cout.precision (3);

try {
    std::complex<double> alpha = std::complex<double>(1.3,0.21);
    std::complex<double> beta = std::complex<double>(0.5,-0.1);
    cmatrix m1(2,3);
    cmatrix m2(3,2);
    schmatrix ms(2);
    cmatrix m(2,3);
    m.randomize_real(-1., 2.); m.randomize_imag(1., 3.); 
    m1.randomize_real(-1., 3.); m1.randomize_imag(1., 2.);
    m2.randomize_real(0., 2.); m2.randomize_imag(-3., -1.);
    ms.randomize_real(-3., 1.); ms.randomize_imag(-1.3, 4.);

    std::cout << ms * m1 * alpha + m * beta << std::endl;
    std::cout << m.hemm (true, ms, m1, alpha, beta) << std::endl;

    m.resize(3,2);
    m.randomize_real(-1.4, 1.3); m.randomize_imag(1.1, 3.); 
    std::cout << m2 * ms * alpha + m * beta << std::endl;
    std::cout << m.hemm (false, ms, m2, alpha, beta);
}
catch (std::exception& e) {
    std::cout << "Exception " << e.what () << std::endl;
}
\end{Verbatim}
prints
\begin{Verbatim}
(1.096e+00,-7.692e+00) (-7.923e+00,-3.909e+00) (-1.324e+01,-5.264e+00)
(2.415e+00,1.240e+00) (4.384e-01,-1.771e+00) (7.495e-01,-2.740e+00)

(1.096e+00,-7.692e+00) (-7.923e+00,-3.909e+00) (-1.324e+01,-5.264e+00)
(2.415e+00,1.240e+00) (4.384e-01,-1.771e+00) (7.495e-01,-2.740e+00)

(-5.007e+00,1.010e+01) (2.341e+00,3.248e+00)
(-8.753e+00,7.854e+00) (3.152e+00,4.491e+00)
(-9.162e+00,6.401e+00) (-1.168e+00,3.973e+00)

(-5.007e+00,1.010e+01) (2.341e+00,3.248e+00)
(-8.753e+00,7.854e+00) (3.152e+00,4.491e+00)
(-9.162e+00,6.401e+00) (-1.168e+00,3.973e+00)
\end{Verbatim}
\newpage



\subsubsection{randomize\_real}
Function%
\pdfdest name {cmatrix.randomizereal} fit
\begin{verbatim}
cmatrix& cmatrix::randomize_real (TR dFrom, TR dTo);
\end{verbatim}
fills  real part of  calling matrix with 
pseudo-random numbers distributed between
\verb"dFrom" and \verb"dTo".
Function
returns  reference to the matrix changed.
Function is \emph{redefined} in the classes
\GOT{scmatrix}{scmatrix}, \GOT{scbmatrix}{scbmatrix} and
\GOT{schmatrix}{schmatrix}.
See also
\GOT{cmatrix}{cmatrix}.
Example:
\begin{Verbatim}
using namespace cvm;
std::cout.setf (std::ios::scientific | std::ios::left); 
std::cout.precision (3);

cmatrix m(2,3);
m.randomize_real(-1., 2.);
std::cout << m;
\end{Verbatim}
prints
\begin{Verbatim}
(1.090e+00,0.000e+00) (-6.375e-01,0.000e+00) (1.248e+00,0.000e+00)
(-1.272e-01,0.000e+00) (-8.557e-01,0.000e+00) (4.848e-01,0.000e+00)
\end{Verbatim}
\newpage


\subsubsection{randomize\_imag}
Function%
\pdfdest name {cmatrix.randomizeimag} fit
\begin{verbatim}
cmatrix& cmatrix::randomize_imag (TR dFrom, TR dTo);
\end{verbatim}
fills  imaginary part of  calling matrix with 
pseudo-random numbers distributed between
\verb"dFrom" and \verb"dTo".
Function
returns  reference to the matrix changed.
Function is \emph{redefined} in the classes
\GOT{scmatrix}{scmatrix}, \GOT{scbmatrix}{scbmatrix} and
\GOT{schmatrix}{schmatrix}.
See also
\GOT{cmatrix}{cmatrix}.
Example:
\begin{Verbatim}
using namespace cvm;
std::cout.setf (std::ios::scientific | std::ios::left); 
std::cout.precision (3);

cmatrix m(2,3);
m.randomize_imag(-1., 2.);
std::cout << m;
\end{Verbatim}
prints
\begin{Verbatim}
(0.000e+00,1.113e+00) (0.000e+00,6.615e-01) (0.000e+00,1.017e+00)
(0.000e+00,-3.397e-01) (0.000e+00,1.577e+00) (0.000e+00,8.071e-01)
\end{Verbatim}
\newpage

