\bigskip
\noindent
\verb"template <typename TR>"\\
\verb"class rmatrix : public Matrix <TR,TR> {"\\
\verb"public:"\\
\verb"    "\GOT{rmatrix}{rmatrix.rmatrix ()}\verb" ();"\\
\verb"    "\GOT{rmatrix}{rmatrix.rmatrix (int,int)}\verb" (int nM, int nN);"\\
\verb"    "\GOT{rmatrix}{rmatrix.rmatrix (TR*,int,int)}\verb" (TR* pD, int nM, int nN);"\\
\verb"    "\GOT{rmatrix}{rmatrix.rmatrix (const TR*,int,int)}\verb" (const TR* pD, int nM, int nN);"\\
\verb"    "\GOT{rmatrix}{rmatrix.rmatrix (const rmatrix&)}\verb" (const rmatrix& m);"\\
\verb"    explicit "\GOT{rmatrix}{rmatrix.rmatrix (const rvector&,bool)}\verb" (const rvector& v, bool bBeColumn = true);"\\
\verb"    "\GOT{rmatrix}{rmatrix.submatrixctr}\verb" (rmatrix& m, int nRow, int nCol, int nHeight, int nWidth);"\\
\verb"    TR& "\GOT{operator ()}{rmatrix.operator (,)}\verb" (int im, int in) throw (cvmexception);"\\
\verb"    TR "\GOT{operator ()}{rmatrix.operator (,)}\verb" (int im, int in) const throw (cvmexception);"\\
\verb"    rvector "\GOT{operator ()}{rmatrix.operator ()}\verb" (int i) throw (cvmexception);"\\
\verb"    const rvector "\GOT{operator ()}{rmatrix.operator ()}\verb" (int i) const throw (cvmexception);"\\
\verb"    rvector "\GOT{operator []}{rmatrix.operator []}\verb" (int i) throw (cvmexception);"\\
\verb"    const rvector "\GOT{operator []}{rmatrix.operator []}\verb" (int i) const throw (cvmexception);"\\
\verb"    rvector "\GOT{diag}{rmatrix.diag}\verb" (int i) throw (cvmexception);"\\
\verb"    const rvector "\GOT{diag}{rmatrix.diag}\verb" (int i) const throw (cvmexception);"\\
\verb"    rmatrix& "\GOT{operator =}{rmatrix.operator = (const rmatrix&)}\verb" (const rmatrix& m) throw (cvmexception);"\\
\verb"    rmatrix& "\GOT{assign}{rmatrix.assign}\verb" (const rvector& v) throw (cvmexception);"\\
\verb"    rmatrix& "\GOT{assign}{rmatrix.assign}\verb" (const TR* pD);"\\
\verb"    rmatrix& "\GOT{assign}{rmatrix.assign (int, int, const rmatrix&)}\verb" (int nRow, int nCol, const rmatrix& m)"\\
\verb"                     throw (cvmexception);"\\
\verb"    rmatrix& "\GOT{set}{rmatrix.set}\verb" (TR x);"\\
\verb"    rmatrix& "\GOT{resize}{rmatrix.resize}\verb" (int nNewM, int nNewN) throw (cvmexception);"\\
\verb"    bool "\GOT{operator ==}{rmatrix.operator ==}\verb" (const rmatrix& m) const;"\\
\verb"    bool "\GOT{operator !=}{rmatrix.operator !=}\verb" (const rmatrix& m) const;"\\
\verb"    rmatrix& "\GOT{operator <{}<}{rmatrix.operator <<}\verb" (const rmatrix& m) throw (cvmexception);"\\
\verb"    rmatrix "\GOT{operator +}{rmatrix.operator +}\verb" (const rmatrix& m) const"\\
\verb"                        throw (cvmexception);"\\
\verb"    rmatrix "\GOT{operator -}{rmatrix.operator -}\verb" (const rmatrix& m) const"\\
\verb"                        throw (cvmexception);"\\
\verb"    rmatrix& "\GOT{sum}{rmatrix.sum}\verb" (const rmatrix& m1,"\\
\verb"                  const rmatrix& m2) throw (cvmexception);"\\
\verb"    rmatrix& "\GOT{diff}{rmatrix.diff}\verb" (const rmatrix& m1,"\\
\verb"                   const rmatrix& m2) throw (cvmexception);"\\
\verb"    rmatrix& "\GOT{operator +=}{rmatrix.operator +=}\verb" (const rmatrix& m) throw (cvmexception);"\\
\verb"    rmatrix& "\GOT{operator -=}{rmatrix.operator -=}\verb" (const rmatrix& m) throw (cvmexception);"\\
\verb"    rmatrix "\GOT{operator -}{rmatrix.operator - ()}\verb" () const;"\\
\verb"    rmatrix "\GOT{operator *}{rmatrix.operator * (TR)}\verb" (TR d) const;"\\
\verb"    rmatrix "\GOT{operator /}{rmatrix.operator / (TR)}\verb" (TR d) const"\\
\verb"                        throw (cvmexception);"\\
\verb"    rmatrix& "\GOT{operator *=}{rmatrix.operator *= (TR)}\verb" (TR d);"\\
\verb"    rmatrix& "\GOT{operator /=}{rmatrix.operator /= (TR)}\verb" (TR d) throw (cvmexception);"\\
\verb"    rmatrix& "\GOT{normalize}{rmatrix.normalize}\verb" ();"\\
\verb"    rmatrix "\GOT{operator \TildaT}{rmatrix.transposition}\verb" () const throw (cvmexception);"\\
\verb"    rmatrix& "\GOT{transpose}{rmatrix.transposition}\verb" (const rmatrix& m) throw (cvmexception);"\\
\verb"    rmatrix& "\GOT{transpose}{rmatrix.transposition}\verb" () throw (cvmexception);"\\
\verb"    rvector "\GOT{operator *}{rmatrix.operator * (const rvector&)}\verb" (const rvector& v) const"\\
\verb"                        throw (cvmexception);"\\
\verb"    rmatrix "\GOT{operator *}{rmatrix.operator * (const rmatrix&)}\verb" (const rmatrix& m) const"\\
\verb"                        throw (cvmexception);"\\
\verb"    rmatrix& "\GOT{mult}{rmatrix.mult}\verb" (const rmatrix& m1, const rmatrix& m2)"\\
\verb"                   throw (cvmexception);"\\
\verb"    rmatrix& "\GOT{rank1update}{rmatrix.rank1update}\verb" (const rvector& vCol,"\\
\verb"                          const rvector& vRow)"\\
\verb"                          throw (cvmexception);"\\
\verb"    rmatrix& "\GOT{swap\_rows}{rmatrix.swap_rows}\verb" (int n1, int n2) throw (cvmexception);"\\
\verb"    rmatrix& "\GOT{swap\_cols}{rmatrix.swap_cols}\verb" (int n1, int n2) throw (cvmexception);"\\
\verb"    rmatrix& "\GOT{solve}{rmatrix.solve}\verb" (const srmatrix& mA,"\\
\verb"                    const rmatrix& mB, TR& dErr)"\\
\verb"                    throw (cvmexception);"\\
\verb"    rmatrix& "\GOT{solve}{rmatrix.solve}\verb" (const srmatrix& mA,"\\
\verb"                    const rmatrix& mB) throw (cvmexception);"\\
\verb"    rmatrix& "\GOT{solve\_tran}{rmatrix.solvetran}\verb" (const srmatrix& mA,"\\
\verb"                         const rmatrix& mB, TR& dErr)"\\
\verb"                         throw (cvmexception);"\\
\verb"    rmatrix& "\GOT{solve\_tran}{rmatrix.solvetran}\verb" (const srmatrix& mA,"\\
\verb"                         const rmatrix& mB) throw (cvmexception);"\\
\verb"    rmatrix& "\GOT{solve\_lu}{rmatrix.solvelu}\verb" (const srmatrix& mA, const srmatrix& mLU,"\\
\verb"                       const int* pPivots, const rmatrix& mB, TR& dErr)"\\
\verb"                       throw (cvmexception);"\\
\verb"    rmatrix& "\GOT{solve\_lu}{rmatrix.solvelu}\verb" (const srmatrix& mA, const srmatrix& mLU,"\\
\verb"                       const int* pPivots, const rmatrix& mB)"\\
\verb"                       throw (cvmexception);"\\
\verb"    rvector "\GOT{svd}{rmatrix.svd}\verb" () const throw (cvmexception);"\\
\verb"    rvector "\GOT{svd}{rmatrix.svd}\verb" (srmatrix& mU, srmatrix& mVH) const"\\
\verb"                 throw (cvmexception);"\\
\verb"    rmatrix "\GOT{pinv}{rmatrix.pinv}\verb" (TR threshold = cvmMachSp()) const"\\
\verb"                  throw (cvmexception);"\\
\verb"    rmatrix& "\GOT{pinv}{rmatrix.pinv}\verb" (const rmatrix& mA, TR threshold = cvmMachSp())"\\
\verb"                   throw (cvmexception);"\\
\verb"    rmatrix "\GOT{gels}{rmatrix.gels}\verb" (bool transpose, const rmatrix& mB,"\\
\verb"                  rvector& vErr) const throw (cvmexception);"\\
\verb"    rmatrix& "\GOT{gels}{rmatrix.gels}\verb" (bool transpose, const rmatrix& mA, const rmatrix& mB,"\\
\verb"                   rvector& vErr) throw (cvmexception);"\\
\verb"    rvector "\GOT{gels}{rmatrix.gels}\verb" (bool transpose, const rvector& mB,"\\
\verb"                   TR& dErr) const throw (cvmexception);"\\
\verb"    rmatrix "\GOT{gelsy}{rmatrix.gelsy}\verb" (const rmatrix& mB, int& rank,"\\
\verb"                   TR tol = cvmMachSp()) const throw (cvmexception);"\\
\verb"    rmatrix& "\GOT{gelsy}{rmatrix.gelsy}\verb" (const rmatrix& mA, const rmatrix& mB, int& rank,"\\
\verb"                   TR tol = cvmMachSp()) throw (cvmexception);"\\
\verb"    rvector "\GOT{gelsy}{rmatrix.gelsy}\verb" (const rvector& mB, int& rank,"\\
\verb"                   TR tol = cvmMachSp()) const throw (cvmexception);"\\
\verb"    rmatrix "\GOT{gelss}{rmatrix.gelss}\verb" (const rmatrix& mB, rvector& sv,"\\
\verb"                   int& rank, TR tol = cvmMachSp()) const"\\
\verb"                   throw (cvmexception);"\\
\verb"    rmatrix& "\GOT{gelss}{rmatrix.gelss}\verb" (const rmatrix& mA, const rmatrix& mB, rvector& sv,"\\
\verb"                   int& rank, TR tol = cvmMachSp())"\\
\verb"                   throw (cvmexception);"\\
\verb"    rvector "\GOT{gelss}{rmatrix.gelss}\verb" (const rvector& mB, rvector& sv,"\\
\verb"                   int& rank, TR tol = cvmMachSp()) const"\\
\verb"                   throw (cvmexception);"\\
\verb"    rmatrix "\GOT{gelsd}{rmatrix.gelsd}\verb" (const rmatrix& mB, rvector& sv,"\\
\verb"                   int& rank, TR tol = cvmMachSp()) const"\\
\verb"                   throw (cvmexception);"\\
\verb"    rmatrix& "\GOT{gelsd}{rmatrix.gelsd}\verb" (const rmatrix& mA, const rmatrix& mB, rvector& sv,"\\
\verb"                   int& rank, TR tol = cvmMachSp())"\\
\verb"                   throw (cvmexception);"\\
\verb"    rvector "\GOT{gelsd}{rmatrix.gelsd}\verb" (const rvector& mB, rvector& sv,"\\
\verb"                   int& rank, TR tol = cvmMachSp()) const"\\
\verb"                   throw (cvmexception);"\\
\verb"    int "\GOT{rank}{rmatrix.rank}\verb" (TR eps = cvmMachSp ()) const throw (cvmexception);"\\
\verb"    rmatrix& "\GOT{ger}{rmatrix.ger}\verb" (TR dAlpha, const rvector& vCol,"\\
\verb"                  const rvector& vRow) throw (cvmexception);"\\
\verb"    rmatrix& "\GOT{gemm}{rmatrix.gemm}\verb" (const rmatrix& m1, bool bTrans1,"\\
\verb"                   const rmatrix& m2, bool bTrans2,"\\
\verb"                   TR dAlpha, TR dBeta) throw (cvmexception);"\\
\verb"    rmatrix& "\GOT{symm}{rmatrix.symm}\verb" (bool bLeft, const srsmatrix& ms, "\\
\verb"                   const rmatrix& m, TR dAlpha, TR dBeta)"\\
\verb"                   throw (cvmexception);"\\
\verb"    void "\GOT{qr}{rmatrix.qr}\verb"(rmatrix& mQ, srmatrix& mR) const throw (cvmexception);"\\
\verb"    void "\GOT{qr}{rmatrix.qr}\verb"(srmatrix& mQ, rmatrix& mR) const throw (cvmexception);"\\
\verb"    void "\GOT{lq}{rmatrix.lq}\verb"(srmatrix& mL, rmatrix& mQ) const throw (cvmexception);"\\
\verb"    void "\GOT{lq}{rmatrix.lq}\verb"(rmatrix& mL, srmatrix& mQ) const throw (cvmexception);"\\
\verb"    void "\GOT{rq}{rmatrix.rq}\verb"(srmatrix& mR, rmatrix& mQ) const throw (cvmexception);"\\
\verb"    void "\GOT{rq}{rmatrix.rq}\verb"(rmatrix& mR, srmatrix& mQ) const throw (cvmexception);"\\
\verb"    void "\GOT{ql}{rmatrix.ql}\verb"(rmatrix& mQ, srmatrix& mL) const throw (cvmexception);"\\
\verb"    void "\GOT{ql}{rmatrix.ql}\verb"(srmatrix& mQ, rmatrix& mL) const throw (cvmexception);"\\
\verb"    rmatrix& "\GOT{vanish}{rmatrix.vanish}\verb" ();"\\
\verb"    rmatrix& "\GOT{randomize}{rmatrix.randomize}\verb" (TR dFrom, TR dTo);"\\
\verb"};"
\newpage


\subsubsection{rmatrix ()}
Default constructor%
\pdfdest name {rmatrix.rmatrix ()} fit
\begin{verbatim}
rmatrix::rmatrix ();
\end{verbatim}
creates  empty \verb"rmatrix" object.
See also \GOT{rmatrix}{rmatrix}.
Example:
\begin{Verbatim}
using namespace cvm;

rmatrix m;
std::cout << m.msize() << std::endl << m.nsize() << std::endl;
std::cout << m.size() << std::endl;

m.resize (2, 3);
std::cout << m;
\end{Verbatim}
prints
\begin{Verbatim}
0
0
0
0 0 0
0 0 0
\end{Verbatim}
\newpage



\subsubsection{rmatrix (int,int)}
Constructor%
\pdfdest name {rmatrix.rmatrix (int,int)} fit
\begin{verbatim}
rmatrix::rmatrix (int nM, int nN);
\end{verbatim}
creates  $m\times n$ \verb"rmatrix" object where $m$ is passed in
\verb"nM" parameter (number of rows) and $n$ is passed in
\verb"nN" (number of columns). Constructor sets all elements to zero.
It throws  \GOT{cvmexception}{cvmexception}
in case of non-positive sizes passed or memory allocation failure.
See also \GOT{rmat\-rix}{rmatrix}.
Example:
\begin{Verbatim}
using namespace cvm;

rmatrix m (3, 4);
std::cout << m.msize() << std::endl << m.nsize()
          << std::endl << m.size() << std::endl << m;
\end{Verbatim}
prints
\begin{Verbatim}
3
4
12
0 0 0 0
0 0 0 0
0 0 0 0
\end{Verbatim}
\newpage




\subsubsection{rmatrix (TR*,int,int)}
Constructor%
\pdfdest name {rmatrix.rmatrix (TR*,int,int)} fit
\begin{verbatim}
rmatrix::rmatrix (TR* pD, int nM, int nN);
\end{verbatim}
creates  $m\times n$ \verb"rmatrix" object where $m$ is passed in
\verb"nM" parameter (number of rows) and $n$ is passed in
\verb"nN" (number of columns).
Unlike others, this constructor \textit{does not allocate  memory}.
It just shares  memory with an array pointed to by \verb"pD".
It throws  \GOT{cvmexception}{cvmexception}
in case of non-positive size passed.
See also \GOT{rmatrix}{rmatrix}, \GOT{rmatrix (const TR*,int,int)}{rmatrix.rmatrix (const TR*,int,int)}.
Example:
\begin{Verbatim}
using namespace cvm;
std::cout.setf (std::ios::scientific | std::ios::left);
std::cout.precision (2);
double a[] = {1., 2., 3., 4., 5., 6.};
rmatrix m (a, 2, 3);

m(1,1) = 7.77;
std::cout << m << std::endl;
std::cout << a[0] << " " << a[1] << " " << a[2] << std::endl;
\end{Verbatim}
prints
\begin{Verbatim}
7.77e+000 3.00e+000 5.00e+000
2.00e+000 4.00e+000 6.00e+000

7.77e+000 2.00e+000 3.00e+000
\end{Verbatim}
\newpage


\subsubsection{rmatrix (const TR*,int,int)}
Constructor%
\pdfdest name {rmatrix.rmatrix (const TR*,int,int)} fit
\begin{verbatim}
rmatrix::rmatrix (const TR* pD, int nM, int nN);
\end{verbatim}
creates  $m\times n$ \verb"rmatrix" object where $m$ is passed in
\verb"nM" parameter (number of rows) and $n$ is passed in
\verb"nN" (number of columns)
and copies $m*n$ elements of an array  \verb"pD" to it \GOT{by colums}{SubSubSectionStorage}.
It throws  \GOT{cvmexception}{cvmexception}
in case of non-positive size passed.
See also \GOT{rmatrix}{rmatrix}, \GOT{rmatrix (TR*,int,int)}{rmatrix.rmatrix (TR*,int,int)}.
Example:
\begin{Verbatim}
using namespace cvm;
std::cout.setf (std::ios::scientific | std::ios::left);
std::cout.precision (2);
const double a[] = {1., 2., 3., 4., 5., 6.};
rmatrix m (a, 2, 3);

m(1,1) = 7.77;
std::cout << m << std::endl;
std::cout << a[0] << " " << a[1] << " " << a[2] << std::endl;
\end{Verbatim}
prints
\begin{Verbatim}
7.77e+000 3.00e+000 5.00e+000
2.00e+000 4.00e+000 6.00e+000

1.00e+000 2.00e+000 3.00e+000
\end{Verbatim}
\newpage



\subsubsection{rmatrix (const rmatrix\&)}
Copy constructor%
\pdfdest name {rmatrix.rmatrix (const rmatrix&)} fit
\begin{verbatim}
rmatrix::rmatrix (const rmatrix& m);
\end{verbatim}
creates  \verb"rmatrix" object as a copy of \verb"m".
It throws  \GOT{cvmexception}{cvmexception}
in case of memory allocation failure.
See also \GOT{rmatrix}{rmatrix}.
Example:
\begin{Verbatim}
using namespace cvm;
std::cout.setf (std::ios::scientific | std::ios::left);
std::cout.precision (2);
double a[] = {1., 2., 3., 4., 5., 6.};
rmatrix m (a, 2, 3);
rmatrix mc(m);

m(1,1) = 7.77;
std::cout << m << std::endl << mc;
\end{Verbatim}
prints
\begin{Verbatim}
7.77e+000 3.00e+000 5.00e+000
2.00e+000 4.00e+000 6.00e+000

1.00e+000 3.00e+000 5.00e+000
2.00e+000 4.00e+000 6.00e+000
\end{Verbatim}
\newpage




\subsubsection{rmatrix (const rvector\&,bool)}
Constructor%
\pdfdest name {rmatrix.rmatrix (const rvector&,bool)} fit
\begin{verbatim}
explicit rmatrix::rmatrix (const rvector& v, bool bBeColumn = true);
\end{verbatim}
creates  \verb"rmatrix" object containing \verb"v.size()" rows and
1 column if \verb"bBeColumn" is \verb"true" or 1 row and
\verb"v.size()" columns otherwise.
After that it copies  vector \verb"v"'s elements to
the matrix created.
Constructor throws  \GOT{cvmexception}{cvmexception}
in case of memory allocation failure.
See also \GOT{rmatrix}{rmatrix}, \GOT{rvector}{rvector}.
Example:
\begin{Verbatim}
using namespace cvm;
std::cout.setf (std::ios::scientific | std::ios::left);
std::cout.precision (2);
rvector v(3);
v(1) = 1.;
v(2) = 2.;
v(3) = 3.;

rmatrix mc (v);
rmatrix mr (v, false);

std::cout << mc << std::endl << mr;
\end{Verbatim}
prints
\begin{Verbatim}
1.00e+000
2.00e+000
3.00e+000

1.00e+000 2.00e+000 3.00e+000
\end{Verbatim}
\newpage



\subsubsection{submatrix}
Submatrix constructor%
\pdfdest name {rmatrix.submatrixctr} fit
\begin{verbatim}
rmatrix::rmatrix (rmatrix& m, int nRow, int nCol,
                              int nHeight, int nWidth);
\end{verbatim}
creates  \verb"rmatrix" object as  \emph{submatrix} of \verb"m".
It means that the matrix object created shares  memory with some part
of \verb"m". This part is defined by its upper left corner (parameters
\verb"nRow" and \verb"nCol", both are \Based)
and its height and width (parameters
\verb"nHeight" and \verb"nWidth").
See also \GOT{rmatrix}{rmatrix}.
Example:
\begin{Verbatim}
using namespace cvm;
rmatrix m(4,5);
rmatrix subm(m, 2, 2, 2, 2);
subm.set(1.);

std::cout << m;
\end{Verbatim}
prints
\begin{Verbatim}
0 0 0 0 0
0 1 1 0 0
0 1 1 0 0
0 0 0 0 0
\end{Verbatim}
\newpage


\subsubsection{operator (,)}
Indexing operators%
\pdfdest name {rmatrix.operator (,)} fit
\begin{verbatim}
TR& rmatrix::operator () (int im, int in) throw (cvmexception);
TR rmatrix::operator () (int im, int in) const throw (cvmexception);
\end{verbatim}
provide access to a particular element of a calling matrix. The first version
is applicable to  non-constant object.
This version returns  \emph{l-value}
in order to make possible write access to an element.
Both operators are \Based.
Operators throw \GOT{cvmexception}{cvmexception}
if \verb"im" is outside of \verb"[1,msize()]" range or
\verb"in" is outside of \verb"[1,nsize()]" range.
Operators are \emph{inherited}
in the classes
\GOT{srmatrix}{srmatrix} and \GOT{srbmatrix}{srbmatrix}.
Operators are \emph{redefined}
in the class \GOT{srsmatrix}{srsmatrix}.
See also \GOT{rmatrix}{rmatrix},
\GOT{Matrix::msize()}{Matrix.msize},
\GOT{Matrix::nsize()}{Matrix.nsize}.
Example:
\begin{Verbatim}
using namespace cvm;
std::cout.setf (std::ios::scientific | std::ios::left);
std::cout.precision (2);
try {
    double a[] = {1., 2., 3., 4., 5., 6.};
    const rmatrix m (a, 2, 3);
    rmatrix ms(m);

    std::cout << m(1,1) << " " << m(2,3) << std::endl << std::endl;

    ms(2,2) = 7.77;
    std::cout << ms;
}
catch (std::exception& e) {
    std::cout << "Exception " << e.what () << std::endl;
}
\end{Verbatim}
prints
\begin{Verbatim}
1.00e+00 6.00e+00

1.00e+00 3.00e+00 5.00e+00
2.00e+00 7.77e+00 6.00e+00
\end{Verbatim}
\newpage


\subsubsection{operator ()}
Indexing operators%
\pdfdest name {rmatrix.operator ()} fit
\begin{verbatim}
rvector rmatrix::operator () (int i) throw (cvmexception);
const rvector rmatrix::operator () (int i) const throw (cvmexception);
\end{verbatim}
provide access to  \hbox{$i$-th} column of a calling matrix. The first version
is applicable to non-constant object and
\emph{returns  l-value}, i.e. vector returned shares  memory
with \hbox{$i$-th} column of the matrix
in order to make possible write access to it.
The second version creates  \emph{copy} of a column and therefore is
\emph{not l-value}.
Both operators are \Based.
Operators throw \GOT{cvmexception}{cvmexception}
if the parameter \verb"i" is outside of \verb"[1,nsize()]" range.
Operators are \emph{inherited}
in  the class
\GOT{srmatrix}{srmatrix}.
Operators are \emph{redefined}
in  the classes \GOT{srbmatrix}{srbmatrix}
and \GOT{srsmatrix}{srsmatrix}.
See also \GOT{rmatrix}{rmatrix},
\GOT{Matrix::nsize()}{Matrix.nsize}.
Example:
\begin{Verbatim}
using namespace cvm;
std::cout.setf (std::ios::scientific | std::ios::left);
std::cout.precision (2);
try {
    double a[] = {1., 2., 3., 4., 5., 6.};
    const rmatrix m (a, 2, 3);
    srmatrix ms(2);

    std::cout << m(2) << std::endl;

    ms(2) = m(3);
    std::cout << ms;
}
catch (std::exception& e) {
    std::cout << "Exception " << e.what () << std::endl;
}
\end{Verbatim}
prints
\begin{Verbatim}
3.00e+00 4.00e+00

0.00e+00 5.00e+00
0.00e+00 6.00e+00
\end{Verbatim}
\newpage


\subsubsection{operator []}
Indexing operators%
\pdfdest name {rmatrix.operator []} fit
\begin{verbatim}
rvector rmatrix::operator [] (int i) throw (cvmexception);
const rvector rmatrix::operator [] (int i) const throw (cvmexception);
\end{verbatim}
provide access to  \hbox{$i$-th} row of  matrix. The first version
is applicable to non-constant object and
\emph{returns  l-value}, i.e. vector returned shares  memory
with  \hbox{$i$-th} row of the matrix
in order to make possible write access to it.
The second version creates \emph{copy} of a row and therefore is
\emph{not  l-value}.
Both operators are \Based.
Operators throw \GOT{cvmexception}{cvmexception}
if \verb"i" is outside of \verb"[1,msize()]" range.
Operators are \emph{inherited}
in  the class
\GOT{srmatrix}{srmatrix}.
Operators are \emph{redefined}
in  the classes \GOT{srbmatrix}{srbmatrix}
and \GOT{srsmatrix}{srsmatrix}.
See also \GOT{rmatrix}{rmatrix},
\GOT{Matrix::msize()}{Matrix.msize}.
Example:
\begin{Verbatim}
using namespace cvm;
std::cout.setf (std::ios::scientific | std::ios::left);
std::cout.precision (2);
try {
    double a[] = {1., 2., 3., 4., 5., 6.};
    const rmatrix m (a, 2, 3);
    srmatrix ms(3);

    std::cout << m[1] << std::endl;

    ms[1] = m[2];
    std::cout << ms;
}
catch (std::exception& e) {
    std::cout << "Exception " << e.what () << std::endl;
}
\end{Verbatim}
prints
\begin{Verbatim}
1.00e+00 3.00e+00 5.00e+00

2.00e+00 4.00e+00 6.00e+00
0.00e+00 0.00e+00 0.00e+00
0.00e+00 0.00e+00 0.00e+00
\end{Verbatim}
\newpage



\subsubsection{diag}
Functions%
\pdfdest name {rmatrix.diag} fit
\begin{verbatim}
rvector rmatrix::diag (int i) throw (cvmexception);
const rvector rmatrix::diag (int i) const throw (cvmexception);
\end{verbatim}
provide access to  \hbox{$i$-th} diagonal of a calling matrix,
where $i=0$ for main diagonal, $i<0$ for lower diagonals
and $i>0$ for upper ones.
The first version
is applicable to  non-constant object and
\emph{returns  l-value}, i.e. vector returned shares  memory
with  \hbox{$i$-th} diagonal of the matrix
in order to make possible write access to it.
The second version creates  \emph{copy} of a diagonal and therefore is
\emph{not  l-value}.
Functions throw 
\GOT{cvmexception}{cvmexception}
if  parameter \verb"i" is outside of
\verb"[-msize()+1,nsize()-1]" range.
Functions are \emph{inherited}
in  the classes
\GOT{srmatrix}{srmatrix} and \GOT{srbmatrix}{srbmatrix}.
Functions are \emph{redefined}
in  the class \GOT{srsmatrix}{srsmatrix}.
See also \GOT{rmatrix}{rmatrix}.
Example:
\begin{Verbatim}
using namespace cvm;
try {
    double a[] = {1., 2., 3., 4., 5., 6., 7., 8., 9.};
    rmatrix  m(2,3);
    const srmatrix ms(a,3);

    m.diag(-1).set(1.);
    m.diag(0).set(2.);
    m.diag(1).set(3.);
    m.diag(2).set(4.);
    std::cout << m << std::endl;

    std::cout << ms << std::endl;
    std::cout << ms.diag(0) << ms.diag(1);
}
catch (std::exception& e) {
    std::cout << "Exception " << e.what () << std::endl;
}
\end{Verbatim}
prints
\begin{Verbatim}
2 3 4
1 2 3

1 4 7
2 5 8
3 6 9

1 5 9
4 8
\end{Verbatim}
\newpage




\subsubsection{operator = (const rmatrix\&)}
Operator%
\pdfdest name {rmatrix.operator = (const rmatrix&)} fit
\begin{verbatim}
rmatrix& rmatrix::operator = (const rmatrix& m) throw (cvmexception);
\end{verbatim}
sets  every element of a calling matrix to be equal to
appropriate element of a matrix \verb"m"
and returns a reference to
the matrix changed.
Operator throws \GOT{cvmexception}{cvmexception}
in case of different matrix sizes.
Operator is \emph{redefined} in  the classes
\GOT{srmatrix}{srmatrix}, \GOT{srbmatrix}{srbmatrix}
and \GOT{srsmatrix}{srsmatrix}.
See also \GOT{rmatrix}{rmatrix}.
Example:
\begin{Verbatim}
using namespace cvm;
std::cout.setf (std::ios::scientific | std::ios::left);
std::cout.precision (2);

try {
    double a[] = {1., 2., 3., 4., 5., 6.};
    const rmatrix m1(a, 3, 2);
    rmatrix m2(3, 2);

    m2 = m1;
    std::cout << m2;
}
catch (std::exception& e) {
    std::cout << "Exception " << e.what () << std::endl;
}
\end{Verbatim}
prints
\begin{Verbatim}
1.00e+000 4.00e+000
2.00e+000 5.00e+000
3.00e+000 6.00e+000
\end{Verbatim}
\newpage



\subsubsection{assign (const TR*)}
Function%
\pdfdest name {rmatrix.assign} fit
\begin{verbatim}
rmatrix& rmatrix::assign (const rvector& v) throw (cvmexception);
rmatrix& rmatrix::assign (const TR* pD);
\end{verbatim}
sets every element of a calling matrix to be equal to
appropriate element of  vector~\verb'v'
or array pointed to by~\verb"pD"
and returns a reference to
the matrix changed.
In first version it's assumed that vector passed is long
enough to fill calling matrix. Function throws \GOT{cvmexception}{cvmexception} 
otherwise.
Function is \emph{redefined} in the classes
\GOT{srmatrix}{srmatrix}, \GOT{srbmatrix}{srbmatrix}
and \GOT{srsmatrix}{srsmatrix}.
See also \GOT{rmatrix}{rmatrix}.
Example:
\begin{Verbatim}
using namespace cvm;
std::cout.setf (std::ios::scientific | std::ios::left);
std::cout.precision (2);
const double a[] = {1., 2., 3., 4., 5., 6.};
rmatrix m(2, 3);

m.assign(a);
std::cout << m;
\end{Verbatim}
prints
\begin{Verbatim}
1.00e+000 3.00e+000 5.00e+000
2.00e+000 4.00e+000 6.00e+000
\end{Verbatim}
\newpage



\subsubsection{assign (int, int, const rmatrix\&)}
Function%
\pdfdest name {rmatrix.assign (int, int, const rmatrix&)} fit
\begin{verbatim}
rmatrix& rmatrix::assign (int nRow, int nCol, const rmatrix& m)
throw (cvmexception);
\end{verbatim}
sets sub-matrix of a calling matrix beginning with \Based row
\verb"nRow" and column \verb"nCol" to a matrix \verb"m" and
returns a reference to the matrix changed. Function throws 
 \GOT{cvmexception}{cvmexception} if \verb"nRow"
or \verb"nCol" are not positive or matrix \verb"m" doesn't fit.
Function is \emph{redefined} in the classes
\GOT{srmatrix}{srmatrix}, \GOT{srbmatrix}{srbmatrix}
and \GOT{srsmatrix}{srsmatrix}.
See also \GOT{rmatrix}{rmatrix}.
Example:
\begin{Verbatim}
using namespace cvm;

rmatrix m1(4,5);
rmatrix m2(2,2);
m1.set(1.);
m2.set(2.);
m1.assign(2,3,m2);
std::cout << m1;
\end{Verbatim}
prints
\begin{Verbatim}
1 1 1 1 1
1 1 2 2 1
1 1 2 2 1
1 1 1 1 1
\end{Verbatim}
\newpage




\subsubsection{set (TR)}
Function%
\pdfdest name {rmatrix.set} fit
\begin{verbatim}
rmatrix& rmatrix::set (TR x);
\end{verbatim}
sets every element of a calling matrix to be equal to
parameter \verb"x" and returns a reference to
the matrix changed.
Use \GOT{vanish}{rmatrix.vanish} to set every element
of a calling matrix to be equal to zero.
Function is \emph{redefined} in the classes
\GOT{srmatrix}{srmatrix}, \GOT{srbmatrix}{srbmatrix}
and \GOT{srsmatrix}{srsmatrix}.
See also \GOT{rmatrix}{rmatrix}.
Example:
\begin{Verbatim}
using namespace cvm;
std::cout.setf (std::ios::scientific | std::ios::left);
std::cout.precision (2);
rmatrix m(2, 3);

m.set(3.);
std::cout << m;
\end{Verbatim}
prints
\begin{Verbatim}
3.00e+000 3.00e+000 3.00e+000
3.00e+000 3.00e+000 3.00e+000
\end{Verbatim}
\newpage



\subsubsection{resize}
Function%
\pdfdest name {rmatrix.resize} fit
\begin{verbatim}
rmatrix& rmatrix::resize (int nNewM, int nNewN)
throw (cvmexception);
\end{verbatim}
changes  size of a calling matrix to \verb"nNewM" by \verb"nNewN"
and returns a reference to
the matrix changed. In case of increasing of its size, the matrix
is filled up with zeroes.
Function throws  \GOT{cvmexception}{cvmexception}
in case of negative size passed or memory allocation failure.
Function is \emph{redefined} in the classes
\GOT{srmatrix}{srmatrix}, \GOT{srbmatrix}{srbmatrix}
and \GOT{srsmatrix}{srsmatrix}.
See also \GOT{rmatrix}{rmatrix}.
Example:
\begin{Verbatim}
using namespace cvm;
std::cout.setf (std::ios::scientific | std::ios::left);
std::cout.precision (2);
try {
    double a[] = {1., 2., 3., 4., 5., 6.};
    rmatrix m(a, 2, 3);
    std::cout << m << std::endl;
    m.resize (2, 2);
    std::cout << m << std::endl;
    m.resize (3, 3);
    std::cout << m;
}
catch (std::exception& e) {
    std::cout << "Exception " << e.what () << std::endl;
}
\end{Verbatim}
prints
\begin{Verbatim}
1.00e+000 3.00e+000 5.00e+000
2.00e+000 4.00e+000 6.00e+000

1.00e+000 3.00e+000
2.00e+000 4.00e+000

1.00e+000 3.00e+000 0.00e+000
2.00e+000 4.00e+000 0.00e+000
0.00e+000 0.00e+000 0.00e+000
\end{Verbatim}
\newpage




\subsubsection{operator ==}
Operator%
\pdfdest name {rmatrix.operator ==} fit
\begin{verbatim}
bool rmatrix::operator == (const rmatrix& m) const;
\end{verbatim}
compares  calling matrix with a matrix \verb"m"
and returns \verb"true" if they have the same sizes
and their appropriate elements differ by not more than the
\GO{smallest normalized positive number}{Utilities.cvmMachMin}.
Returns \verb"false" otherwise.
Operator is \emph{redefined} in the classes
\GOT{srmatrix}{srmatrix}, \GOT{srbmatrix}{srbmatrix}
and \GOT{srsmatrix}{srsmatrix}.
See also \GOT{rmatrix}{rmatrix}.
Example:
\begin{Verbatim}
using namespace cvm;
double a[] = {1., 2., 3., 4.};
rmatrix m1(a, 2, 2);
rmatrix m2(2, 2);

m2(1,1) = 1.; m2(1,2) = 3.;
m2(2,1) = 2.; m2(2,2) = 4.;

std::cout << (m1 == m2) << std::endl;
\end{Verbatim}
prints
\begin{Verbatim}
1
\end{Verbatim}
\newpage




\subsubsection{operator !=}
Operator%
\pdfdest name {rmatrix.operator !=} fit
\begin{verbatim}
bool rmatrix::operator != (const rmatrix& m) const;
\end{verbatim}
compares  calling matrix with a matrix \verb"m"
and returns \verb"true" if they have different sizes
or at least one of their appropriate elements
differs by more than the
\GO{smallest normalized positive number}{Utilities.cvmMachMin}.
Returns \verb"false" otherwise.
Operator is \emph{redefined} in the classes
\GOT{srmatrix}{srmatrix}, \GOT{srbmatrix}{srbmatrix}
and \GOT{srsmatrix}{srsmatrix}.
See also \GOT{rmatrix}{rmatrix}.
Example:
\begin{Verbatim}
using namespace cvm;
double a[] = {1., 2., 3., 4.};
rmatrix m1(a, 2, 2);
rmatrix m2(2, 2);

m2(1,1) = 1.; m2(1,2) = 3.;
m2(2,1) = 2.; m2(2,2) = 4.;

std::cout << (m1 != m2) << std::endl;
\end{Verbatim}
prints
\begin{Verbatim}
0
\end{Verbatim}
\newpage




\subsubsection{operator <{}<}
Operator%
\pdfdest name {rmatrix.operator <<} fit
\begin{verbatim}
rmatrix& rmatrix::operator << (const rmatrix& m)
throw (cvmexception);
\end{verbatim}
destroys  calling matrix, creates  new one as a copy of \verb"m"
and returns a reference to
the matrix changed.
Operator throws  \GOT{cvmexception}{cvmexception}
in case of memory allocation failure.
Operator is \emph{redefined} in the classes
\GOT{srmatrix}{srmatrix}, \GOT{srbmatrix}{srbmatrix}
and \GOT{srsmatrix}{srsmatrix}.
See also \GOT{rmatrix}{rmatrix}.
Example:
\begin{Verbatim}
using namespace cvm;
std::cout.setf (std::ios::scientific | std::ios::left);
std::cout.precision (2);
try {
    rmatrix m(3,4);
    rmatrix mc(1,1);
    m(1,2) = 1.;
    m(3,4) = 2.;
    std::cout << m << mc << std::endl;

    mc << m;
    std::cout << mc;
}
catch (std::exception& e) {
    std::cout << "Exception " << e.what () << std::endl;
}
\end{Verbatim}
prints
\begin{Verbatim}
0.00e+000 1.00e+000 0.00e+000 0.00e+000
0.00e+000 0.00e+000 0.00e+000 0.00e+000
0.00e+000 0.00e+000 0.00e+000 2.00e+000
0.00e+000

0.00e+000 1.00e+000 0.00e+000 0.00e+000
0.00e+000 0.00e+000 0.00e+000 0.00e+000
0.00e+000 0.00e+000 0.00e+000 2.00e+000
\end{Verbatim}
\newpage



\subsubsection{operator +}
Operator%
\pdfdest name {rmatrix.operator +} fit
\begin{verbatim}
rmatrix rmatrix::operator + (const rmatrix& m) const
throw (cvmexception);
\end{verbatim}
creates an object of type \verb"rmatrix" as a sum of
a calling matrix and a matrix \verb"m".
Operator throws  \GOT{cvmexception}{cvmexception}
in case of different sizes of the operands.
Operator is \emph{redefined} in the classes
\GOT{srmatrix}{srmatrix}, \GOT{srbmatrix}{srbmatrix}
and \GOT{srsmatrix}{srsmatrix}.
See also \GOT{rmatrix::sum}{rmatrix.sum}, \GOT{rmatrix}{rmatrix}.
Example:
\begin{Verbatim}
using namespace cvm;
try {
    double a[] = {1., 2., 3., 4., 5., 6.};
    const rmatrix ma(a,2,3);
    rmatrix mb(2,3);
    mb.set(1.);

    std::cout << ma + mb << std::endl;
    std::cout << ma + ma;
}
catch (std::exception& e) {
    std::cout << "Exception " << e.what () << std::endl;
}
\end{Verbatim}
prints
\begin{Verbatim}
2 4 6
3 5 7

2 6 10
4 8 12
\end{Verbatim}
\newpage




\subsubsection{operator -}
Operator%
\pdfdest name {rmatrix.operator -} fit
\begin{verbatim}
rmatrix rmatrix::operator - (const rmatrix& m) const
throw (cvmexception);
\end{verbatim}
creates an object of type \verb"rmatrix" as a difference of
a calling matrix and a matrix \verb"m".
It throws  \GOT{cvmexception}{cvmexception}
in case of different sizes of the operands.
Operator is \emph{redefined} in the classes
\GOT{srmatrix}{srmatrix}, \GOT{srbmatrix}{srbmatrix}
and \GOT{srsmatrix}{srsmatrix}.
See also \GOT{rmatrix::diff}{rmatrix.diff}, \GOT{rmatrix}{rmatrix}.
Example:
\begin{Verbatim}
using namespace cvm;
try {
    double a[] = {1., 2., 3., 4., 5., 6.};
    const rmatrix ma(a,2,3);
    rmatrix mb(2,3);
    mb.set(1.);

    std::cout << ma - mb << std::endl;
    std::cout << ma - ma;
}
catch (std::exception& e) {
    std::cout << "Exception " << e.what () << std::endl;
}
\end{Verbatim}
prints
\begin{Verbatim}
0 2 4
1 3 5

0 0 0
0 0 0
\end{Verbatim}
\newpage




\subsubsection{sum}
Function%
\pdfdest name {rmatrix.sum} fit
\begin{verbatim}
rmatrix& rmatrix::sum (const rmatrix& m1, const rmatrix& m2)
throw (cvmexception);
\end{verbatim}
assigns the result of addition of
matrices \verb"m1" and \verb"m2"  to a calling matrix
and returns a reference to
the matrix changed.
It throws  \GOT{cvmexception}{cvmexception}
in case of different sizes of the operands.
Function is \emph{redefined} in the classes
\GOT{srmatrix}{srmatrix}, \GOT{srbmatrix}{srbmatrix}
and \GOT{srsmatrix}{srsmatrix}.
See also \GOT{rmatrix::operator +~}{rmatrix.operator +},
\GOT{rmatrix}{rmatrix}.
Example:
\begin{Verbatim}
using namespace cvm;

try {
    double a[] = {1., 2., 3., 4., 5., 6.};
    const rmatrix m1(a, 2, 3);
    rmatrix m2(2, 3);
    rmatrix m(2, 3);
    m2.set(1.);

    std::cout << m.sum(m1, m2) << std::endl;
    std::cout << m.sum(m, m2);
}
catch (std::exception& e) {
    std::cout << "Exception " << e.what () << std::endl;
}
\end{Verbatim}
prints
\begin{Verbatim}
2 4 6
3 5 7

3 5 7
4 6 8
\end{Verbatim}
\newpage



\subsubsection{diff}
Function%
\pdfdest name {rmatrix.diff} fit
\begin{verbatim}
rmatrix& rmatrix::diff (const rmatrix& m1, const rmatrix& m2)
throw (cvmexception);
\end{verbatim}
assigns the result of subtraction of
matrices \verb"m1" and \verb"m2" to a calling matrix
and returns a reference to
the matrix changed.
It throws  \GOT{cvmexception}{cvmexception}
in case of different sizes of the operands.
Function is \emph{redefined} in the classes
\GOT{srmatrix}{srmatrix}, \GOT{srbmatrix}{srbmatrix}
and \GOT{srsmatrix}{srsmatrix}.
See also \GOT{rmatrix::operator -~}{rmatrix.operator -},
\GOT{rmatrix}{rmatrix}.
Example:
\begin{Verbatim}
using namespace cvm;

try {
    double a[] = {1., 2., 3., 4., 5., 6.};
    const rmatrix m1(a, 2, 3);
    rmatrix m2(2, 3);
    rmatrix m(2, 3);
    m2.set(1.);

    std::cout << m.diff(m1, m2) << std::endl;
    std::cout << m.diff(m, m2);
}
catch (std::exception& e) {
    std::cout << "Exception " << e.what () << std::endl;
}
\end{Verbatim}
prints
\begin{Verbatim}
0 2 4
1 3 5

-1 1 3
0 2 4
\end{Verbatim}
\newpage



\subsubsection{operator +=}
Operator%
\pdfdest name {rmatrix.operator +=} fit
\begin{verbatim}
rmatrix& rmatrix::operator += (const rmatrix& m) throw (cvmexception);
\end{verbatim}
adds  matrix \verb"m" to a calling matrix
and returns a reference to
the matrix changed.
It throws  \GOT{cvmexception}{cvmexception}
in case of different sizes of the operands.
Operator is \emph{redefined} in the classes
\GOT{srmatrix}{srmatrix}, \GOT{srbmatrix}{srbmatrix}
and \GOT{srsmatrix}{srsmatrix}.
See also \GOT{rmatrix::operator +~}{rmatrix.operator +},
\GOT{rmatrix::sum}{rmatrix.sum},
\GOT{rmatrix}{rmatrix}.
Example:
\begin{Verbatim}
using namespace cvm;
try {
    rmatrix m1(2, 3);
    rmatrix m2(2, 3);
    m1.set(1.);
    m2.set(2.);

    m1 += m2;
    std::cout << m1 << std::endl;

    // well, you can do this too, but temporary object would be created
    m2 += m2;
    std::cout << m2;
}
catch (std::exception& e) {
    std::cout << "Exception " << e.what () << std::endl;
}
\end{Verbatim}
prints
\begin{Verbatim}
3 3 3
3 3 3

4 4 4
4 4 4
\end{Verbatim}
\newpage




\subsubsection{operator -=}
Operator%
\pdfdest name {rmatrix.operator -=} fit
\begin{verbatim}
rmatrix& rmatrix::operator -= (const rmatrix& m) throw (cvmexception);
\end{verbatim}
subtracts  matrix \verb"m" from  calling matrix
and returns a reference to
the matrix changed.
It throws  \GOT{cvmexception}{cvmexception}
in case of different sizes of the operands.
Operator is \emph{redefined} in the classes
\GOT{srmatrix}{srmatrix}, \GOT{srbmatrix}{srbmatrix}
and \GOT{srsmatrix}{srsmatrix}.
See also \GOT{rmatrix::operator -~}{rmatrix.operator -},
\GOT{rmatrix::diff}{rmatrix.diff},
\GOT{rmatrix}{rmatrix}.
Example:
\begin{Verbatim}
using namespace cvm;
try {
    rmatrix m1(2, 3);
    rmatrix m2(2, 3);
    m1.set(1.);
    m2.set(2.);

    m1 -= m2;
    std::cout << m1 << std::endl;

    // well, you can do this too, but temporary object would be created
    m2 -= m2;
    std::cout << m2;
}
catch (std::exception& e) {
    std::cout << "Exception " << e.what () << std::endl;
}
\end{Verbatim}
prints
\begin{Verbatim}
-1 -1 -1
-1 -1 -1

0 0 0
0 0 0
\end{Verbatim}
\newpage




\subsubsection{operator - ()}
Operator%
\pdfdest name {rmatrix.operator - ()} fit
\begin{verbatim}
rmatrix rmatrix::operator - () const throw (cvmexception);
\end{verbatim}
creates an object of type \verb"rmatrix" as
a calling matrix multiplied by $-1$.
Operator is \emph{redefined} in the classes
\GOT{srmatrix}{srmatrix}, \GOT{srbmatrix}{srbmatrix}
and \GOT{srsmatrix}{srsmatrix}.
See also \GOT{rmatrix}{rmatrix}.
Example:
\begin{Verbatim}
using namespace cvm;

double a[] = {1., 2., 3., 4., 5., 6.};
const rmatrix ma(a, 2, 3);

std::cout << - ma;
\end{Verbatim}
prints
\begin{Verbatim}
-1 -3 -5
-2 -4 -6
\end{Verbatim}
\newpage



\subsubsection{operator * (TR)}
Operator%
\pdfdest name {rmatrix.operator * (TR)} fit
\begin{verbatim}
rmatrix rmatrix::operator * (TR d) const;
\end{verbatim}
creates an object of type \verb"rmatrix" as a product of
a calling matrix and  number~\verb"d".
Operator is \emph{redefined} in the classes
\GOT{srmatrix}{srmatrix}, \GOT{srbmatrix}{srbmatrix}
and \GOT{srsmatrix}{srsmatrix}.
See also \GOT{rmatrix::operator *=~}{rmatrix.operator *= (TR)},
\GOT{rmatrix}{rmatrix}.
Example:
\begin{Verbatim}
using namespace cvm;

double a[] = {1., 2., 3., 4., 5., 6.};
rmatrix m(a, 2, 3);

std::cout << m * 2.;
\end{Verbatim}
prints
\begin{Verbatim}
2 6 10
4 8 12
\end{Verbatim}
\newpage



\subsubsection{operator / (TR)}
Operator%
\pdfdest name {rmatrix.operator / (TR)} fit
\begin{verbatim}
rmatrix rmatrix::operator / (TR d) const throw (cvmexception);
\end{verbatim}
creates an object of type \verb"rmatrix" as a quotient of
a calling matrix and  number~\verb"d". It throws
 \GOT{cvmexception}{cvmexception}
if \verb"d" has  absolute value equal or less than the
\GO{smallest normalized positive number}{Utilities.cvmMachMin}.
Operator is \emph{redefined} in the classes
\GOT{srmatrix}{srmatrix}, \GOT{srbmatrix}{srbmatrix}
and \GOT{srsmatrix}{srsmatrix}.
See also \GOT{rmatrix::operator /=~}{rmatrix.operator /= (TR)},
\GOT{rmatrix}{rmatrix}.
Example:
\begin{Verbatim}
using namespace cvm;
std::cout.setf (std::ios::scientific | std::ios::left);
std::cout.precision (2);

try {
    double a[] = {1., 2., 3., 4., 5., 6.};
    rmatrix m(a, 2, 3);

    std::cout << m / 2.;
}
catch (std::exception& e) {
    std::cout << "Exception " << e.what () << std::endl;
}
\end{Verbatim}
prints
\begin{Verbatim}
5.00e-01 1.50e+00 2.50e+00
1.00e+00 2.00e+00 3.00e+00
\end{Verbatim}
\newpage



\subsubsection{operator *= (TR)}
Operator%
\pdfdest name {rmatrix.operator *= (TR)} fit
\begin{verbatim}
rmatrix& rmatrix::operator *= (TR d);
\end{verbatim}
multiplies  calling matrix by  number~\verb"d"
and returns a reference to
the matrix changed.
Operator is \emph{redefined} in the classes
\GOT{srmatrix}{srmatrix}, \GOT{srbmatrix}{srbmatrix}
and \GOT{srsmatrix}{srsmatrix}.
See also \GOT{rmatrix}{rmatrix},
\GOT{rmatrix::operator *}{rmatrix.operator * (TR)}.
Example:
\begin{Verbatim}
using namespace cvm;

double a[] = {1., 2., 3., 4., 5., 6.};
rmatrix m(a, 2, 3);

m *= 2.;
std::cout << m;
\end{Verbatim}
prints
\begin{Verbatim}
2 6 10
4 8 12
\end{Verbatim}
\newpage



\subsubsection{operator /= (TR)}
Operator%
\pdfdest name {rmatrix.operator /= (TR)} fit
\begin{verbatim}
rmatrix& rmatrix::operator /= (TR d) throw (cvmexception);
\end{verbatim}
divides  calling matrix by  number~\verb"d"
and returns a reference to
the matrix changed.
It throws  \GOT{cvmexception}{cvmexception}
if \verb"d" has  absolute value equal or less
than the
\GO{smallest normalized positive number}{Utilities.cvmMachMin}.
Operator is \emph{redefined} in the classes
\GOT{srmatrix}{srmatrix}, \GOT{srbmatrix}{srbmatrix}
and \GOT{srsmatrix}{srsmatrix}.
See also \GOT{rmatrix::operator /~}{rmatrix.operator / (TR)},
\GOT{rmatrix}{rmatrix}.
Example:
\begin{Verbatim}
using namespace cvm;
std::cout.setf (std::ios::scientific | std::ios::left);
std::cout.precision (2);

try {
    double a[] = {1., 2., 3., 4., 5., 6.};
    rmatrix m(a, 2, 3);

    m /= 2.;
    std::cout << m;
}
catch (std::exception& e) {
    std::cout << "Exception " << e.what () << std::endl;
}
\end{Verbatim}
prints
\begin{Verbatim}
5.00e-01 1.50e+00 2.50e+00
1.00e+00 2.00e+00 3.00e+00
\end{Verbatim}
\newpage



\subsubsection{normalize}
Function%
\pdfdest name {rmatrix.normalize} fit
\begin{verbatim}
rmatrix& rmatrix::normalize ();
\end{verbatim}
normalizes  calling matrix so its \GO{Euclidean norm}{Array.norm}
becomes equal to $1$ if it was greater than the
\GO{smallest normalized positive number}{Utilities.cvmMachMin}
before the call (otherwise the function does nothing).
Function is \emph{redefined} in the classes
\GOT{srmatrix}{srmatrix}, \GOT{srbmatrix}{srbmatrix}
and \GOT{srsmatrix}{srsmatrix}.
See also \GOT{rmatrix}{rmatrix}.
Example:
\begin{Verbatim}
using namespace cvm;
std::cout.setf (std::ios::scientific | std::ios::left);
std::cout.precision (2);

double a[] = {1., 2., 3., 4., 5., 6.};
rmatrix m(a, 2, 3);

m.normalize();
std::cout << m;
\end{Verbatim}
prints
\begin{Verbatim}
1.05e-01 3.14e-01 5.24e-01
2.10e-01 4.19e-01 6.29e-01
\end{Verbatim}
\newpage





\subsubsection{transposition}
Operator and functions%
\pdfdest name {rmatrix.transposition} fit
\begin{verbatim}
rmatrix rmatrix::operator ~ () const throw (cvmexception);
rmatrix& rmatrix::transpose (const rmatrix& m) throw (cvmexception);
rmatrix& rmatrix::transpose () throw (cvmexception);
\end{verbatim}
implement matrix transposition.
First operator creates an object of type \verb"rmatrix" as
 transposed calling matrix
(it throws 
\GOT{cvmexception}{cvmexception}
in case of memory allocation failure).
Second function sets  calling matrix to be equal to  matrix
\verb"m" transposed
(it throws 
\GOT{cvmexception}{cvmexception}
in case of not appropriate sizes of the operands),
third one makes it to be equal to
transposed itself (it also throws 
\GOT{cvmexception}{cvmexception}
in case of memory allocation failure).
Functions are \emph{redefined} in the classes
\GOT{srmatrix}{srmatrix}, \GOT{srbmatrix}{srbmatrix}
and \GOT{srsmatrix}{srsmatrix}.
See also \GOT{rmatrix}{rmatrix}.
Example:
\begin{Verbatim}
using namespace cvm;
try {
    double a[] = {1., 2., 3., 4., 5., 6.};
    rmatrix m(a,2,3);
    rmatrix mt(3,2);
    std::cout << m << std::endl << ~m << std::endl ;
    mt.transpose(m);
    std::cout << mt << std::endl;
    mt.transpose();
    std::cout << mt;
}
catch (std::exception& e) {
    std::cout << "Exception " << e.what () << std::endl;
}
\end{Verbatim}
prints
\begin{Verbatim}
1 3 5
2 4 6

1 2
3 4
5 6

1 2
3 4
5 6

1 3 5
2 4 6
\end{Verbatim}
\newpage





\subsubsection{operator * (const rvector\&)}
Operator%
\pdfdest name {rmatrix.operator * (const rvector&)} fit
\begin{verbatim}
rvector rmatrix::operator * (const rvector& v) const
throw (cvmexception);
\end{verbatim}
creates an object of type \verb"rvector"
as a product of a calling matrix and a vector \verb"v".
It throws  \GOT{cvmexception}{cvmexception}
if the number of columns of the calling matrix
differs from size of the vector \verb"v".
Use \GOT{rvector::mult}{rvector.mult (const rmatrix&, const rvector&)}
to avoid new object creation.
Function is \emph{redefined} in the classes
\GOT{srmatrix}{srmatrix}, \GOT{srbmatrix}{srbmatrix}
and \GOT{srsmatrix}{srsmatrix}.
See also
\GOT{rmatrix}{rmatrix}, \GOT{rvector}{rvector}.
Example:
\begin{Verbatim}
using namespace cvm;

try {
    rmatrix m(2, 3);
    rvector v(3);
    m.set(1.);
    v.set(1.);

    std::cout << m * v;
}
catch (std::exception& e) {
    std::cout << "Exception " << e.what () << std::endl;
}
\end{Verbatim}
prints
\begin{Verbatim}
3 3
\end{Verbatim}
\newpage



\subsubsection{operator * (const rmatrix\&)}
Operator%
\pdfdest name {rmatrix.operator * (const rmatrix&)} fit
\begin{verbatim}
rmatrix rmatrix::operator * (const rmatrix& m) const
throw (cvmexception);
\end{verbatim}
creates an object of type \verb"rmatrix"
as a product of a calling matrix and a matrix \verb"m".
It throws  \GOT{cvmexception}{cvmexception}
if  number of columns of a calling matrix
differs from  number of rows of the matrix \verb"m".
Use \GOT{rmatrix::mult}{rmatrix.mult} to avoid new object creation.
Operator is \emph{redefined} in the classes
\GOT{srmatrix}{srmatrix}, \GOT{srbmatrix}{srbmatrix}
and \GOT{srsmatrix}{srsmatrix}.
See also
\GOT{rmatrix}{rmatrix}.
Example:
\begin{Verbatim}
using namespace cvm;

try {
    rmatrix m1(2, 3);
    rmatrix m2(3, 2);
    m1.set(1.);
    m2.set(1.);

    std::cout << m1 * m2;
}
catch (std::exception& e) {
    std::cout << "Exception " << e.what () << std::endl;
}
\end{Verbatim}
prints
\begin{Verbatim}
3 3
3 3
\end{Verbatim}
\newpage



\subsubsection{mult}
Function%
\pdfdest name {rmatrix.mult} fit
\begin{verbatim}
rmatrix& rmatrix::mult (const rmatrix& m1, const rmatrix& m2)
throw (cvmexception);
\end{verbatim}
sets  calling matrix to be equal to  product of  matrix
\verb"m1" by  matrix \verb"m2"
and returns a reference to
the matrix changed.
Function throws  \GOT{cvmexception}{cvmexception}
in case of inappropriate sizes of the operands.
Function is \emph{inherited} in the class \GOT{srmatrix}{srmatrix}
and \emph{redefined} in the classes
\GOT{srbmatrix}{srbmatrix} and
\GOT{srsmatrix}{srsmatrix}.
See also
\GOT{rmatrix}{rmatrix}.
Example:
\begin{Verbatim}
using namespace cvm;

try {
    rmatrix m1(2, 3);
    rmatrix m2(3, 2);
    rmatrix m(2, 2);
    m1.set(1.);
    m2.set(1.);

    std::cout << m.mult(m1, m2) << std::endl;
    std::cout << m1.mult(m, m1);
}
catch (std::exception& e) {
    std::cout << "Exception " << e.what () << std::endl;
}
\end{Verbatim}
prints
\begin{Verbatim}
3 3
3 3

6 6 6
6 6 6
\end{Verbatim}
\newpage



\subsubsection{rank1update}
Function%
\pdfdest name {rmatrix.rank1update} fit
\begin{verbatim}
rmatrix&
rmatrix::rank1update (const rvector& vCol, const rvector& vRow)
throw (cvmexception);
\end{verbatim}
sets  calling matrix to be equal to 
\GO{rank-1 update}{rvector.rank1update} of vectors
\verb"vCol" and \verb"vRow"
and returns a reference to
the matrix changed.
Function throws  \GOT{cvmexception}{cvmexception}
if  number of rows of a calling matrix
is not equal to \verb"vCol.size()"
or
 number of columns of a calling matrix
is not equal to \verb"vRow.size()".
Function is \emph{inherited} in  the class
\GOT{srmatrix}{srmatrix} and
\emph{not applicable} to objects of the classes
\GOT{srbmatrix}{srbmatrix} and
\GOT{srsmatrix}{srsmatrix} (i.e. exception of
type \GOT{cvmexception}{cvmexception} would be thrown
in case of using it for objects of those classes).
See also \GO{rvector::rank1update}{rvector.rank1update},
\GOT{rmatrix}{rmatrix}.
Example:
\begin{Verbatim}
using namespace cvm;

try {
    rvector vc(3), vr(2);
    rmatrix m(3, 2);
    vc(1) = 1.;
    vc(2) = 2.;
    vc(3) = 3.;
    vr(1) = 4.;
    vr(2) = 5.;

    std::cout << m.rank1update (vc, vr);
}
catch (std::exception& e) {
    std::cout << "Exception " << e.what () << std::endl;
}
\end{Verbatim}
prints
\begin{Verbatim}
4 5
8 10
12 15
\end{Verbatim}
\newpage




\subsubsection{swap\_rows}
Function%
\pdfdest name {rmatrix.swap_rows} fit
\begin{verbatim}
rmatrix& rmatrix::swap_rows (int n1, int n2) throw (cvmexception);
\end{verbatim}
swaps two rows of a calling matrix and returns a reference to
the matrix changed. \verb"n1" and \verb"n2" are
 numbers of rows to be swapped,
both are \Based).
Function throws  \GOT{cvmexception}{cvmexception}
if one of the parameters is outside of the range
\verb"[1,msize()]".
Function is \emph{redefined} in  the class
\GOT{srmatrix}{srmatrix} and
\emph{not applicable} to objects of the classes
\GOT{srbmatrix}{srbmatrix} and
\GOT{srsmatrix}{srsmatrix} (i.e. exception of
type \GOT{cvmexception}{cvmexception} would be thrown
in case of using it for objects of those classes).
See also \GOT{rmatrix}{rmatrix}.
Example:
\begin{Verbatim}
using namespace cvm;
try {
    double a[] = {1., 2., 3., 4., 5., 6.};
    rmatrix m (a, 3, 2);

    std::cout << m << std::endl;
    std::cout << m.swap_rows(2,3);
}
catch (std::exception& e) {
    std::cout << "Exception " << e.what () << std::endl;
}
\end{Verbatim}
prints
\begin{Verbatim}
1 4
2 5
3 6

1 4
3 6
2 5
\end{Verbatim}
\newpage





\subsubsection{swap\_cols}
Function%
\pdfdest name {rmatrix.swap_cols} fit
\begin{verbatim}
rmatrix& rmatrix::swap_cols (int n1, int n2) throw (cvmexception);
\end{verbatim}
swaps two columns of a calling matrix and returns a reference to
the matrix changed. \verb"n1" and \verb"n2" are
 numbers of columns to be swapped,
both are \Based).
Function throws  \GOT{cvmexception}{cvmexception}
if one of the parameters is outside of the range
\verb"[1,nsize()]".
Function is \emph{redefined} in  the class
\GOT{srmatrix}{srmatrix} and
\emph{not applicable} to objects of the classes
\GOT{srbmatrix}{srbmatrix} and
\GOT{srsmatrix}{srsmatrix} (i.e. exception of
type \GOT{cvmexception}{cvmexception} would be thrown
in case of using it for objects of those classes).
See also \GOT{rmatrix}{rmatrix}.
Example:
\begin{Verbatim}
using namespace cvm;
try {
    double a[] = {1., 2., 3., 4., 5., 6.};
    rmatrix m (a, 2, 3);

    std::cout << m << std::endl;
    std::cout << m.swap_cols(2,3);
}
catch (std::exception& e) {
    std::cout << "Exception " << e.what () << std::endl;
}
\end{Verbatim}
prints
\begin{Verbatim}
1 3 5
2 4 6

1 5 3
2 6 4
\end{Verbatim}
\newpage




\subsubsection{solve}
Functions%
\pdfdest name {rmatrix.solve} fit
\begin{verbatim}
rmatrix&
rmatrix::solve (const srmatrix& mA,
                const rmatrix& mB, TR& dErr) throw (cvmexception);
rmatrix&
rmatrix::solve (const srmatrix& mA,
                const rmatrix& vB) throw (cvmexception);
\end{verbatim}
set  calling matrix to be equal to  solution $X$ of 
matrix linear equation
$A*X=B$ where parameter \verb"mA" is square matrix $A$
and parameter \verb"vB" is matrix $B$.
Every function returns a reference to the matrix changed.
The first version also sets  output parameter \verb"dErr" to be equal
to the norm of computation error.
Functions throw \GOT{cvmexception}{cvmexception}
in case of inappropriate sizes
of the operands or when the matrix $A$ is close to singular.
Functions are \emph{redefined} in  the class
\GOT{srmatrix}{srmatrix} and
\emph{inherited} thereafter in the classes
\GOT{srbmatrix}{srbmatrix} and
\GOT{srsmatrix}{srsmatrix}.
See also
\GOT{rvector::solve}{rvector.solve},
\GOT{rmatrix}{rmatrix},
\GOT{srmatrix}{srmatrix}.
Example:
\begin{Verbatim}
using namespace cvm;
std::cout.setf (std::ios::scientific | std::ios::showpos);
std::cout.precision (2);
try {
    srmatrix ma(3);
    rmatrix  mb(3,4);
    rmatrix  mx(3,4);
    double dErr;
    ma.randomize(-10., 10.);
    mb.randomize(-10., 10.);

    mx.solve (ma, mb, dErr);

    std::cout << ma * mx - mb << dErr << std::endl;
}
catch (cvmexception& e) {
    std::cout << "Exception " << e.what () << std::endl;
}
\end{Verbatim}
prints
\begin{Verbatim}
-1.78e-015 +1.80e-016 -8.88e-016 +0.00e+000
+0.00e+000 -1.33e-015 +8.88e-016 +0.00e+000
+0.00e+000 +0.00e+000 -1.78e-015 +0.00e+000
+2.02e-015
\end{Verbatim}
\newpage





\subsubsection{solve\_tran}
Functions%
\pdfdest name {rmatrix.solvetran} fit
\begin{verbatim}
rmatrix&
rmatrix::solve_tran (const srmatrix& mA,
                     const rmatrix& mB, TR& dErr) throw (cvmexception);
rmatrix&
rmatrix::solve_tran (const srmatrix& mA,
                     const rmatrix& vB) throw (cvmexception);
\end{verbatim}
set calling matrix to be equal to  solution $X$ of 
matrix linear equation
$A^T*X=B$ (which is equivalent to $X^T*A=B^T$)
where parameter \verb"mA" is square matrix $A$
and parameter \verb"vB" is matrix $B$.
Every function returns a reference to the matrix changed.
The first version also sets  output parameter \verb"dErr" to be equal
to the norm of computation error.
Functions throw \GOT{cvmexception}{cvmexception}
in case of inappropriate sizes
of the operands or when the matrix $A$ is close to singular.
Functions are \emph{redefined} in  the class
\GOT{srmatrix}{srmatrix} and
\emph{inherited} thereafter in the classes
\GOT{srbmatrix}{srbmatrix} and
\GOT{srsmatrix}{srsmatrix}.
See also
\GOT{rmatrix::solve}{rmatrix.solve},
\GOT{rvector::solve\_tran}{rvector.solvetran},
\GOT{rmatrix}{rmatrix},
\GOT{srmatrix}{srmatrix}.
Example:
\begin{Verbatim}
using namespace cvm;
std::cout.setf (std::ios::scientific | std::ios::showpos); 
std::cout.precision (3);

srmatrix ma(3);
rmatrix  mb(3,4);
rmatrix  mx(3,4);
double dErr;
ma.randomize(-10., 10.);
mb.randomize(-10., 10.);

mx.solve_tran (ma, mb, dErr);

std::cout << ~ma * mx - mb << dErr << std::endl;
std::cout << ~mx * ma - ~mb;
\end{Verbatim}
prints
\begin{Verbatim}
-1.776e-015 -8.882e-016 -3.553e-015 +4.441e-016
+0.000e+000 +3.553e-015 +1.243e-014 -8.882e-016
+0.000e+000 +3.553e-015 -1.066e-014 +0.000e+000
+4.357e-015
-1.776e-015 +0.000e+000 +0.000e+000
-8.882e-016 +3.553e-015 +3.553e-015
-3.553e-015 +1.243e-014 -1.066e-014
+4.441e-016 -8.882e-016 +0.000e+000
\end{Verbatim}
\newpage






\subsubsection{solve\_lu}
Functions%
\pdfdest name {rmatrix.solvelu} fit
\begin{verbatim}
rmatrix&
rmatrix::solve_lu (const srmatrix& mA, const srmatrix& mLU,
                   const int* pPivots, const rmatrix& mB, TR& dErr)
                   throw (cvmexception);
rmatrix&
rmatrix::solve_lu (const srmatrix& mA, const srmatrix& mLU,
                   const int* pPivots, const rmatrix& mB)
                   throw (cvmexception);
\end{verbatim}
set  calling matrix to be equal to  solution $X$ of 
matrix linear equation
$A*X=B$ where parameter \verb"mA" is square matrix $A$,
parameter \verb"mLU" is \GO{$LU$ factorization}{srmatrix.low_up}
of the matrix $A$, parameter \verb"pPivots" is  array of pivot numbers
created while factorizing the matrix $A$
and  parameter \verb"mB" is the matrix $B$.
Every function returns a reference to the matrix changed.
The first version also sets output parameter \verb"dErr" to be equal
to  norm of computation error.
These functions are useful when you need to solve few linear equations
of kind $AX=B$ with the same matrix $A$ and different matrices $B$.
In such case you save on matrix $A$ factorization since it's needed to be
performed just one time.
Functions throw \GOT{cvmexception}{cvmexception}
in case of inappropriate sizes
of the operands or when the matrix $A$ is close to singular.
See also
\GOT{rvector::solve}{rvector.solve},
\GOT{rmatrix}{rmatrix},
\GOT{srmatrix}{srmatrix}.
Example:
\begin{Verbatim}
using namespace cvm;
std::cout.setf (std::ios::scientific | std::ios::left);
std::cout.precision (2);
try {
    double a[] = {1., -1., 1., 2., -2., 1., 3., -2., 1.};
    srmatrix ma(a,3);
    srmatrix mLU(3);
    rmatrix  mb1(3,2);
    rmatrix  mb2(3,2);
    rmatrix  mx1(3,2);
    rmatrix  mx2(3,2);
    iarray   nPivots(3);
    double   dErr = 0.;
    mb1.randomize(-1.,3.);
    mb2.randomize(2.,5.);

    mLU.low_up(ma, nPivots);
    std::cout << mx1.solve_lu (ma, mLU, nPivots, mb1, dErr);
    std::cout << dErr << std::endl;
    std::cout << mx2.solve_lu (ma, mLU, nPivots, mb2) << std::endl;
    std::cout << ma * mx1 - mb1 << std::endl << ma * mx2 - mb2;
}
catch (std::exception& e) {
    std::cout << "Exception " << e.what () << std::endl;
}
\end{Verbatim}
prints
\begin{Verbatim}
3.85e+00 5.90e-01
-4.23e+00 -3.67e+00
2.10e+00 2.55e+00
7.04e-15
9.49e+00 8.93e+00
-1.00e+01 -1.42e+01
4.21e+00 7.55e+00

0.00e+00 0.00e+00
0.00e+00 0.00e+00
4.44e-16 -1.11e-16

4.44e-16 0.00e+00
-4.44e-16 0.00e+00
8.88e-16 0.00e+00
\end{Verbatim}
\newpage




\subsubsection{svd}
Functions%
\pdfdest name {rmatrix.svd} fit
\begin{verbatim}
rvector
rmatrix::svd () const throw (cvmexception);
rvector
rmatrix::svd (srmatrix& mU, srmatrix& mVH) const throw (cvmexception);
\end{verbatim}
create  object of type \verb"rvector"
as a vector of
\GO{singular values}{rvector.svd} of a calling matrix
The second version 
sets output parameter \verb"mU" to be equal
to a matrix $U$ of size $m\times m$ and
\verb"mVH" to be equal
to a matrix $V^H$ of size $n\times n$.
Functions throw
 \GOT{cvmexception}{cvmexception}
in case of inappropriate sizes of the operands
or in case of convergence error.
Use \GOT{rvector::svd}{rvector.svd} in order to avoid new vector creation.
Function is \emph{redefined} in the classes \GOT{srmatrix}{srmatrix}, \GOT{srbmatrix}{srbmatrix},
\GOT{srsmatrix}{srsmatrix}.
See also
\GOT{rvector}{rvector},
\GOT{rmatrix}{rmatrix}.
Example:
\begin{Verbatim}
using namespace cvm;
std::cout.setf (std::ios::scientific | std::ios::left);
std::cout.precision (2);
try {
    double m[] = {1., -1., 1., 2., -2., 1.,
                  3., -2., 1., 0., -2., 1.};
    rmatrix mA(m,4,3);
    rmatrix mSigma(4,3);
    rvector v;
    srmatrix mU(4), mVH(3);

    v << mA.svd(mU, mVH);
    mSigma.diag(0) = v;

    std::cout << mU << std::endl;
    std::cout << mVH << std::endl;
    std::cout << mSigma << std::endl;

    std::cout << (mA * ~mVH - mU * mSigma).norm() << std::endl;
    std::cout << (~mA * mU - ~(mSigma * mVH)).norm() << std::endl;
}
catch (std::exception& e) {
    std::cout << "Exception " << e.what () << std::endl;
}
\end{Verbatim}
prints
\begin{Verbatim}
-4.84e-01 1.95e-01 1.15e-02 -8.53e-01
2.17e-01 -3.41e-01 -8.89e-01 -2.13e-01
6.62e-01 7.16e-01 -6.18e-02 -2.13e-01
-5.29e-01 5.78e-01 -4.53e-01 4.26e-01

-2.21e-01 8.54e-01 -4.72e-01
9.59e-01 1.04e-01 -2.62e-01
-1.75e-01 -5.11e-01 -8.42e-01

4.96e+00 0.00e+00 0.00e+00
0.00e+00 2.51e+00 0.00e+00
0.00e+00 0.00e+00 3.77e-01
0.00e+00 0.00e+00 0.00e+00

1.37e-15
2.48e-15
\end{Verbatim}
\newpage



\subsubsection{pinv}
Functions%
\pdfdest name {rmatrix.pinv} fit
\begin{verbatim}
rmatrix  rmatrix::pinv (TR threshold = cvmMachSp()) const 
                        throw (cvmexception);
rmatrix& rmatrix::pinv (const basic_rmatrix& mA, 
                        TR threshold = cvmMachSp()) 
                        throw (cvmexception);
\end{verbatim}
implement matrix pseudo inversion~\GO{\cite{Gantmaher},~p.~33}{biblio}
(or Moore-Penrose generalized inversion~\GO{\cite{Horn},~p.~421}{biblio}).
Strictly defined,  $n\times m$ matrix $A^{+}$ is 
\textit{pseudo inversion} of $m\times n$ matrix $A$ if the following two equations
are satisfied:
\begin{gather*}
AA^{+}A=A,\\
A^{+}=QA^H=A^HP
\end{gather*}
where $Q$ and $P$ are some matrices. To compute the pseudo inversion, we use
\GO{Singular Value Decomposition (SVD)}{rvector.svd} 
\begin{equation*}
A = U\Sigma V^H
\end{equation*}
of a matrix $A$, thus
\begin{equation*}
A^{+} = V\Sigma^{-1}U^{H},
\end{equation*}
where $\Sigma^{-1}$ is  diagonal $n\times m$ matrix having inverted
diagonal values of a matrix $\Sigma$ if they are greater than some threshold,
and zeros otherwise.

First version creates an object of type \verb"rmatrix" as
 pseudo inverted calling matrix
(it throws 
\GOT{cvmexception}{cvmexception}
in case of memory allocation failure).
Second function sets  calling matrix to be equal to  matrix
\verb"mA" pseudo inverted
(it throws 
\GOT{cvmexception}{cvmexception}
in case of not appropriate sizes of the operands).
The threshold parameter sets  the minimum distinguishable
from zero singular value to be used to compute the pseudo inversion.
All values equal or less than the threshold are treated as zeros.
Functions are \emph{inherited} in the classes
\GOT{srmatrix}{srmatrix} and \GOT{srsmatrix}{srsmatrix}
and \emph{redefined} in \GOT{srbmatrix}{srbmatrix}.
See also \GOT{rmatrix}{rmatrix}.
Example:
\begin{Verbatim}
using namespace cvm;

try {
    rmatrix mA(3,4);
    mA(1,1) =  1.; mA(1,2) = -1.; mA(1,3) =  2.; mA(1,4) =  0.;
    mA(2,1) = -1.; mA(2,2) =  2.; mA(2,3) = -3.; mA(2,4) =  1.;
    mA(3,1) =  0.; mA(3,2) =  1.; mA(3,3) = -1.; mA(3,4) =  1.;
    rmatrix mX = mA.pinv(1.e-13);            
    std::cout << mX << (mA * mX * mA - mA).norm2() << std::endl;
}
catch (std::exception& e) {
    std::cout << "Exception " << e.what () << std::endl;
}
\end{Verbatim}
prints
\begin{Verbatim}
+3.3333333e-001 -2.1510571e-016 +3.3333333e-001
+1.1111111e-001 +1.1111111e-001 +2.2222222e-001
+2.2222222e-001 -1.1111111e-001 +1.1111111e-001
+4.4444444e-001 +1.1111111e-001 +5.5555556e-001
+2.5460202e-015
\end{Verbatim}
Band matrix example:
\begin{Verbatim}
using namespace cvm;

try {
    srbmatrix mA (40, 1, 0);
    mA.diag(0).randomize(-1.,1.);
    mA.diag(-1).randomize(5.,10.);

    srmatrix mX (40);
    mX.pinv(mA);
    std::cout << (mA * mX * mA - mA).norm2() << std::endl;
}
catch (std::exception& e) {
    std::cout << "Exception " << e.what () << std::endl;
}
\end{Verbatim}
prints
\begin{Verbatim}
+8.1956952e-14
\end{Verbatim}
\newpage


\subsubsection{gels}
Functions%
\pdfdest name {rmatrix.gels} fit
\begin{verbatim}
rmatrix  rmatrix::gels (bool transpose, const rmatrix& mB,
                        rvector& vErr) const throw (cvmexception);
rmatrix& rmatrix::gels (bool transpose, 
                        const rmatrix& mA, const rmatrix& mB,
                        rvector& vErr) throw (cvmexception);
rvector  rmatrix::gels (bool transpose, const rvector& mB,
                        TR& dErr) const throw (cvmexception);
\end{verbatim}
solve overdetermined or underdetermined linear systems 
\begin{equation*}
A*x=b
\end{equation*}
for $m\times n$ matrix $A$ (or its transpose) where 
$b$ is a vector of length $k$ 
or systems 
\begin{equation*}
A*X=B
\end{equation*}
for multiple vectors $b$ stored as columns of $k\times l$ matrix $B$ where
$k=m$ in non-transposed case and $k=n$ otherwise.
The algorithm uses  QR or LQ factorization of $A$.
It is assumed that $A$ has full rank, infinity returned otherwise.
Internally  functions use \verb"?GELS" LAPACK routines. 
If $m>n$ and \verb"transpose=false" or $m<n$ and \verb"transpose=true", then 
the system is overdetermined, thus the algorithm tries to find the least squares solution $x$
of the problem
\begin{equation*}
\|A*x-b\|_2\to\min\quad\text{or}\quad\|A^H*x-b\|_2\to\min\
\end{equation*}
respectively. Output vector \verb"vErr" of length $l$ 
(or real number \verb"dErr" for single vector $b$) returns residual sum of squares.
The system is underdetermined otherwise, and the algorithm finds its minimum norm solution.
In this case \verb"vErr" (or \verb"bErr") is set to zero. In both cases the solution computed
satisfies $x=\pinv(A)*b$, but this algorithm is faster than pseudo inversion.
Matrix $A$ is passed as  argument \verb"mA" or (in second case) it's a calling object.
First and third versions return the solution as  new matrix or vector object.
Second version sets calling object to be the solution.
Functions throw
 \GOT{cvmexception}{cvmexception}
in case of inappropriate sizes of the operands.
Functions are \emph{inherited} in all real matrix classes.
See also \GOT{rmatrix}{rmatrix}, \GOT{rvector::gels}{rvector.gels},
\GOT{rmatrix::pinv}{rmatrix.pinv}.
Example:
\begin{Verbatim}
using namespace cvm;

rmatrix a(7, 5);
rmatrix b(7, 2);
rmatrix bt(5, 2);
rvector bv(5);
rvector vErr(2);
treal dErr;
a.randomize(-10., 10.);
b.randomize(-10., 10.);
bt.randomize(-10., 10.);
bv.randomize(-10., 10.);

rmatrix x = a.gels(false, b, vErr);
rvector vt = a.gels(true, bv, dErr);
rmatrix xt(7,2);
xt.gels(true, a, bt, vErr);

std::cout << (a.pinv()*b - x).norm() << std::endl;
std::cout << (~a.pinv()*bv - vt).norm() << std::endl;
std::cout << (~a.pinv()*bt - xt).norm() << std::endl;
\end{Verbatim}
prints
\begin{Verbatim}
+2.6295387e-015
+5.8515615e-015
+6.6462869e-015
\end{Verbatim}
\newpage


\subsubsection{gelsy}
Functions%
\pdfdest name {rmatrix.gelsy} fit
\begin{verbatim}
rmatrix  rmatrix::gelsy (const rmatrix& mB, int& rank,
                         TR tol = cvmMachSp()) const throw (cvmexception);
rmatrix& rmatrix::gelsy (const rmatrix& mA, const rmatrix& mB, int& rank,
                         TR tol = cvmMachSp()) throw (cvmexception);
rvector  rmatrix::gelsy (const rvector& mB, int& rank,
                         TR tol = cvmMachSp()) const throw (cvmexception);
\end{verbatim}
compute the minimum-norm solution to  linear
least squares problem 
\begin{equation*}
\|A*x-b\|_2\to\min
\end{equation*}
using  complete orthogonal factorization of $m\times n$ matrix $A$.
Here $b$ is a vector of length $m$.
Multiple vectors $b$ can be stored as columns of $m\times l$ matrix $B$.
Matrix $A$ may be rank-deficient, functions return its effective rank in \verb"rank"
output parameter using \verb"tol" tolerance.
Internally functions use \verb"?GELSY" LAPACK routines, see more details
about the algorithm in those routines' documentation.
Matrix $A$ is passed as  argument \verb"mA" or (in second case) it's a calling object.
First and third versions return the solution as  new matrix or vector object.
Second version sets calling object to be the solution.
Functions throw
 \GOT{cvmexception}{cvmexception}
in case of inappropriate sizes of the operands.
Functions are \emph{inherited} in all real matrix classes.
See also \GOT{rmatrix}{rmatrix}, \GOT{rvector::gelsy}{rvector.gelsy},
\GOT{rmatrix::gelss}{rmatrix.gelss}, \GOT{rmatrix::gelsd}{rmatrix.gelsd}.
Example:
\begin{Verbatim}
using namespace cvm;

rmatrix a(4, 5);
rmatrix b(4, 2);
rvector bv(4);
a.randomize(-10., 10.);
b.randomize(-10., 10.);
bv.randomize(-10., 10.);
tint rank;

rmatrix x = a.gelsy(b, rank);
rvector xv = a.gelsy(bv, rank);
rmatrix x2(5,2);
x2.gelsy(a, b, rank);

std::cout << (a*x - b).norm() << std::endl;
std::cout << (a*xv - bv).norm() << std::endl;
std::cout << (a*x2 - b).norm() << std::endl;
std::cout << rank << " " << a.rank() << std::endl;
\end{Verbatim}
prints
\begin{Verbatim}
+3.3539762e-014
+5.2545344e-015
+3.3539762e-014
+4 +4
\end{Verbatim}
\newpage


\subsubsection{gelss}
Functions%
\pdfdest name {rmatrix.gelss} fit
\begin{verbatim}
rmatrix  rmatrix::gelss (const rmatrix& mB, int& rank, rvector& sv,
                         TR tol = cvmMachSp()) const throw (cvmexception);
rmatrix& rmatrix::gelss (const rmatrix& mA, const rmatrix& mB, 
                         rvector& sv, int& rank, TR tol = cvmMachSp()) 
                         throw (cvmexception);
rvector  rmatrix::gelss (const rvector& mB, rvector& sv, int& rank,
                         TR tol = cvmMachSp()) const throw (cvmexception);
\end{verbatim}
compute  minimum-norm solution to  linear
least squares problem 
\begin{equation*}
\|A*x-b\|_2\to\min
\end{equation*}
using  singular value decomposition of $m\times n$ matrix $A$.
Here $b$ is a vector of length $m$.
Multiple vectors $b$ can be stored as columns of $m\times l$ matrix $B$.
Matrix $A$ may be rank-deficient,  functions return its effective rank in \verb"rank"
output parameter using \verb"tol" tolerance. These functions also compute 
singular values of $A$ in decreasing order and return them in \verb"sv" output parameter
having $\min(m,n)$ size.
Internally  functions use \verb"?GELSS" LAPACK routines, see more details
about the algorithm in those routines' documentation.
Matrix $A$ is passed as  argument \verb"mA" or (in second case) it's a calling object.
First and third versions return the solution as  new matrix or vector object.
Second version sets calling object to be the solution.
Functions throw
 \GOT{cvmexception}{cvmexception}
in case of inappropriate sizes of the operands.
Functions are \emph{inherited} in all real matrix classes.
See also \GOT{rmatrix}{rmatrix}, \GOT{rvector::gelss}{rvector.gelss},
\GOT{rmatrix::gelsy}{rmatrix.gelsy}, \GOT{rmatrix::gelsd}{rmatrix.gelsd}.
Example:
\begin{Verbatim}
using namespace cvm;

rmatrix a(4, 5);
rmatrix b(4, 2);
rvector bv(4);
rvector sv(4);
tint rank;
a.randomize(-10., 10.);
b.randomize(-10., 10.);
bv.randomize(-10., 10.);

rmatrix x = a.gelss(b, sv, rank);
rvector xv = a.gelss(bv, sv, rank);
rmatrix xt(5,2);
xt.gelss(a, b, sv, rank);

std::cout << (a*x - b).norm() << std::endl;
std::cout << (a*xv - bv).norm() << std::endl;
std::cout << (a*xt - b).norm() << std::endl;
std::cout << (sv - a.svd()).norm() << std::endl;
std::cout << rank << " " << a.rank() << std::endl;
\end{Verbatim}
prints
\begin{Verbatim}
+1.9169279e-014
+8.9260826e-015
+1.9169279e-014
+8.1402897e-015
+4 +4
\end{Verbatim}
\newpage



\subsubsection{gelsd}
Functions%
\pdfdest name {rmatrix.gelsd} fit
\begin{verbatim}
rmatrix  rmatrix::gelsd (const rmatrix& mB, int& rank, rvector& sv,
                         TR tol = cvmMachSp()) const throw (cvmexception);
rmatrix& rmatrix::gelsd (const rmatrix& mA, const rmatrix& mB, 
                         rvector& sv, int& rank, TR tol = cvmMachSp()) 
                         throw (cvmexception);
rvector  rmatrix::gelsd (const rvector& mB, rvector& sv, int& rank,
                         TR tol = cvmMachSp()) const throw (cvmexception);
\end{verbatim}
compute minimum-norm solution to  linear
least squares problem 
\begin{equation*}
\|A*x-b\|_2\to\min
\end{equation*}
using singular value decomposition of $m\times n$ matrix $A$
and  divide and conquer method.
Here $b$ is a vector of length $m$.
Multiple vectors $b$ can be stored as columns of $m\times l$ matrix $B$.
Matrix $A$ may be rank-deficient,  functions return its effective rank in \verb"rank"
output parameter using \verb"tol" tolerance. These functions also compute 
singular values of $A$ in decreasing order and return them in \verb"sv" output parameter
having $\min(m,n)$ size.
Internally  functions use \verb"?GELSD" LAPACK routines, see more details
about the algorithm in those routines' documentation.
Matrix $A$ is passed as  argument \verb"mA" or (in second case) it's a calling object.
First and third versions return the solution as  new matrix or vector object.
Second version sets calling object to be the solution.
Functions throw
 \GOT{cvmexception}{cvmexception}
in case of inappropriate sizes of the operands.
Functions are \emph{inherited} in all real matrix classes.
See also \GOT{rmatrix}{rmatrix}, \GOT{rvector::gelsd}{rvector.gelsd},
\GOT{rmatrix::gelsy}{rmatrix.gelsy}, \GOT{rmatrix::gelss}{rmatrix.gelss}.
Example:

\begin{Verbatim}
using namespace cvm;

rmatrix a(4, 5);
rmatrix b(4, 2);
rvector bv(4);
rvector sv(4);
tint rank;
a.randomize(-10., 10.);
b.randomize(-10., 10.);
bv.randomize(-10., 10.);

rmatrix x = a.gelsd(b, sv, rank);
rvector xv = a.gelsd(bv, sv, rank);
rmatrix xt(5,2);
xt.gelsd(a, b, sv, rank);

std::cout << (a*x - b).norm() << std::endl;
std::cout << (a*xv - bv).norm() << std::endl;
std::cout << (a*xt - b).norm() << std::endl;
std::cout << (sv - a.svd()).norm() << std::endl;
std::cout << rank << " " << a.rank() << std::endl;
\end{Verbatim}
prints
\begin{Verbatim}
+1.0341178e-014
+6.0443462e-015
+1.0341178e-014
+8.9260826e-015
+4 +4
\end{Verbatim}
\newpage







\subsubsection{rank}
Function%
\pdfdest name {rmatrix.rank} fit
\begin{verbatim}
int rmatrix::rank (TR eps = cvmMachSp ()) const throw (cvmexception);
\end{verbatim}
returns  rank of a calling matrix as  number of
\GO{singular values}{rvector.svd}
with \GO{normalized}{rvector.normalize}
absolute value greater than or
equal to  parameter \verb"eps" (this is the
\GO{largest relative spacing}{Utilities.cvmMachSp} by default).
Function throws \GOT{cvmexception}{cvmexception}
in case of convergence error.
Function is \emph{inherited} in the classes
\GOT{srmatrix}{srmatrix},
\GOT{srbmatrix}{srbmatrix},
\GOT{srsmatrix}{srsmatrix}.
See also
\GOT{rmatrix}{rmatrix}.
Example:
\begin{Verbatim}
using namespace cvm;
try {
    double a[] = {1., 2., 3., 4., 5., 6.,
                  7., 8., 9., 10., 11., 12.};
    rmatrix m(a,3,4);

    std::cout << m << m.rank() << std::endl;
    m(3,4) = 13.;
    std::cout << m.rank() << std::endl;
}
catch (std::exception& e) {
    std::cout << "Exception " << e.what () << std::endl;
}
\end{Verbatim}
prints
\begin{Verbatim}
1 4 7 10
2 5 8 11
3 6 9 12
2
3
\end{Verbatim}
\newpage




\subsubsection{ger}
Function%
\pdfdest name {rmatrix.ger} fit
\begin{verbatim}
rmatrix&
rmatrix::ger (TR dAlpha, const rvector& vCol, const rvector& vRow)
throw (cvmexception);
\end{verbatim}
calls one of \verb"?GER" routines of the
\URL{BLAS library}{http://www.netlib.org/blas}
performing  \GO{rank-1 update}{rvector.rank1update}
matrix-vector operation defined as
\begin{equation*}
M=\alpha\begin{pmatrix}
x_1 \\
x_2 \\
\vdots \\
x_m
\end{pmatrix}
\begin{pmatrix}
y_1 & y_2 & \cdots & y_n
\end{pmatrix} + M,
\end{equation*}
where $\alpha$ is  real number
(parameter \verb"dAlpha"),
$M$ is  calling matrix
and $x$ and $y$ are real vectors (parameters \verb"vCol"
and \verb"vRow" respectively).
Function
returns a reference to the matrix changed and throws
 \GOT{cvmexception}{cvmexception}
in case of inappropriate sizes of the operands.
Function is \emph{inherited} in the class
\GOT{srmatrix}{srmatrix} and
\emph{not applicable} to objects of the classes
\GOT{srbmatrix}{srbmatrix} and
\GOT{srsmatrix}{srsmatrix} (i.e. \GOT{cvmexception}{cvmexception} would be thrown
in case of using it for objects of those classes).
See also
\GOT{rvector}{rvector},
\GOT{rmatrix}{rmatrix}.
Example:
\begin{Verbatim}
using namespace cvm;
std::cout.setf (std::ios::scientific | std::ios::left);
std::cout.precision (4);
try {
    double alpha = 1.3;
    rmatrix m(3,4);
    rvector vc(3);
    rvector vr(4);
    m.randomize(-1., 2.); vc.randomize(-1., 3.); vr.randomize(0., 2.);

    std::cout << m + vc.rank1update (vr) * alpha << std::endl;
    std::cout << m.ger(alpha, vc, vr);
}
catch (std::exception& e) {
    std::cout << "Exception " << e.what () << std::endl;
}
\end{Verbatim}
prints
\begin{Verbatim}
-1.7127e-01 2.9410e+00 1.3449e+00 3.6055e+00
1.9057e+00 2.6726e+00 1.7134e+00 2.2154e+00
1.7217e-01 1.3508e+00 8.8949e-01 2.2551e+00

-1.7127e-01 2.9410e+00 1.3449e+00 3.6055e+00
1.9057e+00 2.6726e+00 1.7134e+00 2.2154e+00
1.7217e-01 1.3508e+00 8.8949e-01 2.2551e+00
\end{Verbatim}
\newpage



\subsubsection{gemm}
Function%
\pdfdest name {rmatrix.gemm} fit
\begin{verbatim}
rmatrix& rmatrix::gemm (const rmatrix& m1, bool bTrans1,
                        const rmatrix& m2, bool bTrans2,
                        TR dAlpha, TR dBeta) throw (cvmexception);
\end{verbatim}
calls one of \verb"?GEMM" routines of the
\URL{BLAS library}{http://www.netlib.org/blas}
performing 
matrix-matrix operation defined as
\begin{equation*}
M=\alpha\,\mathcal{T}(M_1)\cdot\mathcal{T}(M_2) + \beta M,
\end{equation*}
where $\alpha$ and $\beta$ are real numbers
(parameters \verb"dAlpha" and \verb"dBeta"),
$M$ is  calling matrix
and $M_1$ and $M_2$ are matrices (parameters \verb"m1"
and \verb"m2" respectively). Function $\mathcal{T}(M_i)$
transposes matrix $M_i$ if appropriate boolean
parameter \verb"bTrans*" is equal to \verb"true"
and does nothing otherwise.
Function
returns a reference to the matrix changed and throws
 \GOT{cvmexception}{cvmexception}
in case of inappropriate sizes of the operands.
Function is \emph{inherited} in the class
\GOT{srmatrix}{srmatrix} and
\emph{not applicable} to objects of the classes
\GOT{srbmatrix}{srbmatrix} and
\GOT{srsmatrix}{srsmatrix} (i.e. exception of
type \GOT{cvmexception}{cvmexception} would be thrown
in case of using it for objects of those classes).
See also
\GOT{rmatrix}{rmatrix}.
Example:
\begin{Verbatim}
using namespace cvm;
std::cout.setf (std::ios::scientific | std::ios::left);
std::cout.precision (4);
try {
    double alpha = 1.3;
    double beta = -0.7;
    rmatrix m1(4,3); rmatrix m2(4,3);
    rmatrix m(3,3);
    m.randomize(-1., 2.); m1.randomize(-1., 3.); m2.randomize(0., 2.);
    std::cout << ~m1 * m2 * alpha + m * beta << std::endl;
    std::cout << m.gemm(m1, true, m2, false, alpha, beta);
}
catch (std::exception& e) {
    std::cout << "Exception " << e.what () << std::endl;
}
\end{Verbatim}
prints
\begin{Verbatim}
5.0504e+00 6.8736e+00 3.1171e+00
2.3915e+00 2.2544e+00 3.9205e+00
3.4607e+00 3.5351e+00 4.8622e+00

5.0504e+00 6.8736e+00 3.1171e+00
2.3915e+00 2.2544e+00 3.9205e+00
3.4607e+00 3.5351e+00 4.8622e+00
\end{Verbatim}
\newpage


\subsubsection{symm}
Function%
\pdfdest name {rmatrix.symm} fit
\begin{verbatim}
rmatrix& rmatrix::symm (bool bLeft, const srsmatrix& ms,
                        const rmatrix& m, TR dAlpha, TR dBeta)
                        throw (cvmexception);
\end{verbatim}
calls one of \verb"?SYMM" routines of the
\URL{BLAS library}{http://www.netlib.org/blas}
performing one of
matrix-matrix operations defined as
\begin{equation*}
M=\alpha\,M_s\cdot M_1 + \beta M\quad\text{or}\quad M=\alpha\,M_1\cdot M_s + \beta M,
\end{equation*}
where $\alpha$ and $\beta$ are real numbers
(parameters \verb"dAlpha" and \verb"dBeta"),
$M$ is  calling matrix,
$M_s$ is  symmetric matrix and $M_1$
is  real matrix (parameters \verb"ms"
and \verb"m" respectively).
First operation is performed if \verb"bLeft" passed
is \verb"true" and second one otherwise.
Function
returns a reference to the matrix changed and throws
 \GOT{cvmexception}{cvmexception}
in case of inappropriate sizes of the operands.
Function is \emph{inherited} in the classes
\GOT{srmatrix}{srmatrix} and
\GOT{srsmatrix}{srsmatrix} and
\emph{not applicable} to objects of the class
\GOT{srbmatrix}{srbmatrix} (i.e. \GOT{cvmexception}{cvmexception} would be thrown
in case of using it for objects of that class).
See also \GOT{srsmatrix}{srsmatrix},
\GOT{rmatrix}{rmatrix}.
Example:
\begin{Verbatim}
using namespace cvm;
std::cout.setf (std::ios::scientific | std::ios::left);
std::cout.precision (4);
try {
    double alpha = 1.3;
    double beta = -0.7;
    rmatrix m1(3,4);
    rmatrix m2(4,3);
    srsmatrix ms(3);
    rmatrix m(3,4);
    m.randomize(-1., 2.); m1.randomize(-1., 3.); m2.randomize(0., 2.);
    ms.randomize(-3., 1.);

    std::cout << ms * m1 * alpha + m * beta << std::endl;
    std::cout << m.symm (true, ms, m1, alpha, beta) << std::endl;

    m.resize(4,3);
    std::cout << m2 * ms * alpha + m * beta << std::endl;
    std::cout << m.symm (false, ms, m2, alpha, beta);
}
catch (std::exception& e) {
    std::cout << "Exception " << e.what () << std::endl;
}
\end{Verbatim}
prints
\begin{Verbatim}
-3.3733e+00 -5.0566e+00 -6.3018e+00 -5.4907e+00
-1.8629e+00 -1.5133e+00 -1.1372e+00 -2.5557e+00
-3.5695e+00 -1.0012e+01 -1.4239e+00 -6.1786e-01

-3.3733e+00 -5.0566e+00 -6.3018e+00 -5.4907e+00
-1.8629e+00 -1.5133e+00 -1.1372e+00 -2.5557e+00
-3.5695e+00 -1.0012e+01 -1.4239e+00 -6.1786e-01

-6.4072e+00 7.0534e-01 1.5349e+00
-4.8219e+00 -6.9891e+00 -5.1766e+00
6.8503e-01 3.5828e+00 -3.2174e+00
2.3469e-01 -9.3921e-01 -2.1961e+00

-6.4072e+00 7.0534e-01 1.5349e+00
-4.8219e+00 -6.9891e+00 -5.1766e+00
6.8503e-01 3.5828e+00 -3.2174e+00
2.3469e-01 -9.3921e-01 -2.1961e+00
\end{Verbatim}
\newpage



\subsubsection{qr}
Functions%
\pdfdest name {rmatrix.qr} fit
\begin{verbatim}
void rmatrix::qr (rmatrix& mQ, srmatrix& mR) const throw (cvmexception);
void rmatrix::qr (srmatrix& mQ, rmatrix& mR) const throw (cvmexception);
\end{verbatim}
compute QR factorization as
\begin{equation*}
M=QR
\end{equation*}
where 
$M$ is  calling matrix, orthogonal matrix $Q$ 
and upper triangular (trapezoidal) matrix $R$ are \verb"mQ" and \verb"mR"
respectively. First version implements so-called "economy" algorithm
which for  given $m\times n$ matrix $M$ computes $m\times n$ matrix $Q$
and $n\times n$ matrix $R$. Second version is  "full" mode one
computing $m\times m$ matrix $Q$ and $m\times n$ matrix $R$.
Functions throw
 \GOT{cvmexception}{cvmexception}
in case if inappropriate sizes of the operands passed.
Functions are \emph{redefined} in the class
\GOT{srmatrix}{srmatrix}.
See also \GOT{rmatrix}{rmatrix},
\GOT{srmatrix}{srmatrix}.
Example:
\begin{Verbatim}
using namespace cvm;

treal a[] = {1., 2., 3., 4., 5., 6.};
const cvm::rmatrix mh(a, 2, 3);
const cvm::rmatrix mv(a, 3, 2);
cvm::srmatrix s2(2), s3(3);
cvm::rmatrix  h(2,3), v(3,2);

mh.qr(h,s3);
std::cout << (eye_real(2) - ~rmatrix(h,1,1,2,2)*rmatrix(h,1,1,2,2)).norm()
          << " " << (mh - h * s3).norm() << std::endl;
mh.qr(s2,h);
std::cout << (eye_real(2) - ~s2 * s2).norm()
          << " " << (mh - s2 * h).norm() << std::endl;
mv.qr(v,s2);
std::cout << (eye_real(2) - ~v * v).norm()
          << " " << (mv - v * s2).norm() << std::endl;
mv.qr(s3,v);
std::cout << (eye_real(3) - ~s3 * s3).norm()
          << " " << (mv - s3 * v).norm() << std::endl;
\end{Verbatim}
prints
\begin{Verbatim}
+4.6933177e-16 +2.2342807e-15
+4.6933177e-16 +2.2342807e-15
+5.1302953e-16 +1.4432899e-15
+5.2889959e-16 +1.4432899e-15
\end{Verbatim}
\newpage




\subsubsection{lq}
Functions%
\pdfdest name {rmatrix.lq} fit
\begin{verbatim}
void rmatrix::lq (srmatrix& mL, rmatrix& mQ) const throw (cvmexception);
void rmatrix::lq (rmatrix& mL, srmatrix& mQ) const throw (cvmexception);
\end{verbatim}
compute LQ factorization as
\begin{equation*}
M=LQ
\end{equation*}
where 
$M$ is  calling matrix, lower triangular (trapezoidal) matrix $L$
and orthogonal matrix $Q$  are \verb"mL" and \verb"mQ"
respectively. First version implements so-called "economy" algorithm
which for  given $m\times n$ matrix $M$ computes $m\times m$ matrix $L$
and $m\times n$ matrix $Q$. Second version is "full" mode one
computing $m\times n$ matrix $L$ and $n\times n$ matrix $Q$.
Functions throw
 \GOT{cvmexception}{cvmexception}
in case of inappropriate sizes of the operands passed.
Functions are \emph{redefined} in the class
\GOT{srmatrix}{srmatrix}.
See also \GOT{rmatrix}{rmatrix},
\GOT{srmatrix}{srmatrix}.
Example:
\begin{Verbatim}
using namespace cvm;

treal a[] = {1., 2., 3., 4., 5., 6.};
const cvm::rmatrix mh(a, 2, 3);
const cvm::rmatrix mv(a, 3, 2);
cvm::srmatrix s2(2), s3(3);
cvm::rmatrix h(2,3), v(3,2);

mh.lq(s2,h);
std::cout << (eye_real(2) - h * ~h).norm() << " " << (mh - s2 * h)
             .norm() << std::endl;
mv.lq(s3,v);
std::cout << (eye_real(2) - ~rmatrix(v,1,1,2,2) * rmatrix(v,1,1,2,2))
             .norm() << " " << (mv - s3 * v).norm() << std::endl;
mh.lq(h,s3);
std::cout << (eye_real(3) - s3 * ~s3).norm() << " " << (mh - h * s3)
             .norm() << std::endl;
mv.lq(v,s2);
std::cout << (eye_real(2) - s2 * ~s2).norm() << " " << (mv - v * s2)
             .norm() << std::endl;
\end{Verbatim}
prints
\begin{Verbatim}
+8.6355085e-016 +3.3893638e-015
+1.9229627e-016 +1.6011864e-015
+1.0030483e-015 +3.3893638e-015
+1.9229627e-016 +1.6011864e-015
\end{Verbatim}
\newpage





\subsubsection{rq}
Functions%
\pdfdest name {rmatrix.rq} fit
\begin{verbatim}
void rmatrix::rq (srmatrix& mR, rmatrix& mQ) const throw (cvmexception);
void rmatrix::rq (rmatrix& mR, srmatrix& mQ) const throw (cvmexception);
\end{verbatim}
compute RQ factorization as
\begin{equation*}
M=RQ
\end{equation*}
where 
$M$ is  calling matrix, upper triangular matrix $R$
and orthogonal matrix $Q$ are \verb"mR" and \verb"mQ"
respectively. First version implements so-called "economy" algorithm
which for  given $m\times n$ matrix $M$ computes $m\times m$ matrix $R$
and $m\times n$ matrix $Q$. Second version is  "full" mode one
computing $m\times n$ matrix $R$ and $n\times n$ matrix $Q$.
Following \URL{this definition}{http://www.netlib.org/scalapack/slug/node57.html}
the implementation assumes that $m\le n$ and throws \GOT{cvmexception}{cvmexception}
otherwise.
Functions also throw
 \GOT{cvmexception}{cvmexception}
in case of inappropriate sizes of the operands passed.
Functions are \emph{redefined} in the class
\GOT{srmatrix}{srmatrix}.
See also \GOT{rmatrix}{rmatrix},
\GOT{srmatrix}{srmatrix}.
Example:
\begin{Verbatim}
using namespace cvm;

treal a[] = {1., 2., 3., 4., 5., 6.};
const cvm::rmatrix mh(a, 2, 3);
const cvm::rmatrix mv(a, 3, 2);
cvm::srmatrix s2(2), s3(3);
cvm::rmatrix  h(2,3), v(3,2);

mh.rq(h,s3);
std::cout << (eye_real(2) - rmatrix(s3,2,1,2,3) * ~rmatrix(s3,2,1,2,3))
             .norm() << " " << (mh - h * s3).norm() << std::endl;

mh.rq(s2,h);
std::cout << (eye_real(2) - h * ~h).norm()
          << " " << (mh - s2 * h).norm() << std::endl;
\end{Verbatim}
prints
\begin{Verbatim}
+5.5511151e-016 +2.0471501e-015
+5.5511151e-016 +2.0471501e-015
\end{Verbatim}
\newpage



\subsubsection{ql}
Functions%
\pdfdest name {rmatrix.ql} fit
\begin{verbatim}
void rmatrix::ql (rmatrix& mQ, srmatrix& mL) const throw (cvmexception);
void rmatrix::ql (srmatrix& mQ, rmatrix& mL) const throw (cvmexception);
\end{verbatim}
compute QL factorization as
\begin{equation*}
M=QL
\end{equation*}
where 
$M$ is  calling matrix, orthogonal matrix $Q$ 
and lower triangular matrix $L$ are \verb"mQ" and \verb"mL"
respectively. First version implements so-called "economy" algorithm
which for  given $m\times n$ matrix $M$ computes $m\times n$ matrix $Q$
and $n\times n$ matrix $L$. Second version is  "full" mode one
computing $m\times m$ matrix $Q$ and $m\times n$ matrix $L$.
Following \URL{this definition}{http://www.netlib.org/scalapack/slug/node57.html}
the implementation assumes that $m\ge n$ and throws \GOT{cvmexception}{cvmexception}
otherwise.
Functions also throw
 \GOT{cvmexception}{cvmexception}
in case of inappropriate sizes of the operands passed.
Functions are \emph{redefined} in the class
\GOT{srmatrix}{srmatrix}.
See also \GOT{rmatrix}{rmatrix},
\GOT{srmatrix}{srmatrix}.
Example:
\begin{Verbatim}
using namespace cvm;

treal a[] = {1., 2., 3., 4., 5., 6.};
const cvm::rmatrix mv(a, 3, 2);
cvm::srmatrix s2(2), s3(3);
cvm::rmatrix v(3,2);

mv.ql(v,s2);
std::cout << (eye_real(2) - ~v * v).norm()
          << " " << (mv - v * s2).norm() << std::endl;

mv.ql(s3,v);
std::cout << (eye_real(2) - ~rmatrix(s3,1,2,3,2) * rmatrix(s3,1,2,3,2))
             .norm() << " " << (mv - s3 * v).norm() << std::endl;
\end{Verbatim}
prints
\begin{Verbatim}
+0.0000000e+000 +2.0947646e-015
+0.0000000e+000 +2.0947646e-015
\end{Verbatim}
\newpage



\subsubsection{vanish}
Function%
\pdfdest name {rmatrix.vanish} fit
\begin{verbatim}
rmatrix& rmatrix::vanish();
\end{verbatim}
sets every element of a calling matrix to be equal to zero
and returns a reference to
the matrix changed. This function is faster
than, for example,
\GOT{rmatrix::set(TR)}{rmatrix.set}
with zero operand passed.
Function is \emph{redefined} in the classes
\GOT{srmatrix}{srmatrix}, \GOT{srsmatrix}{srsmatrix},
\GOT{srbmatrix}{srbmatrix}.
See also \GOT{rmatrix}{rmatrix}.
Example:
\begin{Verbatim}
using namespace cvm;

rmatrix m(3, 4);
m.randomize(0.,1.);

std::cout << m << std::endl;
std::cout << m.vanish ();
\end{Verbatim}
prints
\begin{Verbatim}
0.856532 0.938261 0.275704 0.186834
0.651173 0.812159 0.100467 0.536912
0.0726646 0.695914 0.661824 0.554613

0 0 0 0
0 0 0 0
0 0 0 0
\end{Verbatim}
\newpage




\subsubsection{randomize}
Function%
\pdfdest name {rmatrix.randomize} fit
\begin{verbatim}
rmatrix& rmatrix::randomize (TR dFrom, TR dTo);
\end{verbatim}
fills  calling matrix with pseudo-random numbers distributed between
\verb"dFrom" and \verb"dTo".
Function
returns a reference to the matrix changed.
See also
\GOT{rmatrix}{rmatrix}.
Example:
\begin{Verbatim}
using namespace cvm;
std::cout.setf (std::ios::scientific | std::ios::left);
std::cout.precision (7);

rmatrix m(3,4);
m.randomize(-2.,3.);
std::cout << m;
\end{Verbatim}
prints
\begin{Verbatim}
9.6853542e-01 2.7761467e+00 2.3791009e+00 -3.4452345e-01
2.9029511e+00 -9.5519883e-01 -4.9131748e-01 -1.2561113e+00
1.5219886e+00 -1.4494461e+00 2.8193304e+00 4.8817408e-01
\end{Verbatim}
\newpage

