\bigskip
\noindent%
\verb"template <typename TR, typename TC>"\\
\verb"class Matrix : public Array<TR,TC> {"\\
\verb"public:"\\
\verb"    int "\GOT{msize}{Matrix.msize}\verb" () const;"\\
\verb"    int "\GOT{nsize}{Matrix.nsize}\verb" () const;"\\
\verb"    int "\GOT{ld}{Matrix.ld}\verb" () const;"\\
\verb"    int "\GOT{rowofmax}{Matrix.rowofmax}\verb" () const;"\\
\verb"    int "\GOT{rowofmin}{Matrix.rowofmin}\verb" () const;"\\
\verb"    int "\GOT{colofmax}{Matrix.colofmax}\verb" () const;"\\
\verb"    int "\GOT{colofmin}{Matrix.colofmin}\verb" () const;"\\
\verb"    virtual TR "\GOT{norm1}{Matrix.norm1}\verb" () const;"\\
\verb"    <typename TR, typename TC>"\\
\verb"    friend std::ostream& "\GOT{operator >{}> <>}{Matrix.input}\verb" (std::istream& is,"\\
\verb"                                         const Array<TR,TC>& mIn);"\\
\verb"    <typename TR, typename TC>"\\
\verb"    friend std::ostream& "\GOT{operator <{}< <>}{Matrix.output}\verb" (std::ostream& os,"\\
\verb"                                         const Array<TR,TC>& mOut);"\\
\verb"};"
\newpage


\subsubsection{msize}
Function%
\pdfdest name {Matrix.msize} fit
\begin{verbatim}
int Matrix<TR,TC>::msize () const;
\end{verbatim}
returns  number of rows of  calling matrix.
Function is \emph{inherited} in all matrix classes
of the library:
\GOT{rmatrix}{rmatrix},   \GOT{cmatrix}{cmatrix},
\GOT{srmatrix}{srmatrix}, \GOT{scmatrix}{scmatrix},
\GOT{srbmatrix}{srbmatrix}, \GOT{scbmatrix}{scbmatrix},
\GOT{srsmatrix}{srsmatrix} and \GOT{schmatrix}{schmatrix}.
See also \GOT{Matrix}{Matrix}.
Example:
\begin{Verbatim}
using namespace cvm;

rmatrix m (100, 200);
std::cout << m.msize() << std::endl;
\end{Verbatim}
prints
\begin{Verbatim}
100
\end{Verbatim}
\newpage



\subsubsection{nsize}
Function%
\pdfdest name {Matrix.nsize} fit
\begin{verbatim}
int Matrix<TR,TC>::nsize () const;
\end{verbatim}
returns  number of columns of  calling matrix.
Function is \emph{inherited} in all matrix classes
of the library:
\GOT{rmatrix}{rmatrix},   \GOT{cmatrix}{cmatrix},
\GOT{srmatrix}{srmatrix}, \GOT{scmatrix}{scmatrix},
\GOT{srbmatrix}{srbmatrix}, \GOT{scbmatrix}{scbmatrix},
\GOT{srsmatrix}{srsmatrix} and \GOT{schmatrix}{schmatrix}.
See also \GOT{Matrix}{Matrix}.
Example:
\begin{Verbatim}
using namespace cvm;

rmatrix m (100, 200);

std::cout << m.nsize() << std::endl;
\end{Verbatim}
prints
\begin{Verbatim}
200
\end{Verbatim}
\newpage



\subsubsection{ld}
Function%
\pdfdest name {Matrix.ld} fit
\begin{verbatim}
int Matrix<TR,TC>::ld () const;
\end{verbatim}
returns  leading dimension of  calling matrix.
Leading dimension is equal to  number of rows
for every matrix except submatrices. For submatrices
it's equal to  number of rows of parent matrix.
Function is \emph{inherited} in all matrix classes
of the library:
\GOT{rmatrix}{rmatrix},   \GOT{cmatrix}{cmatrix},
\GOT{srmatrix}{srmatrix}, \GOT{scmatrix}{scmatrix},
\GOT{srbmatrix}{srbmatrix}, \GOT{scbmatrix}{scbmatrix},
\GOT{srsmatrix}{srsmatrix} and \GOT{schmatrix}{schmatrix}.
See also \GOT{Matrix}{Matrix}.
Example:
\begin{Verbatim}
using namespace cvm;

rmatrix m (100, 200);
srmatrix ms (m, 30, 40, 5); // 5x5 submatrix
std::cout << ms.ld() << std::endl;
\end{Verbatim}
prints
\begin{Verbatim}
100
\end{Verbatim}
\newpage



\subsubsection{rowofmax}
Function%
\pdfdest name {Matrix.rowofmax} fit
\begin{verbatim}
int Matrix<TR,TC>::rowofmax () const;
\end{verbatim}
returns  \Based number of calling matrix row
where the element with the maximum absolute value is located.
Function is \emph{inherited}%
\footnote{Calls \GO{virtual function}{SubSectionPolymorphism} inside}
in all matrix classes
of the library:
\GOT{rmatrix}{rmatrix},   \GOT{cmatrix}{cmatrix},
\GOT{srmatrix}{srmatrix}, \GOT{scmatrix}{scmatrix},
\GOT{srbmatrix}{srbmatrix}, \GOT{scbmatrix}{scbmatrix},
\GOT{srsmatrix}{srsmatrix} and \GOT{schmatrix}{schmatrix}.
See also \GOT{Matrix}{Matrix}.
Example:
\begin{Verbatim}
using namespace cvm;

double a[] = {1., 0., 2., -3., 1., -1.};
rmatrix m (a, 2, 3);

std::cout << m << std::endl << m.rowofmax() << std::endl;
\end{Verbatim}
prints
\begin{Verbatim}
1 2 1
0 -3 -1

2
\end{Verbatim}
\newpage




\subsubsection{rowofmin}
Function%
\pdfdest name {Matrix.rowofmin} fit
\begin{verbatim}
int Matrix<TR,TC>::rowofmin () const;
\end{verbatim}
returns  \Based number of calling matrix row
where the element with the minimum absolute value is located.
Function is \emph{inherited}%
\footnote{Calls \GO{virtual function}{SubSectionPolymorphism} inside}
in all matrix classes
of the library:
\GOT{rmatrix}{rmatrix},   \GOT{cmatrix}{cmatrix},
\GOT{srmatrix}{srmatrix}, \GOT{scmatrix}{scmatrix},
\GOT{srbmatrix}{srbmatrix}, \GOT{scbmatrix}{scbmatrix},
\GOT{srsmatrix}{srsmatrix} and \GOT{schmatrix}{schmatrix}.
See also \GOT{Matrix}{Matrix}.
Example:
\begin{Verbatim}
using namespace cvm;

double a[] = {1., 0., 2., -3., 1., -1.};
rmatrix m (a, 2, 3);

std::cout << m << std::endl << m.rowofmin() << std::endl;
\end{Verbatim}
prints
\begin{Verbatim}
1 2 1
0 -3 -1

2
\end{Verbatim}
\newpage




\subsubsection{colofmax}
Function%
\pdfdest name {Matrix.colofmax} fit
\begin{verbatim}
int Matrix<TR,TC>::colofmax () const;
\end{verbatim}
returns  \Based number of calling matrix column 
where the element with the maximum absolute value is located.
Function is \emph{inherited}%
\footnote{Calls \GO{virtual function}{SubSectionPolymorphism} inside}
in all matrix classes
of the library:
\GOT{rmatrix}{rmatrix},   \GOT{cmatrix}{cmatrix},
\GOT{srmatrix}{srmatrix}, \GOT{scmatrix}{scmatrix},
\GOT{srbmatrix}{srbmatrix}, \GOT{scbmatrix}{scbmatrix},
\GOT{srsmatrix}{srsmatrix} and \GOT{schmatrix}{schmatrix}.
See also \GOT{Matrix}{Matrix}.
Example:
\begin{Verbatim}
using namespace cvm;

double a[] = {1., 0., 2., -3., 1., -1.};
rmatrix m (a, 2, 3);

std::cout << m << std::endl << m.colofmax() << std::endl;
\end{Verbatim}
prints
\begin{Verbatim}
1 2 1
0 -3 -1

2
\end{Verbatim}
\newpage



\subsubsection{colofmin}
Function%
\pdfdest name {Matrix.colofmin} fit
\begin{verbatim}
int Matrix<TR,TC>::colofmin () const;
\end{verbatim}
returns  \Based number of  calling matrix column 
where the element with the minimum absolute value is located.
Function is \emph{inherited}%
\footnote{Calls \GO{virtual function}{SubSectionPolymorphism} inside}
in all matrix classes
of the library:
\GOT{rmatrix}{rmatrix},   \GOT{cmatrix}{cmatrix},
\GOT{srmatrix}{srmatrix}, \GOT{scmatrix}{scmatrix},
\GOT{srbmatrix}{srbmatrix}, \GOT{scbmatrix}{scbmatrix},
\GOT{srsmatrix}{srsmatrix} and \GOT{schmatrix}{schmatrix}.
See also \GOT{Matrix}{Matrix}.
Example:
\begin{Verbatim}
using namespace cvm;

double a[] = {1., 0., 2., -3., 1., -1.};
rmatrix m (a, 2, 3);

std::cout << m << std::endl << m.colofmin() << std::endl;
\end{Verbatim}
prints
\begin{Verbatim}
1 2 1
0 -3 -1

1
\end{Verbatim}
\newpage




\subsubsection{norm1}
Virtual function%
\pdfdest name {Matrix.norm1} fit
\begin{verbatim}
virtual TR Matrix<TR,TC>::norm1 () const;
\end{verbatim}
returns  1-norm of  calling matrix that is defined as
\begin{equation*}
{\|A\|}_{1}=\max_{j=1,\dots,n} \sum_{i=1}^{m} |a_{ij}|,
\end{equation*}
where $A$ is $m\times n$ matrix.
Function is \emph{inherited}
in the following classes of the library:
\GOT{rmatrix}{rmatrix},   \GOT{cmatrix}{cmatrix},
\GOT{srmatrix}{srmatrix}, \GOT{scmatrix}{scmatrix},
\GOT{srsmatrix}{srsmatrix} and \GOT{schmatrix}{schmatrix}.
It's \emph{redefined} in
\GOT{srbmatrix}{srbmatrix} and \GOT{scbmatrix}{scbmatrix}.
See also \GOT{Array::norminf}{Array.norminf} and
\GOT{Matrix}{Matrix}.
Example:
\begin{Verbatim}
using namespace cvm;

double a[] = {1., 0., 2., -3., 1., 0.};
rmatrix m (a, 2, 3);

std::cout << m << std::endl << m.norm1() 
          << std::endl << m.norminf() << std::endl;
\end{Verbatim}
prints
\begin{Verbatim}
1 2 1
0 -3 0

5
4
\end{Verbatim}
\newpage





\subsubsection{operator >{}> <> (std::istream\& is, Matrix<TR,TC>\& mIn)}
Friend template operator%
\pdfdest name {Matrix.input} fit
\begin{verbatim}
template <typename TR, typename TC>
friend std::istream& operator >> <> (std::istream& is,
                                     Matrix<TR,TC>& mIn);
\end{verbatim}
fills calling Matrix (row by row) referenced by parameter \verb"mIn" with numbers from
\verb"is" stream.
See also \GOT{Array::ope\-ra\-tor >{}>\ }{Array.input},
\GOT{Matrix}{Matrix}.
Example:
\begin{Verbatim}
using namespace cvm;
std::cout.setf (std::ios::scientific | std::ios::left);
std::cout.precision (2);

try {
    std::ofstream os;
    os.open ("in.txt");
    os << 1.2 << " " << 2.3 << std::endl << 3.4;
    os.close ();

    std::ifstream is("in.txt");
    rmatrix m(3,2);
    is >> m;

    std::cout << m;
}
catch (std::exception& e) {
    std::cout << "Exception " << e.what () << std::endl;
}
\end{Verbatim}
prints
\begin{Verbatim}
1.20e+000 2.30e+000
3.40e+000 0.00e+000
0.00e+000 0.00e+000
\end{Verbatim}
\newpage





\subsubsection{operator <{}< <> (std::ostream\& os, const Matrix<TR,TC>\& mOut)}
Friend template operator%
\pdfdest name {Matrix.output} fit
\begin{verbatim}
template <typename TR, typename TC>
friend std::ostream& operator << <> (std::ostream& os,
                                     const Matrix<TR,TC>& mOut);
\end{verbatim}
writes matrix (row by row) referenced by parameter \verb"mOut" into
\verb"os" stream.
See also \GOT{Array::ope\-ra\-tor~<{}<\ }{Array.output},
\GOT{Matrix}{Matrix}.
Example:
\begin{Verbatim}
using namespace cvm;
std::cout.setf (std::ios::scientific | std::ios::left); 
std::cout.precision (2);

srmatrix m(3);
m(1,1) = 1.;
m(2,3) = 3.;

std::cout << m;
\end{Verbatim}
prints
\begin{Verbatim}
1.00e+00 0.00e+00 0.00e+00
0.00e+00 0.00e+00 3.00e+00
0.00e+00 0.00e+00 0.00e+00
\end{Verbatim}
\newpage



