\bigskip
\noindent%
\verb"template <typename TR, typename TC>"\\
\verb"class Array : public basic_array<TC> {"\\
\verb"public:"\\
\verb"    int "\GOT{incr}{Array.incr}\verb" () const;"\\
\verb"    int "\GOT{indofmax}{Array.indofmax}\verb" () const;"\\
\verb"    int "\GOT{indofmin}{Array.indofmin}\verb" () const;"\\
\verb"    virtual TR "\GOT{norm}{Array.norm}\verb" () const;"\\
\verb"    virtual TR "\GOT{norminf}{Array.norminf}\verb" () const;"\\
\verb"    virtual TR "\GOT{norm1}{Array.norm1}\verb" () const;"\\
\verb"    virtual TR "\GOT{norm2}{Array.norm2}\verb" () const;"\\
\verb" "\\
\verb"    <typename TR, typename TC>"\\
\verb"    friend std::istream& "\GOT{operator >{}> <>}{Array.input}\verb" (std::istream& is,"\\
\verb"                                         Array<TR,TC>& aIn);"\\
\verb" "\\
\verb"    <typename TR, typename TC>"\\
\verb"    friend std::ostream& "\GOT{operator <{}< <>}{Array.output}\verb" (std::ostream& os,"\\
\verb"                                         const Array<TR,TC>& aOut);"\\
\verb"};"
\newpage


\subsubsection{incr}
Function%
\pdfdest name {Array.incr} fit
\begin{Verbatim}
int Array<TR,TC>::incr () const;
\end{Verbatim}
returns  increment between elements of calling array.
This function is \emph{inherited} in all classes of the library:
\GOT{rvector}{rvector},   \GOT{cvector}{cvector},
\GOT{rmatrix}{rmatrix},   \GOT{cmatrix}{cmatrix},
\GOT{srmatrix}{srmatrix}, \GOT{scmatrix}{scmatrix},
\GOT{srbmatrix}{srbmatrix}, \GOT{scbmatrix}{scbmatrix},
\GOT{srsmatrix}{srsmatrix} and \GOT{schmatrix}{schmatrix}.
It always returns $1$ for matrices.
See also \GOT{Array}{Array}.
Example:
\begin{Verbatim}
using namespace cvm;

double a[] = {1., 2., 3., 4., 5., 6.};
rvector v1 (a, 3, 2);
rvector v2(10);

std::cout << v1 << v1.incr () << std::endl;
std::cout << v2.incr () << std::endl;
\end{Verbatim}
prints
\begin{Verbatim}
1 3 5
2
1
\end{Verbatim}
\newpage


\subsubsection{indofmax}
Function%
\pdfdest name {Array.indofmax} fit{}
\begin{Verbatim}
int Array<TR,TC>::indofmax () const;
\end{Verbatim}
returns \Based
index of calling array's element with maximum
absolute value.
Function is \emph{inherited}%
\footnote{Calls \GO{virtual function}{SubSectionPolymorphism} inside}
in all classes of the library:
\GOT{rvector}{rvector},   \GOT{cvector}{cvector},
\GOT{rmatrix}{rmatrix},   \GOT{cmatrix}{cmatrix},
\GOT{srmatrix}{srmatrix}, \GOT{scmatrix}{scmatrix},
\GOT{srbmatrix}{srbmatrix}, \GOT{scbmatrix}{scbmatrix},
\GOT{srsmatrix}{srsmatrix} and \GOT{schmatrix}{schmatrix}.
See also \GOT{Array}{Array}.
Example:
\begin{Verbatim}
using namespace cvm;

double a[] = {3., 2., -5., -4., 5., -6.};
const rvector v (a, 4);
const rmatrix m (a, 2, 3);

std::cout << v << v.indofmax () << std::endl << std::endl;
std::cout << m << m.indofmax () << std::endl;
\end{Verbatim}
prints
\begin{Verbatim}
3 2 -5 -4
3

3 -5 5
2 -4 -6
6
\end{Verbatim}
\newpage


\subsubsection{indofmin}
Function%
\pdfdest name {Array.indofmin} fit
\begin{Verbatim}
int Array<TR,TC>::indofmin () const;
\end{Verbatim}
returns  \Based
index of calling array's element with minimum
absolute value.
Function is \emph{inherited}%
\footnote{Calls \GO{virtual function}{SubSectionPolymorphism} inside}
in all classes of the library:
\GOT{rvector}{rvector},   \GOT{cvector}{cvector},
\GOT{rmatrix}{rmatrix},   \GOT{cmatrix}{cmatrix},
\GOT{srmatrix}{srmatrix}, \GOT{scmatrix}{scmatrix},
\GOT{srbmatrix}{srbmatrix}, \GOT{scbmatrix}{scbmatrix},
\GOT{srsmatrix}{srsmatrix} and \GOT{schmatrix}{schmatrix}.
See also \GOT{Array}{Array}.
Example:
\begin{Verbatim}
using namespace cvm;

double a[] = {3., 2., -5., 0., 0., -6.};
const rvector v (a, 4);
const rmatrix m (a, 2, 3);

std::cout << v << v.indofmin () << std::endl << std::endl;
std::cout << m << m.indofmin () << std::endl;
\end{Verbatim}
prints
\begin{Verbatim}
3 2 -5 0
4

3 -5 0
2 0 -6
4
\end{Verbatim}
\newpage


\subsubsection{norm}
Virtual function%
\pdfdest name {Array.norm} fit
\begin{verbatim}
virtual TR Array<TR,TC>::norm() const;
\end{verbatim}
returns Euclidean norm of calling array that for vectors is defined as
\begin{equation*}
{\|x\|}_E=\left(\sum_{i=1}^{n} |x_i|^2\right)^{1/2}
\end{equation*}
and for matrices as
\begin{equation*}
{\|A\|}_E=\left(\sum_{i=1}^{m}\sum_{j=1}^{n} |a_{ij}|^2\right)^{1/2},
\end{equation*}
where $A$ is $m\times n$ matrix.
Function is \emph{inherited}
in the following classes of the library:
\GOT{rvector}{rvector},   \GOT{cvector}{cvector},
\GOT{rmatrix}{rmatrix},   \GOT{cmatrix}{cmatrix},
\GOT{srmatrix}{srmatrix}, \GOT{scmatrix}{scmatrix},
\GOT{srsmatrix}{srsmatrix} and \GOT{schmatrix}{schmatrix}.
It's \emph{redefined} in
\GOT{srbmatrix}{srbmatrix} and \GOT{scbmatrix}{scbmatrix}.
See also \GOT{rvector::nor\-ma\-lize}{rvector.normalize},
\GOT{cvector::nor\-ma\-lize}{cvector.normalize},
\GOT{rmatrix::nor\-ma\-lize}{rmatrix.normalize},
\GOT{cmatrix::nor\-ma\-lize}{cmatrix.normalize} and
\GOT{Array}{Array}.
Example:
\begin{Verbatim}
using namespace cvm;
std::cout.setf (ios::scientific | ios::showpos);
std::cout.precision (12);

double a[] = {1., 2., 3., -4., 5., -6.};
const rvector v (a, 3);
const rmatrix m (a, 2, 3);

std::cout << v << v.norm () << std::endl << std::endl;
std::cout << m << m.norm () << std::endl;
\end{Verbatim}
prints
\begin{Verbatim}
+1.000000000000e+000 +2.000000000000e+000 +3.000000000000e+000
+3.741657386774e+000

+1.000000000000e+000 +3.000000000000e+000 +5.000000000000e+000
+2.000000000000e+000 -4.000000000000e+000 -6.000000000000e+000
+9.539392014169e+000
\end{Verbatim}
\newpage


\subsubsection{norminf}
Virtual function%
\pdfdest name {Array.norminf} fit
\begin{verbatim}
virtual TR Array<TR,TC>::norminf () const;
\end{verbatim}
returns  infinity norm of calling array that for vectors
is defined as
\begin{equation*}
{\|x\|}_\infty=\max_{i=1,\dots,n} |x_i|
\end{equation*}
and for matrices as
\begin{equation*}
{\|A\|}_\infty=\max_{i=1,\dots,m} \sum_{j=1}^{n} |a_{ij}|,
\end{equation*}
where $A$ is $m\times n$ matrix.
Function is \emph{inherited}
in
\GOT{rvector}{rvector} and   \GOT{cvector}{cvector}
classes of the library.
It's \emph{redefined} in \GOT{Matrix}{Matrix},
\GOT{srbmatrix}{srbmatrix}, \GOT{scbmatrix}{scbmatrix},
\GOT{srsmatrix}{srsmatrix} and \GOT{schmatrix}{schmatrix}.
See also \GOT{Matrix::norm1}{Matrix.norm1}, \GOT{Array}{Array}.
Example:
\begin{Verbatim}
using namespace cvm;

double a[] = {1., 2., 3., -4., 5., -6.};
const rvector v (a, 3);
const rmatrix m (a, 2, 3);

std::cout << v << v.norminf () << std::endl;
std::cout << m << m.norminf () << std::endl;
\end{Verbatim}
prints
\begin{Verbatim}
1 2 3
3

1 3 5
2 -4 -6
12
\end{Verbatim}
\newpage



\subsubsection{norm1}
Virtual function%
\pdfdest name {Array.norm1} fit
\begin{verbatim}
virtual TR Array<TR,TC>::norm1 () const;
\end{verbatim}
returns 1-norm of calling array that for vectors
is defined as
\begin{equation*}
{\|x\|}_1=\sum_{i=1}^{n} |x_i|
\end{equation*}
and for matrices as
\begin{equation*}
{\|A\|}_{1}=\max_{j=1,\dots,n} \sum_{i=1}^{m} |a_{ij}|,
\end{equation*}
where $x$ is  vector of size $n$ and $A$ is  $m\times n$ matrix.
Function is \emph{inherited} in
\GOT{rvector}{rvector} and \GOT{cvector}{cvector} classes.
It's \emph{redefined} in
\GOT{rmatrix}{rmatrix} and \GOT{cmatrix}{cmatrix}
and inherited thereafter.
See also \GOT{Array}{Array}.
Example:
\begin{Verbatim}
using namespace cvm;
std::cout.setf (std::ios::scientific | std::ios::showpos);
std::cout.precision (12);

double a[] = {1., 2., 3., -4., 5., -6.};
const rvector v (a, 3);
const rmatrix m (a, 2, 3);

std::cout << v << v.norm1 () << std::endl << std::endl;
std::cout << m << m.norm1 () << std::endl;
\end{Verbatim}
prints
\begin{Verbatim}
+1.000000000000e+000 +2.000000000000e+000 +3.000000000000e+000
+6.000000000000e+000

+1.000000000000e+000 +3.000000000000e+000 +5.000000000000e+000
+2.000000000000e+000 -4.000000000000e+000 -6.000000000000e+000
+1.100000000000e+001
\end{Verbatim}
\newpage




\subsubsection{norm2}
Virtual function%
\pdfdest name {Array.norm2} fit
\begin{verbatim}
virtual TR Array<TR,TC>::norm2 () const;
\end{verbatim}
returns  2-norm of calling array that for vectors
is defined as
\begin{equation*}
{\|x\|}_2={\|x\|}_E=\left(\sum_{i=1}^{n} |x_i|^2\right)^{1/2}
\end{equation*}
and for matrices as
\begin{equation*}
{\|A\|}_2=\max_i\sigma_i = \left(\max_{|x|=1}\,
(Ax\cdot Ax)\right)^{1/2},
\end{equation*}
where $\sigma_i$ is  \hbox{$i$-th} singular value
of $m\times n$ matrix $A,\ i=1,\dots,\min(m,n)$.
Function is \emph{inherited} in
\GOT{rvector}{rvector} and \GOT{cvector}{cvector} classes.
It's \emph{redefined} in
\GOT{rmatrix}{rmatrix} and \GOT{cmatrix}{cmatrix}
and inherited thereafter.
See also \GOT{Array}{Array}.
Example:
\begin{Verbatim}
using namespace cvm;
using namespace cvm;
std::cout.setf (std::ios::scientific | std::ios::showpos);
std::cout.precision (12);

double a[] = {1., 2., 3., -4., 5., -6.};
const rvector v (a, 3);
const rmatrix m (a, 2, 3);

std::cout << v << v.norm2 () << std::endl << std::endl;
std::cout << m << m.norm2 () << std::endl;
\end{Verbatim}
prints
\begin{Verbatim}
+1.000000000000e+000 +2.000000000000e+000 +3.000000000000e+000
+3.741657386774e+000

+1.000000000000e+000 +3.000000000000e+000 +5.000000000000e+000
+2.000000000000e+000 -4.000000000000e+000 -6.000000000000e+000
+9.319612060784e+000
\end{Verbatim}
\newpage


\subsubsection{operator >{}> <> (std::istream\& is, Array<TR,TC>\& aIn)}
Friend template operator%
\pdfdest name {Array.input} fit
\begin{verbatim}
template <typename TR, typename TC>
friend std::istream& operator >> <> (std::istream& is,
                                     Array<TR,TC>& aIn);
\end{verbatim}
fills calling object referenced by parameter \verb"aIn" with numbers from
\verb"is" stream.
See also \GOT{basic\_array::ope\-ra\-tor >{}>\ }{basicarray.input},
\GOT{Array}{Array}.
Example:
\begin{Verbatim}
using namespace cvm;
std::cout.setf (std::ios::scientific | std::ios::left);
std::cout.precision (2);

try {
    std::ofstream os;
    os.open ("in.txt");
    os << 1.2 << " " << 2.3 << std::endl << 3.4;
    os.close ();

    std::ifstream is("in.txt");
    rvector v(5);
    is >> v;

    std::cout << v;
}
catch (std::exception& e) {
    std::cout << "Exception " << e.what () << std::endl;
}
\end{Verbatim}
prints
\begin{Verbatim}
1.20e+000 2.30e+000 3.40e+000 0.00e+000 0.00e+000
\end{Verbatim}
\newpage


\subsubsection{operator <{}< <> (std::ostream\& os, const Array<TR,TC>\& aOut)}
Friend template operator%
\pdfdest name {Array.output} fit
\begin{verbatim}
template <typename TR, typename TC>
friend std::ostream& operator << <> (std::ostream& os,
                                     const Array<TR,TC>& aOut);
\end{verbatim}
outputs calling object referenced by \verb"aOut" into
\verb"os" stream.
Operator is \GO{redefined}{Matrix.output}
in the class
\GOT{Matrix}{Matrix}.
See also \GOT{basic\_array::ope\-ra\-tor <{}<\ }{basicarray.output},
\GOT{Array}{Array}.
Example:
\begin{Verbatim}
using namespace cvm;
std::cout.setf (std::ios::scientific | std::ios::left);
std::cout.precision (2);

cvector v(3);
v(1) = tcomplex (1., 2.);
v(2) = tcomplex (3., 4.);

std::cout << v;
\end{Verbatim}
prints
\begin{Verbatim}
(1.00e+000,2.00e+000) (3.00e+000,4.00e+000) (0.00e+000,0.00e+000)
\end{Verbatim}
\newpage

