\bigskip
\noindent
\verb"class cvmexception : public std::exception"\\
\verb"{"\\
\verb"public:"\\
\verb"    explicit "\GOT{cvmexception}{cvmexception.custom}\verb" (int nCause, ...);"\\
\verb"    int "\GOT{cause}{cvmexception.cause}\verb" () const;"\\
\verb"    virtual const char* "\GOT{what}{cvmexception.what}\verb" () const;"\\
\verb"    static int "\GOT{getNextCause}{cvmexception.custom}\verb" ();"\\
\verb"    static bool "\GOT{add}{cvmexception.custom}\verb" (int nNewCause, const char* szNewMessage);"\\
\verb"};"
\newpage



\subsubsection{cause}
Function%
\pdfdest name {cvmexception.cause} fit
\begin{verbatim}
int cvmexception::cause () const;
\end{verbatim}
returns a numeric code of an exception thrown.
Possible codes can be found in \verb"cvm.h" file.
See also \GOT{cvmexception}{cvmexception}.
Example:
\begin{Verbatim}
using namespace cvm;

try {
    rvector v(10);
    v[11] = 1.;
}
catch (cvmexception& e) {
    std::cout << "Exception " << e.cause () << ": " 
              << e.what () << std::endl;
}
\end{Verbatim}
prints
\begin{Verbatim}
Exception 2: Out of range
\end{Verbatim}
\newpage



\subsubsection{what}
Function%
\pdfdest name {cvmexception.what} fit
\begin{verbatim}
virtual const char* cvmexception::what () const noexcept;
\end{verbatim}
returns a string describing an exception happened.
This function overrides \Code{std::ex\-cep\-ti\-on::what()}.
This allows you to catch just one type of exception
in your application.
See also \GOT{cvmexception}{cvmexception}.
Example:
\begin{Verbatim}
using namespace cvm;

try {
    double a[] = {1., 2., 1., 2.};
    const srsmatrix m(a, 2);
    std::cout << m;
}
catch (std::exception& e) {
    std::cout << "Exception: " << e.what () << std::endl;
}
\end{Verbatim}
prints
\begin{Verbatim}
Exception: The matrix passed doesn't appear to be symmetric
\end{Verbatim}
\newpage



\subsubsection{Customization}
Constructor and functions%
\pdfdest name {cvmexception.custom} fit
\begin{verbatim}
explicit cvmexception (int nCause, ...);
static bool cvmexception::add (int nNewCause, const char* szNewMessage);
static int cvmexception::getNextCause ();
\end{verbatim}
allow to add and use customized exception codes and messages.
See also \GOT{cvmex\-cep\-tion}{cvmexception}.
Example:
\begin{Verbatim}
using namespace cvm;

const int nNextCause = cvmexception::getNextCause();
cvmexception::add (nNextCause, 
                   "My first exception with %d parameter");
cvmexception::add (nNextCause + 1, 
                   "My second exception with %s parameter");

try {
    throw cvmexception (nNextCause, 1234);
}
catch (std::exception& e) {
    std::cout << "Exception: " << e.what () << std::endl;
}

try {
    throw cvmexception (nNextCause + 1, "Hi!");
}
catch (std::exception& e) {
    std::cout << "Exception: " << e.what () << std::endl;
}
\end{Verbatim}
prints
\begin{Verbatim}
Exception: My first exception with 1234 parameter
Exception: My second exception with Hi! parameter
\end{Verbatim}
\newpage


