\bigskip
\noindent
\verb"template <typename TR>"\\
\verb"class srbmatrix : public srmatrix <TR>, public BandMatrix <TR,TR> {"\\
\verb"public:"\\
\verb"    "\GOT{srbmatrix}{srbmatrix.srbmatrix ()}\verb" ();"\\
\verb"    explicit "\GOT{srbmatrix}{srbmatrix.srbmatrix (int)}\verb" (int nMN);"\\
\verb"    "\GOT{srbmatrix}{srbmatrix.srbmatrix (int,int,int)}\verb" (int nMN);"\\
\verb"    "\GOT{srbmatrix}{srbmatrix.srbmatrix (TR*,int,int,int)}\verb" (TR* pD, int nMN, int nKL, int nKU);"\\
\verb"    "\GOT{srbmatrix}{srbmatrix.srbmatrix (const TR*,int,int,int)}\verb" (const TR* pD, int nMN, int nKL, int nKU);"\\
\verb"    "\GOT{srbmatrix}{srbmatrix.srbmatrix (const srbmatrix&)}\verb" (const srbmatrix& m);"\\
\verb"    "\GOT{srbmatrix}{srbmatrix.srbmatrix (const rmatrix&,int,int)}\verb" (const rmatrix& m, int nKL, int nKU);"\\
\verb"    explicit "\GOT{srbmatrix}{srbmatrix.srbmatrix (const rvector&)}\verb" (const rvector& v);"\\
\verb"    TR& "\GOT{operator ()}{srbmatrix.operator (,)}\verb" (int im, int in) throw (cvmexception);"\\
\verb"    TR "\GOT{operator ()}{srbmatrix.operator (,)}\verb" (int im, int in) const throw (cvmexception);"\\
\verb"    const rvector "\GOT{operator ()}{srbmatrix.operator ()}\verb" (int i) const throw (cvmexception);"\\
\verb"    const rvector "\GOT{operator []}{srbmatrix.operator []}\verb" (int i) const throw (cvmexception);"\\
\verb"    srbmatrix& "\GOT{operator =}{srbmatrix.operator = (const srbmatrix&)}\verb" (const srbmatrix& m) throw (cvmexception);"\\
\verb"    srbmatrix& "\GOT{assign}{srbmatrix.assign}\verb" (const rvector& v) throw (cvmexception);"\\
\verb"    srbmatrix& "\GOT{assign}{srbmatrix.assign}\verb" (const TR* pD);"\\
\verb"    srbmatrix& "\GOT{set}{srbmatrix.set}\verb" (TR x);"\\
\verb"    srbmatrix& "\GOT{resize}{srbmatrix.resize}\verb" (int nNewMN) throw (cvmexception);"\\
\verb"    srbmatrix& "\GOT{resize\_lu}{srbmatrix.resizelu}\verb" (int nNewKL, int nNewKU) throw (cvmexception);"\\
\verb"    bool "\GOT{operator ==}{srbmatrix.operator ==}\verb" (const srbmatrix& v) const;"\\
\verb"    bool "\GOT{operator !=}{srbmatrix.operator !=}\verb" (const srbmatrix& v) const;"\\
\verb"    srbmatrix& "\GOT{operator <{}<}{srbmatrix.operator <<}\verb" (const srbmatrix& m) throw (cvmexception);"\\
\verb"    srbmatrix "\GOT{operator +}{srbmatrix.operator +}\verb" (const srbmatrix& m) const"\\
\verb"                          throw (cvmexception);"\\
\verb"    srbmatrix "\GOT{operator -}{srbmatrix.operator -}\verb" (const srbmatrix& m) const"\\
\verb"                          throw (cvmexception);"\\
\verb"    srbmatrix& "\GOT{sum}{srbmatrix.sum}\verb" (const srbmatrix& m1,"\\
\verb"                    const srbmatrix& m2) throw (cvmexception);"\\
\verb"    srbmatrix& "\GOT{diff}{srbmatrix.diff}\verb" (const srbmatrix& m1,"\\
\verb"                     const srbmatrix& m2) throw (cvmexception);"\\
\verb"    srbmatrix& "\GOT{operator +=}{srbmatrix.operator +=}\verb" (const srbmatrix& m) throw (cvmexception);"\\
\verb"    srbmatrix& "\GOT{operator -=}{srbmatrix.operator -=}\verb" (const srbmatrix& m) throw (cvmexception);"\\
\verb"    srbmatrix "\GOT{operator -}{srbmatrix.operator - ()}\verb" () const;"\\
\verb"    srbmatrix& "\GOT{operator ++}{srbmatrix.operator ++}\verb" ();"\\
\verb"    srbmatrix& "\GOT{operator ++}{srbmatrix.operator ++}\verb" (int);"\\
\verb"    srbmatrix& "\GOT{operator -{}-}{srbmatrix.operator --}\verb" ();"\\
\verb"    srbmatrix& "\GOT{operator -{}-}{srbmatrix.operator --}\verb" (int);"\\
\verb"    srbmatrix "\GOT{operator *}{srbmatrix.operator * (TR)}\verb" (TR d) const;"\\
\verb"    srbmatrix "\GOT{operator /}{srbmatrix.operator / (TR)}\verb" (TR d) const throw (cvmexception);"\\
\verb"    srbmatrix& "\GOT{operator *=}{srbmatrix.operator *= (TR)}\verb" (TR d);"\\
\verb"    srbmatrix& "\GOT{operator /=}{srbmatrix.operator /= (TR)}\verb" (TR d) throw (cvmexception);"\\
\verb"    srbmatrix& "\GOT{normalize}{srbmatrix.normalize}\verb" ();"\\
\verb"    srbmatrix "\GOT{operator \TildaT}{srbmatrix.transposition}\verb" () const throw (cvmexception);"\\
\verb"    srbmatrix& "\GOT{transpose}{srbmatrix.transposition}\verb" (const srbmatrix& m) throw (cvmexception);"\\
\verb"    srbmatrix& "\GOT{transpose}{srbmatrix.transposition}\verb" () throw (cvmexception);"\\
\verb"    rvector "\GOT{operator *}{srbmatrix.operator * (const rvector&)}\verb" (const rvector& v) const throw (cvmexception);"\\
\verb"    rmatrix "\GOT{operator *}{srbmatrix.operator * (const rmatrix&)}\verb" (const rmatrix& m) const throw (cvmexception);"\\
\verb"    srmatrix "\GOT{operator *}{srbmatrix.operator * (const srmatrix&)}\verb" (const srmatrix& m) const throw (cvmexception);"\\
\verb"    srbmatrix "\GOT{operator *}{srbmatrix.operator * (const srbmatrix&)}\verb" (const srbmatrix& m) const throw (cvmexception);"\\
\verb"    srbmatrix& "\GOT{low\_up}{srbmatrix.low_up}\verb" (const srbmatrix& m,"\\
\verb"                       int* nPivots) throw (cvmexception);"\\
\verb"    srbmatrix "\GOT{low\_up}{srbmatrix.low_up}\verb" (int* nPivots) const throw (cvmexception);"\\
\verb"    rvector "\GOT{operator /}{srbmatrix.operator / (rvector)}\verb" (const rvector& vB) const throw (cvmexception);"\\
\verb"    srbmatrix& "\GOT{identity}{srbmatrix.identity} ();\\
\verb"    srbmatrix& "\GOT{vanish}{srbmatrix.vanish}\verb" ();"\\
\verb"    srbmatrix& "\GOT{randomize}{srbmatrix.randomize}\verb" (TR dFrom, TR dTo);"\\
\verb"};"
\newpage




\subsubsection{srbmatrix ()}
Constructor%
\pdfdest name {srbmatrix.srbmatrix ()} fit
\begin{verbatim}
srbmatrix::srbmatrix ();
\end{verbatim}
creates  empty \verb"srbmatrix" object.
See also \GOT{srbmatrix}{srbmatrix}.
Example:
\begin{Verbatim}
using namespace cvm;

srbmatrix m;
std::cout << m.msize() << " " << m.nsize() << " " << m.size();
std::cout << " " << m.lsize() << " " << m.usize() << std::endl;

m.resize (3);
m.resize_lu(1,0);
m.set(1.);
std::cout << m;
\end{Verbatim}
prints
\begin{Verbatim}
0 0 0 0 0
1 0 0
1 1 0
0 1 1
\end{Verbatim}
\newpage




\subsubsection{srbmatrix (int)}
Constructor%
\pdfdest name {srbmatrix.srbmatrix (int)} fit
\begin{verbatim}
explicit srbmatrix::srbmatrix (int nMN);
\end{verbatim}
creates $n\times n$ \verb"srbmatrix" object where $n$ is passed in
\verb"nMN" parameter. The matrix created is diagonal, i.e. $k_l=k_u=0$.
Constructor throws  \GOT{cvmexception}{cvmexception}
in case of non-positive size passed or memory allocation failure.
See also \GOT{srbmatrix}{srbmatrix}.
Example:
\begin{Verbatim}
using namespace cvm;

srbmatrix m(4);
std::cout << m.msize() << " " << m.nsize() << " " << m.size();
std::cout << " " << m.lsize() << " " << m.usize() << std::endl;

m.set(1.);
std::cout << m;
\end{Verbatim}
prints
\begin{Verbatim}
4 4 4 0 0
1 0 0 0
0 1 0 0
0 0 1 0
0 0 0 1
\end{Verbatim}
\newpage




\subsubsection{srbmatrix (int,int,int)}
Constructor%
\pdfdest name {srbmatrix.srbmatrix (int,int,int)} fit
\begin{verbatim}
srbmatrix::srbmatrix (int nMN, int nKL, int nKU);
\end{verbatim}
creates $n\times n$ \verb"srbmatrix" object where $n$ is passed in
\verb"nMN" parameter. The matrix created has \verb"nKL" 
sub-diagonals and \verb"nKU" super-diagonals.
Constructor throws  \GOT{cvmexception}{cvmexception}
in case of non-positive size or negative number 
of sub-diagonals or super-diagonals
passed or in case of memory allocation failure.
See also \GOT{srbmatrix}{srbmatrix}.
Example:
\begin{Verbatim}
using namespace cvm;

srbmatrix m(5,1,1);
m.set(1.);
std::cout << m << std::endl
          << m.msize() << " " << m.nsize() << " " << m.size()
          << " " << m.lsize() << " " << m.usize() << std::endl;
\end{Verbatim}
prints
\begin{Verbatim}
1 1 0 0 0
1 1 1 0 0
0 1 1 1 0
0 0 1 1 1
0 0 0 1 1

5 5 15 1 1
\end{Verbatim}
\newpage




\subsubsection{srbmatrix (TR*,int,int,int)}
Constructor%
\pdfdest name {srbmatrix.srbmatrix (TR*,int,int,int)} fit
\begin{verbatim}
srbmatrix::srbmatrix (TR* pD, int nMN, int nKL, int nKU);
\end{verbatim}
creates  $n\times n$ \verb"srbmatrix" object where $n$ is passed in
\verb"nMN" parameter. The matrix created has $k_l$=\verb"nKL" 
sub-diagonals and $k_u$=\verb"nKU" super-diagonals.
Unlike others, this constructor \textit{does not allocate  memory}.
It just shares  memory with an array pointed to by \verb"pD".
Please note that this array must contain at least $(k_l + k_u + 1)*n$ elements.
Constructor throws  \GOT{cvmexception}{cvmexception}
in case of non-positive size or negative number of sub- or super-diagonals 
passed.
See also \GOT{srbmatrix}{srbmatrix},
\GOT{srbmatrix (const TR*,int,int,int)}{srbmatrix.srbmatrix (const TR*,int,int,int)}.
Example:
\begin{Verbatim}
using namespace cvm;
double a[] = {1., 2., 3., 4., 5., 6., 7., 8.};
srbmatrix ml(a,4,1,0);
srbmatrix mu(a,4,0,1);
ml(2,1) = 9.;
std::cout << ml << std::endl << mu << std::endl;
std::cout << a[0] << " " << a[1] << " " << a[2] << " "
          << a[3] << " " << a[4] << " " << a[5] << " " << std::endl;
\end{Verbatim}
prints
\begin{Verbatim}
1 0 0 0
9 3 0 0
0 4 5 0
0 0 6 7

9 3 0 0
0 4 5 0
0 0 6 7
0 0 0 8

1 9 3 4 5 6
\end{Verbatim}
\newpage


\subsubsection{srbmatrix (const TR*,int,int,int)}
Constructor%
\pdfdest name {srbmatrix.srbmatrix (const TR*,int,int,int)} fit
\begin{verbatim}
srbmatrix::srbmatrix (const TR* pD, int nMN, int nKL, int nKU);
\end{verbatim}
creates  $n\times n$ \verb"srbmatrix" object where $n$ is passed in
\verb"nMN" parameter. The matrix created has $k_l$=\verb"nKL"
sub-diagonals and $k_u$=\verb"nKU" super-diagonals.
Then constructor copies $(k_l + k_u + 1)*n$ elements of an array  \verb"pD" to the 
matrix according to \GOT{band storage}{SubSubSectionStorage}.
Constructor throws  \GOT{cvmexception}{cvmexception}
in case of non-positive size or negative number of sub- or super-diagonals 
passed.
See also \GOT{srbmatrix}{srbmatrix},
\GOT{srbmatrix (TR*,int,int,int)}{srbmatrix.srbmatrix (TR*,int,int,int)}.
Example:
\begin{Verbatim}
using namespace cvm;
const double a[] = {1., 2., 3., 4., 5., 6., 7., 8.};
srbmatrix ml(a,4,1,0);
srbmatrix mu(a,4,0,1);
ml(2,1) = 9.;
std::cout << ml << std::endl << mu << std::endl;
std::cout << a[0] << " " << a[1] << " " << a[2] << " "
          << a[3] << " " << a[4] << " " << a[5] << " " << std::endl;
\end{Verbatim}
prints
\begin{Verbatim}
1 0 0 0
9 3 0 0
0 4 5 0
0 0 6 7

2 3 0 0
0 4 5 0
0 0 6 7
0 0 0 8

1 2 3 4 5 6
\end{Verbatim}
\newpage


\subsubsection{srbmatrix (const srbmatrix\&)}
Copy constructor%
\pdfdest name {srbmatrix.srbmatrix (const srbmatrix&)} fit
\begin{verbatim}
srbmatrix::srbmatrix (const srbmatrix& m);
\end{verbatim}
creates  \verb"srbmatrix" object as a copy of \verb"m".
Constructor throws  \GOT{cvmexception}{cvmexception}
in case of memory allocation failure.
See also \GOT{srbmatrix}{srbmatrix}.
Example:
\begin{Verbatim}
using namespace cvm;

double a[] = {1., 2., 3., 4., 5., 6., 7., 8.};
srbmatrix m(a,4,1,0);
srbmatrix mc(m);
m(1,1) = 7.77;
std::cout << m << std::endl << mc;
\end{Verbatim}
prints
\begin{Verbatim}
7.77 0 0 0
2 3 0 0
0 4 5 0
0 0 6 7

1 0 0 0
2 3 0 0
0 4 5 0
0 0 6 7
\end{Verbatim}
\newpage



\subsubsection{srbmatrix (const rmatrix\&,int,int)}
Constructor%
\pdfdest name {srbmatrix.srbmatrix (const rmatrix&,int,int)} fit
\begin{verbatim}
srbmatrix::srbmatrix (const rmatrix& m, int nKL, int nKU);
\end{verbatim}
creates  \verb"srbmatrix" object as a copy of ``sliced'' 
matrix \verb"m", i.e. it copies main diagonal, \verb"nKL" 
sub-diagonals and \verb"nKU" super-diagonals of a matrix \verb"m".
It's assumed that $m\times n$ matrix \verb"m" must have equal
sizes, i.e. $m = n$ is satisfied.
Constructor throws  \GOT{cvmexception}{cvmexception}
if this is not true or in case of memory allocation failure.
See also \GOT{srbmatrix}{srbmatrix}.
Example:
\begin{Verbatim}
using namespace cvm;

double a[] = {1., 2., 3., 4., 5., 6., 7., 8., 9.,
              10., 11., 12., 13., 14., 15., 16.};
srmatrix m(a,4);
srbmatrix mb(m,1,2);
std::cout << m << std::endl << mb;
\end{Verbatim}
prints
\begin{Verbatim}
1 5 9 13
2 6 10 14
3 7 11 15
4 8 12 16

1 5 9 0
2 6 10 14
0 7 11 15
0 0 12 16
\end{Verbatim}
\newpage



\subsubsection{srbmatrix (const rvector\&)}
Constructor%
\pdfdest name {srbmatrix.srbmatrix (const rvector&)} fit
\begin{verbatim}
explicit srbmatrix::srbmatrix (const rvector& v);
\end{verbatim}
creates \verb"srbmatrix" object
of size \verb"v.size()" by \verb"v.size()"
and assigns vector \verb"v" to its main diagonal.
Constructor throws  \GOT{cvmexception}{cvmexception}
in case of memory allocation failure.
See also \GOT{srbmatrix}{srbmatrix}, \GOT{rvector}{rvector}.
Example:
\begin{Verbatim}
using namespace cvm;

double a[] = {1., 2., 3., 4., 5.};
rvector v(a, 5);
srbmatrix m(v);
std::cout << m;
\end{Verbatim}
prints
\begin{Verbatim}
1 0 0 0 0
0 2 0 0 0
0 0 3 0 0
0 0 0 4 0
0 0 0 0 5
\end{Verbatim}
\newpage




\subsubsection{operator (,)}
Indexing operators%
\pdfdest name {srbmatrix.operator (,)} fit
\begin{verbatim}
TR& srbmatrix::operator () (int im, int in) throw (cvmexception);
TR srbmatrix::operator () (int im, int in) const throw (cvmexception);
\end{verbatim}
provide access to a particular element of a calling band matrix. The first version
is applicable to non-constant objects.
This version returns  \emph{l-value}
in order to make possible write access to an element.
Only elements located on main diagonal or on non-zero
sub- or super-diagonals are l-values. All other values
located outside this area are not writable.
Both operators are \Based.
Operators throw \GOT{cvmexception}{cvmexception}
if some of parameters passed
is outside of \verb"[1,msize()]" range or
in case of attempt to write to a non-writable element%
\footnote{Here I use \verb"type_proxy<T>" class originally
described in \GO{\cite{Meyers}}{biblio}, p.~217.}.
See also \GOT{srbmatrix}{srbmatrix},
\GOT{BandMatrix::lsize()}{BandMatrix.lsize} and
\GOT{BandMatrix::usize()}{BandMatrix.usize}.
Example:
\begin{Verbatim}
using namespace cvm;

std::cout.setf (std::ios::scientific | std::ios::left); 
std::cout.precision (2);
try {
    double a[] = {1., 2., 3., 4., 5., 6., 7., 8.};
    srbmatrix m (a,4,1,0);

    m(2,2) = 7.77;
    std::cout << m << std::endl;
    std::cout << m(3,2) << " " << m(1,4) << std::endl;

    m(1,3) = 7.77;
}
catch (std::exception& e) {
    std::cout << "Exception: " << e.what () << std::endl;
}
\end{Verbatim}
prints
\begin{Verbatim}
1.00e+000 0.00e+000 0.00e+000 0.00e+000
2.00e+000 7.77e+000 0.00e+000 0.00e+000
0.00e+000 4.00e+000 5.00e+000 0.00e+000
0.00e+000 0.00e+000 6.00e+000 7.00e+000

4.00e+000 0.00e+000
Exception: Attempt to change a read-only element
\end{Verbatim}
\newpage



\subsubsection{operator ()}
Indexing operator%
\pdfdest name {srbmatrix.operator ()} fit
\begin{verbatim}
const rvector srbmatrix::operator () (int i) const throw (cvmexception);
\end{verbatim}
provides access to \hbox{$i$-th} column of a calling band matrix.
Unlike \GO{srmatrix::operator~()}{srmatrix.operator ()},
this operator creates only \emph{copy} of a column 
and therefore it returns
\emph{not  l-value}.
Operator is \Based.
It throws \GOT{cvmexception}{cvmexception}
if parameter \verb"i" is outside of \verb"[1,nsize()]" range.
See also \GOT{srbmatrix}{srbmatrix}.
Example:
\begin{Verbatim}
using namespace cvm;

double a[] = {1., 2., 3., 4., 5., 6., 7., 8.};
srbmatrix m (a,4,1,0);

std::cout << m << std::endl;
std::cout << m(2);
\end{Verbatim}
prints
\begin{Verbatim}
1 0 0 0
2 3 0 0
0 4 5 0
0 0 6 7

0 3 4 0
\end{Verbatim}
\newpage



\subsubsection{operator []}
Indexing operator%
\pdfdest name {srbmatrix.operator []} fit
\begin{verbatim}
const rvector srbmatrix::operator [] (int i) const throw (cvmexception);
\end{verbatim}
provides access to  \hbox{$i$-th} row of a calling band matrix.
Unlike \GO{srmatrix::operator~[]}{srmatrix.operator []},
this operator creates only  \emph{copy} of a column and therefore 
it returns
\emph{not  l-value}.
Operator is \Based.
It throws \GOT{cvmexception}{cvmexception}
if  parameter \verb"i" is outside of \verb"[1,nsize()]" range.
See also \GOT{srbmatrix}{srbmatrix}.
Example:
\begin{Verbatim}
using namespace cvm;

double a[] = {1., 2., 3., 4., 5., 6., 7., 8.};
srbmatrix m (a,4,1,0);

std::cout << m << std::endl;
std::cout << m[2];
\end{Verbatim}
prints
\begin{Verbatim}
1 0 0 0
2 3 0 0
0 4 5 0
0 0 6 7

2 3 0 0
\end{Verbatim}
\newpage




\subsubsection{operator = (const srbmatrix\&)}
Operator%
\pdfdest name {srbmatrix.operator = (const srbmatrix&)} fit
\begin{verbatim}
srbmatrix& srbmatrix::operator = (const srbmatrix& m)
throw (cvmexception);
\end{verbatim}
sets every element of a calling band matrix to a value of
appropriate element of  band matrix \verb"m"
and returns a reference to
the matrix changed.
Operator throws  \GOT{cvmexception}{cvmexception}
in case of different matrix sizes or in case of different numbers
of sub- or super-diagonals.
See also \GOT{srbmatrix}{srbmatrix}.
Example:
\begin{Verbatim}
using namespace cvm;

try {
    double a[] = {1., 2., 3., 4., 5., 6., 7., 8.};
    const srbmatrix m1(a,4,1,0);
    srbmatrix m2(4,1,0);

    m2 = m1;
    std::cout << m2;
}
catch (std::exception& e) {
    std::cout << "Exception: " << e.what () << std::endl;
}
\end{Verbatim}
prints
\begin{Verbatim}
1 0 0 0
2 3 0 0
0 4 5 0
0 0 6 7
\end{Verbatim}
\newpage



\subsubsection{assign (const TR*)}
Function%
\pdfdest name {srbmatrix.assign} fit
\begin{verbatim}
srbmatrix& srbmatrix::assign (const rvector& v) throw (cvmexception);
srbmatrix& srbmatrix::assign (const TR* pD);
\end{verbatim}
sets every element of a calling band matrix to a value of
appropriate element of  vector~\verb'v'
or  array pointed to by~\verb"pD"
and returns a reference to
the matrix changed.
In first version it's assumed that vector passed is long
enough to fill calling matrix. Function throws \GOT{cvmexception}{cvmexception} 
otherwise.
In other words this array must contain at least $(k_l + k_u + 1)n$ elements.
See also \GOT{srbmatrix}{srbmatrix}.
Example:
\begin{Verbatim}
using namespace cvm;

const double a[] = {1., 2., 3., 4., 5., 6., 7., 8.};
srbmatrix m(4,0,1);

m.assign(a);
std::cout << m;
\end{Verbatim}
prints
\begin{Verbatim}
2 3 0 0
0 4 5 0
0 0 6 7
0 0 0 8
\end{Verbatim}
\newpage


\subsubsection{set (TR)}
Function%
\pdfdest name {srbmatrix.set} fit
\begin{verbatim}
srbmatrix& srbmatrix::set (TR x);
\end{verbatim}
sets every element of a calling band matrix to a value of
parameter~\verb"x" and returns a reference to
the matrix changed.
Use \GOT{vanish}{srbmatrix.vanish} to set every element
of a calling matrix to be equal to zero.
See also \GOT{srbmatrix}{srbmatrix}.
Example:
\begin{Verbatim}
using namespace cvm;

srbmatrix m(8,2,1);
m.set(3.);
std::cout << m;
\end{Verbatim}
prints
\begin{Verbatim}
3 3 0 0 0 0 0 0
3 3 3 0 0 0 0 0
3 3 3 3 0 0 0 0
0 3 3 3 3 0 0 0
0 0 3 3 3 3 0 0
0 0 0 3 3 3 3 0
0 0 0 0 3 3 3 3
0 0 0 0 0 3 3 3
\end{Verbatim}
\newpage



\subsubsection{resize}
Function%
\pdfdest name {srbmatrix.resize} fit
\begin{verbatim}
srbmatrix& srbmatrix::resize (int nNewMN) throw (cvmexception);
\end{verbatim}
changes  size of a calling band matrix to \verb"nNewMN" by \verb"nNewMN"
and returns a reference to
the matrix changed. In case of increasing of its size, calling matrix
is filled up with zeroes. This function doesn't change
 number of sub- ore super-diagonals. Like any band matrix 
class member function, this function doesn't change 
\GO{non-referred elements}{SubSubSectionStorage}. 
See number \verb"8" appearing after
resize in example below.
Function throws  \GOT{cvmexception}{cvmexception}
in case of non-positive size passed or memory allocation failure.
See also \GOT{srbmatrix.resize\_lu}{srbmatrix.resizelu},
\GOT{srbmatrix}{srbmatrix}.
Example:
\begin{Verbatim}
using namespace cvm;

try {
    double a[] = {1., 2., 3., 4., 5., 6., 7., 8.};
    srbmatrix m(a,4,1,0);
    std::cout << m << std::endl;
    m.resize (5);
    std::cout << m;
}
catch (std::exception& e) {
    std::cout << "Exception: " << e.what () << std::endl;
}
\end{Verbatim}
prints
\begin{Verbatim}
1 0 0 0
2 3 0 0
0 4 5 0
0 0 6 7

1 0 0 0 0
2 3 0 0 0
0 4 5 0 0
0 0 6 7 0
0 0 0 8 0
\end{Verbatim}
\newpage





\subsubsection{resize\_lu}
Function%
\pdfdest name {srbmatrix.resizelu} fit
\begin{verbatim}
srbmatrix& srbmatrix::resize_lu (int nNewKL, int nNewKU) 
throw (cvmexception);
\end{verbatim}
changes  number of sub- and super-diagonals 
of a calling band matrix to \verb"nNewKL" and \verb"nNewKU" respectively
and returns a reference to
the matrix changed. In case of increasing of the numbers, calling matrix
is filled up with zeroes. 
Function throws  \GOT{cvmexception}{cvmexception}
in case of negative number passed or memory allocation failure.
See also \GOT{srbmatrix::resize}{srbmatrix.resize},
\GOT{srbmatrix}{srbmatrix}.
Example:
\begin{Verbatim}
using namespace cvm;

try {
    double a[] = {1., 2., 3., 4., 5., 6., 7., 8.};
    srbmatrix m(a,4,1,0);
    std::cout << m << std::endl;
    m.resize_lu (0,1);
    m.diag(1).set(9.);
    std::cout << m;
}
catch (std::exception& e) {
    std::cout << "Exception: " << e.what () << std::endl;
}
\end{Verbatim}
prints
\begin{Verbatim}
1 0 0 0
2 3 0 0
0 4 5 0
0 0 6 7

1 9 0 0
0 3 9 0
0 0 5 9
0 0 0 7
\end{Verbatim}
\newpage




\subsubsection{operator ==}
Operator%
\pdfdest name {srbmatrix.operator ==} fit
\begin{verbatim}
bool srbmatrix::operator == (const srbmatrix& m) const;
\end{verbatim}
compares  calling band matrix with  band matrix \verb"m"
and returns \verb"true" if they have the same sizes, tha same
numbers of sub- and super-diagonals 
and their appropriate elements differ by not more than the
\GO{smallest normalized positive number}{Utilities.cvmMachMin}.
Returns \verb"false" otherwise.
See also \GOT{srbmatrix}{srbmatrix}.
Example:
\begin{Verbatim}
using namespace cvm;

double a[] = {1., 2., 3., 4.};
srbmatrix m1(a,2,1,0);
srbmatrix m2(2,1,0);

std::cout << m1 << std::endl;

m2(1,1) = 1.; 
m2(2,1) = 2.; m2(2,2) = 3.;

std::cout << (m1 == m2) << std::endl;
\end{Verbatim}
prints
\begin{Verbatim}
1 0
2 3

1
\end{Verbatim}
\newpage




\subsubsection{operator !=}
Operator%
\pdfdest name {srbmatrix.operator !=} fit
\begin{verbatim}
bool srbmatrix::operator != (const srbmatrix& m) const;
\end{verbatim}
compares  calling band matrix with  band matrix \verb"m"
and returns \verb"true" if they have different sizes,
different numbers of sub- or super-diagonals
or at least one of their appropriate elements
differs by more than the
\GO{smallest normalized positive number}{Utilities.cvmMachMin}.
Returns \verb"false" otherwise.
See also \GOT{srbmatrix}{srbmatrix}.
Example:
\begin{Verbatim}
using namespace cvm;

double a[] = {1., 2., 3., 4.};
srbmatrix m1(a,2,1,0);
srbmatrix m2(2,1,0);

std::cout << m1 << std::endl;

m2(1,1) = 1.; 
m2(2,1) = 2.; m2(2,2) = 3.;

std::cout << (m1 != m2) << std::endl;
\end{Verbatim}
prints
\begin{Verbatim}
1 0
2 3

0
\end{Verbatim}
\newpage




\subsubsection{operator <{}<}
Operator%
\pdfdest name {srbmatrix.operator <<} fit
\begin{verbatim}
srbmatrix& srbmatrix::operator << (const srbmatrix& m)
throw (cvmexception);
\end{verbatim}
destroys  calling band matrix, creates  new one as a copy of \verb"m"
and returns a reference to
the matrix changed.
Operator throws  \GOT{cvmexception}{cvmexception}
in case of memory allocation failure.
See also \GOT{srbmatrix}{srbmatrix}.
Example:
\begin{Verbatim}
using namespace cvm;

try {
    srbmatrix m(3,1,0);
    srmatrix mc(1);
    m(2,1) = 1.;
    m(2,2) = 2.;
    std::cout << m << std::endl << mc << std::endl;

    mc << m;
    std::cout << mc;
}
catch (std::exception& e) {
    std::cout << "Exception: " << e.what () << std::endl;
}
\end{Verbatim}
prints
\begin{Verbatim}
0 0 0
1 2 0
0 0 0

0

0 0 0
1 2 0
0 0 0
\end{Verbatim}
\newpage




\subsubsection{operator +}
Operator%
\pdfdest name {srbmatrix.operator +} fit
\begin{verbatim}
srbmatrix srbmatrix::operator + (const srbmatrix& m) const
throw (cvmexception);
\end{verbatim}
creates an object of type \verb"srbmatrix" as a sum of
a calling band matrix and  band matrix \verb"m".
It throws  \GOT{cvmexception}{cvmexception}
in case of different sizes or different numbers of sub- or super-diagonals
of the operands.
See also \GOT{srbmatrix::sum}{srbmatrix.sum}, \GOT{srbmatrix}{srbmatrix}.
Example:
\begin{Verbatim}
using namespace cvm;

std::cout.setf (std::ios::scientific | std::ios::left); 
std::cout.precision (1);
try {
    double a[] = {1., 2., 3., 4., 5., 6., 7., 8.};
    double b[] = {10., 20., 30., 40., 50., 60., 70., 80.};
    srbmatrix m1(a,4,1,0);
    srbmatrix m2(b,4,1,0);

    std::cout << m1 + m2 << std::endl << m1 + m1;
}
catch (std::exception& e) {
    std::cout << "Exception: " << e.what () << std::endl;
}
\end{Verbatim}
prints
\begin{Verbatim}
1.1e+01 0.0e+00 0.0e+00 0.0e+00
2.2e+01 3.3e+01 0.0e+00 0.0e+00
0.0e+00 4.4e+01 5.5e+01 0.0e+00
0.0e+00 0.0e+00 6.6e+01 7.7e+01

2.0e+00 0.0e+00 0.0e+00 0.0e+00
4.0e+00 6.0e+00 0.0e+00 0.0e+00
0.0e+00 8.0e+00 1.0e+01 0.0e+00
0.0e+00 0.0e+00 1.2e+01 1.4e+01
\end{Verbatim}
\newpage





\subsubsection{operator -}
Operator%
\pdfdest name {srbmatrix.operator -} fit
\begin{verbatim}
srbmatrix srbmatrix::operator - (const srbmatrix& m) const
throw (cvmexception);
\end{verbatim}
creates an object of type \verb"srbmatrix" as a difference of
a calling band matrix and  band matrix \verb"m".
It throws  \GOT{cvmexception}{cvmexception}
in case of different sizes or different numbers of sub- or super-diagonals
of the operands.
See also \GOT{srbmatrix::diff}{srbmatrix.diff}, \GOT{srbmatrix}{srbmatrix}.
Example:
\begin{Verbatim}
using namespace cvm;

std::cout.setf (std::ios::scientific | std::ios::left); 
std::cout.precision (1);
try {
    double a[] = {1., 2., 3., 4., 5., 6., 7., 8.};
    double b[] = {10., 20., 30., 40., 50., 60., 70., 80.};
    srbmatrix m1(a,4,1,0);
    srbmatrix m2(b,4,1,0);

    std::cout << m2 - m1 << std::endl << m1 - m1;
}
catch (std::exception& e) {
    std::cout << "Exception: " << e.what () << std::endl;
}
\end{Verbatim}
prints
\begin{Verbatim}
9.0e+00 0.0e+00 0.0e+00 0.0e+00
1.8e+01 2.7e+01 0.0e+00 0.0e+00
0.0e+00 3.6e+01 4.5e+01 0.0e+00
0.0e+00 0.0e+00 5.4e+01 6.3e+01

0.0e+00 0.0e+00 0.0e+00 0.0e+00
0.0e+00 0.0e+00 0.0e+00 0.0e+00
0.0e+00 0.0e+00 0.0e+00 0.0e+00
0.0e+00 0.0e+00 0.0e+00 0.0e+00
\end{Verbatim}
\newpage





\subsubsection{sum}
Function%
\pdfdest name {srbmatrix.sum} fit
\begin{verbatim}
srbmatrix& srbmatrix::sum (const srbmatrix& m1, const srbmatrix& m2)
throw (cvmexception);
\end{verbatim}
assigns the result of addition of
band matrices \verb"m1" and \verb"m2"  to a calling band matrix
and returns a reference to
the matrix changed.
It throws  \GOT{cvmexception}{cvmexception}
in case of different sizes or different numbers of sub- or super-diagonals
of the operands.
See also \GOT{srbmatrix::operator +~}{srbmatrix.operator +},
\GOT{srbmatrix}{srbmatrix}.
Example:
\begin{Verbatim}
using namespace cvm;

try {
    double a[] = {1., 2., 3., 4., 5., 6.};
    const srbmatrix m1(a,3,1,0);
    srbmatrix m2(3,1,0);
    srbmatrix m(3,1,0);
    m2.set(1.);
    std::cout << m1 << std::endl << m2 << std::endl;
    std::cout << m.sum(m1, m2) << std::endl;
    std::cout << m.sum(m, m2);
}
catch (std::exception& e) {
    std::cout << "Exception: " << e.what () << std::endl;
}
\end{Verbatim}
prints
\begin{Verbatim}
1 0 0
2 3 0
0 4 5

1 0 0
1 1 0
0 1 1

2 0 0
3 4 0
0 5 6

3 0 0
4 5 0
0 6 7
\end{Verbatim}
\newpage






\subsubsection{diff}
Function%
\pdfdest name {srbmatrix.diff} fit
\begin{verbatim}
srbmatrix& srbmatrix::diff (const srbmatrix& m1, const srbmatrix& m2)
throw (cvmexception);
\end{verbatim}
assigns the result of subtraction of
band matrices \verb"m1" and \verb"m2" to a calling band matrix
and returns a reference to
the matrix changed.
It throws  \GOT{cvmexception}{cvmexception}
in case of different sizes or different numbers of sub- or super-diagonals
of the operands.
See also \GOT{srbmatrix::operator -~}{srbmatrix.operator -},
\GOT{srbmatrix}{srbmatrix}.
Example:
\begin{Verbatim}
using namespace cvm;

try {
    double a[] = {1., 2., 3., 4., 5., 6.};
    const srbmatrix m1(a,3,1,0);
    srbmatrix m2(3,1,0);
    srbmatrix m(3,1,0);
    m2.set(1.);
    std::cout << m1 << std::endl << m2 << std::endl;
    std::cout << m.diff(m1, m2) << std::endl;
    std::cout << m.diff(m, m2);
}
catch (std::exception& e) {
    std::cout << "Exception: " << e.what () << std::endl;
}
\end{Verbatim}
prints
\begin{Verbatim}
1 0 0
2 3 0
0 4 5

1 0 0
1 1 0
0 1 1

0 0 0
1 2 0
0 3 4

-1 0 0
0 1 0
0 2 3
\end{Verbatim}
\newpage





\subsubsection{operator +=}
Operator%
\pdfdest name {srbmatrix.operator +=} fit
\begin{verbatim}
srbmatrix& srbmatrix::operator += (const srbmatrix& m) 
throw (cvmexception);
\end{verbatim}
adds  band matrix \verb"m" to a calling band matrix 
and returns a reference to
the matrix changed.
Operator throws  \GOT{cvmexception}{cvmexception}
in case of different sizes or different numbers of sub- or super-diagonals
of the operands.
See also \GOT{srbmatrix::ope\-ra\-tor +~}{srbmatrix.operator +},
\GOT{srbmatrix::sum}{srbmatrix.sum},
\GOT{srbmatrix}{srbmatrix}.
Example:
\begin{Verbatim}
using namespace cvm;

try {
    srbmatrix m1(4,0,1);
    srbmatrix m2(4,0,1);
    m1.set(1.);
    m2.set(2.);

    m1 += m2;
    std::cout << m1 << std::endl;

    // well, you can do this too, but temporary object would be created
    m2 += m2; 
    std::cout << m2;
}
catch (std::exception& e) {
    std::cout << "Exception: " << e.what () << std::endl;
}
\end{Verbatim}
prints
\begin{Verbatim}
3 3 0 0
0 3 3 0
0 0 3 3
0 0 0 3

4 4 0 0
0 4 4 0
0 0 4 4
0 0 0 4
\end{Verbatim}
\newpage




\subsubsection{operator -=}
Operator%
\pdfdest name {srbmatrix.operator -=} fit
\begin{verbatim}
srbmatrix& srbmatrix::operator -= (const srbmatrix& m) 
throw (cvmexception);
\end{verbatim}
subtracts  band matrix \verb"m" from  calling band matrix
and returns a reference to
the matrix changed.
It throws  \GOT{cvmexception}{cvmexception}
in case of different sizes or different numbers of sub- or super-diagonals
of the operands.
See also \GOT{srbmatrix::ope\-ra\-tor -~}{srbmatrix.operator -},
\GOT{srbmatrix::diff}{srbmatrix.diff},
\GOT{srbmatrix}{srbmatrix}.
Example:
\begin{Verbatim}
using namespace cvm;

try {
    srbmatrix m1(4,0,1);
    srbmatrix m2(4,0,1);
    m1.set(1.);
    m2.set(4.);

    m2 -= m1;
    std::cout << m2 << std::endl;

    // well, you can do this too, but temporary object would be created
    m2 -= m2; 
    std::cout << m2;
}
catch (std::exception& e) {
    std::cout << "Exception: " << e.what () << std::endl;
}
\end{Verbatim}
prints
\begin{Verbatim}
3 3 0 0
0 3 3 0
0 0 3 3
0 0 0 3

0 0 0 0
0 0 0 0
0 0 0 0
0 0 0 0
\end{Verbatim}
\newpage





\subsubsection{operator - ()}
Operator%
\pdfdest name {srbmatrix.operator - ()} fit
\begin{verbatim}
srbmatrix srbmatrix::operator - () const throw (cvmexception);
\end{verbatim}
creates an object of type \verb"srbmatrix" as
a calling band matrix multiplied by $-1$.
See also \GOT{srbmatrix}{srbmatrix}.
Example:
\begin{Verbatim}
using namespace cvm;

std::cout.setf (std::ios::scientific | 
                std::ios::left | 
                std::ios::showpos); 
std::cout.precision (1);
double a[] = {1., 2., 3., 4., 5., 6., 7., 8.};
srbmatrix m(a,4,1,0);

std::cout << -m;
\end{Verbatim}
prints
\begin{Verbatim}
-1.0e+00 +0.0e+00 +0.0e+00 +0.0e+00
-2.0e+00 -3.0e+00 +0.0e+00 +0.0e+00
+0.0e+00 -4.0e+00 -5.0e+00 +0.0e+00
+0.0e+00 +0.0e+00 -6.0e+00 -7.0e+00
\end{Verbatim}
\newpage




\subsubsection{operator ++}
Operator%
\pdfdest name {srbmatrix.operator ++} fit
\begin{verbatim}
srbmatrix& srbmatrix::operator ++ ();
srbmatrix& srbmatrix::operator ++ (int);
\end{verbatim}
adds identity matrix to a calling band matrix
and returns a reference to
the matrix changed.
See also \GOT{srbmatrix}{srbmatrix}.
Example:
\begin{Verbatim}
using namespace cvm;

double a[] = {1., 2., 3., 4., 5., 6., 7., 8.};
srbmatrix m (a,4,1,0);

m++;
std::cout << m << std::endl;
std::cout << ++m;
\end{Verbatim}
prints
\begin{Verbatim}
2 0 0 0
2 4 0 0
0 4 6 0
0 0 6 8

3 0 0 0
2 5 0 0
0 4 7 0
0 0 6 9
\end{Verbatim}
\newpage



\subsubsection{operator -{}-}
Operator%
\pdfdest name {srbmatrix.operator --} fit
\begin{verbatim}
srbmatrix& srbmatrix::operator -- ();
srbmatrix& srbmatrix::operator -- (int);
\end{verbatim}
subtracts identity matrix from  calling band matrix
and returns a reference to
the matrix changed.
See also \GOT{srbmatrix}{srbmatrix}.
Example:
\begin{Verbatim}
using namespace cvm;

double a[] = {1., 2., 3., 4., 5., 6., 7., 8.};
srbmatrix m (a,4,1,0);

m--;
std::cout << m << std::endl;
std::cout << --m;
\end{Verbatim}
prints
\begin{Verbatim}
0 0 0 0
2 2 0 0
0 4 4 0
0 0 6 6

-1 0 0 0
2 1 0 0
0 4 3 0
0 0 6 5
\end{Verbatim}
\newpage



\subsubsection{operator * (TR)}
Operator%
\pdfdest name {srbmatrix.operator * (TR)} fit
\begin{verbatim}
srbmatrix srbmatrix::operator * (TR d) const;
\end{verbatim}
creates an object of type \verb"srbmatrix" as a product of
a calling band matrix and  number~\verb"d".
See also \GOT{srbmatrix::ope\-ra\-tor *=~}{srbmatrix.operator *= (TR)},
\GOT{srbmatrix}{srbmatrix}.
Example:
\begin{Verbatim}
using namespace cvm;

double a[] = {1., 2., 3., 4., 5., 6., 7., 8.};
srbmatrix m (a,4,1,0);
std::cout << m * 5. << std::endl;
\end{Verbatim}
prints
\begin{Verbatim}
5 0 0 0
10 15 0 0
0 20 25 0
0 0 30 35
\end{Verbatim}
\newpage



\subsubsection{operator / (TR)}
Operator%
\pdfdest name {srbmatrix.operator / (TR)} fit
\begin{verbatim}
srbmatrix srbmatrix::operator / (TR d) const throw (cvmexception);
\end{verbatim}
creates an object of type \verb"srbmatrix" as a quotient of
a calling band matrix and  number~\verb"d". It throws
 \GOT{cvmexception}{cvmexception}
if \verb"d" has  absolute value equal or less than the
\GO{smallest normalized positive number}{Utilities.cvmMachMin}.
See also \GOT{srbmatrix::operator /=~}{srbmatrix.operator /= (TR)},
\GOT{srbmatrix}{srbmatrix}.
Example:
\begin{Verbatim}
using namespace cvm;

double a[] = {1., 2., 3., 4., 5., 6., 7., 8.};
srbmatrix m (a,4,1,0);
std::cout << m / 2. << std::endl;
\end{Verbatim}
prints
\begin{Verbatim}
0.5 0 0 0
1 1.5 0 0
0 2 2.5 0
0 0 3 3.5
\end{Verbatim}
\newpage



\subsubsection{operator *= (TR)}
Operator%
\pdfdest name {srbmatrix.operator *= (TR)} fit
\begin{verbatim}
srbmatrix& srbmatrix::operator *= (TR d);
\end{verbatim}
multiplies  calling band matrix by  number~\verb"d"
and returns a reference to
the matrix changed.
See also \GOT{srbmatrix::ope\-ra\-tor *~}{srbmatrix.operator * (TR)},
\GOT{srbmatrix}{srbmatrix}.
Example:
\begin{Verbatim}
using namespace cvm;

double a[] = {1., 2., 3., 4., 5., 6., 7., 8.};
srbmatrix m (a,4,1,0);
m *= 2.;
std::cout << m;
\end{Verbatim}
prints
\begin{Verbatim}
2 0 0 0
4 6 0 0
0 8 10 0
0 0 12 14
\end{Verbatim}
\newpage



\subsubsection{operator /= (TR)}
Operator%
\pdfdest name {srbmatrix.operator /= (TR)} fit
\begin{verbatim}
srbmatrix& srbmatrix::operator /= (TR d) throw (cvmexception);
\end{verbatim}
divides  calling matrix by  number~\verb"d"
and returns a reference to
the matrix changed.
It throws  \GOT{cvmexception}{cvmexception}
if \verb"d" has  absolute value equal or less
than the 
\GO{smallest normalized positive number}{Utilities.cvmMachMin}.
See also \GOT{srbmatrix::operator /~}{srbmatrix.operator / (TR)},
\GOT{srbmatrix}{srbmatrix}.
Example:
\begin{Verbatim}
using namespace cvm;

double a[] = {1., 2., 3., 4., 5., 6., 7., 8.};
srbmatrix m (a,4,1,0);
m /= 2.;
std::cout << m;
\end{Verbatim}
prints
\begin{Verbatim}
0.5 0 0 0
1 1.5 0 0
0 2 2.5 0
0 0 3 3.5
\end{Verbatim}
\newpage



\subsubsection{normalize}
Function%
\pdfdest name {srbmatrix.normalize} fit
\begin{verbatim}
srbmatrix& srbmatrix::normalize ();
\end{verbatim}
normalizes  calling band matrix so its \GO{Euclidean norm}{Array.norm}
becomes equal to $1$ if it was greater than the 
\GO{smallest normalized positive number}{Utilities.cvmMachMin}
before the call (otherwise function does nothing).
See also \GOT{srbmatrix}{srbmatrix}.
Example:
\begin{Verbatim}
using namespace cvm;

std::cout.setf (std::ios::scientific | std::ios::left); 
std::cout.precision (3);
double a[] = {1., 2., 3., 4., 5., 6., 7., 8.};
srbmatrix m (a,4,1,0);

m.normalize();
std::cout << m << m.norm() << std::endl;
\end{Verbatim}
prints
\begin{Verbatim}
8.452e-02 0.000e+00 0.000e+00 0.000e+00
1.690e-01 2.535e-01 0.000e+00 0.000e+00
0.000e+00 3.381e-01 4.226e-01 0.000e+00
0.000e+00 0.000e+00 5.071e-01 5.916e-01
1.000e+00
\end{Verbatim}
\newpage


\subsubsection{transposition}
Operator and functions%
\pdfdest name {srbmatrix.transposition} fit
\begin{verbatim}
srbmatrix srbmatrix::operator ~ () const throw (cvmexception);
srbmatrix& srbmatrix::transpose (const srbmatrix& m) throw (cvmexception);
srbmatrix& srbmatrix::transpose () throw (cvmexception);
\end{verbatim}
implement band matrix transposition.
First operator creates an object of type \verb"srbmatrix" as
 transposed calling band matrix
(it throws  
\GOT{cvmexception}{cvmexception}
in case of memory allocation failure). 
Second function sets calling band matrix to be 
equal to  band matrix
\verb"m" transposed
(it throws  
\GOT{cvmexception}{cvmexception}
in case of not appropriate sizes of the operands or
in case of memory allocation failure), 
third one makes it to be equal to
transposed itself. 
See also \GOT{srbmatrix}{srbmatrix}.
Example:
\begin{Verbatim}
using namespace cvm;

double a[] = {1., 2., 3., 4., 5., 6., 7., 8.};
srbmatrix m (a,4,1,0);
srbmatrix mt(4,0,1);
std::cout << ~m << std::endl ;
mt.transpose(m);
std::cout << mt << std::endl;
mt.transpose();
std::cout << mt;
\end{Verbatim}
prints
\begin{Verbatim}
1 2 0 0
0 3 4 0
0 0 5 6
0 0 0 7

1 2 0 0
0 3 4 0
0 0 5 6
0 0 0 7

1 0 0 0
2 3 0 0
0 4 5 0
0 0 6 7
\end{Verbatim}
\newpage




\subsubsection{operator * (const rvector\&)}
Operator%
\pdfdest name {srbmatrix.operator * (const rvector&)} fit
\begin{verbatim}
rvector srbmatrix::operator * (const rvector& v) const
throw (cvmexception);
\end{verbatim}
creates an object of type \verb"rvector"
as a product of a calling band matrix and a vector \verb"v".
It throws  \GOT{cvmexception}{cvmexception}
if number of columns of a calling matrix
differs from  size of a vector \verb"v".
Use \GOT{rvector::mult}{rvector.mult (const rmatrix&, const rvector&)}
in order to avoid new object creation.
See also
\GOT{srbmatrix}{srbmatrix} and \GOT{rvector}{rvector}.
Example:
\begin{Verbatim}
using namespace cvm;

try {
    srbmatrix m (4,1,0);
    rvector v(4);
    m.set(1.);
    v.set(1.);

    std::cout << m * v;
}
catch (std::exception& e) {
    std::cout << "Exception: " << e.what () << std::endl;
}
\end{Verbatim}
prints
\begin{Verbatim}
1 2 2 2
\end{Verbatim}
\newpage



\subsubsection{operator * (const rmatrix\&)}
Operator%
\pdfdest name {srbmatrix.operator * (const rmatrix&)} fit
\begin{verbatim}
rmatrix srbmatrix::operator * (const rmatrix& m) const
throw (cvmexception);
\end{verbatim}
creates an object of type \verb"rmatrix"
as a product of a calling band matrix and a matrix \verb"m".
It throws  \GOT{cvmexception}{cvmexception}
if  number of columns of a calling matrix
differs from  number of rows of a matrix \verb"m".
Use \GOT{rmatrix::mult}{rmatrix.mult} in order to avoid
 new object creation.
See also
\GOT{rmatrix}{rmatrix} and \GOT{srbmatrix}{srbmatrix}.
Example:
\begin{Verbatim}
using namespace cvm;

try {
    srbmatrix mb (4,1,0);
    rmatrix m(4,2);
    mb.set(1.);
    m.set(1.);

    std::cout << mb * m;
}
catch (std::exception& e) {
    std::cout << "Exception " << e.what () << std::endl;
}
\end{Verbatim}
prints
\begin{Verbatim}
1 1
2 2
2 2
2 2
\end{Verbatim}
\newpage




\subsubsection{operator * (const srmatrix\&)}
Operator%
\pdfdest name {srbmatrix.operator * (const srmatrix&)} fit
\begin{verbatim}
srmatrix srbmatrix::operator * (const srmatrix& m) const
throw (cvmexception);
\end{verbatim}
creates an object of type \verb"srmatrix"
as a product of a calling band matrix and a matrix \verb"m".
It throws  \GOT{cvmexception}{cvmexception}
if the operands have different sizes.
Use \GOT{rmatrix::mult}{rmatrix.mult} in order to avoid
 new object creation.
See also
\GOT{srmatrix}{srmatrix} and \GOT{srbmatrix}{srbmatrix}.
Example:
\begin{Verbatim}
using namespace cvm;

try {
    srbmatrix mb(4,1,0);
    srmatrix ms(4);
    mb.set(1.);
    ms.set(1.);

    std::cout << mb * ms;
}
catch (std::exception& e) {
    std::cout << "Exception " << e.what () << std::endl;
}
\end{Verbatim}
prints
\begin{Verbatim}
1 1 1 1
2 2 2 2
2 2 2 2
2 2 2 2
\end{Verbatim}
\newpage



\subsubsection{operator * (const srbmatrix\&)}
Operator%
\pdfdest name {srbmatrix.operator * (const srbmatrix&)} fit
\begin{verbatim}
srbmatrix srbmatrix::operator * (const srbmatrix& m) const
throw (cvmexception);
\end{verbatim}
creates an object of type \verb"srbmatrix"
as a product of a calling band matrix and band matrix \verb"m".
It throws  \GOT{cvmexception}{cvmexception}
if the operands have different sizes.
Use \GOT{rmatrix::mult}{rmatrix.mult} in order to avoid
 new object creation.
See also
\GOT{srbmatrix}{srbmatrix}.
Example:
\begin{Verbatim}
using namespace cvm;

try {
    srbmatrix m1(7,1,0);
    srbmatrix m2(7,1,1);
    m1.set(1.);
    m2.set(1.);

    std::cout << m1 * m2;
}
catch (std::exception& e) {
    std::cout << "Exception: " << e.what () << std::endl;
}
\end{Verbatim}
prints
\begin{Verbatim}
1 1 0 0 0 0 0
2 2 1 0 0 0 0
1 2 2 1 0 0 0
0 1 2 2 1 0 0
0 0 1 2 2 1 0
0 0 0 1 2 2 1
0 0 0 0 1 2 2
\end{Verbatim}
\newpage




\subsubsection{low\_up}
Functions%
\pdfdest name {srbmatrix.low_up} fit
\begin{verbatim}
srbmatrix& 
srbmatrix::low_up (const srbmatrix& m, int* nPivots) throw (cvmexception);
srbmatrix
srbmatrix::low_up (int* nPivots) const throw (cvmexception);
\end{verbatim}
compute  $LU$ factorization of a calling band matrix as
\begin{equation*}
A=PLU
\end{equation*}
where $P$ is  permutation matrix, $L$ is  lower
triangular matrix with unit diagonal
elements and $U$ is  upper triangular matrix.
All  functions store  result as  matrix $L$ without
main diagonal combined with $U$. All  functions
return pivot indices as  array of integers
(it should support at least \verb"msize()" elements)
pointed to by \verb"nPivots" so \hbox{$i$-th} row
was interchanged with \hbox{\verb"nPivots["$i$\verb"]"-th} row.
The first version sets  calling matrix to be equal to the
\verb"m"'s $LU$ factorization and the second one
creates an object of type \verb"srbmatrix" as  calling band matrix's
$LU$ factorization.
Functions throw \GOT{cvmexception}{cvmexception}
in case of inappropriate
sizes of the operands or when  matrix to be factorized is close to
singular. 
The first version also changes number of 
super-diagonals to be equal to $k_l+k_u$
in order to keep the result of factorization.
It is recommended to use \GOT{iarray}{iarray}
for pivot values.
This function is provided mostly for solving multiple
systems of linear equations using 
\GOT{srmatrix::solve\_lu}{srmatrix.solvelu} function,
See also
\GOT{srbmatrix}{srbmatrix}.
Example:
\begin{Verbatim}
using namespace cvm;

std::cout.setf (std::ios::scientific | std::ios::left); 
std::cout.precision (4);
try {
    double a[] = {1., 2., 3., 4., 5., 6., 7., 8.};
    srbmatrix ma(a,4,1,0);
    srbmatrix mLU(4,1,0);
    rmatrix  mb1(4,2); rvector vb1(4);
    rmatrix  mb2(4,2); rvector vb2(4);
    rmatrix  mx1(4,2); rvector vx1(4);
    rmatrix  mx2(4,2); rvector vx2(4);
    iarray   nPivots(4);
    double   dErr = 0.;
    mb1.randomize(-1.,3.); vb1.randomize(-2.,4.);
    mb2.randomize(-2.,5.); vb2.randomize(-3.,1.);

    mLU.low_up(ma, nPivots);
    mx1 = ma.solve_lu (mLU, nPivots, mb1, dErr);
    std::cout << mx1 << dErr << std::endl << std::endl;
    mx2 = ma.solve_lu (mLU, nPivots, mb2);
    std::cout << mx2 << std::endl;;
    std::cout << ma * mx1 - mb1 << std::endl << ma * mx2 - mb2;

    vx1 = ma.solve_lu (mLU, nPivots, vb1, dErr);
    std::cout << vx1 << dErr << std::endl;
    vx2 = ma.solve_lu (mLU, nPivots, vb2);
    std::cout << vx2 << std::endl;;
    std::cout << ma * vx1 - vb1 << std::endl << ma * vx2 - vb2;
}
catch (std::exception& e) {
    std::cout << "Exception: " << e.what () << std::endl;
}
\end{Verbatim}
prints
\begin{Verbatim}
7.6327e-02 -4.7386e-01
-2.9523e-01 9.7577e-01
1.7288e-01 -3.5093e-01
1.0595e-01 4.7363e-01
1.1832e-15

3.1963e+00 4.8622e+00
-4.9904e-01 -2.6575e+00
8.2183e-01 2.3294e+00
-6.1693e-01 -1.8015e+00

0.0000e+00 0.0000e+00
0.0000e+00 -2.2204e-16
0.0000e+00 -4.4409e-16
0.0000e+00 0.0000e+00

0.0000e+00 0.0000e+00
8.8818e-16 0.0000e+00
0.0000e+00 -4.4409e-16
-4.4409e-16 4.4409e-16
7.8933e-01 7.0543e-01 -1.6338e-02 -2.6206e-01
1.4832e-15
-1.5505e+00 5.8987e-01 -8.4977e-01 7.3059e-01

0.0000e+00 0.0000e+00 0.0000e+00 0.0000e+00

-2.2204e-16 0.0000e+00 0.0000e+00 4.4409e-16
\end{Verbatim}
\newpage


\subsubsection{operator / (const rvector\&)}
Operator%
\pdfdest name {srbmatrix.operator / (rvector)} fit
\begin{verbatim}
rvector operator / (const rvector& vB) const throw (cvmexception);
\end{verbatim}
returns solution $x$ of linear equation
$A*x=b$ where calling matrix is square band matrix $A$
and a vector $b$ is passed in parameter \verb"vB".
This operator throws exception 
of type \GOT{cvmexception}{cvmexception}
in case of inappropriate sizes
of the objects or when  matrix $A$ is close to singular.
See also \GOT{rvector::solve}{rvector.solve}, 
\GOT{srmatrix::solve}{srmatrix.solve}, 
\GOT{rvector.operator~\%}{rvector.operator percent (srmatrix)}, 
\GOT{rvector}{rvector}, \GOT{srmatrix}{srmatrix}, \GOT{srbmatrix}{srbmatrix}.
Example:
\begin{Verbatim}
using namespace cvm;
std::cout.setf (std::ios::scientific | std::ios::showpos);
std::cout.precision (5);

try {
    srbmatrix ma(4,2,1);
    rvector  vb(4);
    rvector  vx(4);
    ma.randomize(-1.,1.);
    vb.randomize(-2.,2.);

    vx = ma / vb;

    std::cout << ma * vx - vb;
}
catch (std::exception& e) {
    std::cout << "Exception " << e.what () << std::endl;
}
\end{Verbatim}
prints
\begin{Verbatim}
+0.00000e+000 +2.22045e-016 +2.22045e-016 +0.00000e+000
\end{Verbatim}
\newpage




\subsubsection{identity}
Function%
\pdfdest name {srbmatrix.identity} fit
\begin{verbatim}
srbmatrix& srbmatrix::identity();
\end{verbatim}
sets  calling band matrix to be equal to identity matrix
and returns a reference to
the matrix changed. 
Function doesn't change
numbers of sub- and super-diagonals.
See also \GOT{srbmatrix}{srbmatrix}.
Example:
\begin{Verbatim}
using namespace cvm;

srbmatrix m(4);
m.randomize(0.,1.);
std::cout << m << std::endl;
std::cout << m.identity();
\end{Verbatim}
prints
\begin{Verbatim}
0.327372 0 0 0
0 0.955718 0 0
0 0 0.0960723 0
0 0 0 0.291818

1 0 0 0
0 1 0 0
0 0 1 0
0 0 0 1
\end{Verbatim}
\newpage





\subsubsection{vanish}
Function%
\pdfdest name {srbmatrix.vanish} fit
\begin{verbatim}
srbmatrix& srbmatrix::vanish();
\end{verbatim}
sets every element of a calling band matrix to be equal to zero
and returns a reference to
the matrix changed. This function is faster
than
\GOT{srbmatrix::set(TR)}{srbmatrix.set}
with zero operand passed.
See also \GOT{srbmatrix}{srbmatrix}.
Example:
\begin{Verbatim}
using namespace cvm;

srbmatrix m(4,1,0);
m.randomize(0.,1.);
std::cout << m << std::endl;
std::cout << m.vanish();
\end{Verbatim}
prints
\begin{Verbatim}
0.337138 0 0 0
0.101199 0.522843 0 0
0 0.258522 0.123447 0
0 0 0.591723 0.661489

0 0 0 0
0 0 0 0
0 0 0 0
0 0 0 0
\end{Verbatim}
\newpage




\subsubsection{randomize}
Function%
\pdfdest name {srbmatrix.randomize} fit
\begin{verbatim}
srbmatrix& srbmatrix::randomize (TR dFrom, TR dTo);
\end{verbatim}
fills  calling band matrix with pseudo-random numbers distributed between
\verb"dFrom" and \verb"dTo".
Function
returns a reference to the matrix changed.
See also
\GOT{srbmatrix}{srbmatrix}.
Example:
\begin{Verbatim}
using namespace cvm;

std::cout.setf (std::ios::scientific | std::ios::showpos);
std::cout.precision (7);
srbmatrix m(4,1,0);
m.randomize(0.,1.);
std::cout << m << std::endl;
\end{Verbatim}
prints
\begin{Verbatim}
+3.4281442e-01 +0.0000000e+00 +0.0000000e+00 +0.0000000e+00
+7.9808954e-01 +5.9761345e-01 +0.0000000e+00 +0.0000000e+00
+0.0000000e+00 +1.1670278e-01 +6.5645314e-02 +0.0000000e+00
+0.0000000e+00 +0.0000000e+00 +4.2225410e-01 +7.5563829e-02
\end{Verbatim}
\newpage

