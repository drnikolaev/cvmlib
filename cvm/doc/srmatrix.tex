\bigskip
\noindent
\verb"template <typename TR>"\\
\verb"class srmatrix : public rmatrix <TR>, public SqMatrix <TR,TR> {"\\
\verb"public:"\\
\verb"    "\GOT{srmatrix}{srmatrix.srmatrix ()}\verb" ();"\\
\verb"    explicit "\GOT{srmatrix}{srmatrix.srmatrix (int)}\verb" (int nMN);"\\
\verb"    "\GOT{srmatrix}{srmatrix.srmatrix (TR*,int)}\verb" (TR* pD, int nMN);"\\
\verb"    "\GOT{srmatrix}{srmatrix.srmatrix (const TR*,int)}\verb" (const TR* pD, int nMN);"\\
\verb"    "\GOT{srmatrix}{srmatrix.srmatrix (const srmatrix&)}\verb" (const srmatrix& m);"\\
\verb"    "\GOT{srmatrix}{srmatrix.srmatrix (const rmatrix&)}\verb" (const rmatrix& m);"\\
\verb"    explicit "\GOT{srmatrix}{srmatrix.srmatrix (const rvector&)}\verb" (const rvector& v);"\\
\verb"    "\GOT{srmatrix}{srmatrix.submatrixctr}\verb" (rmatrix& m, int nRow, int nCol, int size);"\\
\verb"    TR& "\GOT{operator ()}{srmatrix.operator (,)}\verb" (int im, int in) throw (cvmexception);"\\
\verb"    TR "\GOT{operator ()}{srmatrix.operator (,)}\verb" (int im, int in) const throw (cvmexception);"\\
\verb"    rvector "\GOT{operator ()}{srmatrix.operator ()}\verb" (int i) throw (cvmexception);"\\
\verb"    const rvector "\GOT{operator ()}{srmatrix.operator ()}\verb" (int i) const throw (cvmexception);"\\
\verb"    rvector "\GOT{operator []}{srmatrix.operator []}\verb" (int i) throw (cvmexception);"\\
\verb"    const rvector "\GOT{operator []}{srmatrix.operator []}\verb" (int i) const throw (cvmexception);"\\
\verb"    srmatrix& "\GOT{operator =}{srmatrix.operator = (const srmatrix&)}\verb" (const srmatrix& m) throw (cvmexception);"\\
\verb"    srmatrix& "\GOT{assign}{srmatrix.assign}\verb" (const rvector& v) throw (cvmexception);"\\
\verb"    srmatrix& "\GOT{assign}{srmatrix.assign}\verb" (const TR* pD);"\\
\verb"    srmatrix& "\GOT{assign}{srmatrix.assign (int, int, const rmatrix&)}\verb" (int nRow, int nCol, const rmatrix& m)"\\
\verb"                      throw (cvmexception);"\\
\verb"    srmatrix& "\GOT{set}{srmatrix.set}\verb" (TR x);"\\
\verb"    srmatrix& "\GOT{resize}{srmatrix.resize}\verb" (int nNewMN) throw (cvmexception);"\\
\verb"    srmatrix& "\GOT{operator <{}<}{srmatrix.operator <<}\verb" (const srmatrix& m) throw (cvmexception);"\\
\verb"    srmatrix "\GOT{operator +}{srmatrix.operator +}\verb" (const srmatrix& m) const"\\
\verb"                         throw (cvmexception);"\\
\verb"    srmatrix "\GOT{operator -}{srmatrix.operator -}\verb" (const srmatrix& m) const"\\
\verb"                         throw (cvmexception);"\\
\verb"    srmatrix& "\GOT{sum}{srmatrix.sum}\verb" (const srmatrix& m1,"\\
\verb"                   const srmatrix& m2) throw (cvmexception);"\\
\verb"    srmatrix& "\GOT{diff}{srmatrix.diff}\verb" (const srmatrix& m1,"\\
\verb"                    const srmatrix& m2) throw (cvmexception);"\\
\verb"    srmatrix& "\GOT{operator +=}{srmatrix.operator +=}\verb" (const srmatrix& m) throw (cvmexception);"\\
\verb"    srmatrix& "\GOT{operator -=}{srmatrix.operator -=}\verb" (const srmatrix& m) throw (cvmexception);"\\
\verb"    srmatrix "\GOT{operator -}{srmatrix.operator - ()}\verb" () const;"\\
\verb"    srmatrix& "\GOT{operator ++}{srmatrix.operator ++}\verb" ();"\\
\verb"    srmatrix& "\GOT{operator ++}{srmatrix.operator ++}\verb" (int);"\\
\verb"    srmatrix& "\GOT{operator -{}-}{srmatrix.operator --}\verb" ();"\\
\verb"    srmatrix& "\GOT{operator -{}-}{srmatrix.operator --}\verb" (int);"\\
\verb"    srmatrix "\GOT{operator *}{srmatrix.operator * (TR)}\verb" (TR d) const;"\\
\verb"    srmatrix "\GOT{operator /}{srmatrix.operator / (TR)}\verb" (TR d) const throw (cvmexception);"\\
\verb"    srmatrix& "\GOT{operator *=}{srmatrix.operator *= (TR)}\verb" (TR d);"\\
\verb"    srmatrix& "\GOT{operator /=}{srmatrix.operator /= (TR)}\verb" (TR d) throw (cvmexception);"\\
\verb"    srmatrix& "\GOT{normalize}{srmatrix.normalize}\verb" ();"\\
\verb"    srmatrix "\GOT{operator \TildaT}{srmatrix.transposition}\verb" () const throw (cvmexception);"\\
\verb"    srmatrix& "\GOT{transpose}{srmatrix.transposition}\verb" (const srmatrix& m) throw (cvmexception);"\\
\verb"    srmatrix& "\GOT{transpose}{srmatrix.transposition}\verb" ();"\\
\verb"    rvector "\GOT{operator *}{srmatrix.operator * (const rvector&)}\verb" (const rvector& v) const throw (cvmexception);"\\
\verb"    rmatrix "\GOT{operator *}{srmatrix.operator * (const rmatrix&)}\verb" (const rmatrix& m) const throw (cvmexception);"\\
\verb"    srmatrix "\GOT{operator *}{srmatrix.operator * (const srmatrix&)}\verb" (const srmatrix& m) const throw (cvmexception);"\\
\verb"    srmatrix& "\GOT{operator *=}{srmatrix.operator *= (const srmatrix&)}\verb" (const srmatrix& m) throw (cvmexception);"\\
\verb"    srmatrix& "\GOT{swap\_rows}{srmatrix.swap_rows}\verb" (int n1, int n2) throw (cvmexception);"\\
\verb"    srmatrix& "\GOT{swap\_cols}{srmatrix.swap_cols}\verb" (int n1, int n2) throw (cvmexception);"\\
\verb"    rvector "\GOT{solve}{srmatrix.solve}\verb" (const rvector& vB) const throw (cvmexception);"\\
\verb"    rmatrix "\GOT{solve}{srmatrix.solve}\verb" (const rmatrix& mB) const throw (cvmexception);"\\
\verb"    rvector "\GOT{solve}{srmatrix.solve}\verb" (const rvector& vB, TR& dErr) const"\\
\verb"                   throw (cvmexception);"\\
\verb"    rmatrix "\GOT{solve}{srmatrix.solve}\verb" (const rmatrix& mB, TR& dErr) const"\\
\verb"                   throw (cvmexception);"\\
\verb"    rvector "\GOT{solve\_tran}{srmatrix.solvetran}\verb" (const rvector& vB) const throw (cvmexception);"\\
\verb"    rmatrix "\GOT{solve\_tran}{srmatrix.solvetran}\verb" (const rmatrix& mB) const throw (cvmexception);"\\
\verb"    rvector "\GOT{solve\_tran}{srmatrix.solvetran}\verb" (const rvector& vB, TR& dErr) const"\\
\verb"                        throw (cvmexception);"\\
\verb"    rmatrix "\GOT{solve\_tran}{srmatrix.solvetran}\verb" (const rmatrix& mB, TR& dErr) const"\\
\verb"                        throw (cvmexception);"\\
\verb"    rvector "\GOT{operator \%}{srmatrix.operator percent (rvector)}\verb" (const rvector& vB) const throw (cvmexception);"\\
\verb"    rvector "\GOT{operator /}{srmatrix.operator / (rvector)}\verb" (const rvector& vB) const throw (cvmexception);"\\
\verb"    rvector "\GOT{solve\_lu}{srmatrix.solvelu}\verb" (const srmatrix& mLU, const int* pPivots,"\\
\verb"                      const rvector& vB, TR& dErr) throw (cvmexception);"\\
\verb"    rvector "\GOT{solve\_lu}{srmatrix.solvelu}\verb" (const srmatrix& mLU, const int* pPivots,"\\
\verb"                      const rvector& vB) throw (cvmexception);"\\
\verb"    rmatrix "\GOT{solve\_lu}{srmatrix.solvelu}\verb" (const srmatrix& mLU, const int* pPivots,"\\
\verb"                      const rmatrix& mB, TR& dErr) throw (cvmexception);"\\
\verb"    rmatrix "\GOT{solve\_lu}{srmatrix.solvelu}\verb" (const srmatrix& mLU, const int* pPivots,"\\
\verb"                      const rmatrix& mB) throw (cvmexception);"\\
\verb"    TR "\GOT{det}{srmatrix.det}\verb" () const throw (cvmexception);"\\
\verb"    srmatrix& "\GOT{low\_up}{srmatrix.low_up}\verb" (const srmatrix& m,"\\
\verb"                      int* nPivots) throw (cvmexception);"\\
\verb"    srmatrix "\GOT{low\_up}{srmatrix.low_up}\verb" (int* nPivots) const throw (cvmexception);"\\
\verb"    TR "\GOT{cond}{srmatrix.cond}\verb" () const throw (cvmexception);"\\
\verb"    srmatrix& "\GOT{inv}{srmatrix.inv}\verb" (const srmatrix& mArg) throw (cvmexception);"\\
\verb"    srmatrix "\GOT{inv}{srmatrix.inv}\verb" () const throw (cvmexception);"\\
\verb"    srmatrix& "\GOT{exp}{srmatrix.exp}\verb" (const srmatrix& m,"\\
\verb"                   TR tol = cvmMachSp ()) throw (cvmexception);"\\
\verb"    srmatrix "\GOT{exp}{srmatrix.exp}\verb" (TR tol = cvmMachSp ()) const throw (cvmexception);"\\
\verb"    srmatrix& "\GOT{polynom}{srmatrix.polynom}\verb" (const srmatrix& m, const rvector& v)"\\
\verb"                       throw (cvmexception);"\\
\verb"    srmatrix "\GOT{polynom}{srmatrix.polynom}\verb" (const rvector& v) const throw (cvmexception);"\\
\verb"    cvector "\GOT{eig}{srmatrix.eig}\verb" (scmatrix& mEigVect,"\\
\verb"                 bool bRightVect = true) const throw (cvmexception);"\\
\verb"    cvector "\GOT{eig}{srmatrix.eig}\verb" () const throw (cvmexception);"\\
\verb"    srmatrix& "\GOT{cholesky}{srmatrix.cholesky}\verb" (const srsmatrix& m) throw (cvmexception);"\\
\verb"    srmatrix& "\GOT{bunch\_kaufman}{srmatrix.bunch_kaufman}\verb" (const srsmatrix& m,"\\
\verb"                             int* pivots) throw (cvmexception);"\\
\verb"    void "\GOT{qr}{srmatrix.qr}\verb"(srmatrix& mQ, srmatrix& mR) const throw (cvmexception);"\\
\verb"    void "\GOT{lq}{srmatrix.lq}\verb"(srmatrix& mL, srmatrix& mQ) const throw (cvmexception);"\\
\verb"    void "\GOT{rq}{srmatrix.rq}\verb"(srmatrix& mR, srmatrix& mQ) const throw (cvmexception);"\\
\verb"    void "\GOT{ql}{srmatrix.ql}\verb"(srmatrix& mQ, srmatrix& mL) const throw (cvmexception);"\\
\verb"    srmatrix& "\GOT{identity}{srmatrix.identity}\verb" ();"\\
\verb"    srmatrix& "\GOT{vanish}{srmatrix.vanish}\verb" ();"\\
\verb"    srmatrix& "\GOT{randomize}{srmatrix.randomize}\verb" (TR dFrom, TR dTo);"\\
\verb"};"
\newpage




\subsubsection{srmatrix ()}
Constructor%
\pdfdest name {srmatrix.srmatrix ()} fit
\begin{verbatim}
srmatrix::srmatrix ();
\end{verbatim}
creates  empty \verb"srmatrix" object.
See also \GOT{srmatrix}{srmatrix}.
Example:
\begin{Verbatim}
using namespace cvm;

srmatrix m;
std::cout << m.msize() << std::endl << m.nsize() << std::endl;
std::cout << m.size() << std::endl;

m.resize (3);
std::cout << m;
\end{Verbatim}
prints
\begin{Verbatim}
0
0
0
0 0 0
0 0 0
0 0 0
\end{Verbatim}
\newpage



\subsubsection{srmatrix (int)}
Constructor%
\pdfdest name {srmatrix.srmatrix (int)} fit
\begin{verbatim}
explicit srmatrix::srmatrix (int nMN);
\end{verbatim}
creates  $n\times n$ \verb"srmatrix" object where $n$ is passed in
\verb"nMN" parameter. Constructor sets all elements to zero.
Constructor throws  \GOT{cvmexception}{cvmexception}
in case of non-positive size passed or memory allocation failure.
See also \GOT{srmat\-rix}{srmatrix}.
Example:
\begin{Verbatim}
using namespace cvm;

srmatrix m (4);
std::cout << m.msize() << std::endl << m.nsize() 
          << std::endl << m.size() << std::endl << m;
\end{Verbatim}
prints
\begin{Verbatim}
4
4
16
0 0 0 0
0 0 0 0
0 0 0 0
0 0 0 0
\end{Verbatim}
\newpage



\subsubsection{srmatrix (TR*,int)}
Constructor%
\pdfdest name {srmatrix.srmatrix (TR*,int)} fit
\begin{verbatim}
srmatrix::srmatrix (TR* pD, int nMN);
\end{verbatim}
creates  $n\times n$ \verb"srmatrix" object where $n$ is passed in
\verb"nMN" parameter.
Unlike others, this constructor \textit{does not allocate  memory}.
It just shares  memory with an array pointed to by \verb"pD".
Constructor throws  \GOT{cvmexception}{cvmexception}
in case of non-positive size passed.
See also \GOT{srmatrix}{srmatrix}, 
\GOT{srmatrix (const TR*,int)}{srmatrix.srmatrix (const TR*,int)}.
Example:
\begin{Verbatim}
using namespace cvm;
double a[] = {1., 1., 1., 1., 1., 1., 1., 1., 1.};
srmatrix m (a, 3);

m(1,1) = 5.;
std::cout << m << std::endl;
std::cout << a[0] << " " << a[1] << " " << a[2] << std::endl;
\end{Verbatim}
prints
\begin{Verbatim}
5 1 1
1 1 1
1 1 1

5 1 1
\end{Verbatim}
\newpage


\subsubsection{srmatrix (const TR*,int)}
Constructor%
\pdfdest name {srmatrix.srmatrix (const TR*,int)} fit
\begin{verbatim}
srmatrix::srmatrix (const TR* pD, int nMN);
\end{verbatim}
creates  $n\times n$ \verb"srmatrix" object where $n$ is passed in
\verb"nMN" parameter
and copies $n*n$ elements of an array  \verb"pD" to it \GOT{by colums}{SubSubSectionStorage}.
Constructor throws  \GOT{cvmexception}{cvmexception}
in case of non-positive size passed.
See also \GOT{srmatrix}{srmatrix}, 
\GOT{srmatrix (TR*,int)}{srmatrix.srmatrix (TR*,int)}.
Example:
\begin{Verbatim}
using namespace cvm;
const double a[] = {1., 1., 1., 1., 1., 1., 1., 1., 1.};
srmatrix m (a, 3);

m(1,1) = 5.;
std::cout << m << std::endl;
std::cout << a[0] << " " << a[1] << " " << a[2] << std::endl;
\end{Verbatim}
prints
\begin{Verbatim}
5 1 1
1 1 1
1 1 1

1 1 1
\end{Verbatim}
\newpage




\subsubsection{srmatrix (const srmatrix\&)}
Copy constructor%
\pdfdest name {srmatrix.srmatrix (const srmatrix&)} fit
\begin{verbatim}
srmatrix::srmatrix (const srmatrix& m);
\end{verbatim}
creates  \verb"srmatrix" object as a copy of \verb"m".
Constructor throws  \GOT{cvmexception}{cvmexception}
in case of memory allocation failure.
See also \GOT{srmatrix}{srmatrix}.
Example:
\begin{Verbatim}
using namespace cvm;

double a[] = {1., 2., 3., 4., 5., 6., 7., 8., 9.};
srmatrix m(a, 3);
srmatrix mc(m);

m(1,1) = 7.77;
std::cout << m << std::endl << mc;
\end{Verbatim}
prints
\begin{Verbatim}
7.77 4 7
2 5 8
3 6 9

1 4 7
2 5 8
3 6 9
\end{Verbatim}
\newpage



\subsubsection{srmatrix (const rmatrix\&)}
Constructor%
\pdfdest name {srmatrix.srmatrix (const rmatrix&)} fit
\begin{verbatim}
srmatrix::srmatrix (const rmatrix& m);
\end{verbatim}
creates  \verb"srmatrix" object as a copy of a matrix \verb"m".
It's assumed that $m\times n$ matrix \verb"m" must have equal
sizes, i.e. $m = n$ is satisfied.
Constructor throws  \GOT{cvmexception}{cvmexception}
if this is not true or in case of memory allocation failure.
Please note that this constructor is \emph{not explicit} anymore.
See also \GOT{srmatrix}{srmatrix}.
Example:
\begin{Verbatim}
using namespace cvm;

double a[] = {1., 2., 3., 4., 5., 6.};
rmatrix m(a, 2, 3);
std::cout << m << std::endl;

m.resize(3, 3);
srmatrix ms (m);
std::cout << ms;
\end{Verbatim}
prints
\begin{Verbatim}
1 3 5
2 4 6

1 3 5
2 4 6
0 0 0
\end{Verbatim}
\newpage



\subsubsection{srmatrix (const rvector\&)}
Constructor%
\pdfdest name {srmatrix.srmatrix (const rvector&)} fit
\begin{verbatim}
explicit srmatrix::srmatrix (const rvector& v);
\end{verbatim}
creates  \verb"srmatrix" object
of size \verb"v.size()" by \verb"v.size()"
and assigns vector \verb"v" to its main diagonal.
Constructor throws  \GOT{cvmexception}{cvmexception}
in case of memory allocation failure.
See also \GOT{srmatrix}{srmatrix}, \GOT{rvector}{rvector}.
Example:
\begin{Verbatim}
using namespace cvm;

double a[] = {1., 2., 3., 4., 5.};
rvector v(a, 5);
srmatrix m(v);
std::cout << m;
\end{Verbatim}
prints
\begin{Verbatim}
1 0 0 0 0
0 2 0 0 0
0 0 3 0 0
0 0 0 4 0
0 0 0 0 5
\end{Verbatim}
\newpage



\subsubsection{submatrix}
Submatrix constructor%
\pdfdest name {srmatrix.submatrixctr} fit
\begin{verbatim}
srmatrix::srmatrix (rmatrix& m, int nRow, int nCol, int size);
\end{verbatim}
creates  \verb"srmatrix" object as  \emph{submatrix} of \verb"m".
It means that the matrix object created shares  memory with some part
of \verb"m". This part is defined by its upper left corner (parameters 
\verb"nRow" and \verb"nCol", both are \Based)
and its size (parameter \verb"size").
See also \GOT{srmatrix}{srmatrix}.
Example:
\begin{Verbatim}
using namespace cvm;

rmatrix m(4,5);
srmatrix subm(m, 2, 2, 2);
subm.set(1.);

std::cout << m;
\end{Verbatim}
prints
\begin{Verbatim}
0 0 0 0 0
0 1 1 0 0
0 1 1 0 0
0 0 0 0 0
\end{Verbatim}
\newpage



\subsubsection{operator (,)}
Indexing operators%
\pdfdest name {srmatrix.operator (,)} fit
\begin{verbatim}
TR& srmatrix::operator () (int im, int in) throw (cvmexception);
TR srmatrix::operator () (int im, int in) const throw (cvmexception);
\end{verbatim}
provide access to a particular element of a calling matrix. The first version
is applicable to non-constant object.
This version returns  \emph{l-value}
in order to make possible write access to a particular element.
Both operators are \Based.
Operators throw \GOT{cvmexception}{cvmexception}
if some of parameters passed
is outside of \verb"[1,msize()]" range.
Operators are \emph{inherited}
in the class
\GOT{srbmatrix}{srbmatrix} 
and \emph{redefined}
in the class \GOT{srsmatrix}{srsmatrix}.
See also \GOT{srmatrix}{srmatrix},
\GOT{Matrix::msize()}{Matrix.msize},
\GOT{Matrix::nsize()}{Matrix.nsize}.
Example:
\begin{Verbatim}
using namespace cvm;

try {
    double a[] = {1., 2., 3., 4., 5., 6., 7., 8., 9.};
    const srmatrix m (a, 3);
    srmatrix ms(m);

    std::cout << m(1,1) << " " << m(2,3) << std::endl << std::endl;

    ms(2,2) = 7.77;
    std::cout << ms;
}
catch (std::exception& e) {
    std::cout << "Exception " << e.what () << std::endl;
}
\end{Verbatim}
prints
\begin{Verbatim}
1 8

1 4 7
2 7.77 8
3 6 9
\end{Verbatim}
\newpage


\subsubsection{operator ()}
Indexing operators%
\pdfdest name {srmatrix.operator ()} fit
\begin{verbatim}
rvector srmatrix::operator () (int i) throw (cvmexception);
const rvector srmatrix::operator () (int i) const throw (cvmexception);
\end{verbatim}
provide access to \hbox{$i$-th} column of a calling matrix. The first version
is applicable to  non-constant object and
\emph{returns  l-value}, i.e. the vector returned shares  memory
with \hbox{$i$-th} column of a calling matrix
in order to make possible write access to it.
The second version creates  \emph{copy} of a column and therefore it's
\emph{not  l-value}.
Both operators are \Based.
Operators throw \GOT{cvmexception}{cvmexception}
if parameter \verb"i" is outside of \verb"[1,nsize()]" range.
Operators are \emph{redefined}
in the classes \GOT{srbmatrix}{srbmatrix}
and \GOT{srsmatrix}{srsmatrix}.
See also \GOT{srmatrix}{srmatrix}.
Example:
\begin{Verbatim}
using namespace cvm;

std::cout.setf (std::ios::scientific | std::ios::left); 
std::cout.precision (2);
try {
    double a[] = {1., 2., 3., 4., 5., 6., 7., 8., 9.};
    const srmatrix m (a, 3);
    srmatrix ms(3);

    std::cout << m(2) << std::endl;

    ms(2) = m(3);
    std::cout << ms;
}
catch (std::exception& e) {
    std::cout << "Exception " << e.what () << std::endl;
}
\end{Verbatim}
prints
\begin{Verbatim}
4.00e+00 5.00e+00 6.00e+00

0.00e+00 7.00e+00 0.00e+00
0.00e+00 8.00e+00 0.00e+00
0.00e+00 9.00e+00 0.00e+00
\end{Verbatim}
\newpage



\subsubsection{operator []}
Indexing operators%
\pdfdest name {srmatrix.operator []} fit
\begin{verbatim}
rvector srmatrix::operator [] (int i) throw (cvmexception);
const rvector srmatrix::operator [] (int i) const throw (cvmexception);
\end{verbatim}
provide access to \hbox{$i$-th} row of a calling matrix. The first version
is applicable to non-constant object and
\emph{returns  l-value}, i.e. the vector returned shares  memory
with \hbox{$i$-th} row of a calling matrix
in order to make possible write access to it.
The second version creates  \emph{copy} of a row and therefore it's
\emph{not  l-value}.
Both operators are \Based.
Operators throw \GOT{cvmexception}{cvmexception}
if  parameter \verb"i" is outside of \verb"[1,msize()]" range.
Operators are \emph{redefined}
in the classes \GOT{srbmatrix}{srbmatrix}
and \GOT{srsmatrix}{srsmatrix}.
See also \GOT{srmatrix}{srmatrix},
\GOT{Matrix::msize()}{Matrix.msize}.
Example:
\begin{Verbatim}
using namespace cvm;

std::cout.setf (std::ios::scientific | std::ios::left); 
std::cout.precision (2);
try {
    double a[] = {1., 2., 3., 4., 5., 6., 7., 8., 9.};
    const srmatrix m (a, 3);
    srmatrix ms(3);

    std::cout << m[2] << std::endl;

    ms[2] = m[3];
    std::cout << ms;
}
catch (std::exception& e) {
    std::cout << "Exception " << e.what () << std::endl;
}
\end{Verbatim}
prints
\begin{Verbatim}
2.00e+00 5.00e+00 8.00e+00

0.00e+00 0.00e+00 0.00e+00
3.00e+00 6.00e+00 9.00e+00
0.00e+00 0.00e+00 0.00e+00
\end{Verbatim}
\newpage



\subsubsection{operator = (const srmatrix\&)}
Operator%
\pdfdest name {srmatrix.operator = (const srmatrix&)} fit
\begin{verbatim}
srmatrix& srmatrix::operator = (const srmatrix& m)
throw (cvmexception);
\end{verbatim}
sets  every element of a calling matrix to a value of
appropriate element of a matrix \verb"m"
and returns a reference to
the matrix changed.
Operator throws  \GOT{cvmexception}{cvmexception}
in case of different sizes of the operands.
Operator is \emph{redefined} in the classes
\GOT{srbmatrix}{srbmatrix} 
and \GOT{srsmatrix}{srsmatrix}.
See also \GOT{srmatrix}{srmatrix}.
Example:
\begin{Verbatim}
using namespace cvm;

std::cout.setf (std::ios::scientific | std::ios::left); 
std::cout.precision (2);
try {
    double a[] = {1., 2., 3., 4., 5., 6., 7., 8., 9.};
    const srmatrix m1(a, 3);
    srmatrix m2(3);

    m2 = m1;
    std::cout << m2;
}
catch (std::exception& e) {
    std::cout << "Exception " << e.what () << std::endl;
}
\end{Verbatim}
prints
\begin{Verbatim}
1.00e+00 4.00e+00 7.00e+00
2.00e+00 5.00e+00 8.00e+00
3.00e+00 6.00e+00 9.00e+00
\end{Verbatim}
\newpage



\subsubsection{assign (const TR*)}
Function%
\pdfdest name {srmatrix.assign} fit
\begin{verbatim}
srmatrix& srmatrix::assign (const rvector& v) throw (cvmexception);
srmatrix& srmatrix::assign (const TR* pD);
\end{verbatim}
sets every element of a calling matrix to a value of
appropriate element of  vector~\verb'v'
or  array pointed to by~\verb"pD"
and returns a reference to
the matrix changed.
In first version it's assumed that vector passed is long
enough to fill calling matrix. Function throws \GOT{cvmexception}{cvmexception} 
otherwise.
Function is \emph{redefined} in the classes
\GOT{srbmatrix}{srbmatrix} 
and \GOT{srsmatrix}{srsmatrix}.
See also \GOT{srmatrix}{srmatrix}.
Example:
\begin{Verbatim}
using namespace cvm;
std::cout.setf (std::ios::scientific | std::ios::left); 
std::cout.precision (2);

const double a[] = {1., 2., 3., 4., 5., 6., 7., 8., 9.};
srmatrix m(3);

m.assign(a);
std::cout << m;
\end{Verbatim}
prints
\begin{Verbatim}
1.00e+00 4.00e+00 7.00e+00
2.00e+00 5.00e+00 8.00e+00
3.00e+00 6.00e+00 9.00e+00
\end{Verbatim}
\newpage




\subsubsection{assign (int, int, const rmatrix\&)}
Function%
\pdfdest name {srmatrix.assign (int, int, const rmatrix&)} fit
\begin{verbatim}
srmatrix& srmatrix::assign (int nRow, int nCol, const rmatrix& m)
throw (cvmexception);
\end{verbatim}
sets sub-matrix of a calling matrix beginning with \Based row
\verb"nRow" and column \verb"nCol" to a matrix \verb"m" and
returns a reference to the matrix changed. Function throws 
 \GOT{cvmexception}{cvmexception} if \verb"nRow"
or \verb"nCol" are not positive or matrix \verb"m" doesn't fit.
Function is \emph{redefined} in the class
\GOT{srsmatrix}{srsmatrix}.
See also \GOT{rmatrix}{rmatrix}, \GOT{srmatrix}{srmatrix}.
Example:
\begin{Verbatim}
using namespace cvm;

srmatrix m1(5);
rmatrix m2(2,3);
m1.set(1.);
m2.set(2.);
m1.assign(2,3,m2);
std::cout << m1;
\end{Verbatim}
prints
\begin{Verbatim}
1 1 1 1 1
1 1 2 2 2
1 1 2 2 2
1 1 1 1 1
1 1 1 1 1
\end{Verbatim}
\newpage





\subsubsection{set (TR)}
Function%
\pdfdest name {srmatrix.set} fit
\begin{verbatim}
srmatrix& srmatrix::set (TR x);
\end{verbatim}
sets every element of a calling matrix to a value of
parameter \verb"x" and returns a reference to
the matrix changed.
Use \GOT{vanish}{srmatrix.vanish} to set every element
of a calling matrix to be equal to zero.
Function is \emph{redefined} in the classes
\GOT{srbmatrix}{srbmatrix} 
and \GOT{srsmatrix}{srsmatrix}.
See also \GOT{srmatrix}{srmatrix}.
Example:
\begin{Verbatim}
using namespace cvm;
std::cout.setf (std::ios::scientific | std::ios::left); 
std::cout.precision (2);
srmatrix m(3);

m.set(3.);
std::cout << m;
\end{Verbatim}
prints
\begin{Verbatim}
3.00e+00 3.00e+00 3.00e+00
3.00e+00 3.00e+00 3.00e+00
3.00e+00 3.00e+00 3.00e+00
\end{Verbatim}
\newpage



\subsubsection{resize}
Function%
\pdfdest name {srmatrix.resize} fit
\begin{verbatim}
srmatrix& srmatrix::resize (int nNewMN) throw (cvmexception);
\end{verbatim}
changes  size of a calling matrix to \verb"nNewMN" by \verb"nNewMN"
and returns a reference to
the matrix changed. In case of increasing of its size, calling matrix
is filled up with zeroes.
Function throws  \GOT{cvmexception}{cvmexception}
in case of negative size passed or memory allocation failure.
Function is \emph{redefined} in the classes
\GOT{srbmatrix}{srbmatrix} 
and \GOT{srsmatrix}{srsmatrix}.
See also \GOT{srmatrix}{srmatrix}.
Example:
\begin{Verbatim}
using namespace cvm;

try {
    double a[] = {1., 2., 3., 4.};
    srmatrix m(a, 2);

    std::cout << m << std::endl;

    m.resize (3);
    std::cout << m;
}
catch (std::exception& e) {
    std::cout << "Exception " << e.what () << std::endl;
}
\end{Verbatim}
prints
\begin{Verbatim}
1 3
2 4

1 3 0
2 4 0
0 0 0
\end{Verbatim}
\newpage



\subsubsection{operator <{}<}
Operator%
\pdfdest name {srmatrix.operator <<} fit
\begin{verbatim}
srmatrix& srmatrix::operator << (const srmatrix& m)
throw (cvmexception);
\end{verbatim}
destroys  calling matrix, creates  new one as a copy of \verb"m"
and returns a reference to
the matrix changed.
Operator throws  \GOT{cvmexception}{cvmexception}
in case of memory allocation failure.
Operator is \emph{redefined} in the classes
\GOT{srbmatrix}{srbmatrix} 
and \GOT{srsmatrix}{srsmatrix}.
See also \GOT{srmatrix}{srmatrix}.
Example:
\begin{Verbatim}
using namespace cvm;
std::cout.setf (std::ios::scientific | std::ios::left); 
std::cout.precision (2);

try {
    srmatrix m(3);
    srmatrix mc(1);
    m(1,2) = 1.;
    m(2,3) = 2.;
    std::cout << m << std::endl << mc << std::endl;

    mc << m;
    std::cout << mc;
}
catch (std::exception& e) {
    std::cout << "Exception " << e.what () << std::endl;
}
\end{Verbatim}
prints
\begin{Verbatim}
0.00e+00 1.00e+00 0.00e+00
0.00e+00 0.00e+00 2.00e+00
0.00e+00 0.00e+00 0.00e+00

0.00e+00

0.00e+00 1.00e+00 0.00e+00
0.00e+00 0.00e+00 2.00e+00
0.00e+00 0.00e+00 0.00e+00
\end{Verbatim}
\newpage



\subsubsection{operator +}
Operator%
\pdfdest name {srmatrix.operator +} fit
\begin{verbatim}
srmatrix srmatrix::operator + (const srmatrix& m) const
throw (cvmexception);
\end{verbatim}
creates an object of type \verb"srmatrix" as a sum of
a calling matrix and a matrix \verb"m".
Operator throws  \GOT{cvmexception}{cvmexception}
in case of different sizes of the operands.
Operator is \emph{redefined} in the classes
\GOT{srbmatrix}{srbmatrix} 
and \GOT{srsmatrix}{srsmatrix}.
See also \GOT{srmatrix::sum}{srmatrix.sum}, \GOT{srmatrix}{srmatrix}.
Example:
\begin{Verbatim}
using namespace cvm;

try {
    double a[] = {1., 2., 3., 4., 5., 6., 7., 8., 9.};
    double b[] = {10., 20., 30., 40., 50., 60., 70., 80., 90.};
    srmatrix m1(a, 3);
    srmatrix m2(b, 3);

    std::cout << m1 + m2 << std::endl << m1 + m1;
}
catch (std::exception& e) {
    std::cout << "Exception " << e.what () << std::endl;
}
\end{Verbatim}
prints
\begin{Verbatim}
11 44 77
22 55 88
33 66 99

2 8 14
4 10 16
6 12 18
\end{Verbatim}
\newpage




\subsubsection{operator -}
Operator%
\pdfdest name {srmatrix.operator -} fit
\begin{verbatim}
srmatrix srmatrix::operator - (const srmatrix& m) const
throw (cvmexception);
\end{verbatim}
creates an object of type \verb"srmatrix" as a difference of
a calling matrix and a matrix \verb"m".
It throws  \GOT{cvmexception}{cvmexception}
in case of different sizes of the operands.
Operator is \emph{redefined} in the classes
\GOT{srbmatrix}{srbmatrix} 
and \GOT{srsmatrix}{srsmatrix}.
See also \GOT{srmatrix::diff}{srmatrix.diff}, \GOT{srmatrix}{srmatrix}.
Example:
\begin{Verbatim}
using namespace cvm;

try {
    double a[] = {1., 2., 3., 4., 5., 6., 7., 8., 9.};
    double b[] = {10., 20., 30., 40., 50., 60., 70., 80., 90.};
    srmatrix m1(a, 3);
    srmatrix m2(b, 3);

    std::cout << m2 - m1 << std::endl << m1 - m1;
}
catch (std::exception& e) {
    std::cout << "Exception " << e.what () << std::endl;
}
\end{Verbatim}
prints
\begin{Verbatim}
9 36 63
18 45 72
27 54 81

0 0 0
0 0 0
0 0 0
\end{Verbatim}
\newpage


\subsubsection{sum}
Function%
\pdfdest name {srmatrix.sum} fit
\begin{verbatim}
srmatrix& srmatrix::sum (const srmatrix& m1, const srmatrix& m2)
throw (cvmexception);
\end{verbatim}
assigns the result of addition of
matrices \verb"m1" and \verb"m2"  to a calling matrix
and returns a reference to
the matrix changed.
It throws  \GOT{cvmexception}{cvmexception}
in case of different sizes of the operands.
Function is \emph{redefined} in the classes
\GOT{srbmatrix}{srbmatrix} 
and \GOT{srsmatrix}{srsmatrix}.
See also \GOT{srmatrix::operator +~}{srmatrix.operator +},
\GOT{srmatrix}{srmatrix}.
Example:
\begin{Verbatim}
using namespace cvm;

try {
    double a[] = {1., 2., 3., 4., 5., 6., 7., 8., 9.};
    const srmatrix m1(a, 3);
    srmatrix m2(3);
    srmatrix m(3);
    m2.set(1.);

    std::cout << m.sum(m1, m2) << std::endl;
    std::cout << m.sum(m, m2);
}
catch (std::exception& e) {
    std::cout << "Exception " << e.what () << std::endl;
}
\end{Verbatim}
prints
\begin{Verbatim}
2 5 8
3 6 9
4 7 10

3 6 9
4 7 10
5 8 11
\end{Verbatim}
\newpage



\subsubsection{diff}
Function%
\pdfdest name {srmatrix.diff} fit
\begin{verbatim}
srmatrix& srmatrix::diff (const srmatrix& m1, const srmatrix& m2)
throw (cvmexception);
\end{verbatim}
assigns the result of subtraction of
matrices \verb"m1" and \verb"m2" to a calling matrix
and returns a reference to
the matrix changed.
It throws  \GOT{cvmexception}{cvmexception}
in case of different sizes of the operands.
Function is \emph{redefined} in the classes
\GOT{srbmatrix}{srbmatrix} 
and \GOT{srsmatrix}{srsmatrix}.
See also \GOT{srmatrix::operator -~}{srmatrix.operator -},
\GOT{srmatrix}{srmatrix}.
Example:
\begin{Verbatim}
using namespace cvm;

try {
    double a[] = {1., 2., 3., 4., 5., 6., 7., 8., 9.};
    const srmatrix m1(a, 3);
    srmatrix m2(3);
    srmatrix m(3);
    m2.set(1.);

    std::cout << m.diff(m1, m2) << std::endl;
    std::cout << m.diff(m, m2);
}
catch (std::exception& e) {
    std::cout << "Exception " << e.what () << std::endl;
}
\end{Verbatim}
prints
\begin{Verbatim}
0 3 6
1 4 7
2 5 8

-1 2 5
0 3 6
1 4 7
\end{Verbatim}
\newpage



\subsubsection{operator +=}
Operator%
\pdfdest name {srmatrix.operator +=} fit
\begin{verbatim}
srmatrix& srmatrix::operator += (const srmatrix& m) throw (cvmexception);
\end{verbatim}
adds  matrix \verb"m" to a calling matrix 
and returns a reference to
the matrix changed.
It throws  \GOT{cvmexception}{cvmexception}
in case of different sizes of the operands.
Operator is \emph{redefined} in the classes
\GOT{srbmatrix}{srbmatrix} 
and \GOT{srsmatrix}{srsmatrix}.
See also \GOT{srmatrix::ope\-ra\-tor +~}{srmatrix.operator +},
\GOT{srmatrix::sum}{srmatrix.sum},
\GOT{srmatrix}{srmatrix}.
Example:
\begin{Verbatim}
using namespace cvm;
try {
    srmatrix m1(3);
    srmatrix m2(3);
    m1.set(1.);
    m2.set(2.);

    m1 += m2;
    std::cout << m1 << std::endl;

    // well, you can do this too, but temporary object would be created
    m2 += m2; 
    std::cout << m2;
}
catch (std::exception& e) {
    std::cout << "Exception " << e.what () << std::endl;
}
\end{Verbatim}
prints
\begin{Verbatim}
3 3 3
3 3 3
3 3 3

4 4 4
4 4 4
4 4 4
\end{Verbatim}
\newpage




\subsubsection{operator -=}
Operator%
\pdfdest name {srmatrix.operator -=} fit
\begin{verbatim}
srmatrix& srmatrix::operator -= (const srmatrix& m) throw (cvmexception);
\end{verbatim}
subtracts  matrix \verb"m" from  calling matrix
and returns a reference to
the matrix changed.
It throws  \GOT{cvmexception}{cvmexception}
in case of different sizes of the operands.
Operator is \emph{redefined} in the classes
\GOT{srbmatrix}{srbmatrix} 
and \GOT{srsmatrix}{srsmatrix}.
See also \GOT{srmatrix::ope\-ra\-tor -~}{srmatrix.operator -},
\GOT{srmatrix::diff}{srmatrix.diff},
\GOT{srmatrix}{srmatrix}.
Example:
\begin{Verbatim}
using namespace cvm;

try {
    srmatrix m1(3);
    srmatrix m2(3);
    m1.set(1.);
    m2.set(2.);

    m1 -= m2;
    std::cout << m1 << std::endl;

    // well, you can do this too, but temporary object would be created
    m2 -= m2; 
    std::cout << m2;
}
catch (std::exception& e) {
    std::cout << "Exception " << e.what () << std::endl;
}
\end{Verbatim}
prints
\begin{Verbatim}
-1 -1 -1
-1 -1 -1
-1 -1 -1

0 0 0
0 0 0
0 0 0
\end{Verbatim}
\newpage



\subsubsection{operator - ()}
Operator%
\pdfdest name {srmatrix.operator - ()} fit
\begin{verbatim}
srmatrix srmatrix::operator - () const throw (cvmexception);
\end{verbatim}
creates an object of type \verb"srmatrix" as
a calling matrix multiplied by $-1$.
Operator is \emph{redefined} in the classes
\GOT{srbmatrix}{srbmatrix} 
and \GOT{srsmatrix}{srsmatrix}.
See also \GOT{srmatrix}{srmatrix}.
Example:
\begin{Verbatim}
using namespace cvm;

double a[] = {1., 2., 3., 4., 5., 6., 7., 8., 9.};
srmatrix m(a, 3);

std::cout << -m;
\end{Verbatim}
prints
\begin{Verbatim}
-1 -4 -7
-2 -5 -8
-3 -6 -9
\end{Verbatim}
\newpage



\subsubsection{operator ++}
Operator%
\pdfdest name {srmatrix.operator ++} fit
\begin{verbatim}
srmatrix& srmatrix::operator ++ ();
srmatrix& srmatrix::operator ++ (int);
\end{verbatim}
adds identity matrix to a calling matrix
and returns a reference to
the matrix changed.
Operator is \emph{redefined} in the classes
\GOT{srbmatrix}{srbmatrix} and \GOT{srsmatrix}{srsmatrix}.
See also \GOT{srmatrix}{srmatrix}.
Example:
\begin{Verbatim}
using namespace cvm;

double a[] = {1., 2., 3., 4., 5., 6., 7., 8., 9.};
srmatrix m (a, 3);

m++;
std::cout << m << std::endl;
std::cout << ++m;
\end{Verbatim}
prints
\begin{Verbatim}
2 4 7
2 6 8
3 6 10

3 4 7
2 7 8
3 6 11
\end{Verbatim}
\newpage



\subsubsection{operator -{}-}
Operator%
\pdfdest name {srmatrix.operator --} fit
\begin{verbatim}
srmatrix& srmatrix::operator -- ();
srmatrix& srmatrix::operator -- (int);
\end{verbatim}
subtracts identity matrix from  calling matrix
and returns a reference to
the matrix changed.
Operator is \emph{redefined} in the classes
\GOT{srbmatrix}{srbmatrix} and \GOT{srsmatrix}{srsmatrix}.
See also \GOT{srmatrix}{srmatrix}.
Example:
\begin{Verbatim}
using namespace cvm;

double a[] = {1., 2., 3., 4., 5., 6., 7., 8., 9.};
srmatrix m (a, 3);

m--;
std::cout << m << std::endl;
std::cout << --m;
\end{Verbatim}
prints
\begin{Verbatim}
0 4 7
2 4 8
3 6 8

-1 4 7
2 3 8
3 6 7
\end{Verbatim}
\newpage



\subsubsection{operator * (TR)}
Operator%
\pdfdest name {srmatrix.operator * (TR)} fit
\begin{verbatim}
srmatrix srmatrix::operator * (TR d) const;
\end{verbatim}
creates an object of type \verb"srmatrix" as a product of
a calling matrix and  number~\verb"d".
Operator is \emph{redefined} in the classes
\GOT{srbmatrix}{srbmatrix} 
and \GOT{srsmatrix}{srsmatrix}.
See also \GOT{srmatrix::ope\-ra\-tor *=~}{srmatrix.operator *= (TR)},
\GOT{srmatrix}{srmatrix}.
Example:
\begin{Verbatim}
using namespace cvm;

double a[] = {1., 2., 3., 4., 5., 6., 7., 8., 9.};
srmatrix m(a, 3);

std::cout << m * 5.;
\end{Verbatim}
prints
\begin{Verbatim}
5 20 35
10 25 40
15 30 45
\end{Verbatim}
\newpage



\subsubsection{operator / (TR)}
Operator%
\pdfdest name {srmatrix.operator / (TR)} fit
\begin{verbatim}
srmatrix srmatrix::operator / (TR d) const throw (cvmexception);
\end{verbatim}
creates an object of type \verb"srmatrix" as a quotient of
a calling matrix and  number~\verb"d". It throws
 \GOT{cvmexception}{cvmexception}
if \verb"d" has  absolute value equal or less than the
\GO{smallest normalized positive number}{Utilities.cvmMachMin}.
)perator is \emph{redefined} in the classes
\GOT{srbmatrix}{srbmatrix} 
and \GOT{srsmatrix}{srsmatrix}.
See also \GOT{srmatrix::operator /=~}{srmatrix.operator /= (TR)},
\GOT{srmatrix}{srmatrix}.
Example:
\begin{Verbatim}
using namespace cvm;

try {
    double a[] = {1., 2., 3., 4., 5., 6., 7., 8., 9.};
    srmatrix m(a, 3);

    std::cout << m / 4.;
}
catch (std::exception& e) {
    std::cout << "Exception " << e.what () << std::endl;
}
\end{Verbatim}
prints
\begin{Verbatim}
0.25 1 1.75
0.5 1.25 2
0.75 1.5 2.25
\end{Verbatim}
\newpage



\subsubsection{operator *= (TR)}
Operator%
\pdfdest name {srmatrix.operator *= (TR)} fit
\begin{verbatim}
srmatrix& srmatrix::operator *= (TR d);
\end{verbatim}
multiplies  calling matrix by  number~\verb"d"
and returns a reference to
the matrix changed.
Operator is \emph{redefined} in the classes
\GOT{srbmatrix}{srbmatrix} 
and \GOT{srsmatrix}{srsmatrix}.
See also \GOT{srmatrix::ope\-ra\-tor *~}{srmatrix.operator * (TR)},
\GOT{srmatrix}{srmatrix}.
Example:
\begin{Verbatim}
using namespace cvm;

double a[] = {1., 2., 3., 4., 5., 6., 7., 8., 9.};
srmatrix m(a, 3);

m *= 2.;
std::cout << m;
\end{Verbatim}
prints
\begin{Verbatim}
2 8 14
4 10 16
6 12 18
\end{Verbatim}
\newpage



\subsubsection{operator /= (TR)}
Operator%
\pdfdest name {srmatrix.operator /= (TR)} fit
\begin{verbatim}
srmatrix& srmatrix::operator /= (TR d) throw (cvmexception);
\end{verbatim}
divides  calling matrix by  number~\verb"d"
and returns a reference to
the matrix changed.
It throws  \GOT{cvmexception}{cvmexception}
if \verb"d" has  absolute value equal or less
than the 
\GO{smallest normalized positive number}{Utilities.cvmMachMin}.
Operator is \emph{redefined} in the classes
\GOT{srbmatrix}{srbmatrix} 
and \GOT{srsmatrix}{srsmatrix}.
See also \GOT{srmatrix::operator /~}{srmatrix.operator / (TR)},
\GOT{srmatrix}{srmatrix}.
Example:
\begin{Verbatim}
using namespace cvm;

try {
    double a[] = {1., 2., 3., 4., 5., 6., 7., 8., 9.};
    srmatrix m(a, 3);

    m /= 2.;
    std::cout << m;
}
catch (std::exception& e) {
    std::cout << "Exception " << e.what () << std::endl;
}
\end{Verbatim}
prints
\begin{Verbatim}
0.5 2 3.5
1 2.5 4
1.5 3 4.5
\end{Verbatim}
\newpage


\subsubsection{normalize}
Function%
\pdfdest name {srmatrix.normalize} fit
\begin{verbatim}
srmatrix& srmatrix::normalize ();
\end{verbatim}
normalizes  calling matrix so its \GO{Euclidean norm}{Array.norm}
becomes equal to $1$ if it was greater than the 
\GO{smallest normalized positive number}{Utilities.cvmMachMin}
before the call (otherwise function does nothing).
Function is \emph{redefined} in the classes
\GOT{srbmatrix}{srbmatrix} 
and \GOT{srsmatrix}{srsmatrix}.
See also \GOT{srmatrix}{srmatrix}.
Example:
\begin{Verbatim}
using namespace cvm;

std::cout.setf (std::ios::scientific | std::ios::left); 
std::cout.precision (3);

double a[] = {1., 2., 3., 4., 5., 6., 7., 8., 9.};
srmatrix m(a, 3);

m.normalize();
std::cout << m << m.norm() << std::endl;
\end{Verbatim}
prints
\begin{Verbatim}
5.923e-02 2.369e-01 4.146e-01
1.185e-01 2.962e-01 4.739e-01
1.777e-01 3.554e-01 5.331e-01
1.000e+00
\end{Verbatim}
\newpage



\subsubsection{transposition}
Operator and functions%
\pdfdest name {srmatrix.transposition} fit
\begin{verbatim}
srmatrix srmatrix::operator ~ () const throw (cvmexception);
srmatrix& srmatrix::transpose (const srmatrix& m) throw (cvmexception);
srmatrix& srmatrix::transpose ();
\end{verbatim}
implement matrix transposition.
First operator creates an object of type \verb"srmatrix" as
 transposed calling matrix
(it throws  
\GOT{cvmexception}{cvmexception}
in case of memory allocation failure). 
Second function sets  calling matrix to be equal to  matrix
\verb"m" transposed
(it throws  
\GOT{cvmexception}{cvmexception}
in case of not appropriate sizes of the operands), 
third one makes it to be equal to
transposed itself. 
Functions are \emph{redefined} in the classes
\GOT{srbmatrix}{srbmatrix} 
and \GOT{srsmatrix}{srsmatrix}.
See also \GOT{srmatrix}{srmatrix}.
Example:
\begin{Verbatim}
using namespace cvm;
try {
    double a[] = {1., 2., 3., 4., 5., 6., 7., 8., 9.};
    srmatrix m(a,3);
    srmatrix mt(3);
    std::cout << ~m << std::endl ;
    mt.transpose(m);
    std::cout << mt << std::endl;
    mt.transpose();
    std::cout << mt;
}
catch (std::exception& e) {
    std::cout << "Exception " << e.what () << std::endl;
}
\end{Verbatim}
prints
\begin{Verbatim}
1 2 3
4 5 6
7 8 9

1 2 3
4 5 6
7 8 9

1 4 7
2 5 8
3 6 9
\end{Verbatim}
\newpage



\subsubsection{operator * (const rvector\&)}
Operator%
\pdfdest name {srmatrix.operator * (const rvector&)} fit
\begin{verbatim}
rvector srmatrix::operator * (const rvector& v) const
throw (cvmexception);
\end{verbatim}
creates an object of type \verb"rvector"
as a product of a calling matrix and a vector \verb"v".
It throws  \GOT{cvmexception}{cvmexception}
if  number of columns of a calling matrix
differs from  size of a vector \verb"v".
Use \GOT{rvector::mult}{rvector.mult (const rmatrix&, const rvector&)}
in order to avoid new object creation.
Function is \emph{redefined} in the classes
\GOT{srbmatrix}{srbmatrix} 
and \GOT{srsmatrix}{srsmatrix}.
See also
\GOT{srmatrix}{srmatrix}, \GOT{rvector}{rvector}.
Example:
\begin{Verbatim}
using namespace cvm;

try {
    srmatrix m(3);
    rvector v(3);
    m.set(1.);
    v.set(1.);

    std::cout << m * v;
}
catch (std::exception& e) {
    std::cout << "Exception " << e.what () << std::endl;
}
\end{Verbatim}
prints
\begin{Verbatim}
3 3 3
\end{Verbatim}
\newpage



\subsubsection{operator * (const rmatrix\&)}
Operator%
\pdfdest name {srmatrix.operator * (const rmatrix&)} fit
\begin{verbatim}
rmatrix srmatrix::operator * (const rmatrix& m) const
throw (cvmexception);
\end{verbatim}
creates an object of type \verb"rmatrix"
as a product of a calling matrix and a matrix \verb"m".
It throws  \GOT{cvmexception}{cvmexception}
if  number of columns of a calling matrix
differs from  number of rows of a matrix \verb"m".
Use \GOT{rmatrix::mult}{rmatrix.mult} in order to avoid
 new object creation.
Operator is \emph{redefined} in the classes
\GOT{srbmatrix}{srbmatrix} 
and \GOT{srsmatrix}{srsmatrix}.
See also
\GOT{rmatrix}{rmatrix}, \GOT{srmatrix}{srmatrix}.
Example:
\begin{Verbatim}
using namespace cvm;

try {
    srmatrix ms(3);
    rmatrix m(3,2);
    ms.set(1.);
    m.set(1.);

    std::cout << ms * m;
}
catch (std::exception& e) {
    std::cout << "Exception " << e.what () << std::endl;
}
\end{Verbatim}
prints
\begin{Verbatim}
3 3
3 3
3 3
\end{Verbatim}
\newpage




\subsubsection{operator * (const srmatrix\&)}
Operator%
\pdfdest name {srmatrix.operator * (const srmatrix&)} fit
\begin{verbatim}
srmatrix srmatrix::operator * (const srmatrix& m) const
throw (cvmexception);
\end{verbatim}
creates an object of type \verb"srmatrix"
as a product of a calling matrix and a matrix \verb"m".
It throws  \GOT{cvmexception}{cvmexception}
if the operands have different sizes.
Use \GOT{rmatrix::mult}{rmatrix.mult} in order to avoid
 new object creation.
Operator is \emph{inherited} in the class
\GOT{srbmatrix}{srbmatrix} 
and \emph{redefined} in \GOT{srsmatrix}{srsmatrix}.
See also
\GOT{srmatrix}{srmatrix}.
Example:
\begin{Verbatim}
using namespace cvm;

try {
    srmatrix m1(3);
    srmatrix m2(3);
    m1.set(1.);
    m2.set(1.);

    std::cout << m1 * m2;
}
catch (std::exception& e) {
    std::cout << "Exception " << e.what () << std::endl;
}
\end{Verbatim}
prints
\begin{Verbatim}
3 3 3
3 3 3
3 3 3
\end{Verbatim}
\newpage



\subsubsection{operator *= (const srmatrix\&)}
Operator%
\pdfdest name {srmatrix.operator *= (const srmatrix&)} fit
\begin{verbatim}
srmatrix& srmatrix::operator *= (const srmatrix& m)
throw (cvmexception);
\end{verbatim}
sets  calling matrix to be equal to  product
of itself and square matrix \verb"m"
and returns a reference to
the object it changes.
Operator throws \GOT{cvmexception}{cvmexception}
in case of different sizes of the operands.
Operator is \emph{inherited} in the class
\GOT{srbmatrix}{srbmatrix} 
and \emph{redefined} in \GOT{srsmatrix}{srsmatrix}.
See also \GOT{srmatrix}{srmatrix}.
Example:
\begin{Verbatim}
using namespace cvm;

try {
    srmatrix m1(3);
    srmatrix m2(3);
    m1.set(1.);
    m2.set(1.);

    m1 *= m2;
    std::cout << m1 << std::endl;
    m1 *= m1;
    std::cout << m1;
}
catch (std::exception& e) {
    std::cout << "Exception " << e.what () << std::endl;
}
\end{Verbatim}
prints
\begin{Verbatim}
3 3 3
3 3 3
3 3 3

27 27 27
27 27 27
27 27 27
\end{Verbatim}
\newpage




\subsubsection{swap\_rows}
Function%
\pdfdest name {srmatrix.swap_rows} fit
\begin{verbatim}
srmatrix& srmatrix::swap_rows (int n1, int n2) throw (cvmexception);
\end{verbatim}
swaps two rows of a calling matrix and returns a reference to
the matrix changed. \verb"n1" and \verb"n2" are
numbers of rows to be swapped, 
both are \Based).
Function throws  \GOT{cvmexception}{cvmexception}
if one of the parameters is outside of the range
\verb"[1,msize()]".
Function is 
\emph{not applicable} to objects of the classes
\GOT{srbmatrix}{srbmatrix} and
\GOT{srsmatrix}{srsmatrix} (i.e. \GOT{cvmexception}{cvmexception} would be thrown
in case of using it for objects of those classes).
See also \GOT{srmatrix}{srmatrix}.
Example:
\begin{Verbatim}
using namespace cvm;

try {
    double a[] = {1., 2., 3., 4., 5., 6., 7., 8., 9.};
    srmatrix m (a, 3);

    std::cout << m << std::endl;
    std::cout << m.swap_rows(2,3);
}
catch (std::exception& e) {
    std::cout << "Exception " << e.what () << std::endl;
}
\end{Verbatim}
prints
\begin{Verbatim}
1 4 7
2 5 8
3 6 9

1 4 7
3 6 9
2 5 8
\end{Verbatim}
\newpage


\subsubsection{swap\_cols}
Function%
\pdfdest name {srmatrix.swap_cols} fit
\begin{verbatim}
srmatrix& srmatrix::swap_cols (int n1, int n2) throw (cvmexception);
\end{verbatim}
swaps two columns of a calling matrix and returns a reference to
the matrix changed. \verb"n1" and \verb"n2" are
 numbers of columns to be swapped, 
both are \Based).
Function throws  \GOT{cvmexception}{cvmexception}
if one of the parameters is outside of the range
\verb"[1,nsize()]".
Function is 
\emph{not applicable} to objects of the classes
\GOT{srbmatrix}{srbmatrix} and
\GOT{srsmatrix}{srsmatrix} (i.e. \GOT{cvmexception}{cvmexception} would be thrown
in case of using it for objects of those classes).
See also \GOT{srmatrix}{srmatrix}.
Example:
\begin{Verbatim}
using namespace cvm;

try {
    double a[] = {1., 2., 3., 4., 5., 6., 7., 8., 9.};
    srmatrix m (a, 3);

    std::cout << m << std::endl;
    std::cout << m.swap_cols(2,3);
}
catch (std::exception& e) {
    std::cout << "Exception " << e.what () << std::endl;
}
\end{Verbatim}
prints
\begin{Verbatim}
1 4 7
2 5 8
3 6 9

1 7 4
2 8 5
3 9 6
\end{Verbatim}
\newpage





\subsubsection{solve}
Functions%
\pdfdest name {srmatrix.solve} fit
\begin{verbatim}
rvector srmatrix::solve (const rvector& vB) const throw (cvmexception);
rmatrix srmatrix::solve (const rmatrix& mB) const throw (cvmexception);
rvector srmatrix::solve (const rvector& vB, TR& dErr) const
throw (cvmexception);
rmatrix srmatrix::solve (const rmatrix& mB, TR& dErr) const
throw (cvmexception);
\end{verbatim}
return  solution of  linear equation of kind $A*x=b$ or $A*X=B$
where $A$ is  calling matrix. The first and third versions
solve equation $A*x=b$ where vector $b$ is passed in 
 parameter \verb"vB"
and the second and fourth versions
solve equation $A*X=B$ where matrix $B$ is passed 
in parameter \verb"mB".
The last two versions also set output parameter \verb"dErr" to be equal
to the norm of computation error.
Functions throw \GOT{cvmexception}{cvmexception}
in case of inappropriate sizes of the operands.
Function is \emph{inherited} in the classes
\GOT{srbmatrix}{srbmatrix}
and \GOT{srsmatrix}{srsmatrix}.
See also
\GOT{rvector::solve}{rvector.solve},
\GOT{rmatrix::solve}{rmatrix.solve},
\GOT{rmatrix::solve\_tran}{rmatrix.solvetran},
\GOT{srmatrix::solve\_tran}{srmatrix.solvetran},
\GOT{srmatrix}{srmatrix}.
Example:
\begin{Verbatim}
using namespace cvm;

std::cout.setf (std::ios::scientific | std::ios::left); 
std::cout.precision (3);
try {
    double a[] = {1., 2., 3., 4., 5., 6., 7., 8., 10.};
    srmatrix ma(a, 3);
    rmatrix  mb(3,4);
    rmatrix  mx(3,4);
    double dErr;

    mb(1).set(1.);
    mb(2).set(2.);
    mb(3).set(3.);
    mb(1,4) = 1.; mb(2,4) = 2.; mb(3,4) = 3.;

    mx = ma.solve (mb, dErr);
    std::cout << mx << dErr 
              << std::endl << ma * mx - mb << std::endl;

    rvector vb(3), vx(3);
    vb = mb(2);
    vx = ma.solve (vb, dErr);
    std::cout << vx << dErr << std::endl << ma * vx - vb;
}
catch (std::exception& e) {
    std::cout << "Exception " << e.what () << std::endl;
}
\end{Verbatim}
prints
\begin{Verbatim}
-3.333e-01 -6.667e-01 -1.000e+00 1.000e+00
3.333e-01 6.667e-01 1.000e+00 0.000e+00
6.661e-16 1.332e-15 0.000e+00 0.000e+00
3.301e-14
0.000e+00 0.000e+00 0.000e+00 0.000e+00
-1.110e-16 -2.220e-16 0.000e+00 0.000e+00
2.220e-16 4.441e-16 0.000e+00 0.000e+00

-6.667e-01 6.667e-01 1.332e-15
3.301e-14
0.000e+00 -2.220e-16 4.441e-16
\end{Verbatim}
\newpage







\subsubsection{solve\_tran}
Functions%
\pdfdest name {srmatrix.solvetran} fit
\begin{verbatim}
rvector srmatrix::solve_tran (const rvector& vB)
const throw (cvmexception);
rmatrix srmatrix::solve_tran (const rmatrix& mB)
const throw (cvmexception);
rvector srmatrix::solve_tran (const rvector& vB, TR& dErr) const
throw (cvmexception);
rmatrix srmatrix::solve_tran (const rmatrix& mB, TR& dErr) const
throw (cvmexception);
\end{verbatim}
return  solution of  linear equation of kind $A^T*x=b$ or $A^T*X=B$
where $A$ is  calling matrix. The first and third versions
solve  equation $A^T*x=b$ where vector $b$ is passed in 
 parameter \verb"vB"
and the second and fourth versions
solve equation $A^T*X=B$ where matrix $B$ is passed 
in parameter \verb"mB".
The last two versions also set output parameter \verb"dErr" to be equal
to the norm of computation error.
Functions throw \GOT{cvmexception}{cvmexception}
in case of inappropriate sizes of the operands.
Function is \emph{inherited} in the classes
\GOT{srbmatrix}{srbmatrix}
and \GOT{srsmatrix}{srsmatrix}.
See also
\GOT{rvector::solve}{rvector.solve},
\GOT{rmatrix::solve}{rmatrix.solve},
\GOT{srmatrix::solve}{srmatrix.solve},
\GOT{srmatrix}{srmatrix}.
Example:
\begin{Verbatim}
using namespace cvm;

std::cout.setf (std::ios::scientific | std::ios::showpos); 
std::cout.precision (3);
try {
    double a[] = {1., 2., 3., 4., 5., 6., 7., 8., 10.};
    srmatrix ma(a, 3);
    rmatrix  mb(3,4);
    rmatrix  mx(3,4);
    double dErr;

    mb(1).set(1.);
    mb(2).set(2.);
    mb(3).set(3.);
    mb(1,4) = 1.; mb(2,4) = 2.; mb(3,4) = 3.;

    mx = ma.solve_tran (mb, dErr);
    std::cout << mx << dErr 
              << std::endl << ~ma * mx - mb << std::endl;

    rvector vb(3), vx(3);
    vb = mb(2);
    vx = ma.solve_tran (vb, dErr);
    std::cout << vx << dErr << std::endl << vx * ma - vb;
}
catch (std::exception& e) {
    std::cout << "Exception " << e.what () << std::endl;
}
\end{Verbatim}
prints
\begin{Verbatim}
-1.000e+000 -2.000e+000 -3.000e+000 -3.333e-001
+1.000e+000 +2.000e+000 +3.000e+000 +6.667e-001
+0.000e+000 +0.000e+000 -1.332e-015 +0.000e+000
+3.513e-014
+0.000e+000 +0.000e+000 +0.000e+000 +0.000e+000
+0.000e+000 +0.000e+000 +8.882e-016 -2.220e-016
+0.000e+000 +0.000e+000 -2.665e-015 +0.000e+000

-2.000e+000 +2.000e+000 +0.000e+000
+3.168e-014
+0.000e+000 +0.000e+000 +0.000e+000
\end{Verbatim}
\newpage



\subsubsection{operator \% (const rvector\&)}
Operator%
\pdfdest name {srmatrix.operator percent (rvector)} fit
\begin{verbatim}
rvector operator % (const rvector& vB) const throw (cvmexception);
\end{verbatim}
returns solution $x$ of  linear equation
$A^T*x=b$ (which is equivalent to $x*A=b$) 
where calling matrix is square matrix $A$
and a vector $b$ is passed in parameter \verb"vB".
This operator throws exception 
of type \GOT{cvmexception}{cvmexception}
in case of inappropriate sizes
of the objects or when  matrix $A$ is close to singular.
See also \GOT{rvector::solve\_tran}{rvector.solvetran}, 
\GOT{srmatrix::solve\_tran}{srmatrix.solvetran}, 
\GOT{rvector.operator~/}{rvector.operator / (srmatrix)}, 
\GOT{rvector}{rvector}, \GOT{srmatrix}{srmatrix}.
Example:
\begin{Verbatim}
using namespace cvm;
std::cout.setf (std::ios::scientific | std::ios::showpos);
std::cout.precision (12);

try {
    double m[] = {1., -1., 1., 2., -2., 1., 3., -2., 1.};
    double b[] = {1., 2., 3.};
    srmatrix ma(m, 3);
    rvector  vb(b, 3);
    rvector  vx(3);

    vx = ma % vb;

    std::cout << vx * ma - vb;
}
catch (std::exception& e) {
    std::cout << "Exception " << e.what () << std::endl;
}
\end{Verbatim}
prints
\begin{Verbatim}
+0.000000000000e+000 +0.000000000000e+000 +0.000000000000e+000
\end{Verbatim}
\newpage





\subsubsection{operator / (const rvector\&)}
Operator%
\pdfdest name {srmatrix.operator / (rvector)} fit
\begin{verbatim}
rvector operator / (const rvector& vB) const throw (cvmexception);
\end{verbatim}
returns solution $x$ of linear equation
$A*x=b$ where calling matrix is square matrix $A$
and a vector $b$ is passed in parameter \verb"vB".
This operator throws exception 
of type \GOT{cvmexception}{cvmexception}
in case of inappropriate sizes
of the objects or when  matrix $A$ is close to singular.
It's \emph{redefined} in classes \GOT{srbmatrix}{srbmatrix} and \GOT{srsmatrix}{srsmatrix}.
See also \GOT{rvector::solve}{rvector.solve}, 
\GOT{srmatrix::solve}{srmatrix.solve}, 
\GOT{rvector.operator~\%}{rvector.operator percent (srmatrix)}, 
\GOT{rvector}{rvector}, \GOT{srmatrix}{srmatrix}.
Example:
\begin{Verbatim}
using namespace cvm;
std::cout.setf (std::ios::scientific | std::ios::showpos);
std::cout.precision (12);

try {
    double m[] = {1., -1., 1., 2., -2., 1., 3., -2., 1.};
    double b[] = {1., 2., 3.};
    srmatrix ma(m, 3);
    rvector  vb(b, 3);
    rvector  vx(3);

    vx = ma / vb;

    std::cout << ma * vx - vb;
}
catch (std::exception& e) {
    std::cout << "Exception " << e.what () << std::endl;
}
\end{Verbatim}
prints
\begin{Verbatim}
+0.000000000000e+000 +0.000000000000e+000 +0.000000000000e+000
\end{Verbatim}
\newpage




\subsubsection{solve\_lu}
Functions%
\pdfdest name {srmatrix.solvelu} fit
\begin{verbatim}
rvector
srmatrix::solve_lu (const srmatrix& mLU, const int* pPivots,
                    const rvector& vB, TR& dErr) throw (cvmexception);
rvector
srmatrix::solve_lu (const srmatrix& mLU, const int* pPivots,
                    const rvector& vB) throw (cvmexception);

rmatrix
srmatrix::solve_lu (const srmatrix& mLU, const int* pPivots,
                    const rmatrix& mB, TR& dErr) throw (cvmexception);
rmatrix
srmatrix::solve_lu (const srmatrix& mLU, const int* pPivots,
                    const rmatrix& mB) throw (cvmexception);
\end{verbatim}
create  object of type \verb"rvector" or \verb"rmatrix" as
 solution $x$ or $X$ of a matrix linear equation
$A*x=b$ or $A*X=B$ respectively. Here $A$ is  calling
matrix, 
parameter \verb"mLU" is \GO{$LU$ factorization}{srmatrix.low_up}
of a matrix $A$, parameter \verb"pPivots" is an array of pivot numbers
created while factorizing the matrix $A$
and parameters \verb"vB" and \verb"mB" are  vector $b$ and
matrix $B$ respectively.
The first and third version also set output parameter \verb"dErr" to be equal
to the norm of computation error.
These functions are useful when you need to solve few linear equations
of kind $A*x=b$ or $A*X=B$ with the same matrix $A$ and different vectors $b$ 
or matrices $B$.
In such case you save on matrix $A$ factorization since it's needed to be
performed just one time.
Functions throw \GOT{cvmexception}{cvmexception}
in case of inappropriate sizes
of the operands or when the matrix $A$ is close to singular.
Function is \emph{inherited} in the classes
\GOT{srbmatrix}{srbmatrix}
and \GOT{srsmatrix}{srsmatrix}.
See also
\GOT{rvector::solve}{rvector.solve},
\GOT{rmatrix}{rmatrix},
\GOT{srmatrix}{srmatrix}.
Example:
\begin{Verbatim}
using namespace cvm;

std::cout.setf (std::ios::scientific | std::ios::left); 
std::cout.precision (3);
try {
    double a[] = {1., -1., 1., 2., -2., 1., 3., -2., 1.};
    srmatrix ma(a,3);
    srmatrix mLU(3);
    rmatrix  mb1(3,2); rvector vb1(3);
    rmatrix  mb2(3,2); rvector vb2(3);
    rmatrix  mx1(3,2); rvector vx1(3);
    rmatrix  mx2(3,2); rvector vx2(3);
    iarray   nPivots(3);
    double   dErr = 0.;
    mb1.randomize(-1.,3.); vb1.randomize(-2.,4.);
    mb2.randomize(-2.,5.); vb2.randomize(-3.,1.);

    mLU.low_up(ma, nPivots);
    mx1 = ma.solve_lu (mLU, nPivots, mb1, dErr);
    std::cout << mx1 << dErr << std::endl;
    mx2 = ma.solve_lu (mLU, nPivots, mb2);
    std::cout << mx2 << std::endl;;
    std::cout << ma * mx1 - mb1 << std::endl << ma * mx2 - mb2;

    vx1 = ma.solve_lu (mLU, nPivots, vb1, dErr);
    std::cout << vx1 << dErr << std::endl;
    vx2 = ma.solve_lu (mLU, nPivots, vb2);
    std::cout << vx2 << std::endl;;
    std::cout << ma * vx1 - vb1 << std::endl << ma * vx2 - vb2;
}
catch (std::exception& e) {
    std::cout << "Exception " << e.what () << std::endl;
}
\end{Verbatim}
prints
\begin{Verbatim}
2.807e+00 1.107e+00
-3.651e-01 -4.843e+00
-5.412e-01 3.095e+00
6.438e-15
-7.639e-01 1.082e+01
-2.869e-01 -1.110e+01
4.890e-01 3.443e+00

0.000e+00 -4.441e-16
1.110e-16 -4.441e-16
-4.441e-16 4.441e-16

0.000e+00 -4.441e-16
0.000e+00 8.882e-16
0.000e+00 -4.441e-16
-1.651e+00 2.361e-01 -6.384e-02
3.828e-15
-5.886e+00 7.038e+00 -3.125e+00

0.000e+00 0.000e+00 0.000e+00

0.000e+00 0.000e+00 2.220e-16
\end{Verbatim}
\newpage



\subsubsection{det}
Function%
\pdfdest name {srmatrix.det} fit
\begin{verbatim}
TR srmatrix::det () const throw (cvmexception);
\end{verbatim}
returns  determinant of a calling matrix.
It uses the \GO{$LU$ factorization}{srmatrix.low_up} internally
and may throw the same exceptions as the factorizer.
Function is \emph{inherited} in the classes
\GOT{srbmatrix}{srbmatrix}
and \GOT{srsmatrix}{srsmatrix}.
See also
\GOT{srmatrix}{srmatrix}.
Example:
\begin{Verbatim}
using namespace cvm;

std::cout.setf (std::ios::scientific | std::ios::left); 
std::cout.precision (3);
try {
    double a[] = {1., 2., 3., 4., 5., 6., 7., 8., 10.};
    srmatrix m(a, 3);

    std::cout << m << std::endl << m.det() << std::endl;
}
catch (std::exception& e) {
    std::cout << "Exception " << e.what () << std::endl;
}
\end{Verbatim}
prints
\begin{Verbatim}
1.000e+00 4.000e+00 7.000e+00
2.000e+00 5.000e+00 8.000e+00
3.000e+00 6.000e+00 1.000e+01

-3.000e+00
\end{Verbatim}
\newpage



\subsubsection{low\_up}
Functions%
\pdfdest name {srmatrix.low_up} fit
\begin{verbatim}
srmatrix& 
srmatrix::low_up (const srmatrix& m, int* nPivots) throw (cvmexception);
srmatrix
srmatrix::low_up (int* nPivots) const throw (cvmexception);
\end{verbatim}
compute the $LU$ factorization of square matrix $A$ as
\begin{equation*}
A=PLU
\end{equation*}
where $P$ is  permutation matrix, $L$ is  lower
triangular matrix with unit diagonal
elements and $U$ is  upper triangular matrix.
All  functions store  result as the matrix $L$ without
main diagonal combined with $U$. All  functions
return pivot indices as  array of integers
(it should support at least \verb"msize()" elements)
pointed to by \verb"nPivots" so \hbox{$i$-th} row
was interchanged with \hbox{\verb"nPivots["$i$\verb"]"-th} row.
The first version sets  calling matrix to be equal to matrix
\verb"m"'s $LU$ factorization and the second one
creates an object of type \verb"srmatrix" as  calling matrix
$LU$ factorization.
Functions throw \GOT{cvmexception}{cvmexception}
in case of inappropriate
sizes of the operands or when the matrix to be factorized is close to
singular. It is recommended to use \GOT{iarray}{iarray}
for pivot values.
Function is \emph{redefined} in the class
\GOT{srbmatrix}{srbmatrix}
and \emph{inherited} in \GOT{srsmatrix}{srsmatrix}.
See also
\GOT{srmatrix}{srmatrix}.
Example:
\begin{Verbatim}
using namespace cvm;

std::cout.setf (std::ios::scientific | std::ios::left); 
std::cout.precision (3);
try {
    double a[] = {1., 2., 3., 4., 5., 6., 7., 8., 10.};
    srmatrix m(a, 3);
    srmatrix mLU(3), mLo(3), mUp(3);
    iarray naPivots(3);

    mLU.low_up (m, naPivots);

    mLo.identity ();

    mLo.diag(-2) = mLU.diag(-2);
    mLo.diag(-1) = mLU.diag(-1);
    mUp.diag(0) = mLU.diag(0);
    mUp.diag(1) = mLU.diag(1);
    mUp.diag(2) = mLU.diag(2);

    std::cout << mLo << std::endl << mUp 
              << std::endl << naPivots << std::endl;

    mLU = mLo * mUp;
    for (int i = 3; i >= 1; i--) {
        mLU.swap_rows (i, naPivots[i]);
    }
    std::cout << mLU;
}
catch (std::exception& e) {
    std::cout << "Exception " << e.what () << std::endl;
}
\end{Verbatim}
prints
\begin{Verbatim}
1.000e+00 0.000e+00 0.000e+00
3.333e-01 1.000e+00 0.000e+00
6.667e-01 5.000e-01 1.000e+00

3.000e+00 6.000e+00 1.000e+01
0.000e+00 2.000e+00 3.667e+00
0.000e+00 0.000e+00 -5.000e-01

3 3 3

1.000e+00 4.000e+00 7.000e+00
2.000e+00 5.000e+00 8.000e+00
3.000e+00 6.000e+00 1.000e+01
\end{Verbatim}
\newpage





\subsubsection{cond}
Function%
\pdfdest name {srmatrix.cond} fit
\begin{verbatim}
TR srmatrix::cond () const throw (cvmexception);
\end{verbatim}
returns  condition number reciprocal of 
a calling matrix $A$ in the \GO{infinity-norm}{Array.norminf}:
\begin{equation*}
\kappa_\infty=\|A\|_\infty \|A^{-1}\|_\infty.
\end{equation*}
Less value returned means that matrix $A$ is closer to singular.
Zero value returned means estimation underflow or
that matrix $A$ is singular.
The condition number is used for error analysis
of systems of linear equations.
Function throws \GOT{cvmexception}{cvmexception} in case of LAPACK
subroutines failure.
Function is \emph{inherited} in the classes
\GOT{srbmatrix}{srbmatrix}
and \GOT{srsmatrix}{srsmatrix}.
See also
\GOT{srmatrix::solve}{srmatrix.solve}, \GOT{srmatrix}{srmatrix}.
Example:
\begin{Verbatim}
using namespace cvm;

std::cout.setf (std::ios::scientific | std::ios::left); 
std::cout.precision (3);
try {
    double a[] = {1., 2., 3., 4., 5., 6., 7., 8., 9.};
    srmatrix m(a, 3);
    std::cout << m.cond() << std::endl 
              << m.det() << std::endl << std::endl;
    m(3,3) = 10.;
    std::cout << m.cond() << std::endl << m.det() << std::endl;
}
catch (std::exception& e) {
    std::cout << "Exception " << e.what () << std::endl;
}
\end{Verbatim}
prints
\begin{Verbatim}
0.000e+00
0.000e+00

7.519e-03
-3.000e+00
\end{Verbatim}
\newpage



\subsubsection{inv}
Functions%
\pdfdest name {srmatrix.inv} fit
\begin{verbatim}
srmatrix& srmatrix::inv (const srmatrix& m) throw (cvmexception);
srmatrix srmatrix::inv () const throw (cvmexception);
\end{verbatim}
implement matrix inversion.
The first version sets  calling matrix to be equal to 
\verb"m" inverted and the second one
creates an object of type \verb"srmatrix" as inverted calling matrix.
Functions throw  \GOT{cvmexception}{cvmexception} 
in case of inappropriate
sizes of the operands or when the matrix to be inverted is close to
singular.
Function is \emph{redefined} in the class
\GOT{srsmatrix}{srsmatrix}
and \emph{inherited} in \GOT{srbmatrix}{srbmatrix}.
See also
\GOT{srmatrix}{srmatrix}.
Example:
\begin{Verbatim}
using namespace cvm;

std::cout.setf (std::ios::scientific | std::ios::left); 
std::cout.precision (10);
try {
    double a[] = {1., 2., 3., 4., 5., 6., 7., 8., 10.};
    srmatrix m(a, 3);
    srmatrix mi(3);
    mi.inv (m);
    std::cout << mi << std::endl << mi * m - eye_real(3);
    std::cout << std::endl << mi.inv() * mi - eye_real(3);
}
catch (std::exception& e) {
    std::cout << "Exception " << e.what () << std::endl;
}
\end{Verbatim}
prints
\begin{Verbatim}
-6.6666666667e-01 -6.6666666667e-01 1.0000000000e+00
-1.3333333333e+00 3.6666666667e+00 -2.0000000000e+00
1.0000000000e+00 -2.0000000000e+00 1.0000000000e+00

0.0000000000e+00 0.0000000000e+00 1.7763568394e-15
1.7763568394e-15 3.5527136788e-15 0.0000000000e+00
0.0000000000e+00 0.0000000000e+00 1.7763568394e-15

0.0000000000e+00 1.7763568394e-15 -1.7763568394e-15
-8.8817841970e-16 3.5527136788e-15 -3.5527136788e-15
0.0000000000e+00 0.0000000000e+00 -1.7763568394e-15
\end{Verbatim}
\newpage





\subsubsection{exp}
Functions%
\pdfdest name {srmatrix.exp} fit
\begin{verbatim}
srmatrix& srmatrix::exp (const srmatrix& m, TR tol = cvmMachSp ())
throw (cvmexception);

srmatrix srmatrix::exp (TR tol = cvmMachSp ()) const
throw (cvmexception);
\end{verbatim}
compute exponent of square matrix using Pad\'e approximation
defined as
\begin{equation*}
R_{pq}(z)=D_{pq}(z)^{-1}N_{pq}(z)=1+z+\dots+z^p/p!\,,
\end{equation*}
where
\begin{align*}
N_{pq}(z)&=\sum_{k=0}^p\frac{(p+q-k)!p!}{(p+q)!k!(q-k)!}z^k,\\
D_{pq}(z)&=\sum_{k=0}^q\frac{(p+q-k)!p!}{(p+q)!k!(q-k)!}(-z)^k
\end{align*}
along with the matrix normalizing as described in
\GO{\cite{Golub}}{biblio}, p.~572.
Functions use \verb"DMEXP" (or \verb"SMEXP" for \verb"float" version)
\FORTRAN subroutine implementing the algorithm.
The first version sets calling matrix to be equal to
exponent of \verb"m" and returns a reference to the matrix
changed. The second version
creates an object of type \verb"srmatrix" as 
exponent of a calling matrix.
The algorithm uses parameter \verb"tol"
as $\varepsilon(p,q)$ in order to choose constants $p$ and $q$
so that
\begin{equation*}
\varepsilon(p,q)\ge 2^{3-(p+q)}\frac{p!q!}{(p+q)!(p+q+1)!}.
\end{equation*}
This parameter is equal to the
\GO{largest relative spacing}{Utilities.cvmMachSp} by default.
Functions throw  \GOT{cvmexception}{cvmexception} 
in case of
inappropriate sizes of the operands or when LAPACK subroutine fails.
Functions are \emph{inherited} in the classes
\GOT{srbmatrix}{srbmatrix}
and \GOT{srsmatrix}{srsmatrix}. The second version
is \emph{redefined} in \GOT{srbmatrix}{srbmatrix}.
See also
\GOT{srmatrix}{srmatrix}.
Example (see \GO{\cite{Golub}}{biblio}, p.~567,
example 11.2.2):
\begin{Verbatim}
using namespace cvm;

std::cout.setf (std::ios::scientific | std::ios::left); 
std::cout.precision (15);
try {
    srmatrix m(2);
    m(1,1) = -49.;
    m(1,2) = 24.;
    m(2,1) = -64.;
    m(2,2) = 31.;

    std::cout << m << std::endl << m.exp();
}
catch (std::exception& e) {
    std::cout << "Exception " << e.what () << std::endl;
}
\end{Verbatim}
prints
\begin{Verbatim}
-4.900000000000000e+01 2.400000000000000e+01
-6.400000000000000e+01 3.100000000000000e+01

-7.357587581448284e-01 5.518190996581556e-01
-1.471517599088415e+00 1.103638240715692e+00
\end{Verbatim}
Matlab output:
\begin{Verbatim}
-7.357587581446907e-001    5.518190996580505e-001
-1.471517599088136e+000    1.103638240715478e+000
\end{Verbatim}
\newpage





\subsubsection{polynomial}
Functions%
\pdfdest name {srmatrix.polynom} fit
\begin{verbatim}
srmatrix& srmatrix::polynom (const srmatrix& m, const rvector& v)
throw (cvmexception);

srmatrix srmatrix::polynom (const rvector& v) const
throw (cvmexception);
\end{verbatim}
compute  matrix polynomial defined as
\begin{equation*}
p(A)=b_0I+b_1A+\dots+b_qA^q
\end{equation*}
using the Horner's rule:
\begin{equation*}
p(A)=\sum_{k=0}^r B_k(A^s)^k,\quad s=\floor(\!\sqrt{q}\,),\ r=\floor(q/s)
\end{equation*}
where
\begin{equation*}
B_k=\begin{cases}
\sum\limits_{i=0}^{s-1}b_{sk+i} A^i, & k=0,1,\dots,r-1\\
\sum\limits_{i=0}^{q-sr}b_{sr+i} A^i, & k=r.
\end{cases}
\end{equation*}
See also \GO{\cite{Golub}}{biblio}, p.~568.
The coefficients $b_0,b_1,\dots,b_q$ are passed in the parameter 
\verb"v",
where $q$ is equal to \verb"v.size()-1", so the functions
compute matrix polynomial equal to
\begin{equation*}
\mathtt{v[1]*I + v[2]*m +\cdots + v[v.size()]*m^{v.size()-1}}
\end{equation*}
The first version sets  calling matrix to be equal to the polynomial of
\verb"m"
and the second one
creates an object of type \verb"srmatrix" as the 
polynomial of a calling matrix.
Functions use \verb"DPOLY" (or \verb"SPOLY" for \verb"float" version)
\FORTRAN subroutine implementing the Horner's algorithm.
Functions throw  \GOT{cvmexception}{cvmexception} 
in case of inappropriate sizes of the operands.
Functions are \emph{inherited} in the class
\GOT{srbmatrix}{srbmatrix}
and \emph{redefined} in \GOT{srsmatrix}{srsmatrix}.
See also
\GOT{srmatrix}{srmatrix}.
Example:
\begin{Verbatim}
using namespace cvm;

std::cout.setf (std::ios::scientific | std::ios::left); 
std::cout.precision (15);
try {
    double a[] = {2.2, 1.3, 1.1, -0.9, 0.2,
                 -0.45, 45, -30, 10, 3, 3.2};
    const rvector v(a, 11);
    srmatrix m(2), mp(2);
    m(1,1) = 1.;
    m(1,2) = 0.5;
    m(2,1) = -1.;
    m(2,2) = 0.3;

    mp.polynom (m, v);
    std::cout << mp;
}
catch (std::exception& e) {
    std::cout << "Exception " << e.what () << std::endl;
}
\end{Verbatim}
prints
\begin{Verbatim}
-7.963641665999998e+00 -7.551532476200001e+00
1.510306495240000e+01 2.608503800680002e+00
\end{Verbatim}
Matlab output:
\begin{Verbatim}
-7.963641665999999e+000   -7.551532476200002e+000
 1.510306495240000e+001    2.608503800680002e+000
\end{Verbatim}
\newpage







\subsubsection{eig}
Functions%
\pdfdest name {srmatrix.eig} fit
\begin{verbatim}
cvector srmatrix::eig (scmatrix& mEigVect, bool bRightVect = true) const
throw (cvmexception);

cvector srmatrix::eig () const throw (cvmexception);
\end{verbatim}
solve the
\GO{nonsymmetric eigenvalue problem}{cvector.eig}
and return  complex vector with eigenvalues
of a calling matrix.
The first version
sets  output parameter \verb"mEigVect" to be equal
to  square matrix containing right (if parameter
\verb"bRightVect" is true, which is default value)
or left (if parameter
\verb"bRightVect" is false)
eigenvectors as columns.
All  functions
throw 
 \GOT{cvmexception}{cvmexception} 
in case of inappropriate sizes of the operands
or in case of convergence error.
Functions are \emph{inherited} in the class
\GOT{srbmatrix}{srbmatrix}
and \emph{redefined} in \GOT{srsmatrix}{srsmatrix}.
See also
\GOT{cvector}{cvector},
\GOT{scmatrix}{scmatrix} and
\GOT{srmatrix}{srmatrix}.
Example:
\begin{Verbatim}
using namespace cvm;
try {
    scmatrix m(3), me(3);
    cvector vl(3);

    m(1,1) = 0.1;  m(1,2) = 0.2;  m(1,3) = 0.1;
    m(2,1) = 0.11; m(2,2) = -2.9; m(2,3) = -8.4;
    m(3,1) = 0.;   m(3,2) = 2.91; m(3,3) = 8.2;

    vl = m.eig (me);
    std::cout << vl;

    m(2,2) = 2.9;
    vl = m.eig (me);
    std::cout << vl << std::endl;

    std::cout.setf (std::ios::scientific | std::ios::showpos);
    std::cout.precision (1);

    std::cout << m * me(1) - me(1) * vl(1);
    std::cout << m * me(2) - me(2) * vl(2);
    std::cout << m * me(3) - me(3) * vl(3);
}
catch (std::exception& e) {
    std::cout << "Exception: " << e.what () << std::endl;
}
\end{Verbatim}
prints
\begin{Verbatim}
(-0.0555784,0) (0.285327,0) (5.17025,0)
(0.0968985,-1.38778e-017) (5.55155,4.1733) (5.55155,-4.1733)

(+1.4e-017,+2.8e-017) (-1.0e-016,-5.4e-017) (+3.9e-017,+4.3e-017)
(-5.0e-016,+1.7e-016) (-7.1e-015,+2.9e-015) (+3.6e-015,+2.7e-015)
(-3.1e-016,+2.8e-017) (-2.7e-015,-8.9e-016) (+8.9e-016,+3.6e-015)
\end{Verbatim}
\newpage



\subsubsection{Cholesky}
Function%
\pdfdest name {srmatrix.cholesky} fit
\begin{verbatim}
srmatrix& srmatrix::cholesky (const srsmatrix& m)
throw (cvmexception);
\end{verbatim}
forms the Cholesky factorization of  symmetric positive-definite 
matrix $A$ defined as
\begin{equation*}
A=U^T U,
\end{equation*}
where $U$ is upper triangular matrix.
It utilizes one of \verb"?POTRF" routines of the
\URL{LAPACK library}{http://www.netlib.org/lapack}.
Function
sets  calling matrix to be equal to the factorization
of  symmetric 
positive-definite matrix \verb"m".
Function throws 
 \GOT{cvmexception}{cvmexception} 
in case of inappropriate sizes of the operands
or in case of convergence error.
See also
\GOT{srmatrix}{srmatrix} and
\GOT{srsmatrix}{srsmatrix}.
Example:
\begin{Verbatim}
using namespace cvm;

try {
    double a[] = {1., 2., 1., 2., 5., -1., 1., -1., 20.};
    const srsmatrix m(a, 3);
    srmatrix h(3);

    h.cholesky(m);
    std::cout << h << std::endl;
    std::cout << ~h * h - m;
}
catch (std::exception& e) {
    std::cout << "Exception: " << e.what () << std::endl;
}
\end{Verbatim}
prints
\begin{Verbatim}
1 2 1
0 1 -3
0 0 3.16228

0 0 0
0 0 0
0 0 0
\end{Verbatim}
\newpage




\subsubsection{Bunch-Kaufman}
Function%
\pdfdest name {srmatrix.bunch_kaufman} fit
\begin{verbatim}
srmatrix& srmatrix::bunch_kaufman (const srsmatrix& m, int* pivots)
throw (cvmexception);
\end{verbatim}
forms the Bunch-Kaufman factorization of  symmetric matrix 
(cited from the MKL library documentation):
\begin{equation*}
A=PUDU^TP^T,
\end{equation*}
where $A$ is the input matrix passed in parameter \verb"m",
$P$ is a permutation matrix, $U$ and $L$ are upper and lower triangular
matrices with unit diagonal, and $D$ is a symmetric 
block-diagonal matrix with $1$-by-$1$ and $2$-by-$2$
diagonal blocks. $U$ and $L$ have $2$-by-$2$ unit diagonal 
blocks corresponding to the $2$-by-$2$ blocks of $D$.
It utilizes one of \verb"?SYTRF" routines of the
\URL{LAPACK library}{http://www.netlib.org/lapack}.
Function
sets  calling matrix to be equal to the factorization
of  symmetric 
positive-definite matrix \verb"m".
Function throws 
 \GOT{cvmexception}{cvmexception} 
in case of inappropriate sizes of the operands
or in case of convergence error.
See also
\GOT{srmatrix}{srmatrix} and
\GOT{srsmatrix}{srsmatrix}.
Function is mostly designed to be used for subsequent calls
of \verb"?SYTRS", \verb"?SYCON" and \verb"?SYTRI" routines of the
\URL{LAPACK library}{http://www.netlib.org/lapack}.
Currently it's used internally in \GOT{srmatrix::det}{srmatrix.det} flow
when argument is symmetric but not positive-definite.
\newpage


\subsubsection{qr}
Function%
\pdfdest name {srmatrix.qr} fit
\begin{verbatim}
void srmatrix::qr (srmatrix& mQ, srmatrix& mR) const throw (cvmexception);
\end{verbatim}
computes QR factorization as
\begin{equation*}
M=QR
\end{equation*}
where 
$M$ is  calling square matrix, orthogonal matrix $Q$ 
and upper triangular matrix $R$ are \verb"mQ" and \verb"mR"
respectively. 
Function throws
 \GOT{cvmexception}{cvmexception}
in case if inappropriate sizes of the operands passed.
See also \GOT{rmatrix}{rmatrix}, \GOT{rmatrix::qr}{rmatrix.qr},
\GOT{srmatrix}{srmatrix}.
Example:
\begin{Verbatim}
using namespace cvm;

treal a[] = {1., 2., 3., 4., 5., 6., 7., 8., 9.};
const cvm::srmatrix m(a, 3);
cvm::srmatrix q(3), r(3);

m.qr(q,r);
std::cout << (eye_real(3) - ~q * q).norm()
          << " " << (m - q * r).norm() << std::endl;
\end{Verbatim}
prints
\begin{Verbatim}
+5.2889959e-16 +7.0854500e-15
\end{Verbatim}
\newpage



\subsubsection{lq}
Function%
\pdfdest name {srmatrix.lq} fit
\begin{verbatim}
void srmatrix::lq (srmatrix& mL, srmatrix& mQ) const throw (cvmexception);
\end{verbatim}
computes LQ factorization as
\begin{equation*}
M=LQ
\end{equation*}
where 
$M$ is  calling square matrix, lower triangular matrix $L$ 
and orthogonal matrix $Q$ are \verb"mL" and \verb"mQ"
respectively. 
Function throws
 \GOT{cvmexception}{cvmexception}
in case if inappropriate sizes of the operands passed.
See also \GOT{rmatrix}{rmatrix}, \GOT{rmatrix::lq}{rmatrix.lq},
\GOT{srmatrix}{srmatrix}.
Example:
\begin{Verbatim}
using namespace cvm;

treal a[] = {1., 2., 3., 4., 5., 6., 7., 8., 9.};
const cvm::srmatrix m(a, 3);
cvm::srmatrix l(3), q(3);

m.lq(l,q);
std::cout << (eye_real(3) - ~q * q).norm()
          << " " << (m - l * q).norm() << std::endl;
\end{Verbatim}
prints
\begin{Verbatim}
+7.3329369e-016 +8.1523942e-015
\end{Verbatim}
\newpage


\subsubsection{ql}
Function%
\pdfdest name {srmatrix.ql} fit
\begin{verbatim}
void srmatrix::ql (srmatrix& mQ, srmatrix& mL) const throw (cvmexception);
\end{verbatim}
computes QL factorization as
\begin{equation*}
M=QL
\end{equation*}
where 
$M$ is  calling square matrix, orthogonal matrix $Q$ 
and lower triangular matrix $L$ are \verb"mQ" and \verb"mL"
respectively. 
Function throws
 \GOT{cvmexception}{cvmexception}
in case if inappropriate sizes of the operands passed.
See also \GOT{rmatrix}{rmatrix}, \GOT{rmatrix::ql}{rmatrix.ql},
\GOT{srmatrix}{srmatrix}.
Example:
\begin{Verbatim}
using namespace cvm;

treal a[] = {1., 2., 3., 4., 5., 6., 7., 8., 9.};
const cvm::srmatrix m(a, 3);
cvm::srmatrix q(3), l(3);

m.ql(q,l);
std::cout << (eye_real(3) - ~q * q).norm()
          << " " << (m - q * l).norm() << std::endl;
\end{Verbatim}
prints
\begin{Verbatim}
+1.6146017e-015 +4.3341378e-015
\end{Verbatim}
\newpage



\subsubsection{rq}
Function%
\pdfdest name {srmatrix.rq} fit
\begin{verbatim}
void srmatrix::rq (srmatrix& mR, srmatrix& mQ) const throw (cvmexception);
\end{verbatim}
computes RQ factorization as
\begin{equation*}
M=RQ
\end{equation*}
where 
$M$ is  calling square matrix, upper triangular matrix $R$ 
and orthogonal matrix $Q$ are \verb"mR" and \verb"mQ"
respectively. 
Function throws
 \GOT{cvmexception}{cvmexception}
in case if inappropriate sizes of the operands passed.
See also \GOT{rmatrix}{rmatrix}, \GOT{rmatrix::rq}{rmatrix.rq},
\GOT{srmatrix}{srmatrix}.
Example:
\begin{Verbatim}
using namespace cvm;

treal a[] = {1., 2., 3., 4., 5., 6., 7., 8., 9.};
const cvm::srmatrix m(a, 3);
cvm::srmatrix r(3), q(3);

m.rq(r,q);
std::cout << (eye_real(3) - ~q * q).norm()
          << " " << (m - r * q).norm() << std::endl;
\end{Verbatim}
prints
\begin{Verbatim}
+3.7030557e-016 +2.2752801e-015
\end{Verbatim}
\newpage


\subsubsection{identity}
Function%
\pdfdest name {srmatrix.identity} fit
\begin{verbatim}
srmatrix& srmatrix::identity();
\end{verbatim}
sets  calling matrix to be equal to identity matrix
and returns a reference to
the matrix changed. 
Function is \emph{redefined} in the classes
\GOT{srbmatrix}{srbmatrix} and
\GOT{srsmatrix}{srsmatrix}.
See also \GOT{srmatrix}{srmatrix}.
Example:
\begin{Verbatim}
using namespace cvm;

std::cout.setf (std::ios::scientific | std::ios::left); 
std::cout.precision (3);
srmatrix m(3);
m.randomize(0.,1.);

std::cout << m << std::endl;
std::cout << m.identity();
\end{Verbatim}
prints
\begin{Verbatim}
9.423e-01 2.950e-01 8.429e-01
2.013e-01 3.250e-01 2.904e-01
7.920e-01 2.405e-02 7.801e-01

1.000e+00 0.000e+00 0.000e+00
0.000e+00 1.000e+00 0.000e+00
0.000e+00 0.000e+00 1.000e+00
\end{Verbatim}
\newpage




\subsubsection{vanish}
Function%
\pdfdest name {srmatrix.vanish} fit
\begin{verbatim}
srmatrix& srmatrix::vanish();
\end{verbatim}
sets every element of a calling matrix to be equal to zero
and returns a reference to
the matrix changed. This function is faster
than
\GOT{srmatrix::set(TR)}{srmatrix.set}
with zero operand passed.
Function is \emph{redefined} in the classes
\GOT{srsmatrix}{srsmatrix} and
\GOT{srbmatrix}{srbmatrix}.
See also \GOT{srmatrix}{srmatrix}.
Example:
\begin{Verbatim}
using namespace cvm;

std::cout.setf (std::ios::scientific | std::ios::left); 
std::cout.precision (3);
srmatrix m(3);
m.randomize(0.,1.);

std::cout << m << std::endl;
std::cout << m.vanish ();
\end{Verbatim}
prints
\begin{Verbatim}
1.747e-01 7.563e-01 5.163e-01
9.657e-01 6.619e-01 8.036e-01
6.392e-01 6.658e-01 6.495e-01

0.000e+00 0.000e+00 0.000e+00
0.000e+00 0.000e+00 0.000e+00
0.000e+00 0.000e+00 0.000e+00
\end{Verbatim}
\newpage




\subsubsection{randomize}
Function%
\pdfdest name {srmatrix.randomize} fit
\begin{verbatim}
srmatrix& srmatrix::randomize (TR dFrom, TR dTo);
\end{verbatim}
fills  calling matrix with pseudo-random numbers distributed between
\verb"dFrom" and \verb"dTo".
Function
returns a reference to the matrix changed.
See also
\GOT{srmatrix}{srmatrix}.
Example:
\begin{Verbatim}
using namespace cvm;
std::cout.setf (std::ios::scientific | std::ios::left); 
std::cout.precision (7);

srmatrix m(3);
m.randomize(-2.,3.);
std::cout << m;
\end{Verbatim}
prints
\begin{Verbatim}
-1.6790979e+00 5.0233467e-02 -1.9559008e+00
-1.7987609e-01 -5.2092044e-01 -1.8211615e+00
6.8242439e-01 9.0688803e-01 -1.7891171e+00
\end{Verbatim}
\newpage

