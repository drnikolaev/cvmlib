%\bigskip
\noindent%
\verb"template <typename T>"\\
\verb"class basic_array {"\\
\verb"public:"\\
\verb"    int "\GOT{basic\_array}{basicarray.ctr}\verb" ();"\\
\verb"    explicit "\GOT{basic\_array}{basicarray.ctr2}\verb" (int nSize, bool bZeroMemory = true);"\\
\verb"    "\GOT{basic\_array}{basicarray.ctr2.5}\verb" (T* p, int nSize);"\\
\verb"    "\GOT{basic\_array}{basicarray.ctr3}\verb" (const T* p, int nSize);"\\
\verb"    "\GOT{basic\_array}{basicarray.ctrstl}\verb" (const T* first, const T* last);"\\
\verb"    "\GOT{basic\_array}{basicarray.copyctr}\verb" (const basic_array& a);"\\
\verb"    int "\GOT{size}{basicarray.size}\verb" () const;"\\
\verb"    T* "\GOT{get}{basicarray.get}\verb" ();"\\
\verb"    const T* "\GOT{get}{basicarray.get}\verb" () const;"\\
\verb"    "\GOT{operator T*}{basicarray.get}\verb" ();"\\
\verb"    "\GOT{operator const T*}{basicarray.get}\verb" () const;"\\
\verb"    T& "\GOT{operator ()}{basicarray.indexing}\verb" (int i) throw (cvmexception);"\\
\verb"    T "\GOT{operator ()}{basicarray.indexing}\verb" (int i) const throw (cvmexception);"\\
\verb"    T& "\GOT{operator []}{basicarray.indexing}\verb" (size_type i) throw (cvmexception);"\\
\verb"    T "\GOT{operator []}{basicarray.indexing}\verb" (size_type i) const throw (cvmexception);"\\
\verb"    T& "\GOT{operator []}{basicarray.indexing}\verb" (int i) throw (cvmexception);"\\
\verb"    T "\GOT{operator []}{basicarray.indexing}\verb" (int i) const throw (cvmexception);"\\
\verb"    basic_array& "\GOT{operator =}{basicarray.operator =}\verb" (const basic_array& a) throw (cvmexception);"\\
\verb"    basic_array& "\GOT{assign}{basicarray.assign}\verb" (const T* p);"\\
\verb"    basic_array& "\GOT{set}{basicarray.set}\verb" (T x);"\\
\verb"    basic_array& "\GOT{resize}{basicarray.resize}\verb" (int nNewSize) throw (cvmexception);"\\
\verb"    // "\GOT{STL-specific type definitions}{basicarray.typedefs}\\
\verb"    typedef T value_type;"\\
\verb"    typedef value_type* pointer;"\\
\verb"    typedef const value_type* const_pointer;"\\
\verb"    typedef value_type* iterator;"\\
\verb"    typedef const value_type* const_iterator;"\\
\verb"    typedef value_type& reference;"\\
\verb"    typedef const value_type& const_reference;"\\
\verb"    typedef size_t size_type;"\\
\verb"    typedef ptrdiff_t difference_type;"\\
\verb"    typedef std::reverse_iterator<const_iterator> const_reverse_iterator;"\\
\verb"    typedef std::reverse_iterator<iterator> reverse_iterator;"\\
\verb"    // STL-specific functions"\\
\verb"    iterator "\GOT{begin}{basicarray.beginend}\verb" ();"\\
\verb"    const_iterator "\GOT{begin}{basicarray.beginend}\verb" () const;"\\
\verb"    iterator "\GOT{end}{basicarray.beginend}\verb" ();"\\
\verb"    const_iterator "\GOT{end}{basicarray.beginend}\verb" () const;"\\
\verb"    reverse_iterator "\GOT{rbegin}{basicarray.beginend}\verb" ();"\\
\verb"    const_reverse_iterator "\GOT{rbegin}{basicarray.beginend}\verb" () const;"\\
\verb"    reverse_iterator "\GOT{rend}{basicarray.beginend}\verb" ();"\\
\verb"    const_reverse_iterator "\GOT{rend}{basicarray.beginend}\verb" () const;"\\
\verb"    size_type "\GOT{max\_size}{basicarray.beginend}\verb" () const;"\\
\verb"    size_type "\GOT{capacity}{basicarray.beginend}\verb" () const;"\\
\verb"    bool "\GOT{empty}{basicarray.beginend}\verb" () const;"\\
\verb"    reference "\GOT{front}{basicarray.beginend}\verb" ();"\\
\verb"    const_reference "\GOT{front}{basicarray.beginend}\verb" ();"\\
\verb"    reference "\GOT{back}{basicarray.beginend}\verb" ();"\\
\verb"    const_reference "\GOT{back}{basicarray.beginend}\verb" () const;"\\
%\verb"    void "\GOT{reserve}{basicarray.beginend}\verb" (size_type n) throw (cvmexception);"\\
\verb"    void "\GOT{assign}{basicarray.beginend}\verb" (size_type n, const T& val) throw (cvmexception);"\\
\verb"    void "\GOT{assign}{basicarray.beginend}\verb" (const_iterator first,"\\
\verb"                 const_iterator last) throw (cvmexception);"\\
\verb"    void "\GOT{resize}{basicarray.beginend}\verb" (size_type nNewSize) throw (cvmexception);"\\
\verb"    void "\GOT{clear}{basicarray.beginend}\verb" ();"\\
\verb"    void "\GOT{swap}{basicarray.beginend}\verb" (basic_array& v);"\\
\verb"    reference "\GOT{at}{basicarray.at}\verb" (size_type n) throw (cvmexception);"\\
\verb"    const_reference "\GOT{at}{basicarray.at}\verb" (size_type n) const throw (cvmexception);"\\
\verb"    void "\GOT{push\_back}{basicarray.pushpop}\verb" (const T& x) throw (cvmexception);"\\
\verb"    void "\GOT{pop\_back}{basicarray.pushpop}\verb" () throw (cvmexception);"\\
\verb"    iterator "\GOT{insert}{basicarray.inserterase}\verb" (iterator position, const T& x) throw (cvmexception);"\\
\verb"    iterator "\GOT{erase}{basicarray.inserterase}\verb" (iterator position) throw (cvmexception);"\\
\verb" "\\
\verb"    template <typename T>"\\
\verb"    friend std::istream& "\GOT{operator >{}> <>}{basicarray.input}\verb" (const std::istream& is,"\\
\verb"                                         basic_array<T>& aIn);"\\
\verb" "\\
\verb"    template <typename T>"\\
\verb"    friend std::ostream& "\GOT{operator <{}< <>}{basicarray.output}\verb" (std::ostream& os,"\\
\verb"                                         const basic_array<T>& aOut);"\\
\verb"};"
\newpage


\subsubsection{basic\_array()}
Default constructor%
\pdfdest name {basicarray.ctr} fit
\begin{verbatim}
basic_array<T>::basic_array();
\end{verbatim}
creates  empty \verb"basic_array" object. See also 
\GOT{basic\_array}{basicarray}.
Example:
\begin{Verbatim}
using namespace cvm;

iarray a;
std::cout << a.size() << std::endl;
a.resize(10);
std::cout << a.size() << std::endl;
\end{Verbatim}
prints
\begin{Verbatim}
0
10
\end{Verbatim}
\newpage


\subsubsection{basic\_array(int, bool)}
Constructor%
\pdfdest name {basicarray.ctr2} fit
\begin{verbatim}
explicit basic_array<T>::basic_array(int nSize, bool bZeroMemory = true);
\end{verbatim}
creates  \verb"basic_array" object of size equal to \verb"nSize". 
Allocated memory is initialized with zero values by default 
(you can pass \verb"false" in second argument in order to 
avoid this initialization). 
Constructor throws \GOT{cvmexception}{cvmexception}
in case of non-positive size passed or memory allocation failure.
See also
\GOT{basic\_array}{basicarray}.
Example:
\begin{Verbatim}
using namespace cvm;

iarray a(5);
std::cout << a.size() << " " << a[1] << std::endl;
\end{Verbatim}
prints
\begin{Verbatim}
5 0
\end{Verbatim}
\newpage


\subsubsection{basic\_array(T*, int)}
Constructor%
\pdfdest name {basicarray.ctr2.5} fit
\begin{verbatim}
basic_array<T>::basic_array (T* p, int nSize);
\end{verbatim}
creates  \verb"basic_array" object of size equal to 
\verb"nSize" and shares memory (shallow copy) with an array pointed to by \verb"p" parameter.
Constructor throws  \GOT{cvmexception}{cvmexception}
in case of non-positive size passed.
See also \GOT{basic\_array}{basicarray}.
Example:
\begin{Verbatim}
using namespace cvm;

int a[] = {1, 2, 3, 4};
iarray v (a, 3);
std::cout << v;
a[0] = 77;
std::cout << v;
\end{Verbatim}
prints
\begin{Verbatim}
1 2 3
77 2 3
\end{Verbatim}
\newpage


\subsubsection{basic\_array(const T*, int)}
Constructor%
\pdfdest name {basicarray.ctr3} fit
\begin{verbatim}
basic_array<T>::basic_array (const T* p, int nSize);
\end{verbatim}
creates  \verb"basic_array" object of size equal to 
\verb"nSize" and copies \verb"nSize" elements of
an array \verb"p" to it.
Constructor throws  \GOT{cvmexception}{cvmexception}
in case of non-positive size passed or memory allocation failure.
See also \GOT{basic\_array}{basicarray}.
Example:
\begin{Verbatim}
using namespace cvm;

const int a[] = {1, 2, 3, 4};
iarray v (a, 3);
std::cout << v;
\end{Verbatim}
prints
\begin{Verbatim}
1 2 3
\end{Verbatim}
\newpage


\subsubsection{basic\_array(const T*, const T*)}
Constructor%
\pdfdest name {basicarray.ctrstl} fit
\begin{verbatim}
basic_array<T>::basic_array (const T* first, const T* last);
\end{verbatim}
creates  \verb"basic_array" object of size equal to 
\verb"last-first" and copies all elements in the range
of \verb"[first,last)" to it.
Constructor throws \GOT{cvmexception}{cvmexception}
in case of wrong range passed or memory allocation failure.
See also \GOT{basic\_array}{basicarray}.
Example:
\begin{Verbatim}
using namespace cvm;

const int a[] = {1, 2, 3, 4};
const iarray v (a+1, a+3);
std::cout << v << std::endl;
\end{Verbatim}
prints
\begin{Verbatim}
2 3
\end{Verbatim}
\newpage


\subsubsection{basic\_array(const basic\_array\&)}
Copy constructor%
\pdfdest name {basicarray.copyctr} fit
\begin{verbatim}
basic_array<T>::basic_array (const basic_array& a);
\end{verbatim}
creates  \verb"basic_array" object of size equal to size of
vector \verb"a" and 
sets every element of created array to a value of
appropriate element of  \verb"a".
See also \GOT{basic\_array}{basicarray}.
Example:
\begin{Verbatim}
using namespace cvm;

iarray a(5);
a.set(3);
iarray b(a);
std::cout << b;
\end{Verbatim}
prints
\begin{Verbatim}
3 3 3 3 3
\end{Verbatim}
\newpage


\subsubsection{size()}
Function%
\pdfdest name {basicarray.size} fit
\begin{verbatim}
int basic_array<T>::size () const;
\end{verbatim}
returns  number of elements of a calling array.
This function is \emph{inherited} in all classes of the library:
\GOT{rvector}{rvector},   \GOT{cvector}{cvector},
\GOT{rmatrix}{rmatrix},   \GOT{cmatrix}{cmatrix},
\GOT{srmatrix}{srmatrix}, \GOT{scmatrix}{scmatrix},
\GOT{srbmatrix}{srbmatrix}, \GOT{scbmatrix}{scbmatrix},
\GOT{srsmatrix}{srsmatrix} and \GOT{schmatrix}{schmatrix}.
See also \GOT{basic\_array}{basicarray}.
Example:
\begin{Verbatim}
using namespace cvm;

rvector v(3);
cmatrix m(10,20);
cout << v.size() << " " << m.size() << endl;
\end{Verbatim}
prints
\begin{Verbatim}
3 200
\end{Verbatim}
\newpage


\subsubsection{get(), operator T*()}
Functions and operators%
\pdfdest name {basicarray.get} fit
\begin{verbatim}
T* basic_array<T>::get ();
const T* basic_array<T>::get () const;
basic_array<T>::operator T* ();
basic_array<T>::operator const T* () const;
\end{verbatim}
return  pointer to beginning (first element) of a calling array.
These functions and operators are \emph{inherited} in all 
classes of the library:
\GOT{rvector}{rvector},   \GOT{cvector}{cvector},
\GOT{rmatrix}{rmatrix},   \GOT{cmatrix}{cmatrix},
\GOT{srmatrix}{srmatrix}, \GOT{scmatrix}{scmatrix},
\GOT{srbmatrix}{srbmatrix}, \GOT{scbmatrix}{scbmatrix},
\GOT{srsmatrix}{srsmatrix} and \GOT{schmatrix}{schmatrix}.
See also \GOT{basic\_array}{basicarray}.
Example:
\begin{Verbatim}
using namespace cvm;
std::cout.setf (std::ios::scientific | std::ios::left); 
std::cout.precision (2);

void cprint (const std::complex<double>* p, int size)
{
    for (int i = 0; i < size; ++i)
        std::cout << p[i] << " ";
    std::cout << std::endl;
}
...
iarray a(10);
scmatrix m(3);
a[2] = 1;
m(3,1) = std::complex<double>(1., 2.);
std::cout << a.get()[1] << std::endl;
cprint(m, 3);
\end{Verbatim}
prints
\begin{Verbatim}
1
(0.00e+00,0.00e+00) (0.00e+00,0.00e+00) (1.00e+00,2.00e+00)
\end{Verbatim}
\newpage


\subsubsection{Indexing operators}
Indexing operators%
\pdfdest name {basicarray.indexing} fit
\begin{verbatim}
T& basic_array<T>::operator () (int i) throw (cvmexception);
T  basic_array<T>::operator () (int i) const throw (cvmexception);
T& basic_array<T>::operator [] (size_type i) throw (cvmexception);
T  basic_array<T>::operator [] (size_type i) const throw (cvmexception);
T& basic_array<T>::operator [] (int i) throw (cvmexception);
T  basic_array<T>::operator [] (int i) const throw (cvmexception);
\end{verbatim}
return  reference (or value for constant versions)
to \verb"i"-th element of a calling array. Please note that \emph{all indexing 
operators of the library are \Based.}
These operators are \emph{inherited} in 
\GOT{rvector}{rvector} and \GOT{cvector}{cvector} 
classes of the library but \emph{reimplemented} in other ones:
\GOT{rmatrix}{rmatrix},   \GOT{cmatrix}{cmatrix},
\GOT{srmatrix}{srmatrix}, \GOT{scmatrix}{scmatrix},
\GOT{srbmatrix}{srbmatrix}, \GOT{scbmatrix}{scbmatrix},
\GOT{srsmatrix}{srsmatrix} and \GOT{schmatrix}{schmatrix}.
See also \GOT{basic\_array}{basicarray}.
Example:
\begin{Verbatim}
using namespace cvm;

try {
    rvector v (10);
    v[1] = 1.;
    v(2) = 2.;
    std::cout << v;

    double a[] = {1., 2., 3., 4.};
    const rvector vc (a, 4);
    std::cout << vc(1) << " " << vc[2] << std::endl;
}
catch (std::exception& e) {
    std::cout << "Exception: " << e.what () << std::endl;
}
\end{Verbatim}
prints
\begin{Verbatim}
1 2 0 0 0 0 0 0 0 0
1 2
\end{Verbatim}
\newpage


\subsubsection{operator = (const basic\_array\&)}
Assignment operator%
\pdfdest name {basicarray.operator =} fit
\begin{verbatim}
basic_array<T>&
basic_array<T>::operator = (const basic_array& a) throw (cvmexception);
\end{verbatim}
sets every element of a calling array to a value of
appropriate element of an array  \verb"a"
and returns a reference to the object changed.
This operator is \emph{reimplemented} in all classes of the library:
\GOT{rvector}{rvector}, \GOT{cvector}{cvector} ,
\GOT{rmatrix}{rmatrix},   \GOT{cmatrix}{cmatrix},
\GOT{srmatrix}{srmatrix}, \GOT{scmatrix}{scmatrix},
\GOT{srbmatrix}{srbmatrix}, \GOT{scbmatrix}{scbmatrix},
\GOT{srsmatrix}{srsmatrix} and \GOT{schmatrix}{schmatrix}.
See also \GOT{basic\_array}{basicarray}.
Operator throws \GOT{cvmexception}{cvmexception}
in case of different sizes of the arrays.
Example:
\begin{Verbatim}
using namespace cvm;

iarray a(5), b(5);
a.set(3);
b = a;
std::cout << b;
\end{Verbatim}
prints
\begin{Verbatim}
3 3 3 3 3
\end{Verbatim}
\newpage


\subsubsection{assign(const T*)}
Function%
\pdfdest name {basicarray.assign} fit
\begin{verbatim}
basic_array<T>& basic_array<T>::assign (const T* p);
\end{verbatim}
sets every element of a calling array to a value of
appropriate element of an array  pointed to by parameter \verb"p"
and returns a reference to the object changed.
This function is \emph{reimplemented} in all classes of the library:
\GOT{rvector}{rvector}, \GOT{cvector}{cvector} ,
\GOT{rmatrix}{rmatrix},   \GOT{cmatrix}{cmatrix},
\GOT{srmatrix}{srmatrix}, \GOT{scmatrix}{scmatrix},
\GOT{srbmatrix}{srbmatrix}, \GOT{scbmatrix}{scbmatrix},
\GOT{srsmatrix}{srsmatrix} and \GOT{schmatrix}{schmatrix}.
See also \GOT{basic\_array}{basicarray}.
Example:
\begin{Verbatim}
using namespace cvm;

const int a[] = {1, 2, 3, 4, 5, 6, 7};
iarray v (5);

v.assign(a);
std::cout << v;
\end{Verbatim}
prints
\begin{Verbatim}
1 2 3 4 5
\end{Verbatim}
\newpage


\subsubsection{set(T)}
Function%
\pdfdest name {basicarray.set} fit
\begin{verbatim}
basic_array<T>& basic_array<T>::set (T x);
\end{verbatim}
sets every element of a calling array to a value of
parameter \verb"x"
and returns a reference to the object changed.
This function is \emph{reimplemented} in all classes of the library:
\GOT{rvector}{rvector}, \GOT{cvector}{cvector} ,
\GOT{rmatrix}{rmatrix},   \GOT{cmatrix}{cmatrix},
\GOT{srmatrix}{srmatrix}, \GOT{scmatrix}{scmatrix},
\GOT{srbmatrix}{srbmatrix}, \GOT{scbmatrix}{scbmatrix},
\GOT{srsmatrix}{srsmatrix} and \GOT{schmatrix}{schmatrix}.
See also \GOT{basic\_array}{basicarray}.
Example:
\begin{Verbatim}
using namespace cvm;

iarray a(5);
a.set(3);
std::cout << a;
\end{Verbatim}
prints
\begin{Verbatim}
3 3 3 3 3
\end{Verbatim}
\newpage


\subsubsection{resize}
Function%
\pdfdest name {basicarray.resize} fit
\begin{verbatim}
basic_array<T>&
basic_array<T>::resize (int nNewSize) throw (cvmexception);
\end{verbatim}
changes  size of a calling array to be equal to
\verb"nNewSize" and returns a reference to
the array changed. Calling array
will be filled with zeroes in case of increasing of its size.
This function is \emph{reimplemented} in all classes of the library:
\GOT{rvector}{rvector}, \GOT{cvector}{cvector} ,
\GOT{rmatrix}{rmatrix},   \GOT{cmatrix}{cmatrix},
\GOT{srmatrix}{srmatrix}, \GOT{scmatrix}{scmatrix},
\GOT{srbmatrix}{srbmatrix}, \GOT{scbmatrix}{scbmatrix},
\GOT{srsmatrix}{srsmatrix} and \GOT{schmatrix}{schmatrix}.
Function throws \GOT{cvmexception}{cvmexception}
in case of non-positive size passed or memory allocation failure.
See also \GOT{basic\_array}{basicarray}.
Example:
\begin{Verbatim}
using namespace cvm;

try {
    const int a[] = {1, 2, 3, 4};
    iarray v (a, 3);
    std::cout << v;
    v.resize(2);
    std::cout << v;
    v.resize(4);
    std::cout << v;
}
catch (std::exception& e) {
    std::cout << "Exception " << e.what () << std::endl;
}
\end{Verbatim}
prints
\begin{Verbatim}
1 2 3
1 2
1 2 0 0
\end{Verbatim}
\newpage



\subsubsection{STL-specific type definitions}
Type definitions%
\pdfdest name {basicarray.typedefs} fit
\begin{verbatim}
typedef T value_type;
typedef value_type* pointer;
typedef const value_type* const_pointer;
typedef value_type* iterator;
typedef const value_type* const_iterator;
typedef value_type& reference;
typedef const value_type& const_reference;
typedef size_t size_type;
typedef ptrdiff_t difference_type;
typedef std::reverse_iterator<const_iterator> const_reverse_iterator;
typedef std::reverse_iterator<iterator> reverse_iterator;
\end{verbatim}
are provided for every class of the library to be compatible 
with STL algorithms and methods. See also 
\GOT{basic\_array}{basicarray}.
Example:
\begin{Verbatim}
using namespace cvm;

rvector vs1(5);
vs1[1] = 1.; vs1[2] = 2.; vs1[3] = 3.; vs1[4] = 4.; vs1[5] = 5.;
std::cout << vs1;

rvector::iterator it = vs1.begin() + 1;
rvector::iterator ite = vs1.erase(it);
std::cout << vs1;
std::cout << *ite << std::endl;

ite = vs1.insert(ite, 10.);
std::cout << vs1;
std::cout << *ite << std::endl;

vs1.push_back(11.);
std::cout << vs1;

vs1.randomize(0., 2.);
std::cout << vs1;
std::sort(vs1.begin(), vs1.end());
std::cout << vs1;
std::cout << *std::max_element(vs1.begin(), vs1.end()) << std::endl;

std::reverse(vs1.begin(), vs1.end());
std::cout << vs1;
\end{Verbatim}
prints
\begin{Verbatim}
1.00e+000 2.00e+000 3.00e+000 4.00e+000 5.00e+000
1.00e+000 3.00e+000 4.00e+000 5.00e+000
3.00e+000
1.00e+000 1.00e+001 3.00e+000 4.00e+000 5.00e+000
1.00e+001
1.00e+000 1.00e+001 3.00e+000 4.00e+000 5.00e+000 1.10e+001
1.11e-001 4.96e-001 1.70e+000 1.91e+000 1.19e-001 1.11e+000
1.11e-001 1.19e-001 4.96e-001 1.11e+000 1.70e+000 1.91e+000
1.91e+000
1.91e+000 1.70e+000 1.11e+000 4.96e-001 1.19e-001 1.11e-001
\end{Verbatim}
\newpage


\subsubsection{STL-specific functions:\\
begin(), end(), rbegin(), rend(), \\
max\_size(), capacity(), empty(), front(), back(),\\
%reserve(),\\
assign(), resize(), clear(), swap()}
Functions%     %%%void basic_array<T>::reserve (size_type n);
\pdfdest name {basicarray.beginend} fit
\begin{verbatim}
basic_array<T>::iterator basic_array<T>::begin();
basic_array<T>::const_iterator basic_array<T>::begin() const;
basic_array<T>::iterator basic_array<T>::end();
basic_array<T>::const_iterator basic_array<T>::end() const;
basic_array<T>::reverse_iterator basic_array<T>::rbegin();
basic_array<T>::const_reverse_iterator basic_array<T>::rbegin() const;
basic_array<T>::reverse_iterator basic_array<T>::rend();
basic_array<T>::const_reverse_iterator basic_array<T>::rend();
basic_array<T>::size_type basic_array<T>::max_size() const;
basic_array<T>::size_type basic_array<T>::capacity() const;
bool basic_array<T>::empty() const;
basic_array<T>::reference basic_array<T>::front();
basic_array<T>::const_reference basic_array<T>::front();
basic_array<T>::reference basic_array<T>::back();
basic_array<T>::const_reference basic_array<T>::back();
void basic_array<T>::assign (size_type n, 
                             const T& val) throw (cvmexception);
void basic_array<T>::assign (const_iterator first, 
                             const_iterator last) throw (cvmexception);
void basic_array<T>::resize (size_type nNewSize) throw (cvmexception);
void basic_array<T>::clear();
void basic_array<T>::swap (basic_array& v) throw (cvmexception);
\end{verbatim}
are provided for every class of the library to be compatible 
with STL algorithms and methods. See also 
\GOT{basic\_array}{basicarray} 
and \URL{STL documentation}{http://www.sgi.com/tech/stl/}
for further details.
Example:
\begin{Verbatim}
using namespace cvm;

iarray a(5);
a[1] = 1; a[2] = 2; a[3] = 3; a[4] = 4; a[5] = 5;

for (iarray::reverse_iterator it = a.rbegin(); it != a.rend(); ++it)
{
    std::cout << *it << " ";
}
std::cout << std::endl;
std::cout << a.front() << std::endl;
std::cout << a.back() << std::endl;
\end{Verbatim}
prints
\begin{Verbatim}
5 4 3 2 1
1
5
\end{Verbatim}
\newpage


\subsubsection{at()}
Functions%
\pdfdest name {basicarray.at} fit
\begin{verbatim}
basic_array<T>::reference
basic_array<T>::at(size_type n) throw (cvmexception);

basic_array<T>::const_reference 
basic_array<T>::at(size_type n) throw (cvmexception);
\end{verbatim}
return  reference to  \verb"(n-1)"-th element of a calling array, i.e.,
unlike \GO{indexing operators}{basicarray.indexing},
these functions are \emph{$0$-based}.
They are provided for every class of the library to be compatible 
with STL algorithms and methods. 
Functions throw  \GOT{cvmexception}{cvmexception}
in case of negative parameter passed.
See also 
\GOT{basic\_array}{basicarray} 
and \URL{STL documentation}{http://www.sgi.com/tech/stl/}
for further details.
Example:
\begin{Verbatim}
using namespace cvm;

iarray a(5);
a[1] = 1; a[2] = 2; a[3] = 3; a[4] = 4; a[5] = 5;
std::cout << a.at(0) << " " << a.at(1) << std::endl;
\end{Verbatim}
prints
\begin{Verbatim}
1 2
\end{Verbatim}
\newpage


\subsubsection{push\_back(const T\&), pop\_back()}
Functions%
\pdfdest name {basicarray.pushpop} fit
\begin{verbatim}
void basic_array<T>::push_back (const T& x) throw (cvmexception);
void basic_array<T>::pop_back () throw (cvmexception);
\end{verbatim}
add and remove an element to/from calling array.
They are provided for every class of the library to be compatible 
with STL algorithms and methods. Since CVM doesn't 
pre-allocate  memory for extra storage, these functions
\emph{will require memory reallocation every time they are being executed}
and may slow down your application. Please consider usage of
\verb"std::vector<T>" in such cases.
Functions throw  \GOT{cvmexception}{cvmexception}
in case of memory allocation failure.
See also 
\GOT{basic\_array}{basicarray} 
and \URL{STL documentation}{http://www.sgi.com/tech/stl/}
for further details.
Example:
\begin{Verbatim}
using namespace cvm;

iarray a(5);
a.push_back(77);
std::cout << a;
a.pop_back();
std::cout << a;
\end{Verbatim}
prints
\begin{Verbatim}
0 0 0 0 0 77
0 0 0 0 0
\end{Verbatim}
\newpage


\subsubsection{insert (iterator, const T\&), erase (iterator)}
Functions%
\pdfdest name {basicarray.inserterase} fit
\begin{verbatim}
basic_array<T>::iterator 
basic_array<T>::insert (iterator pos, const T& x) throw (cvmexception);

basic_array<T>::iterator 
basic_array<T>::erase (iterator pos) throw (cvmexception);
\end{verbatim}
insert and remove an element to/from calling array at given position \verb"pos".
They are provided for every class of the library to be compatible 
with STL algorithms and methods. Since CVM doesn't 
pre-allocate  memory for extra storage, these functions
\emph{will require memory reallocation every time they are executed}
and may slow down your application. Please consider usage of
\verb"std::vector<T>" in such cases.
Functions throw  \GOT{cvmexception}{cvmexception}
in case of memory allocation failure.
See also 
\GOT{basic\_array}{basicarray} 
and \URL{STL documentation}{http://www.sgi.com/tech/stl/}
for further details.
Example:
\begin{Verbatim}
using namespace cvm;

iarray a(5);
iarray::iterator pos = a.begin() + 2;
a.insert(pos, 77);
std::cout << a;
pos = a.begin() + 1;
a.erase(pos);
std::cout << a;
\end{Verbatim}
prints
\begin{Verbatim}
0 0 77 0 0 0
0 77 0 0 0
\end{Verbatim}
\newpage




\subsubsection{operator >{}> <> (std::istream\& is, basic\_array<T>\& aIn)}
Friend template operator%
\pdfdest name {basicarray.input} fit
\begin{verbatim}
template <typename T>
friend std::istream& operator >> <> (std::istream& is, 
                                     basic_array<T>& aIn);
\end{verbatim}
fills  object referenced by \verb"aIn" with numbers from
\verb"is" stream.
Operator is \GO{redefined}{Array.input} in the class
\GOT{Array}{Array}.
See also \GOT{basic\_array}{basicarray} 
Example:
\begin{Verbatim}
using namespace cvm;

try {
    std::ofstream os;
    os.open ("in.txt");
    os << 1 << " " << 2 << std::endl << 3;
    os.close ();

    std::ifstream is("in.txt");
    iarray v(5);
    is >> v;

    std::cout << v;
}
catch (std::exception& e) {
    std::cout << "Exception " << e.what () << std::endl;
}
\end{Verbatim}
prints
\begin{Verbatim}
1 2 3 0 0
\end{Verbatim}
\newpage



\subsubsection{operator <{}< <> (std::ostream\& os, const basic\_array<T>\& aOut)}
Friend template operator%
\pdfdest name {basicarray.output} fit
\begin{verbatim}
template <typename T>
friend std::ostream& operator << <> (std::ostream& os,
                                     const basic_array<T>& aOut);
\end{verbatim}
outputs  object referenced by \verb"aOut" into
\verb"os" stream.
Operator is \GO{redefined}{Array.output} in the class
\GOT{Array}{Array}.
See also \GOT{basic\_array}{basicarray} 
Example:
\begin{Verbatim}
using namespace cvm;

iarray v(5);
v(1) = 1;
v(2) = 2;

std::cout << v;
\end{Verbatim}
prints
\begin{Verbatim}
1 2 0 0 0
\end{Verbatim}
\newpage

