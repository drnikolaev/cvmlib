\subsection{Utilities}
These%
\pdfdest name {Utilities} fit{}
functions have \verb"cvm" namespace scope and
can be used for different purposes.

\bigskip
\noindent
\verb"template <typename T>"\\
\verb"T* "\GOT{cvmMalloc}{Utilities.cvmMalloc}\verb" (size_t nEls) throw (cvmexception);"\\
\verb"template <typename T>"\\
\verb"T* "\GOT{cvmAddRef}{Utilities.cvmAddRef}\verb" (const T* pD);"\\
\verb"template <typename T>"\\
\verb"int "\GOT{cvmFree}{Utilities.cvmFree}\verb" (T*& pD);"\\
\verb"void "\GOT{cvmExit}{Utilities.cvmExit}\verb" ();"\\
\verb"treal "\GOT{cvmMachMin}{Utilities.cvmMachMin}\verb" ();"\\
\verb"treal "\GOT{cvmMachSp}{Utilities.cvmMachSp}\verb" ();"\\
\verb"srmatrix "\GOT{eye\_real}{Utilities.eye_real}\verb" (int nM);"\\
\verb"scmatrix "\GOT{eye\_complex}{Utilities.eye_complex}\verb" (int nM);"\\
\GOT{operator * (,)}{Utilities.operator *}\verb";"\\
\verb""
\newpage




\subsubsection{cvmMalloc}
\textcolor{red}{Since version 6.0} if \verb"CVM_USE_POOL_MANAGER" macro is not defined (this is the default)
this function simply calls \verb"::new T[nEls]" operator.%
\pdfdest name {Utilities.cvmMalloc} fit
\begin{verbatim}
template <typename T>
T* cvmMalloc (size_t nEls) throw (cvmexception);
\end{verbatim}
allocates \verb"nEls" units of type \verb"T"
from the CVM library's \GO{memory pool}{SubSectionFeatures} and returns
a pointer to the memory allocated. 
% It can throw
%  \GOT{cvmexception}{cvmexception} 
% if there is not enough memory in the global pool.
% This is the preferable way to allocate memory in case of
% using the CVM library because it is faster and more robust.
See also \GOT{cvmAddRef}{Utilities.cvmAddRef} and
\GOT{cvmFree}{Utilities.cvmFree}.
Example:
\begin{Verbatim}
using namespace cvm;

double* p = cvmMalloc<double> (10);
p[0] = 1.;
p[1] = 2.;
p[2] = 3.;

rvector v(3);
v.assign(p);
std::cout << v;

cvmFree (p);
\end{Verbatim}
prints
\begin{Verbatim}
1 2 3
\end{Verbatim}
\newpage




\subsubsection{cvmAddRef}
\textcolor{red}{Since version 6.0 it's available only if \texttt{CVM\_USE\_POOL\_MANAGER} is defined.}%
\pdfdest name {Utilities.cvmAddRef} fit
\begin{verbatim}
template <typename T>
void cvmAddRef (const T* pD);
\end{verbatim}
increments a reference counter for a memory block pointed to
by \verb"pD" if this block was allocated from the CVM library's
memory pool (using \GOT{cvmMalloc}{Utilities.cvmMalloc} function).
If \verb"pD" points to a foreign memory block then
the function does nothing.
See also \GOT{cvmAddRef}{Utilities.cvmAddRef}.
Example:
\begin{Verbatim}
using namespace cvm;

double* p = cvmMalloc<double> (10);
p[0] = 1.;
p[1] = 2.;
p[2] = 3.;

cvmAddRef (p);
cvmFree (p);    // this call doesn't deallocate memory
cvmFree (p);    // this one does
\end{Verbatim}
\newpage




\subsubsection{cvmFree}
\textcolor{red}{Since version 6.0} if \verb"CVM_USE_POOL_MANAGER" macro is not defined (this is the default)
then it simply calls \verb"::delete[] pD" operator if \verb"pD!=NULL" and returns zero.%
\pdfdest name {Utilities.cvmFree} fit
\begin{verbatim}
template <typename T>
int cvmFree (T*& pD);
\end{verbatim}
decrements a reference counter for a memory block pointed to
by \verb"pD" if this block was allocated from the CVM library's
memory pool (using \GOT{cvmMalloc}{Utilities.cvmMalloc} function)
and returns the reference counter it changed.
If the function returns zero then it sets the pointer
\verb"pD" to be equal to \verb"NULL" and "frees" the memory, i.e.
returns the memory block to a list of free ones (see \GO{CVM memory
management}{SubSectionFeatures} for details).
If \verb"pD" points to a foreign memory block then
the function does nothing and returns $-1$.
See also \GOT{cvmAddRef}{Utilities.cvmAddRef}.
Example:
\begin{Verbatim}
using namespace cvm;

double* pf = new double[10];
double* p  = cvmMalloc<double> (10);

cvmAddRef (p);

std::cout << cvmFree (p) << " ";
std::cout << p << std::endl;

std::cout << cvmFree (p) << " ";
std::cout << p << std::endl;

std::cout << cvmFree (pf) << " ";
std::cout << pf << std::endl;

delete[] pf;
\end{Verbatim}
prints
\begin{Verbatim}
1 003C66B0
0 00000000
-1 003C7A40
\end{Verbatim}
\newpage





\subsubsection{cvmExit}
\textcolor{red}{Since version 6.0 it's available only if \texttt{CVM\_USE\_POOL\_MANAGER} is defined.}%
\pdfdest name {Utilities.cvmExit} fit
\begin{verbatim}
void cvmExit ();
\end{verbatim}
destroys the CVM library's \GO{memory pool}{SubSectionFeatures}
if \verb"CVM_USE_POOL_MANAGER" is defined. Otherwise does nothing.
All objects created using this pool are not accessible
after calling of this function.
Call this function in the last expression only
if you have problems with memory deallocation
while finishing execution of your program
(I was not able to experience such problems,
this function is provided just in case).
See also \GOT{cvmMalloc}{Utilities.cvmMalloc}.
Example:
\begin{Verbatim}
using namespace cvm;

int main(int argc, char* argv[])
    try {
        rvector v(3);
        v[1] = v[2] = v[3] = 1.;
        std::cout << v.norm2() << std::endl;
    }
    catch (cvmexception& e) {
        std::cout << "Exception " << e.cause ()
                  << ": " << e.what () << std::endl;
    }

    cvmExit ();
    return 1;
}
\end{Verbatim}
prints
\begin{Verbatim}
1.73205
\end{Verbatim}
\newpage




\subsubsection{cvmMachMin}
Function%
\pdfdest name {Utilities.cvmMachMin} fit
\begin{verbatim}
treal cvmMachMin ();
\end{verbatim}
returns the smallest normalized positive number,
i.e. \verb"numeric_limits<treal>::min()"
where \verb"treal" is \verb"typedef"'ed as \verb"double"
by default or as \verb"float" for float version of the library.
See also \GOT{cvmMachSp}{Utilities.cvmMachSp}.
Example:
\begin{Verbatim}
using namespace cvm;

std::cout.setf (std::ios::scientific | std::ios::showpos);
std::cout.precision (15);
std::cout << cvmMachMin() << std::endl;
\end{Verbatim}
on Intel Pentium\textcircled{\scriptsize{R}} III machine prints
\begin{Verbatim}
+2.225073858507201e-308
\end{Verbatim}
\newpage




\subsubsection{cvmMachSp}
Function%
\pdfdest name {Utilities.cvmMachSp} fit
\begin{verbatim}
treal cvmMachSp ();
\end{verbatim}
returns the largest relative spacing or, in other words,
the difference between $1$ and the least value greater 
than $1$ that is representable,
i.e. \verb"numeric_limits<treal>::epsilon()"
where \verb"treal" is \verb"typedef"'ed as \verb"double"
by default or as \verb"float" for float version of the library.
See also \GOT{cvmMachMin}{Utilities.cvmMachMin}.
Example:
\begin{Verbatim}
using namespace cvm;

std::cout.setf (std::ios::scientific | std::ios::showpos);
std::cout.precision (15);
std::cout << cvmMachSp() << std::endl;
\end{Verbatim}
on Intel Pentium\textcircled{\scriptsize{R}} III machine prints
\begin{Verbatim}
+2.220446049250313e-016
\end{Verbatim}
\newpage




\subsubsection{eye\_real}
Function%
\pdfdest name {Utilities.eye_real} fit
\begin{verbatim}
srmatrix eye_real (int nM);
\end{verbatim}
creates a \verb"nM" by \verb"nM" object of type 
\verb"srmatrix" 
equal to identity matrix.
See also \GOT{srmatrix}{srmatrix}.
Example:
\begin{Verbatim}
using namespace cvm;

std::cout << eye_real (4);
\end{Verbatim}
prints
\begin{Verbatim}
1 0 0 0
0 1 0 0
0 0 1 0
0 0 0 1
\end{Verbatim}
\newpage



\subsubsection{eye\_complex}
Function%
\pdfdest name {Utilities.eye_complex} fit
\begin{verbatim}
scmatrix eye_complex (int nM);
\end{verbatim}
creates a \verb"nM" by \verb"nM" object of type 
\verb"scmatrix" 
equal to identity matrix.
See also \GOT{scmatrix}{scmatrix}.
Example:
\begin{Verbatim}
using namespace cvm;

std::cout << eye_complex (4);
\end{Verbatim}
prints
\begin{Verbatim}
(1,0) (0,0) (0,0) (0,0)
(0,0) (1,0) (0,0) (0,0)
(0,0) (0,0) (1,0) (0,0)
(0,0) (0,0) (0,0) (1,0)
\end{Verbatim}
\newpage





\subsubsection{operator *}
Operators%
\pdfdest name {Utilities.operator *} fit
\begin{verbatim}
inline rvector operator * (TR d, const rvector& v);
inline rmatrix operator * (TR d, const rmatrix& m);
inline srmatrix operator * (TR d, const srmatrix& m);
inline srbmatrix operator * (TR d, const srbmatrix& m);
inline srsmatrix operator * (TR d, const srsmatrix& m);
inline cvector operator * (TR d, const cvector& v);
inline cmatrix operator * (TR d, const cmatrix& m);
inline scmatrix operator * (TR d, const scmatrix& m);
inline scbmatrix operator * (TR d, const scbmatrix& m);
inline schmatrix operator * (TR d, const schmatrix& m);
inline cvector operator * (std::complex<TR> c, const cvector& v);
inline cmatrix operator * (std::complex<TR> c, const cmatrix& m);
inline scmatrix operator * (std::complex<TR> c, const scmatrix& m);
inline scbmatrix operator * (std::complex<TR> c, const scbmatrix& m);
inline schmatrix operator * (std::complex<TR> c, const schmatrix& m);
inline rvector operator * (CVM_LONGEST_INT d, const rvector& v);
inline rmatrix operator * (CVM_LONGEST_INT d, const rmatrix& m);
inline srmatrix operator * (CVM_LONGEST_INT d, const srmatrix& m);
inline srbmatrix operator * (CVM_LONGEST_INT d, const srbmatrix& m);
inline srsmatrix operator * (CVM_LONGEST_INT d, const srsmatrix& m);
inline cvector operator * (CVM_LONGEST_INT d, const cvector& v);
inline cmatrix operator * (CVM_LONGEST_INT d, const cmatrix& m);
inline scmatrix operator * (CVM_LONGEST_INT d, const scmatrix& m);
inline scbmatrix operator * (CVM_LONGEST_INT d, const scbmatrix& m);
inline schmatrix operator * (CVM_LONGEST_INT d, const schmatrix& m);
\end{verbatim}
provide an ability to make left-sided multiplication of numbers
by different CVM objects.
Example:
\begin{Verbatim}
using namespace cvm;

const schmatrix scm = eye_complex (4);
std::cout << std::complex<double>(2.,1.) * scm << std::endl;

rvector v(3);
v(1) = 1.;
v(2) = 2.;
v(3) = 3.;

std::cout << 3. * v;
\end{Verbatim}
prints
\begin{Verbatim}
(2,1) (0,0) (0,0) (0,0)
(0,0) (2,1) (0,0) (0,0)
(0,0) (0,0) (2,1) (0,0)
(0,0) (0,0) (0,0) (2,1)

3 6 9
\end{Verbatim}
\newpage



\subsection{Pool Manager}
\textit{\textcolor{red}{The%
\pdfdest name {SubSectionFeatures} fit{}
memory management mechanism described below is no longer supported
by default.} It was a good solution more than 20 years ago when memory allocation
operator was relatively expensive, but now standard allocator does this job
much faster. However, the algorithm described below may be useful as
an error detection helper. You will need to rebuild the library in debug mode
with}
\verb"CVM_USE_POOL_MANAGER"\, \textit{defined in order to use it}.

Since its third release the CVM library implements
nontrivial memory management
which should be described in detail. Earlier memory was being
allocated using operator \verb"new" in every constructor, and
freed with help of \verb"delete" in every destructor. Let us
consider an operation of multiplying of a vector \verb'a' by matrix
\verb'A' and assignment of result to a vector \verb'b':
\begin{Verbatim}
b = a * A;
\end{Verbatim}
This harmless code calls two constructors (a constructor allocating
memory for output and a copy constructor, returning output to
a calling function), as well as two destructors deleting those
temporary objects\footnote{\,In order to avoid those memory
allocations you can use \verb"b.mult(a,A);"}. If sizes of the
objects are relatively small, your processor will be longer
allocating memory than multiplying\footnote{\,At least under Win32}.
The idea of nontrivial memory management came from Jeff Alger's book
\GO{\cite{JAlger}}{biblio}. Author suggests some approaches to
memory allocation (overloading of operators \verb"new" and
\verb"delete" or implementation of a class controlling a pool), and
also some ways of a pool compaction (Baker's algorithm and in-place
compaction) and references counting (using master or ``genius''
pointers). I decided to implement
a class controlling a pool in the CVM library.

Pool is controlled by \verb"MemoryPool" class. It allocates memory
by blocks (so-called ``outer blocks'').
The memory block concept is implemented in a class
\verb"MemoryBlocks". Size of an outer block equals to the nearest
degree of $2$ of a requested number of bytes multiplied by two. This
can be illustrated on the following example. Let us suppose that it's
required to allocate memory for storage of a vector consisting
of 1000 units:
\begin{Verbatim}
rvector v(1000);
\end{Verbatim}
If at the moment of execution of this statement there is no free
block of size 8000 or greater in a list of free memory blocks, then
the following (simplified) code will be executed:
\begin{Verbatim}
const int nUpBytes = up_value (nBytes);
const int nRest    = nUpBytes - nBytes;
try {
    pB = ::new tbyte[nUpBytes];
}
catch (const std::bad_alloc&) {
    throw (cvmexception (CVM_OUTOFMEMORY));
}
m_lOutBlocks.push_back (pB);
m_blocks.AddPair (pB, nBytes, nRest);
\end{Verbatim}
where variable \verb"nBytes" has value of 8000
(\verb"1000*sizeof(treal)"), and function \verb"up_value" returns
the nearest degree of two multiplied by two, i.e. 16384. Thus, the
allocated outer block can be represented as follows:

\noindent
\unitlength=1mm
\begin{picture}(120,13)(0,0)
\put(0,0){\includegraphics[width=340mm]{block1.pdf}}
    \put(0,7){\texttt{pB}}
    \put(50,7){\texttt{pB+8000}}
    \put(118,7){\texttt{pB+16384}}
\end{picture}\\
\unitlength=1pt The block used for storage 1000 units of a vector
\verb"v" (its start address here is stored in the pointer \verb"pB")
is filled. The remaining block (named as ``free block'') consists of
\verb"nRest=8384" bytes. Further, if application needs one more
block of the same size (it happens in most cases while execution of
a copy constructor), the memory will be allocated from this free
block without calling of operator \verb"new". The result will be the
following:

\noindent
\unitlength=1mm
\begin{picture}(120,20)(0,0)
\put(0,0){\includegraphics[width=340mm]{block2.pdf}}
    \put(0,7){\texttt{pB}}
    \put(50,7){\texttt{pB+8000}}
    \put(106,14){\texttt{pB+16000}}
    \put(118,7){\texttt{pB+16384}}
    \put(115.5,13){\vector(0, -1){7}}
\end{picture}\\
\unitlength=1pt Remaining free block of 384 bytes will be utilized
for memory allocation of small objects. In case of creation of an
object of size greater than 384 bytes one more outer block will be
created, etc. While using of this scheme sooner or later memory
begins to be like a sieve of free and occupied blocks. To avoid this
chaos, I have applied the logic of blocks releasing, which acts like
an algorithm of free space compaction. Difference of this logic from
the algorithms of compaction described in \GO{\cite{JAlger}}{biblio}
is that occupied blocks are never being moved.

Let's say one of outer blocks looks like

\noindent
\unitlength=1mm
\begin{picture}(130,7)(0,0)
\put(0,0){\includegraphics[width=340mm]{block3.pdf}}
    \put(5.5,1.5){\textsf{a}}
    \put(27,1.5){\textsf{b}}
    \put(63,1.5){\textsf{c}}
    \put(97.5,1.5){\textsf{d}}
    \put(117.5,1.5){\textsf{e}}
\end{picture}\\
\unitlength=1pt And the block \textsf{d} is going to be released. In
this case neighbor blocks \textsf{c} and \textsf{e} will be checked
and if one of them (or both, as in this case) will appear as free,
then it will be joined with the block being released. The result of
release of the block \textsf{d} will be the following:

\noindent
\unitlength=1mm
\begin{picture}(130,7)(0,0)
\put(0,0){\includegraphics[width=340mm]{block4.pdf}}
    \put(5.5,1.5){\textsf{a}}
    \put(27,1.5){\textsf{b}}
    \put(85,1.5){\textsf{c}}
\end{picture}\\
\unitlength=1pt
And if the following object to be created will not
find a room in the block \textsf{a},
but find a room in the block \textsf{c}, the result of memory
allocation will be the following:

\noindent
\unitlength=1mm
\begin{picture}(130,7)(0,0)
\put(0,0){\includegraphics[width=340mm]{block5.pdf}}
    \put(5.5,1.5){\textsf{a}}
    \put(27,1.5){\textsf{b}}
    \put(63,1.5){\textsf{c}}
    \put(105,1.5){\textsf{d}}
\end{picture}\\
\unitlength=1pt
So, there is no any chaos.
\newpage

%\subsubsection{Allocator}
%Since%
%\pdfdest name {SubSubSectionAllocator} fit{}
%version $5.1$ the library uses \verb"std::allocator".
%However, you can rebuild the whole library using you own
%customized allocator. Just add \verb"-DCVM_ALLOCATOR=MyAllocator" to your
%compiler command line or define
%\begin{Verbatim}
%#define CVM_ALLOCATOR MyAllocator
%#include <cvm.h>
%\end{Verbatim}

%\if 0
%\subsubsection{CVMAllocator}
%This mechanism is wrapped by%
%\pdfdest name {SubSubSectionCVMAllocator} fit{} \verb"CVMAllocator"
%since version 5.0. You can reuse this allocator in your own projects
%as well. For example:
%\begin{Verbatim}
%typedef std::basic_string <char,
%                           std::char_traits<char>,
%                           CVMAllocator<char> >  mystring;
%....
%mystring str ("Hello, World!");
%std::cout << str << std::endl;
%\end{Verbatim}
%
%Moreover, you can specify your own allocator (or use a standard one)
%for a particular object of CVM library:
%\begin{Verbatim}
%typedef cvm::basic_rvector <double, std::allocator>  myvector;
%...
%myvector v(10);
%\end{Verbatim}
%Please note that type name is not required to be specified for
%allocator in this case, since CVM library uses first template parameter.
%This way prevents from wrong allocation sizes. Please also note that
%this feature works for MS~.NET~2003 and GCC~3.x compilers only.
%Finally, you can rebuild the whole library for using you own
%allocator. Just add \verb"-DCVM_ALLOCATOR=MyAllocator" to your
%compiler command line or define
%\begin{Verbatim}
%#define CVM_ALLOCATOR MyAllocator
%#include <cvm.h>
%\end{Verbatim}
%This feature is also available for MS~.NET~2003 and GCC~3.x
%compilers only.
%\fi



