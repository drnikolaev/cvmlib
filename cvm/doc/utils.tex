\subsection{Utilities}
These%
\pdfdest name {Utilities} fit{}
functions have \verb"cvm" namespace scope and
can be used for different purposes.

\bigskip
\noindent
\verb"template <typename T>"\\
\verb"T* "\GOT{cvmMalloc}{Utilities.cvmMalloc}\verb" (size_t nEls) throw (cvmexception);"\\
\verb"template <typename T>"\\
\verb"int "\GOT{cvmFree}{Utilities.cvmFree}\verb" (T*& pD);"\\
\verb"treal "\GOT{cvmMachMin}{Utilities.cvmMachMin}\verb" ();"\\
\verb"treal "\GOT{cvmMachSp}{Utilities.cvmMachSp}\verb" ();"\\
\verb"srmatrix "\GOT{eye\_real}{Utilities.eye_real}\verb" (int nM);"\\
\verb"scmatrix "\GOT{eye\_complex}{Utilities.eye_complex}\verb" (int nM);"\\
\GOT{operator * (,)}{Utilities.operator *}\verb";"\\
\verb""
\newpage




\subsubsection{cvmMalloc}
\textcolor{red}{Since version 6.0} %if \verb"CVM_USE_POOL_MANAGER" macro is not defined (this is the default)
this function simply calls \verb"::new T[nEls]" operator.%
\pdfdest name {Utilities.cvmMalloc} fit
\begin{verbatim}
template <typename T>
T* cvmMalloc (size_t nEls) throw (cvmexception);
\end{verbatim}
allocates \verb"nEls" units of type \verb"T"
from the CVM library's \GO{memory pool}{SubSectionFeatures} and returns
a pointer to the memory allocated.
See also \GOT{cvmAddRef}{Utilities.cvmAddRef} and
\GOT{cvmFree}{Utilities.cvmFree}.
Example:
\begin{Verbatim}
using namespace cvm;

double* p = cvmMalloc<double> (10);
p[0] = 1.;
p[1] = 2.;
p[2] = 3.;

rvector v(3);
v.assign(p);
std::cout << v;

cvmFree (p);
\end{Verbatim}
prints
\begin{Verbatim}
1 2 3
\end{Verbatim}
\newpage




\subsubsection{cvmFree}
\textcolor{red}{Since version 6.0} %if \verb"CVM_USE_POOL_MANAGER" macro is not defined (this is the default) then it
this function simply calls \verb"::delete[] pD" operator if \verb"pD!=NULL" and returns zero.%
\pdfdest name {Utilities.cvmFree} fit
\begin{verbatim}
template <typename T>
int cvmFree (T*& pD);
\end{verbatim}
decrements a reference counter for a memory block pointed to
by \verb"pD" if this block was allocated from the CVM library's
memory pool (using \GOT{cvmMalloc}{Utilities.cvmMalloc} function)
and returns the reference counter it changed.
If the function returns zero then it sets the pointer
\verb"pD" to be equal to \verb"NULL" and "frees" the memory, i.e.
returns the memory block to a list of free ones (see \GO{CVM memory
management}{SubSectionFeatures} for details).
If \verb"pD" points to a foreign memory block then
the function does nothing and returns $-1$.
See also \GOT{cvmAddRef}{Utilities.cvmAddRef}.
Example:
\begin{Verbatim}
using namespace cvm;

double* pf = new double[10];
double* p  = cvmMalloc<double> (10);

cvmAddRef (p);

std::cout << cvmFree (p) << " ";
std::cout << p << std::endl;

std::cout << cvmFree (p) << " ";
std::cout << p << std::endl;

std::cout << cvmFree (pf) << " ";
std::cout << pf << std::endl;

delete[] pf;
\end{Verbatim}
prints
\begin{Verbatim}
1 003C66B0
0 00000000
-1 003C7A40
\end{Verbatim}
\newpage



\subsubsection{cvmMachMin}
Function%
\pdfdest name {Utilities.cvmMachMin} fit
\begin{verbatim}
treal cvmMachMin ();
\end{verbatim}
returns the smallest normalized positive number,
i.e. \verb"numeric_limits<treal>::min()"
where \verb"treal" is \verb"typedef"'ed as \verb"double"
by default or as \verb"float" for float version of the library.
See also \GOT{cvmMachSp}{Utilities.cvmMachSp}.
Example:
\begin{Verbatim}
using namespace cvm;

std::cout.setf (std::ios::scientific | std::ios::showpos);
std::cout.precision (15);
std::cout << cvmMachMin() << std::endl;
\end{Verbatim}
on Intel Pentium\textcircled{\scriptsize{R}} III machine prints
\begin{Verbatim}
+2.225073858507201e-308
\end{Verbatim}
\newpage


\subsubsection{cvmMachSp}
Function%
\pdfdest name {Utilities.cvmMachSp} fit
\begin{verbatim}
treal cvmMachSp ();
\end{verbatim}
returns the largest relative spacing or, in other words,
the difference between $1$ and the least value greater 
than $1$ that is representable,
i.e. \verb"numeric_limits<treal>::epsilon()"
where \verb"treal" is \verb"typedef"'ed as \verb"double"
by default or as \verb"float" for float version of the library.
See also \GOT{cvmMachMin}{Utilities.cvmMachMin}.
Example:
\begin{Verbatim}
using namespace cvm;

std::cout.setf (std::ios::scientific | std::ios::showpos);
std::cout.precision (15);
std::cout << cvmMachSp() << std::endl;
\end{Verbatim}
on Intel Pentium\textcircled{\scriptsize{R}} III machine prints
\begin{Verbatim}
+2.220446049250313e-016
\end{Verbatim}
\newpage


\subsubsection{eye\_real}
Function%
\pdfdest name {Utilities.eye_real} fit
\begin{verbatim}
srmatrix eye_real (int nM);
\end{verbatim}
creates a \verb"nM" by \verb"nM" object of type 
\verb"srmatrix" 
equal to identity matrix.
See also \GOT{srmatrix}{srmatrix}.
Example:
\begin{Verbatim}
using namespace cvm;

std::cout << eye_real (4);
\end{Verbatim}
prints
\begin{Verbatim}
1 0 0 0
0 1 0 0
0 0 1 0
0 0 0 1
\end{Verbatim}
\newpage



\subsubsection{eye\_complex}
Function%
\pdfdest name {Utilities.eye_complex} fit
\begin{verbatim}
scmatrix eye_complex (int nM);
\end{verbatim}
creates a \verb"nM" by \verb"nM" object of type 
\verb"scmatrix" 
equal to identity matrix.
See also \GOT{scmatrix}{scmatrix}.
Example:
\begin{Verbatim}
using namespace cvm;

std::cout << eye_complex (4);
\end{Verbatim}
prints
\begin{Verbatim}
(1,0) (0,0) (0,0) (0,0)
(0,0) (1,0) (0,0) (0,0)
(0,0) (0,0) (1,0) (0,0)
(0,0) (0,0) (0,0) (1,0)
\end{Verbatim}
\newpage


\subsubsection{operator *}
Operators%
\pdfdest name {Utilities.operator *} fit
\begin{verbatim}
inline rvector operator * (TR d, const rvector& v);
inline rmatrix operator * (TR d, const rmatrix& m);
inline srmatrix operator * (TR d, const srmatrix& m);
inline srbmatrix operator * (TR d, const srbmatrix& m);
inline srsmatrix operator * (TR d, const srsmatrix& m);
inline cvector operator * (TR d, const cvector& v);
inline cmatrix operator * (TR d, const cmatrix& m);
inline scmatrix operator * (TR d, const scmatrix& m);
inline scbmatrix operator * (TR d, const scbmatrix& m);
inline schmatrix operator * (TR d, const schmatrix& m);
inline cvector operator * (std::complex<TR> c, const cvector& v);
inline cmatrix operator * (std::complex<TR> c, const cmatrix& m);
inline scmatrix operator * (std::complex<TR> c, const scmatrix& m);
inline scbmatrix operator * (std::complex<TR> c, const scbmatrix& m);
inline schmatrix operator * (std::complex<TR> c, const schmatrix& m);
inline rvector operator * (CVM_LONGEST_INT d, const rvector& v);
inline rmatrix operator * (CVM_LONGEST_INT d, const rmatrix& m);
inline srmatrix operator * (CVM_LONGEST_INT d, const srmatrix& m);
inline srbmatrix operator * (CVM_LONGEST_INT d, const srbmatrix& m);
inline srsmatrix operator * (CVM_LONGEST_INT d, const srsmatrix& m);
inline cvector operator * (CVM_LONGEST_INT d, const cvector& v);
inline cmatrix operator * (CVM_LONGEST_INT d, const cmatrix& m);
inline scmatrix operator * (CVM_LONGEST_INT d, const scmatrix& m);
inline scbmatrix operator * (CVM_LONGEST_INT d, const scbmatrix& m);
inline schmatrix operator * (CVM_LONGEST_INT d, const schmatrix& m);
\end{verbatim}
provide an ability to make left-sided multiplication of numbers
by different CVM objects.
Example:
\begin{Verbatim}
using namespace cvm;

const schmatrix scm = eye_complex (4);
std::cout << std::complex<double>(2.,1.) * scm << std::endl;

rvector v(3);
v(1) = 1.;
v(2) = 2.;
v(3) = 3.;

std::cout << 3. * v;
\end{Verbatim}
prints
\begin{Verbatim}
(2,1) (0,0) (0,0) (0,0)
(0,0) (2,1) (0,0) (0,0)
(0,0) (0,0) (2,1) (0,0)
(0,0) (0,0) (0,0) (2,1)

3 6 9
\end{Verbatim}
\newpage
