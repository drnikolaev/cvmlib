\bigskip
\noindent
\verb"template <typename TR, typename TC>"\\
\verb"class schmatrix : public scmatrix <TR,TC> {"\\
\verb"public:"\\
\verb"    "\GOT{schmatrix}{schmatrix.schmatrix ()}\verb" ();"\\
\verb"    explicit "\GOT{schmatrix}{schmatrix.schmatrix (int)}\verb" (int nMN);"\\
\verb"    "\GOT{schmatrix}{schmatrix.schmatrix (TC*,int)}\verb" (TC* pD, int nMN, TR tol = cvmMachSp());"\\
\verb"    "\GOT{schmatrix}{schmatrix.schmatrix (const TC*,int)}\verb" (const TC* pD, int nMN, TR tol = cvmMachSp());"\\
\verb"    "\GOT{schmatrix}{schmatrix.schmatrix (const schmatrix&)}\verb" (const schmatrix& m);"\\
\verb"    explicit "\GOT{schmatrix}{schmatrix.schmatrix (const cmatrix&)}\verb" (const cmatrix& m, TR tol = cvmMachSp());"\\
\verb"    explicit "\GOT{schmatrix}{schmatrix.schmatrix (const rvector&)}\verb" (const rvector& v);"\\
\verb"    explicit "\GOT{schmatrix}{schmatrix.schmatrix (const srsmatrix&)}\verb" (const srsmatrix& m);"\\
\verb"    "\GOT{schmatrix}{schmatrix.schmatrix (TR*,TR*,int)}\verb" (const TR* pRe, const TR* pIm, int nMN,"\\
\verb"               TR tol = cvmMachSp());"\\
\verb"    "\GOT{schmatrix}{schmatrix.schmatrix (const srmatrix&, const srmatrix&)}\verb" (const srmatrix& mRe, const srmatrix& mIm,"\\
\verb"               TR tol = cvmMachSp());"\\
\verb"    "\GOT{schmatrix}{schmatrix.submatrixctr}\verb" (schmatrix& m, int nRowCol, int size);"\\
\verb"    TC "\GOT{operator ()}{schmatrix.operator (,)}\verb" (int im, int in) const throw (cvmexception);"\\
\verb"    const cvector "\GOT{operator ()}{schmatrix.operator ()}\verb" (int i) const throw (cvmexception);"\\
\verb"    const cvector "\GOT{operator []}{schmatrix.operator []}\verb" (int i) const throw (cvmexception);"\\
\verb"    const cvector "\GOT{diag}{schmatrix.diag}\verb" (int i) const throw (cvmexception);"\\
\verb"    const srsmatrix "\GOT{real}{schmatrix.real}\verb" () const;"\\
\verb"    const srmatrix "\GOT{imag}{schmatrix.imag}\verb" () const;"\\
\verb"    schmatrix& "\GOT{operator =}{schmatrix.operator = (const schmatrix&)}\verb" (const schmatrix& m) throw (cvmexception);"\\
\verb"    schmatrix& "\GOT{assign}{schmatrix.assign}\verb" (const cvector& v, TR tol = cvmMachSp())"\\
\verb"                       throw (cvmexception);"\\
\verb"    schmatrix& "\GOT{assign}{schmatrix.assign}\verb" (const TC* pD, TR tol = cvmMachSp())"\\
\verb"                       throw (cvmexception);"\\
\verb"    schmatrix& "\GOT{assign}{schmatrix.assign (int, const schmatrix&)}\verb" (int nRowCol, const schmatrix& m)"\\
\verb"                       throw (cvmexception);"\\
\verb"    schmatrix& "\GOT{set}{schmatrix.set(int,int,TC)}\verb" (int nRow, int nCol, TC z);"\\
\verb"    schmatrix& "\GOT{set\_diag}{schmatrix.set_diag}\verb" (int i, const cvector& v) throw (cvmexception);"\\
\verb"    schmatrix& "\GOT{set\_main\_diag}{schmatrix.set_main_diag}\verb" (const rvector& v) throw (cvmexception);"\\
\verb"    schmatrix& "\GOT{assign\_real}{schmatrix.assign_real}\verb" (const srsmatrix& m) throw (cvmexception);"\\
\verb"    schmatrix& "\GOT{set\_real}{schmatrix.set_real}\verb" (TR d) throw (cvmexception);"\\
\verb"    schmatrix& "\GOT{resize}{schmatrix.resize}\verb" (int nNewMN) throw (cvmexception);"\\
\verb"    bool "\GOT{operator ==}{schmatrix.operator ==}\verb" (const schmatrix& v) const;"\\
\verb"    bool "\GOT{operator !=}{schmatrix.operator !=}\verb" (const schmatrix& v) const;"\\
\verb"    schmatrix& "\GOT{operator <{}<}{schmatrix.operator <<}\verb" (const schmatrix& m) throw (cvmexception);"\\
\verb"    schmatrix "\GOT{operator +}{schmatrix.operator +}\verb" (const schmatrix& m) const"\\
\verb"                          throw (cvmexception);"\\
\verb"    schmatrix "\GOT{operator -}{schmatrix.operator -}\verb" (const schmatrix& m) const"\\
\verb"                          throw (cvmexception);"\\
\verb"    schmatrix& "\GOT{sum}{schmatrix.sum}\verb" (const schmatrix& m1,"\\
\verb"                    const schmatrix& m2) throw (cvmexception);"\\
\verb"    schmatrix& "\GOT{diff}{schmatrix.diff}\verb" (const schmatrix& m1,"\\
\verb"                     const schmatrix& m2) throw (cvmexception);"\\
\verb"    schmatrix& "\GOT{operator +=}{schmatrix.operator +=}\verb" (const schmatrix& m) throw (cvmexception);"\\
\verb"    schmatrix& "\GOT{operator -=}{schmatrix.operator -=}\verb" (const schmatrix& m) throw (cvmexception);"\\
\verb"    schmatrix "\GOT{operator -}{schmatrix.operator - ()}\verb" () const;"\\
\verb"    schmatrix& "\GOT{operator ++}{schmatrix.operator ++}\verb" ();"\\
\verb"    schmatrix& "\GOT{operator ++}{schmatrix.operator ++}\verb" (int);"\\
\verb"    schmatrix& "\GOT{operator -{}-}{schmatrix.operator --}\verb" ();"\\
\verb"    schmatrix& "\GOT{operator -{}-}{schmatrix.operator --}\verb" (int);"\\
\verb"    schmatrix "\GOT{operator *}{schmatrix.operator * (TR)}\verb" (TR d) const;"\\
\verb"    schmatrix "\GOT{operator /}{schmatrix.operator / (TR)}\verb" (TR d) const throw (cvmexception);"\\
\verb"    scmatrix "\GOT{operator *}{schmatrix.operator * (TC)}\verb" (TC z) const;"\\
\verb"    scmatrix "\GOT{operator /}{schmatrix.operator / (TC)}\verb" (TC z) const throw (cvmexception);"\\
\verb"    schmatrix& "\GOT{operator *=}{schmatrix.operator *= (TR)}\verb" (TR d);"\\
\verb"    schmatrix& "\GOT{operator /=}{schmatrix.operator /= (TR)}\verb" (TR d) throw (cvmexception);"\\
\verb"    schmatrix& "\GOT{normalize}{schmatrix.normalize}\verb" ();"\\
\verb"    schmatrix "\GOT{operator \TildaT}{schmatrix.conj}\verb" () const;"\\
\verb"    schmatrix "\GOT{operator !}{schmatrix.transpose}\verb" () const;"\\
\verb"    schmatrix& "\GOT{conj}{schmatrix.conj}\verb" (const schmatrix& m) throw (cvmexception);"\\
\verb"    schmatrix& "\GOT{conj}{schmatrix.conj}\verb" ();"\\
\verb"    schmatrix& "\GOT{transpose}{schmatrix.transpose}\verb" (const schmatrix& m) throw (cvmexception);"\\
\verb"    schmatrix& "\GOT{transpose}{schmatrix.transpose}\verb" ();"\\
\verb"    cvector "\GOT{operator *}{schmatrix.operator * (const cvector&)}\verb" (const cvector& v) const throw (cvmexception);"\\
\verb"    cmatrix "\GOT{operator *}{schmatrix.operator * (const cmatrix&)}\verb" (const cmatrix& m) const throw (cvmexception);"\\
\verb"    scmatrix "\GOT{operator *}{schmatrix.operator * (const scmatrix&)}\verb" (const scmatrix& m) const throw (cvmexception);"\\
\verb"    cvector "\GOT{operator /}{schmatrix.operator / (cvector)}\verb" (const cvector& vB) const throw (cvmexception);"\\
\verb"    schmatrix& "\GOT{herk}{schmatrix.herk}\verb" (TR alpha,"\\
\verb"                     const cvector& v, TR beta) throw (cvmexception);"\\
\verb"    schmatrix& "\GOT{herk}{schmatrix.herk}\verb" (bool bTransp, TR alpha,"\\
\verb"                     const cmatrix& m, TR beta) throw (cvmexception);"\\
\verb"    schmatrix& "\GOT{her2k}{schmatrix.her2k}\verb" (TC alpha,"\\
\verb"                     const cvector& v1, const cvector& v2, TR beta)"\\
\verb"                     throw (cvmexception);"\\
\verb"    schmatrix& "\GOT{her2k}{schmatrix.her2k}\verb" (bool bTransp, TC alpha,"\\
\verb"                     const cmatrix& m1, const cmatrix& m2, TR beta)"\\
\verb"                     throw (cvmexception);"\\
\verb"    schmatrix& "\GOT{inv}{schmatrix.inv}\verb" (const schmatrix& mArg) throw (cvmexception);"\\
\verb"    schmatrix "\GOT{inv}{schmatrix.inv}\verb" () const throw (cvmexception);"\\
\verb"    schmatrix& "\GOT{exp}{schmatrix.exp}\verb" (const schmatrix& m,"\\
\verb"                    TR tol = cvmMachSp ()) throw (cvmexception);"\\
\verb"    schmatrix "\GOT{exp}{schmatrix.exp}\verb" (TR tol = cvmMachSp ()) const throw (cvmexception);"\\
\verb"    schmatrix& "\GOT{polynom}{schmatrix.polynom}\verb" (const schmatrix& m, const rvector& v)"\\
\verb"                        throw (cvmexception);"\\
\verb"    schmatrix "\GOT{polynom}{schmatrix.polynom}\verb" (const rvector& v) const throw (cvmexception);"\\
\verb"    rvector "\GOT{eig}{schmatrix.eig}\verb" (scmatrix& mEigVect) const throw (cvmexception);"\\
\verb"    rvector "\GOT{eig}{schmatrix.eig}\verb" () const throw (cvmexception);"\\
\verb"    scmatrix "\GOT{cholesky}{schmatrix.cholesky}\verb" () const throw (cvmexception);"\\
\verb"    scmatrix "\GOT{bunch\_kaufman}{schmatrix.bunch_kaufman}\verb" () const throw (cvmexception);"\\
\verb"    schmatrix& "\GOT{identity}{schmatrix.identity}\verb" ();"\\
\verb"    schmatrix& "\GOT{vanish}{schmatrix.vanish}\verb" ();"\\
\verb"    schmatrix& "\GOT{randomize\_real}{schmatrix.randomizereal}\verb" (TR dFrom, TR dTo);"\\
\verb"    schmatrix& "\GOT{randomize\_imag}{schmatrix.randomizeimag}\verb" (TR dFrom, TR dTo);"\\
\verb"};"
\newpage


\subsubsection{schmatrix ()}
Constructor%
\pdfdest name {schmatrix.schmatrix ()} fit
\begin{verbatim}
schmatrix::schmatrix ();
\end{verbatim}
creates  empty \verb"schmatrix" object.
See also \GOT{schmatrix}{schmatrix}.
Example:
\begin{Verbatim}
using namespace cvm;

schmatrix m;
std::cout << m.msize() << " " << m.nsize() << " " 
          << m.size() << std::endl;

m.resize (3);
std::cout << m;
\end{Verbatim}
prints
\begin{Verbatim}
0 0 0
(0,0) (0,0) (0,0)
(0,0) (0,0) (0,0)
(0,0) (0,0) (0,0)
\end{Verbatim}
\newpage



\subsubsection{schmatrix (int)}
Constructor%
\pdfdest name {schmatrix.schmatrix (int)} fit
\begin{verbatim}
explicit schmatrix::schmatrix (int nMN);
\end{verbatim}
creates  $n\times n$ \verb"schmatrix" object where $n$ is passed in
\verb"nMN" parameter.
Constructor throws  \GOT{cvmexception}{cvmexception}
in case of non-positive size passed or memory allocation failure.
See also \GOT{schmat\-rix}{schmatrix}.
Example:
\begin{Verbatim}
using namespace cvm;

schmatrix m (4);
std::cout << m.msize() << std::endl 
          << m.nsize() << std::endl
          << m.size() << std::endl << m;
\end{Verbatim}
prints
\begin{Verbatim}
4
4
16
(0,0) (0,0) (0,0) (0,0)
(0,0) (0,0) (0,0) (0,0)
(0,0) (0,0) (0,0) (0,0)
(0,0) (0,0) (0,0) (0,0)
\end{Verbatim}
\newpage




\subsubsection{schmatrix (TC*,int)}
Constructor%
\pdfdest name {schmatrix.schmatrix (TC*,int)} fit
\begin{verbatim}
schmatrix::schmatrix (TC* pD, int nMN, TR tol = cvmMachSp());
\end{verbatim}
creates  $n\times n$ \verb"scmatrix" object where $n$ is passed in
\verb"nMN" parameter.
Unlike others, this constructor \textit{does not allocate  memory}.
It just shares  memory with an array pointed to by \verb"pD".
Constructor throws  \GOT{cvmexception}{cvmexception}
if the matrix created doesn't appear to be hermitian (tolerance is set by
parameter \verb'tol') or in case of non-positive size passed. 
If subsequent application flow would change the array passed so
it becomes not  hermitian matrix anymore then results are not predictable.
See also \GOT{schmatrix}{schmatrix},
\GOT{schmatrix (const TC*,int)}{schmatrix.schmatrix (const TC*,int)}.
Example:
\begin{Verbatim}
using namespace cvm;
double a[] = {1., 0., 2., 1., -1., 2., 2., -1., 2., 0.,
              0., 3., -1., -2., 0., -3., 3., 0.};
schmatrix m ((std::complex<double>*)a, 3);
m.set(2,1,std::complex<double>(8.,8.));
std::cout << m << std::endl;
std::cout << a[0] << " " << a[1] << " " << a[2] << " "
          << a[3] << " " << a[4] << " " << a[5] << " " << std::endl;
\end{Verbatim}
prints
\begin{Verbatim}
(1,0) (8,-8) (-1,-2)
(8,8) (2,0) (0,-3)
(-1,2) (0,3) (3,0)

1 0 8 8 -1 2
\end{Verbatim}
\newpage


\subsubsection{schmatrix (const TC*,int)}
Constructor%
\pdfdest name {schmatrix.schmatrix (const TC*,int)} fit
\begin{verbatim}
schmatrix::schmatrix (const TC* pD, int nMN, TR tol = cvmMachSp());
\end{verbatim}
creates  $n\times n$ \verb"scmatrix" object where $n$ is passed in
\verb"nMN" parameter
and copies $n*n$ elements of an array  \verb"pD" to it \GOT{by colums}{SubSubSectionStorage}.
Constructor throws  \GOT{cvmexception}{cvmexception}
if the matrix created doesn't appear to be hermitian (tolerance is set by
parameter \verb'tol') or in case of non-positive size passed. 
See also \GOT{schmatrix}{schmatrix},
\GOT{schmatrix (TC*,int)}{schmatrix.schmatrix (TC*,int)}.
Example:
\begin{Verbatim}
using namespace cvm;
const double a[] = {1., 0., 2., 1., -1., 2., 2., -1., 2., 0.,
                    0., 3., -1., -2., 0., -3., 3., 0.};
schmatrix m ((const std::complex<double>*)a, 3);
m.set(2,1,std::complex<double>(8.,8.));
std::cout << m << std::endl;
std::cout << a[0] << " " << a[1] << " " << a[2] << " "
          << a[3] << " " << a[4] << " " << a[5] << " " << std::endl;
\end{Verbatim}
prints
\begin{Verbatim}
(1,0) (8,-8) (-1,-2)
(8,8) (2,0) (0,-3)
(-1,2) (0,3) (3,0)

1 0 2 1 -1 2
\end{Verbatim}
\newpage



\subsubsection{schmatrix (const schmatrix\&)}
Copy constructor%
\pdfdest name {schmatrix.schmatrix (const schmatrix&)} fit
\begin{verbatim}
schmatrix::schmatrix (const schmatrix& m)
\end{verbatim}
creates  \verb"schmatrix" object as a copy of \verb"m".
Constructor throws  \GOT{cvmexception}{cvmexception}
in case of memory allocation failure.
See also \GOT{schmatrix}{schmatrix}.
Example:
\begin{Verbatim}
using namespace cvm;

std::cout.setf (std::ios::scientific | std::ios::left |
                std::ios::showpos); 
std::cout.precision (1);
double a[] = {1., 0., 2., 1., -1., 2., 2., -1., 2., 0.,
              0., 3., -1., -2., 0., -3., 3., 0.};
schmatrix m ((std::complex<double>*)a, 3);
scmatrix mc(m);

m.set(1,2, std::complex<double>(7.7,7.7));
std::cout << m << std::endl << mc;
\end{Verbatim}
prints
\begin{Verbatim}
(+1.0e+000,+0.0e+000) (+7.7e+000,+7.7e+000) (-1.0e+000,-2.0e+000)
(+7.7e+000,-7.7e+000) (+2.0e+000,+0.0e+000) (+0.0e+000,-3.0e+000)
(-1.0e+000,+2.0e+000) (+0.0e+000,+3.0e+000) (+3.0e+000,+0.0e+000)

(+1.0e+000,+0.0e+000) (+2.0e+000,-1.0e+000) (-1.0e+000,-2.0e+000)
(+2.0e+000,+1.0e+000) (+2.0e+000,+0.0e+000) (+0.0e+000,-3.0e+000)
(-1.0e+000,+2.0e+000) (+0.0e+000,+3.0e+000) (+3.0e+000,+0.0e+000)
\end{Verbatim}
\newpage





\subsubsection{schmatrix (const cmatrix\&)}
Constructor%
\pdfdest name {schmatrix.schmatrix (const cmatrix&)} fit
\begin{verbatim}
explicit schmatrix::schmatrix (const cmatrix& m, TR tol = cvmMachSp())
\end{verbatim}
creates  \verb"schmatrix" object as a copy of a matrix \verb"m".
It's assumed that $m\times n$ matrix \verb"m" must have equal
sizes, i.e. $m = n$ is satisfied and it has to be  hermitian one
(tolerance is set by parameter \verb'tol').
Constructor throws  \GOT{cvmexception}{cvmexception}
if this is not true or in case of memory allocation failure.
See also \GOT{schmatrix}{schmatrix} and \GOT{cmatrix}{cmatrix}.
Example:
\begin{Verbatim}
using namespace cvm;

double a[] = {1., 0., 2., 1., -1., 2., 2., -1., 2., 0.,
              0., 3., -1., -2., 0., -3., 3., 0.};
cmatrix m ((std::complex<double>*)a, 3, 3);
scmatrix mch(m);
std::cout << mch;
\end{Verbatim}
prints
\begin{Verbatim}
(1,0) (2,-1) (-1,-2)
(2,1) (2,0) (0,-3)
(-1,2) (0,3) (3,0)
\end{Verbatim}
\newpage




\subsubsection{schmatrix (const rvector\&)}
Constructor%
\pdfdest name {schmatrix.schmatrix (const rvector&)} fit
\begin{verbatim}
explicit schmatrix::schmatrix (const rvector& v);
\end{verbatim}
creates  \verb"schmatrix" object
of size \verb"v.size()" by \verb"v.size()"
and assigns vector \verb"v" to its main diagonal.
Constructor throws  \GOT{cvmexception}{cvmexception}
in case of memory allocation failure.
See also \GOT{schmatrix}{schmatrix} and \GOT{rvector}{rvector}.
Example:
\begin{Verbatim}
using namespace cvm;

double a[] = {1., 2., 3., 4., 5.};
const rvector v (a, 5);
schmatrix m(v);
std::cout << m.msize() << " " << m.nsize() << " " 
          << m.size() << std::endl << m;
\end{Verbatim}
prints
\begin{Verbatim}
5 5 25
(1,0) (0,0) (0,0) (0,0) (0,0)
(0,0) (2,0) (0,0) (0,0) (0,0)
(0,0) (0,0) (3,0) (0,0) (0,0)
(0,0) (0,0) (0,0) (4,0) (0,0)
(0,0) (0,0) (0,0) (0,0) (5,0)
\end{Verbatim}
\newpage



\subsubsection{schmatrix (const srsmatrix\&)}
Constructor%
\pdfdest name {schmatrix.schmatrix (const srsmatrix&)} fit
\begin{verbatim}
explicit schmatrix::schmatrix (const srsmatrix& m);
\end{verbatim}
creates  \verb"schmatrix" object
having the same dimension as real symmetric matrix \verb"m"
and copies  matrix \verb"m" to its real part.
See also \GOT{schmatrix}{schmatrix} and 
\GOT{srsmatrix}{srsmatrix}.
Example:
\begin{Verbatim}
using namespace cvm;

double a[] = {1., 2., 3., 2., 5., 6., 3., 6., 9.};
srsmatrix m(a, 3);
schmatrix mch(m);
std::cout << mch;
\end{Verbatim}
prints
\begin{Verbatim}
(1,0) (2,0) (3,0)
(2,0) (5,0) (6,0)
(3,0) (6,0) (9,0)
\end{Verbatim}
\newpage




\subsubsection{schmatrix (const TR*,const TR*,int)}
Constructor%
\pdfdest name {schmatrix.schmatrix (TR*,TR*,int)} fit
\begin{verbatim}
schmatrix::schmatrix (const TR* pRe, const TRl* pIm, int nMN,
                      TR tol = cvmMachSp());
\end{verbatim}
creates  \verb"schmatrix" object
of size \verb"nMN" by \verb"nMN" and copies every
element of arrays pointed to by \verb"pRe" and \verb"pIm"
to  real and imaginary part of the matrix created respectively.
Use \verb"NULL" pointer to fill up appropriate
part with zero values.
Constructor throws  \GOT{cvmexception}{cvmexception}
if the matrix created doesn't appear to be hermitian (tolerance is et by parameter \verb'tol')
or in case of memory allocation failure.
See also \GOT{schmatrix}{schmatrix}.
Example:
\begin{Verbatim}
using namespace cvm;

double re[] = {1., 2., -1., 2., 2., 0., -1., 0., 3.};
double im[] = {0., 1., 2., -1., 0., 3., -2., -3., 0.};
schmatrix m(re, im, 3);
std::cout << m;
\end{Verbatim}
prints
\begin{Verbatim}
(1,0) (2,-1) (-1,-2)
(2,1) (2,0) (0,-3)
(-1,2) (0,3) (3,0)
\end{Verbatim}
\newpage





\subsubsection{schmatrix (const srmatrix\&, const srmatrix\&)}
Constructor%
\pdfdest name {schmatrix.schmatrix (const srmatrix&, const srmatrix&)} fit
\begin{verbatim}
schmatrix::schmatrix (const srmatrix& mRe, const srmatrix& mIm,
                      TR tol = cvmMachSp());
\end{verbatim}
creates  \verb"schmatrix" object
of the same size as \verb"mRe" and \verb"mIm" has
(it throws \GOT{cvmexception}{cvmexception}
if \verb"mRe" and
\verb"mIm" have different sizes)
and copies matrices \verb"mRe" and \verb"mIm"
to  real and imaginary part of the matrix created respectively.
Constructor throws  \GOT{cvmexception}{cvmexception}
if the matrix created doesn't appear to be hermitian (tolerance is et by parameter \verb'tol')
or in case of memory allocation failure.
See also \GOT{schmatrix}{schmatrix}, \GOT{srmatrix}{srmatrix}.
Example:
\begin{Verbatim}
using namespace cvm;

double re[] = {1., 2., -1., 2., 2., 0., -1., 0., 3.};
double im[] = {0., 1., 2., -1., 0., 3., -2., -3., 0.};
srmatrix mr(re, 3);
srmatrix mi(im, 3);
schmatrix m(mr, mi);
std::cout << m;
\end{Verbatim}
prints
\begin{Verbatim}
(1,0) (2,-1) (-1,-2)
(2,1) (2,0) (0,-3)
(-1,2) (0,3) (3,0)
\end{Verbatim}
\newpage



\subsubsection{submatrix}
Submatrix constructor%
\pdfdest name {schmatrix.submatrixctr} fit
\begin{verbatim}
schmatrix::schmatrix (schmatrix& m, int nRowCol, int size);
\end{verbatim}
creates  \verb"scmatrix" object as  \emph{submatrix} of \verb"m".
It means that the matrix object created shares  memory with some part
of \verb"m". This part is defined by its upper left corner (parameter 
\verb"nRowCol", \Based)
and its size (parameter \verb"size").
See also \GOT{schmatrix}{schmatrix}.
Example:
\begin{Verbatim}
using namespace cvm;

double a[] = {1., 0., 2., 1., -1., 2., 2., -1., 2., 0.,
              0., 3., -1., -2., 0., -3., 3., 0.};
schmatrix m ((std::complex<double>*)a, 3);
schmatrix m2 (m, 1, 2);

std::cout << m2 << std::endl;
m2.set_real(7.7);
std::cout << m;
\end{Verbatim}
prints
\begin{Verbatim}
(1,0) (2,-1)
(2,1) (2,0)

(7.7,0) (7.7,-1) (-1,-2)
(7.7,1) (7.7,0) (0,-3)
(-1,2) (0,3) (3,0)
\end{Verbatim}
\newpage




\subsubsection{operator (,)}
Indexing operator%
\pdfdest name {schmatrix.operator (,)} fit
\begin{verbatim}
TC schmatrix::operator () (int im, int in) const throw (cvmexception);
\end{verbatim}
returns value of element of a calling hermitian matrix located on
\verb"im"-row and \verb"in"-th column. 
Operator is \Based.
It throws \GOT{cvmexception}{cvmexception}
if some of parameters passed
is outside of \verb"[1,msize()]" range.
See also \GOT{schmatrix}{schmatrix}.
Example:
\begin{Verbatim}
using namespace cvm;

double a[] = {1., 0., 2., 1., -1., 2., 2., -1., 2., 0.,
              0., 3., -1., -2., 0., -3., 3., 0.};
schmatrix m ((std::complex<double>*)a, 3);
std::cout << m << std::endl;
std::cout << m(3,2) << " " << m(1,3) << std::endl;
\end{Verbatim}
prints
\begin{Verbatim}
(1,0) (2,-1) (-1,-2)
(2,1) (2,0) (0,-3)
(-1,2) (0,3) (3,0)

(0,3) (-1,-2)
\end{Verbatim}
\newpage




\subsubsection{operator ()}
Indexing operator%
\pdfdest name {schmatrix.operator ()} fit
\begin{verbatim}
const cvector schmatrix::operator () (int i) const throw (cvmexception);
\end{verbatim}
provides access to \hbox{$i$-th} column of a calling hermitian matrix.
Unlike \GO{scmatrix::operator~()}{scmatrix.operator ()},
this operator creates only \emph{copy} of a column and therefore
it returns
\emph{not  l-value}.
Operator is \Based.
It throws \GOT{cvmexception}{cvmexception}
if parameter \verb"i" is outside of \verb"[1,nsize()]" range.
See also \GOT{schmatrix}{schmatrix}.
Example:
\begin{Verbatim}
using namespace cvm;

double a[] = {1., 0., 2., 1., -1., 2., 2., -1., 2., 0.,
              0., 3., -1., -2., 0., -3., 3., 0.};
schmatrix m ((std::complex<double>*)a, 3);
std::cout << m << std::endl;
std::cout << m(1) << std::endl << m(3);
\end{Verbatim}
prints
\begin{Verbatim}
(1,0) (2,-1) (-1,-2)
(2,1) (2,0) (0,-3)
(-1,2) (0,3) (3,0)

(1,0) (2,1) (-1,2)

(-1,-2) (0,-3) (3,0)
\end{Verbatim}
\newpage



\subsubsection{operator []}
Indexing operator%
\pdfdest name {schmatrix.operator []} fit
\begin{verbatim}
const cvector schmatrix::operator [] (int i) const throw (cvmexception);
\end{verbatim}
provides access to \hbox{$i$-th} row of a calling hermitian matrix.
Unlike \GO{scmatrix::operator~[]}{scmatrix.operator []},
this operator creates only \emph{copy} of a column and therefore
it returns
\emph{not  l-value}.
Operator is \Based.
It throws \GOT{cvmexception}{cvmexception}
if parameter \verb"i" is outside of \verb"[1,nsize()]" range.
See also \GOT{schmatrix}{schmatrix}.
Example:
\begin{Verbatim}
using namespace cvm;

double a[] = {1., 0., 2., 1., -1., 2., 2., -1., 2., 0.,
              0., 3., -1., -2., 0., -3., 3., 0.};
schmatrix m ((std::complex<double>*)a, 3);
std::cout << m << std::endl;
std::cout << m[1] << std::endl << m[3];
\end{Verbatim}
prints
\begin{Verbatim}
(1,0) (2,-1) (-1,-2)
(2,1) (2,0) (0,-3)
(-1,2) (0,3) (3,0)

(1,0) (2,-1) (-1,-2)

(-1,2) (0,3) (3,0)
\end{Verbatim}
\newpage



\subsubsection{diag}
Functions%
\pdfdest name {schmatrix.diag} fit
\begin{verbatim}
const cvector schmatrix::diag (int i) const throw (cvmexception);
\end{verbatim}
provide access to \hbox{$i$-th} diagonal of a calling matrix,
where $i=0$ for main diagonal, $i<0$ for lower diagonals 
and $i>0$ for upper ones.
Unlike \GO{cmatrix::diag}{cmatrix.diag},
this operator creates only  \emph{copy} of a diagonal and therefore 
it returns \emph{not  l-value}.
Function throws \GOT{cvmexception}{cvmexception}
if  parameter \verb"i" is outside of 
\verb"[-msize()+1,nsize()-1]" range.
See also \GOT{schmatrix}{schmatrix}.
Example:
\begin{Verbatim}
using namespace cvm;

double a[] = {1., 0., 2., 1., -1., 2., 2., -1., 2., 0.,
              0., 3., -1., -2., 0., -3., 3., 0.};
schmatrix m ((std::complex<double>*)a, 3);
std::cout << m << std::endl;
std::cout << m.diag(-2)
          << m.diag(-1)
          << m.diag(0)
          << m.diag(1)
          << m.diag(2);
\end{Verbatim}
prints
\begin{Verbatim}
(1,0) (2,-1) (-1,-2)
(2,1) (2,0) (0,-3)
(-1,2) (0,3) (3,0)

(-1,2)
(2,1) (0,3)
(1,0) (2,0) (3,0)
(2,-1) (0,-3)
(-1,-2)
\end{Verbatim}
\newpage



\subsubsection{real}
Function%
\pdfdest name {schmatrix.real} fit
\begin{verbatim}
const srsmatrix schmatrix::real () const;
\end{verbatim}
creates an object of type \verb"const srsmatrix"
as  real part
of a calling hermitian matrix.
Please note that, unlike
\GO{cvector::real}{cvector.real}, this
function creates new object \emph{not sharing}  memory
with  real part of a calling matrix, i.e.
the matrix returned is \emph{not  l-value}.
See also
\GOT{srsmatrix}{srsmatrix},
\GOT{schmatrix}{schmatrix}.
Example:
\begin{Verbatim}
using namespace cvm;

double a[] = {1., 0., 2., 1., -1., 2., 2., -1., 2., 0.,
              0., 3., -1., -2., 0., -3., 3., 0.};
schmatrix m ((std::complex<double>*)a, 3);
std::cout << m << std::endl;
std::cout << m.real();
\end{Verbatim}
prints
\begin{Verbatim}
(1,0) (2,-1) (-1,-2)
(2,1) (2,0) (0,-3)
(-1,2) (0,3) (3,0)

1 2 -1
2 2 0
-1 0 3
\end{Verbatim}
\newpage




\subsubsection{imag}
Function%
\pdfdest name {schmatrix.imag} fit
\begin{verbatim}
const srmatrix schmatrix::imag () const;
\end{verbatim}
creates an object of type \verb"const srmatrix"
as imaginary part
of a calling matrix.
Please note that, unlike
\GO{cvector::imag}{cvector.imag}, this
function creates new object \emph{not sharing}  memory
with  imaginary part of a calling matrix, i.e.
the matrix returned is \emph{not  l-value}.
See also
\GOT{srmatrix}{srmatrix},
\GOT{schmatrix}{schmatrix}.
Example:
\begin{Verbatim}
using namespace cvm;

double a[] = {1., 0., 2., 1., -1., 2., 2., -1., 2., 0.,
              0., 3., -1., -2., 0., -3., 3., 0.};
schmatrix m ((std::complex<double>*)a, 3);
std::cout << m << std::endl;
std::cout << m.imag();
\end{Verbatim}
prints
\begin{Verbatim}
(1,0) (2,-1) (-1,-2)
(2,1) (2,0) (0,-3)
(-1,2) (0,3) (3,0)

0 -1 -2
1 0 -3
2 3 0
\end{Verbatim}
\newpage



\subsubsection{operator = (const schmatrix\&)}
Operator%
\pdfdest name {schmatrix.operator = (const schmatrix&)} fit
\begin{verbatim}
schmatrix& schmatrix::operator = (const schmatrix& m)
throw (cvmexception);
\end{verbatim}
sets  every element of a calling hermitian matrix to a value of
appropriate element of  hermitian matrix \verb"m"
and returns a reference to
the matrix changed.
Operator throws  \GOT{cvmexception}{cvmexception}
in case of different matrix sizes.
See also \GOT{schmatrix}{schmatrix}.
Example:
\begin{Verbatim}
using namespace cvm;

try {
    double a[] = {1., 0., 2., 1., -1., 2., 2., -1., 2., 0.,
                  0., 3., -1., -2., 0., -3., 3., 0.};
    schmatrix m1((std::complex<double>*)a, 3);
    schmatrix m2(3);
    
    m2 = m1;
    std::cout << m2;
}
catch (std::exception& e) {
    std::cout << "Exception " << e.what () << std::endl;
}
\end{Verbatim}
prints
\begin{Verbatim}
(1,0) (2,-1) (-1,-2)
(2,1) (2,0) (0,-3)
(-1,2) (0,3) (3,0)
\end{Verbatim}
\newpage



\subsubsection{assign (const TC*)}
Function%
\pdfdest name {schmatrix.assign} fit
\begin{verbatim}
schmatrix& schmatrix::assign (const cvector& v, TR tol = cvmMachSp())
throw (cvmexception);

schmatrix& schmatrix::assign (const TC* pD, TR tol = cvmMachSp())
throw (cvmexception);
\end{verbatim}
sets every element of a calling hermitian matrix to a value of
appropriate element of  vector~\verb'v'
or  array pointed to by~\verb"pD"
and returns a reference to
the matrix changed.
In first version it's assumed that vector passed is long
enough to fill calling matrix. Function throws \GOT{cvmexception}{cvmexception} 
otherwise.
Function throws  \GOT{cvmexception}{cvmexception}
if the matrix changed doesn't appear to be hermitian 
(tolerance is set by parameter \verb'tol').
See also \GOT{schmatrix}{schmatrix}.
Example:
\begin{Verbatim}
using namespace cvm;

double a[] = {1., 0., 2., 1., -1., 2., 2., -1., 2., 0.,
              0., 3., -1., -2., 0., -3., 3., 0.};
schmatrix m(3);
m.assign((std::complex<double>*)a);
std::cout << m;
\end{Verbatim}
prints
\begin{Verbatim}
(1,0) (2,-1) (-1,-2)
(2,1) (2,0) (0,-3)
(-1,2) (0,3) (3,0)
\end{Verbatim}
\newpage



\subsubsection{assign (int, int, const schmatrix\&)}
Function%
\pdfdest name {schmatrix.assign (int, const schmatrix&)} fit
\begin{verbatim}
schmatrix& schmatrix::assign (int nRowCol, const schmatrix& m)
throw (cvmexception);
\end{verbatim}
sets main sub-matrix of a calling hermitian matrix beginning with \Based row
\verb"nRowCol" to  hermitian matrix \verb"m" and
returns a reference to the matrix changed.
Function throws 
 \GOT{cvmexception}{cvmexception} if \verb"nRowCol"
is not positive or matrix \verb"m" doesn't fit.
See also \GOT{schmatrix}{schmatrix}.
Example:
\begin{Verbatim}
using namespace cvm;

schmatrix m1(5);
schmatrix m2(2);
m2.set_main_diag(rvector(2,2.));
m2.set(1,2,std::complex<double>(2.,2.));
m1.assign(2,m2);
std::cout << m1;
\end{Verbatim}
prints
\begin{Verbatim}
(0,0) (0,0) (0,0) (0,0) (0,0)
(0,0) (2,0) (2,2) (0,0) (0,0)
(0,0) (2,-2) (2,0) (0,0) (0,0)
(0,0) (0,0) (0,0) (0,0) (0,0)
(0,0) (0,0) (0,0) (0,0) (0,0)
\end{Verbatim}
\newpage



\subsubsection{set (int,int,TC)}
Function%
\pdfdest name {schmatrix.set(int,int,TC)} fit
\begin{verbatim}
schmatrix& schmatrix::set (int nRow, int nCol, TC z) throw (cvmexception);
\end{verbatim}
sets an element located on \verb"nRow"'s row (\Based)
and \verb"nCol"'s column (\Based) to a value of
parameter \verb"z" and
an element located
on \verb"nCol"'s row
and \verb"nRow"'s column to a value of
parameter \verb"z" conjugated (keeping
 calling matrix to be hermitian).
Function returns a reference to
the matrix changed.
It throws \GOT{cvmexception}{cvmexception}
if any of parameters passed
is outside of \verb"[1,msize()]" range.
It also throws in case of assigning
complex number with non-zero imaginary part
to any element on the main diagonal.
See also \GOT{schmatrix}{schmatrix}.
Example:
\begin{Verbatim}
using namespace cvm;

double a[] = {1., 0., 2., 1., -1., 2., 2., -1., 2., 0.,
              0., 3., -1., -2., 0., -3., 3., 0.};
schmatrix m((std::complex<double>*)a,3);

m.set(1,2,std::complex<double>(7.7,7.7));
m.set(3,3,11.11);
std::cout << m;
\end{Verbatim}
prints
\begin{Verbatim}
(1,0) (7.7,7.7) (-1,-2)
(7.7,-7.7) (2,0) (0,-3)
(-1,2) (0,3) (11.11,0)
\end{Verbatim}
\newpage



\subsubsection{set\_diag}
Function%
\pdfdest name {schmatrix.set_diag} fit
\begin{verbatim}
schmatrix& schmatrix::set_diag (int i, const cvector& v)
throw (cvmexception);
\end{verbatim}
sets  \hbox{$i$-th} diagonal of a calling hermitian matrix,
$i<0$ for lower diagonals 
and $i>0$ for upper ones, to be equal to  complex vector
passed in parameter \verb"v".
Function also sets $-i$-th diagonal to
be equal to a vector \verb"v" conjugated,
thus keeping calling matrix to be hermitian.
Parameter \verb"i" is \emph{not} \Based.
Function returns a reference to
the matrix changed
and throws \GOT{cvmexception}{cvmexception}
if parameter \verb"i" passed
is outside of \verb"[-msize()+1,nsize()-1]" range
or equal to zero. 
It also throws if vector \verb"v" passed has size not equal to
\verb"msize()-abs(i)".
Use \GOT{schmatrix::set\_main\_diag}{schmatrix.set_main_diag}
to set  main diagonal.
See also \GOT{schmatrix}{schmatrix}.
Example:
\begin{Verbatim}
using namespace cvm;

double a[] = {1., 0., 2., 1., -1., 2., 2., -1., 2., 0.,
              0., 3., -1., -2., 0., -3., 3., 0.};
schmatrix m((std::complex<double>*)a,3);
cvector v(2);
v.set(std::complex<double>(7.7,7.7));
m.set_diag(1, v);
std::cout << m;
\end{Verbatim}
prints
\begin{Verbatim}
(1,0) (7.7,7.7) (-1,-2)
(7.7,-7.7) (2,0) (7.7,7.7)
(-1,2) (7.7,-7.7) (3,0)
\end{Verbatim}
\newpage



\subsubsection{set\_main\_diag}
Function%
\pdfdest name {schmatrix.set_main_diag} fit
\begin{verbatim}
schmatrix& schmatrix::set_main_diag (const rvector& v)
throw (cvmexception);
\end{verbatim}
sets  main diagonal of a calling hermitian 
matrix to be equal to  real vector
passed in parameter \verb"v".
Function returns a reference to
the matrix changed.
It throws \GOT{cvmexception}{cvmexception}
if vector \verb"v" passed has  size not equal to
\verb"msize()".
See also \GOT{schmatrix}{schmatrix}.
Example:
\begin{Verbatim}
using namespace cvm;

double a[] = {1., 0., 2., 1., -1., 2., 2., -1., 2., 0.,
              0., 3., -1., -2., 0., -3., 3., 0.};
schmatrix m((std::complex<double>*)a,3);
rvector v(3);
v.set(7.7);
m.set_main_diag(v);
std::cout << m;
\end{Verbatim}
prints
\begin{Verbatim}
(7.7,0) (2,-1) (-1,-2)
(2,1) (7.7,0) (0,-3)
(-1,2) (0,3) (7.7,0)
\end{Verbatim}
\newpage




\subsubsection{assign\_real}
Function%
\pdfdest name {schmatrix.assign_real} fit
\begin{verbatim}
schmatrix& schmatrix::assign_real (const srsmatrix& m) 
throw (cvmexception);
\end{verbatim}
sets real part of a calling hermitian matrix to be equal to
real symmetric matrix \verb"m".
Function returns a reference to
the matrix changed.
It throws  \GOT{cvmexception}{cvmexception}
in case of different sizes of the operands.
See also \GOT{schmatrix}{schmatrix}.
Example:
\begin{Verbatim}
using namespace cvm;

double a[] = {1., 0., 2., 1., -1., 2., 2., -1., 2., 0.,
              0., 3., -1., -2., 0., -3., 3., 0.};
schmatrix m((std::complex<double>*)a,3);
srsmatrix ms(3);
ms.set(7.);
m.assign_real(ms);
std::cout << m;
\end{Verbatim}
prints
\begin{Verbatim}
(7,0) (7,-1) (7,-2)
(7,1) (7,0) (7,-3)
(7,2) (7,3) (7,0)
\end{Verbatim}
\newpage



\subsubsection{set\_real}
Function%
\pdfdest name {schmatrix.set_real} fit
\begin{verbatim}
schmatrix& schmatrix::set_real (TR d);
\end{verbatim}
sets every element of  real part of a calling 
hermitian matrix to be equal to
 real number \verb"d".
Function returns a reference to
the matrix changed.
See also \GOT{schmatrix}{schmatrix}.
Example:
\begin{Verbatim}
using namespace cvm;

double a[] = {1., 0., 2., 1., -1., 2., 2., -1., 2., 0.,
              0., 3., -1., -2., 0., -3., 3., 0.};
schmatrix m((std::complex<double>*)a,3);
m.set_real(7.);
std::cout << m;
\end{Verbatim}
prints
\begin{Verbatim}
(7,0) (7,-1) (7,-2)
(7,1) (7,0) (7,-3)
(7,2) (7,3) (7,0)
\end{Verbatim}
\newpage



\subsubsection{resize}
Function%
\pdfdest name {schmatrix.resize} fit
\begin{verbatim}
schmatrix& schmatrix::resize (int nNewMN) throw (cvmexception);
\end{verbatim}
changes  size of a calling hermitian matrix to \verb"nNewMN" by \verb"nNewMN"
and returns a reference to
the matrix changed. In case of increasing of its size, calling matrix
is filled up with zeroes. 
Function throws  \GOT{cvmexception}{cvmexception}
in case of non-positive size passed or memory allocation failure.
See also 
\GOT{schmatrix}{schmatrix}.
Example:
\begin{Verbatim}
using namespace cvm;

double a[] = {1., 0., 2., 1., -1., 2., 2., -1., 2., 0.,
              0., 3., -1., -2., 0., -3., 3., 0.};
schmatrix m((std::complex<double>*)a,3);
std::cout << m << std::endl;
m.resize(4);
std::cout << m;
\end{Verbatim}
prints
\begin{Verbatim}
(1,0) (2,-1) (-1,-2)
(2,1) (2,0) (0,-3)
(-1,2) (0,3) (3,0)

(1,0) (2,-1) (-1,-2) (0,0)
(2,1) (2,0) (0,-3) (0,0)
(-1,2) (0,3) (3,0) (0,0)
(0,0) (0,0) (0,0) (0,0)
\end{Verbatim}
\newpage




\subsubsection{operator ==}
Operator%
\pdfdest name {schmatrix.operator ==} fit
\begin{verbatim}
bool schmatrix::operator == (const schmatrix& m) const;
\end{verbatim}
compares  calling hermitian matrix with  hermitian matrix \verb"m"
and returns \verb"true" if they have the same sizes
and their appropriate elements differ by not more than the
\GO{smallest normalized positive number}{Utilities.cvmMachMin}.
Returns \verb"false" otherwise.
See also \GOT{schmatrix}{schmatrix}.
Example:
\begin{Verbatim}
using namespace cvm;

double a[] = {1., 0., 2., 1., -1., 2., 2., -1., 2., 0.,
              0., 3., -1., -2., 0., -3., 3., 0.};
schmatrix m1((std::complex<double>*)a,3);
schmatrix m2(3);
m2.assign((std::complex<double>*)a);
std::cout << (m1 == m2) << std::endl;
\end{Verbatim}
prints
\begin{Verbatim}
1
\end{Verbatim}
\newpage




\subsubsection{operator !=}
Operator%
\pdfdest name {schmatrix.operator !=} fit
\begin{verbatim}
bool schmatrix::operator != (const schmatrix& m) const;
\end{verbatim}
compares  calling hermitian matrix with  hermitian matrix \verb"m"
and returns \verb"true" if they have different sizes
or at least one of their appropriate elements
differs by more than the
\GO{smallest normalized positive number}{Utilities.cvmMachMin}.
Returns \verb"false" otherwise.
See also \GOT{schmatrix}{schmatrix}.
Example:
\begin{Verbatim}
using namespace cvm;

double a[] = {1., 0., 2., 1., -1., 2., 2., -1., 2., 0.,
              0., 3., -1., -2., 0., -3., 3., 0.};
schmatrix m1((std::complex<double>*)a,3);
schmatrix m2(3);
m2.assign((std::complex<double>*)a);
m2.set(2,1,std::complex<double>(2.,1.000001));
std::cout << (m1 != m2) << std::endl;
\end{Verbatim}
prints
\begin{Verbatim}
1
\end{Verbatim}
\newpage



\subsubsection{operator <{}<}
Operator%
\pdfdest name {schmatrix.operator <<} fit
\begin{verbatim}
schmatrix& schmatrix::operator << (const schmatrix& m)
throw (cvmexception);
\end{verbatim}
destroys  calling hermitian matrix, creates  new one as a copy of \verb"m"
and returns a reference to
the matrix changed.
Operator throws  \GOT{cvmexception}{cvmexception}
in case of memory allocation failure.
See also \GOT{schmatrix}{schmatrix}.
Example:
\begin{Verbatim}
using namespace cvm;

double a[] = {1., 0., 2., 1., -1., 2., 2., -1., 2., 0.,
              0., 3., -1., -2., 0., -3., 3., 0.};
schmatrix m((std::complex<double>*)a,3);
schmatrix mc(1);
std::cout << m << std::endl;
std::cout << mc << std::endl;
mc << m;
std::cout << mc;
\end{Verbatim}
prints
\begin{Verbatim}
(1,0) (2,-1) (-1,-2)
(2,1) (2,0) (0,-3)
(-1,2) (0,3) (3,0)

(0,0)

(1,0) (2,-1) (-1,-2)
(2,1) (2,0) (0,-3)
(-1,2) (0,3) (3,0)
\end{Verbatim}
\newpage



\subsubsection{operator +}
Operator%
\pdfdest name {schmatrix.operator +} fit
\begin{verbatim}
schmatrix schmatrix::operator + (const schmatrix& m) const
throw (cvmexception);
\end{verbatim}
creates an object of type \verb"schmatrix" as a sum of
a calling hermitian matrix and  hermitian matrix \verb"m".
It throws  \GOT{cvmexception}{cvmexception}
in case of different sizes 
of the operands.
See also \GOT{schmatrix::sum}{schmatrix.sum}, \GOT{schmatrix}{schmatrix}.
Example:
\begin{Verbatim}
using namespace cvm;

double a[] = {1., 0., 2., 1., -1., 2., 2., -1., 2., 0.,
              0., 3., -1., -2., 0., -3., 3., 0.};
double b[] = {1., 0., 1., 1., 1., 1., 1., -1., 1., 0.,
              1., 1., 1., -1., 1., -1., 1., 0.};
schmatrix m1((std::complex<double>*)a,3);
schmatrix m2((std::complex<double>*)b,3);
std::cout << m1 + m2;
\end{Verbatim}
prints
\begin{Verbatim}
(2,0) (3,-2) (0,-3)
(3,2) (3,0) (1,-4)
(0,3) (1,4) (4,0)
\end{Verbatim}
\newpage



\subsubsection{operator -}
Operator%
\pdfdest name {schmatrix.operator -} fit
\begin{verbatim}
schmatrix schmatrix::operator - (const schmatrix& m) const
throw (cvmexception);
\end{verbatim}
creates an object of type \verb"schmatrix" as a difference of
a calling hermitian matrix and  hermitian matrix \verb"m".
It throws  \GOT{cvmexception}{cvmexception}
in case of different sizes 
of the operands.
See also \GOT{schmatrix::diff}{schmatrix.diff}, \GOT{schmatrix}{schmatrix}.
Example:
\begin{Verbatim}
using namespace cvm;

double a[] = {1., 0., 2., 1., -1., 2., 2., -1., 2., 0.,
              0., 3., -1., -2., 0., -3., 3., 0.};
double b[] = {1., 0., 1., 1., 1., 1., 1., -1., 1., 0.,
              1., 1., 1., -1., 1., -1., 1., 0.};
schmatrix m1((std::complex<double>*)a,3);
schmatrix m2((std::complex<double>*)b,3);
std::cout << m1 - m2;
\end{Verbatim}
prints
\begin{Verbatim}
(0,0) (1,0) (-2,-1)
(1,0) (1,0) (-1,-2)
(-2,1) (-1,2) (2,0)
\end{Verbatim}
\newpage




\subsubsection{sum}
Function%
\pdfdest name {schmatrix.sum} fit
\begin{verbatim}
schmatrix& schmatrix::sum (const schmatrix& m1, const schmatrix& m2)
throw (cvmexception);
\end{verbatim}
assigns the result of addition of
hermitian matrices \verb"m1" and \verb"m2"  to a calling hermitian matrix
and returns a reference to
the matrix changed.
It throws  \GOT{cvmexception}{cvmexception}
in case of different sizes 
of the operands.
See also \GOT{schmatrix::operator +~}{schmatrix.operator +},
\GOT{schmatrix}{schmatrix}.
Example:
\begin{Verbatim}
using namespace cvm;

double a[] = {1., 0., 2., 1., -1., 2., 2., -1., 2., 0.,
              0., 3., -1., -2., 0., -3., 3., 0.};
double b[] = {1., 0., 1., 1., 1., 1., 1., -1., 1., 0.,
              1., 1., 1., -1., 1., -1., 1., 0.};
schmatrix m1((std::complex<double>*)a,3);
schmatrix m2((std::complex<double>*)b,3);
schmatrix m(3);
std::cout << m.sum(m1, m2);
\end{Verbatim}
prints
\begin{Verbatim}
(2,0) (3,-2) (0,-3)
(3,2) (3,0) (1,-4)
(0,3) (1,4) (4,0)
\end{Verbatim}
\newpage




\subsubsection{diff}
Function%
\pdfdest name {schmatrix.diff} fit
\begin{verbatim}
schmatrix& schmatrix::diff (const schmatrix& m1, const schmatrix& m2)
throw (cvmexception);
\end{verbatim}
assigns the result of subtraction of
hermitian matrices \verb"m1" and \verb"m2" to a calling hermitian matrix
and returns a reference to
the matrix changed.
It throws  \GOT{cvmexception}{cvmexception}
in case of different sizes 
of the operands.
See also \GOT{schmatrix::operator -~}{schmatrix.operator -},
\GOT{schmatrix}{schmatrix}.
Example:
\begin{Verbatim}
using namespace cvm;

double a[] = {1., 0., 2., 1., -1., 2., 2., -1., 2., 0.,
              0., 3., -1., -2., 0., -3., 3., 0.};
double b[] = {1., 0., 1., 1., 1., 1., 1., -1., 1., 0.,
              1., 1., 1., -1., 1., -1., 1., 0.};
schmatrix m1((std::complex<double>*)a,3);
schmatrix m2((std::complex<double>*)b,3);
schmatrix m(3);
std::cout << m.diff(m1, m2);
\end{Verbatim}
prints
\begin{Verbatim}
(0,0) (1,0) (-2,-1)
(1,0) (1,0) (-1,-2)
(-2,1) (-1,2) (2,0)
\end{Verbatim}
\newpage




\subsubsection{operator +=}
Operator%
\pdfdest name {schmatrix.operator +=} fit
\begin{verbatim}
schmatrix& schmatrix::operator += (const schmatrix& m) 
throw (cvmexception);
\end{verbatim}
adds  hermitian matrix \verb"m" to a calling hermitian matrix 
and returns a reference to
the matrix changed.
It throws  \GOT{cvmexception}{cvmexception}
in case of different sizes 
of the operands.
See also \GOT{schmatrix::ope\-ra\-tor +~}{schmatrix.operator +},
\GOT{schmatrix::sum}{schmatrix.sum},
\GOT{schmatrix}{schmatrix}.
Example:
\begin{Verbatim}
using namespace cvm;

double a[] = {1., 0., 2., 1., -1., 2., 2., -1., 2., 0.,
              0., 3., -1., -2., 0., -3., 3., 0.};
double b[] = {1., 0., 1., 1., 1., 1., 1., -1., 1., 0.,
              1., 1., 1., -1., 1., -1., 1., 0.};
schmatrix m1((std::complex<double>*)a,3);
schmatrix m2((std::complex<double>*)b,3);
m1 += m2;
m2 += m2;
std::cout << m1 << std::endl;
std::cout << m2;
\end{Verbatim}
prints
\begin{Verbatim}
(2,0) (3,-2) (0,-3)
(3,2) (3,0) (1,-4)
(0,3) (1,4) (4,0)

(2,0) (2,-2) (2,-2)
(2,2) (2,0) (2,-2)
(2,2) (2,2) (2,0)
\end{Verbatim}
\newpage




\subsubsection{operator -=}
Operator%
\pdfdest name {schmatrix.operator -=} fit
\begin{verbatim}
schmatrix& schmatrix::operator -= (const schmatrix& m) 
throw (cvmexception);
\end{verbatim}
subtracts  hermitian matrix \verb"m" from  calling hermitian matrix
and returns a reference to
the matrix changed.
It throws  \GOT{cvmexception}{cvmexception}
in case of different sizes 
of the operands.
See also \GOT{schmatrix::ope\-ra\-tor -~}{schmatrix.operator -},
\GOT{schmatrix::diff}{schmatrix.diff},
\GOT{schmatrix}{schmatrix}.
Example:
\begin{Verbatim}
using namespace cvm;

double a[] = {1., 0., 2., 1., -1., 2., 2., -1., 2., 0.,
              0., 3., -1., -2., 0., -3., 3., 0.};
double b[] = {1., 0., 1., 1., 1., 1., 1., -1., 1., 0.,
              1., 1., 1., -1., 1., -1., 1., 0.};
schmatrix m1((std::complex<double>*)a,3);
schmatrix m2((std::complex<double>*)b,3);
m1 -= m2;
m2 -= m2;
std::cout << m1 << std::endl;
std::cout << m2;
\end{Verbatim}
prints
\begin{Verbatim}
(0,0) (1,0) (-2,-1)
(1,0) (1,0) (-1,-2)
(-2,1) (-1,2) (2,0)

(0,0) (0,0) (0,0)
(0,0) (0,0) (0,0)
(0,0) (0,0) (0,0)
\end{Verbatim}
\newpage




\subsubsection{operator - ()}
Operator%
\pdfdest name {schmatrix.operator - ()} fit
\begin{verbatim}
schmatrix schmatrix::operator - () const throw (cvmexception);
\end{verbatim}
creates an object of type \verb"schmatrix" as
a calling hermitian matrix multiplied by $-1$.
See also \GOT{schmatrix}{schmatrix}.
Example:
\begin{Verbatim}
using namespace cvm;

double a[] = {1., 0., 2., 1., -1., 2., 2., -1., 2., 0.,
              0., 3., -1., -2., 0., -3., 3., 0.};
schmatrix m((std::complex<double>*)a,3);
std::cout << -m;
\end{Verbatim}
prints
\begin{Verbatim}
(-1,0) (-2,1) (1,2)
(-2,-1) (-2,0) (0,3)
(1,-2) (0,-3) (-3,0)
\end{Verbatim}
\newpage



\subsubsection{operator ++}
Operator%
\pdfdest name {schmatrix.operator ++} fit
\begin{verbatim}
schmatrix& schmatrix::operator ++ ();
schmatrix& schmatrix::operator ++ (int);
\end{verbatim}
adds identity matrix to a calling hermitian matrix
and returns a reference to
the matrix changed.
See also \GOT{schmatrix}{schmatrix}.
Example:
\begin{Verbatim}
using namespace cvm;

double a[] = {1., 0., 2., 1., -1., 2., 2., -1., 2., 0.,
              0., 3., -1., -2., 0., -3., 3., 0.};
schmatrix m((std::complex<double>*)a,3);
m++;
std::cout << m << std::endl;
std::cout << ++m;
\end{Verbatim}
prints
\begin{Verbatim}
(2,0) (2,-1) (-1,-2)
(2,1) (3,0) (0,-3)
(-1,2) (0,3) (4,0)

(3,0) (2,-1) (-1,-2)
(2,1) (4,0) (0,-3)
(-1,2) (0,3) (5,0)
\end{Verbatim}
\newpage



\subsubsection{operator -{}-}
Operator%
\pdfdest name {schmatrix.operator --} fit
\begin{verbatim}
schmatrix& schmatrix::operator -- ();
schmatrix& schmatrix::operator -- (int);
\end{verbatim}
subtracts identity matrix from  calling hermitian matrix
and returns a reference to
the matrix changed.
See also \GOT{schmatrix}{schmatrix}.
Example:
\begin{Verbatim}
using namespace cvm;

double a[] = {1., 0., 2., 1., -1., 2., 2., -1., 2., 0.,
              0., 3., -1., -2., 0., -3., 3., 0.};
schmatrix m((std::complex<double>*)a,3);
m--;
std::cout << m << std::endl;
std::cout << --m;
\end{Verbatim}
prints
\begin{Verbatim}
(0,0) (2,-1) (-1,-2)
(2,1) (1,0) (0,-3)
(-1,2) (0,3) (2,0)

(-1,0) (2,-1) (-1,-2)
(2,1) (0,0) (0,-3)
(-1,2) (0,3) (1,0)
\end{Verbatim}
\newpage




\subsubsection{operator * (TR)}
Operator%
\pdfdest name {schmatrix.operator * (TR)} fit
\begin{verbatim}
schmatrix schmatrix::operator * (TR d) const;
\end{verbatim}
creates an object of type \verb"schmatrix" as a product of
a calling hermitian matrix and  real number~\verb"d".
See also \GOT{schmatrix::ope\-ra\-tor~*=~}{schmatrix.operator *= (TR)},
\GOT{schmatrix}{schmatrix}.
Example:
\begin{Verbatim}
using namespace cvm;

double a[] = {1., 0., 2., 1., -1., 2., 2., -1., 2., 0.,
              0., 3., -1., -2., 0., -3., 3., 0.};
schmatrix m((std::complex<double>*)a,3);
std::cout << m * 5.;
\end{Verbatim}
prints
\begin{Verbatim}
(5,0) (10,-5) (-5,-10)
(10,5) (10,0) (0,-15)
(-5,10) (0,15) (15,0)
\end{Verbatim}
\newpage



\subsubsection{operator / (TR)}
Operator%
\pdfdest name {schmatrix.operator / (TR)} fit
\begin{verbatim}
schmatrix schmatrix::operator / (TR d) const throw (cvmexception);
\end{verbatim}
creates an object of type \verb"schmatrix" as a quotient of
a calling hermitian matrix and  real number~\verb"d". It throws
 \GOT{cvmexception}{cvmexception}
if \verb"d" has  absolute value equal or less than the
\GO{smallest normalized positive number}{Utilities.cvmMachMin}.
See also \GOT{schmatrix::operator~/=~}{schmatrix.operator /= (TR)},
\GOT{schmatrix}{schmatrix}.
Example:
\begin{Verbatim}
using namespace cvm;

double a[] = {1., 0., 2., 1., -1., 2., 2., -1., 2., 0.,
              0., 3., -1., -2., 0., -3., 3., 0.};
schmatrix m((std::complex<double>*)a,3);
std::cout << m / 2.;
\end{Verbatim}
prints
\begin{Verbatim}
(0.5,0) (1,-0.5) (-0.5,-1)
(1,0.5) (1,0) (0,-1.5)
(-0.5,1) (0,1.5) (1.5,0)
\end{Verbatim}
\newpage



\subsubsection{operator * (TC)}
Operator%
\pdfdest name {schmatrix.operator * (TC)} fit
\begin{verbatim}
scmatrix schmatrix::operator * (TC z) const;
\end{verbatim}
creates an object of type \verb"scmatrix" as a product of
a calling hermitian matrix and  complex number~\verb"z".
See also \GOT{schmatrix}{schmatrix}.
Example:
\begin{Verbatim}
using namespace cvm;

double a[] = {1., 0., 2., 1., -1., 2., 2., -1., 2., 0.,
              0., 3., -1., -2., 0., -3., 3., 0.};
schmatrix m((std::complex<double>*)a,3);
std::cout << m * std::complex<double>(1.,1.);
\end{Verbatim}
prints
\begin{Verbatim}
(1,1) (3,1) (1,-3)
(1,3) (2,2) (3,-3)
(-3,1) (-3,3) (3,3)
\end{Verbatim}
\newpage



\subsubsection{operator / (TC)}
Operator%
\pdfdest name {schmatrix.operator / (TC)} fit
\begin{verbatim}
scmatrix schmatrix::operator / (TC z) const throw (cvmexception);
\end{verbatim}
creates an object of type \verb"scmatrix" as a quotient of
a calling hermitian matrix and  complex number~\verb"z". 
It throws
 \GOT{cvmexception}{cvmexception}
if \verb"z" has  absolute value equal or less than the
\GO{smallest normalized positive number}{Utilities.cvmMachMin}.
See also \GOT{schmatrix}{schmatrix}.
Example:
\begin{Verbatim}
using namespace cvm;

double a[] = {1., 0., 2., 1., -1., 2., 2., -1., 2., 0.,
              0., 3., -1., -2., 0., -3., 3., 0.};
schmatrix m((std::complex<double>*)a,3);
std::cout << m / std::complex<double>(1.,1.);
\end{Verbatim}
prints
\begin{Verbatim}
(0.5,-0.5) (0.5,-1.5) (-1.5,-0.5)
(1.5,-0.5) (1,-1) (-1.5,-1.5)
(0.5,1.5) (1.5,1.5) (1.5,-1.5)
\end{Verbatim}
\newpage



\subsubsection{operator *= (TR)}
Operator%
\pdfdest name {schmatrix.operator *= (TR)} fit
\begin{verbatim}
schmatrix& schmatrix::operator *= (TR d);
\end{verbatim}
multiplies  calling hermitian matrix by  real number~\verb"d"
and returns a reference to
the matrix changed.
See also \GOT{schmatrix::operator~*~}{schmatrix.operator * (TR)},
\GOT{schmatrix}{schmatrix}.
Example:
\begin{Verbatim}
using namespace cvm;

double a[] = {1., 0., 2., 1., -1., 2., 2., -1., 2., 0.,
              0., 3., -1., -2., 0., -3., 3., 0.};
schmatrix m((std::complex<double>*)a,3);
m *= 5.;
std::cout << m;
\end{Verbatim}
prints
\begin{Verbatim}
(5,0) (10,-5) (-5,-10)
(10,5) (10,0) (0,-15)
(-5,10) (0,15) (15,0)
\end{Verbatim}
\newpage



\subsubsection{operator /= (TR)}
Operator%
\pdfdest name {schmatrix.operator /= (TR)} fit
\begin{verbatim}
schmatrix& schmatrix::operator /= (TR d) throw (cvmexception);
\end{verbatim}
divides  calling hermitian matrix by  real number~\verb"d"
and returns a reference to
the matrix changed.
It throws  \GOT{cvmexception}{cvmexception}
if \verb"d" has  absolute value equal or less
than the 
\GO{smallest normalized positive number}{Utilities.cvmMachMin}.
See also \GOT{schmatrix::operator /~}{schmatrix.operator / (TR)},
\GOT{schmatrix}{schmatrix}.
Example:
\begin{Verbatim}
using namespace cvm;

double a[] = {1., 0., 2., 1., -1., 2., 2., -1., 2., 0.,
              0., 3., -1., -2., 0., -3., 3., 0.};
schmatrix m((std::complex<double>*)a,3);
m /= 2.;
std::cout << m;
\end{Verbatim}
prints
\begin{Verbatim}
(0.5,0) (1,-0.5) (-0.5,-1)
(1,0.5) (1,0) (0,-1.5)
(-0.5,1) (0,1.5) (1.5,0)
\end{Verbatim}
\newpage



\subsubsection{normalize}
Function%
\pdfdest name {schmatrix.normalize} fit
\begin{verbatim}
schmatrix& schmatrix::normalize ();
\end{verbatim}
normalizes  calling hermitian matrix so 
its \GO{Euclidean norm}{Array.norm}
becomes equal to $1$ if it was greater than the 
\GO{smallest normalized positive number}{Utilities.cvmMachMin}
before the call (otherwise function does nothing).
See also \GOT{schmatrix}{schmatrix}.
Example:
\begin{Verbatim}
using namespace cvm;

std::cout.setf (std::ios::scientific | std::ios::left); 
std::cout.precision (2);
double a[] = {1., 0., 2., 1., -1., 2., 2., -1., 2., 0.,
              0., 3., -1., -2., 0., -3., 3., 0.};
schmatrix m((std::complex<double>*)a,3);

m.normalize();
std::cout << m << m.norm() << std::endl;
\end{Verbatim}
prints
\begin{Verbatim}
(1.39e-001,0.00e+000) (2.77e-001,-1.39e-001) (-1.39e-001,-2.77e-001)
(2.77e-001,1.39e-001) (2.77e-001,0.00e+000) (0.00e+000,-4.16e-001)
(-1.39e-001,2.77e-001) (0.00e+000,4.16e-001) (4.16e-001,0.00e+000)
1.00e+000
\end{Verbatim}
\newpage



\subsubsection{conjugation}
Operator and functions%
\pdfdest name {schmatrix.conj} fit
\begin{verbatim}
schmatrix schmatrix::operator ~ () const throw (cvmexception);
schmatrix& schmatrix::conj (const schmatrix& m) throw (cvmexception);
schmatrix& schmatrix::conj () throw (cvmexception);
\end{verbatim}
do nothing since  matrix calling is hermitian.
They are provided to reimplement similar member functions and operator
of the class \GOT{scmatrix}{scmatrix}.
See also \GOT{schmatrix}{schmatrix}.
Example:
\begin{Verbatim}
using namespace cvm;

double a[] = {1., 0., 2., 1., -1., 2., 2., -1., 2., 0.,
              0., 3., -1., -2., 0., -3., 3., 0.};
schmatrix m((std::complex<double>*)a,3);

std::cout << m - ~m;
\end{Verbatim}
prints
\begin{Verbatim}
(0,0) (0,0) (0,0)
(0,0) (0,0) (0,0)
(0,0) (0,0) (0,0)
\end{Verbatim}
\newpage




\subsubsection{transposition}
Operator and functions%
\pdfdest name {schmatrix.transpose} fit
\begin{verbatim}
schmatrix schmatrix::operator ! () const throw (cvmexception);
schmatrix& schmatrix::transpose (const schmatrix& m) throw (cvmexception);
schmatrix& schmatrix::transpose () throw (cvmexception);
\end{verbatim}
implement complex hermitian matrix transposition (\emph{not} conjugation).
First operator creates an object of type \verb"schmatrix" as
 transposed calling matrix
(it throws  
\GOT{cvmexception}{cvmexception}
in case of memory allocation failure). 
Second function sets  calling matrix to be equal to  matrix
\verb"m" transposed
(it throws  
\GOT{cvmexception}{cvmexception}
in case of not appropriate sizes of the operands), 
third one makes it to be equal to
transposed itself (it also throws  
\GOT{cvmexception}{cvmexception}
in case of memory allocation failure). 
See also \GOT{schmatrix}{schmatrix}.
Example:
\begin{Verbatim}
using namespace cvm;

double a[] = {1., 0., 2., 1., -1., 2., 2., -1., 2., 0.,
              0., 3., -1., -2., 0., -3., 3., 0.};
schmatrix m((std::complex<double>*)a,3);
schmatrix mt(3);
std::cout << m << std::endl << !m << std::endl ;
mt.transpose(m);
std::cout << mt << std::endl;
mt.transpose();
std::cout << mt;
\end{Verbatim}
prints
\begin{Verbatim}
(1,0) (2,-1) (-1,-2)
(2,1) (2,0) (0,-3)
(-1,2) (0,3) (3,0)

(1,0) (2,1) (-1,2)
(2,-1) (2,0) (0,3)
(-1,-2) (0,-3) (3,0)

(1,0) (2,1) (-1,2)
(2,-1) (2,0) (0,3)
(-1,-2) (0,-3) (3,0)

(1,0) (2,-1) (-1,-2)
(2,1) (2,0) (0,-3)
(-1,2) (0,3) (3,0)
\end{Verbatim}
\newpage




\subsubsection{operator * (const cvector\&)}
Operator%
\pdfdest name {schmatrix.operator * (const cvector&)} fit
\begin{verbatim}
cvector schmatrix::operator * (const cvector& v) const
throw (cvmexception);
\end{verbatim}
creates an object of type \verb"cvector"
as a product of a calling hermitian matrix and a vector \verb"v".
It throws  \GOT{cvmexception}{cvmexception}
if  number of columns of a calling matrix
differs from  size of a vector \verb"v".
Use \GOT{cvector::mult}{cvector.mult (const cmatrix&, const cvector&)}
in order to avoid new object creation.
See also
\GOT{schmatrix}{schmatrix} and \GOT{cvector}{cvector}.
Example:
\begin{Verbatim}
using namespace cvm;

double a[] = {1., 0., 2., 1., -1., 2., 2., -1., 2., 0.,
              0., 3., -1., -2., 0., -3., 3., 0.};
try {
    schmatrix m((std::complex<double>*)a,3);
    cvector v(3);
    v.set(std::complex<double>(1.,1.));

    std::cout << m * v;
}
catch (std::exception& e) {
    std::cout << "Exception: " << e.what () << std::endl;
}
\end{Verbatim}
prints
\begin{Verbatim}
(5,-1) (6,2) (-3,7)
\end{Verbatim}
\newpage



\subsubsection{operator * (const cmatrix\&)}
Operator%
\pdfdest name {schmatrix.operator * (const cmatrix&)} fit
\begin{verbatim}
cmatrix schmatrix::operator * (const cmatrix& m) const
throw (cvmexception);
\end{verbatim}
creates an object of type \verb"cmatrix"
as a product of a calling hermitian matrix and a matrix \verb"m".
It throws  \GOT{cvmexception}{cvmexception}
if  number of columns of a calling matrix
differs from  number of rows of a matrix \verb"m".
Use \GOT{cmatrix::mult}{cmatrix.mult} in order to avoid
 new object creation.
See also
\GOT{cmatrix}{cmatrix} and \GOT{schmatrix}{schmatrix}.
Example:
\begin{Verbatim}
using namespace cvm;

double a[] = {1., 0., 2., 1., -1., 2., 2., -1., 2., 0.,
              0., 3., -1., -2., 0., -3., 3., 0.};
try {
    schmatrix ms((std::complex<double>*)a,3);
    cmatrix m(3,2);
    m.set(std::complex<double>(1.,1.));

    std::cout << ms * m;
}
catch (std::exception& e) {
    std::cout << "Exception: " << e.what () << std::endl;
}
\end{Verbatim}
prints
\begin{Verbatim}
(5,-1) (5,-1)
(6,2) (6,2)
(-3,7) (-3,7)
\end{Verbatim}
\newpage




\subsubsection{operator * (const scmatrix\&)}
Operator%
\pdfdest name {schmatrix.operator * (const scmatrix&)} fit
\begin{verbatim}
scmatrix schmatrix::operator * (const scmatrix& m) const
throw (cvmexception);
\end{verbatim}
creates an object of type \verb"scmatrix"
as a product of a calling hermitian matrix and a matrix \verb"m".
It throws  \GOT{cvmexception}{cvmexception}
if the operands have different sizes.
Use \GOT{cmatrix::mult}{cmatrix.mult} in order to avoid
 new object creation.
See also
\GOT{scmatrix}{scmatrix} and \GOT{schmatrix}{schmatrix}.
Example:
\begin{Verbatim}
using namespace cvm;

double a[] = {1., 0., 2., 1., -1., 2., 2., -1., 2., 0.,
              0., 3., -1., -2., 0., -3., 3., 0.};
try {
    schmatrix ms((std::complex<double>*)a,3);
    scmatrix m(3);
    m.set(std::complex<double>(1.,1.));

    std::cout << ms * m;
}
catch (std::exception& e) {
    std::cout << "Exception: " << e.what () << std::endl;
}
\end{Verbatim}
prints
\begin{Verbatim}
(5,-1) (5,-1) (5,-1)
(6,2) (6,2) (6,2)
(-3,7) (-3,7) (-3,7)
\end{Verbatim}
\newpage





\subsubsection{operator / (const cvector\&)}
Operator%
\pdfdest name {schmatrix.operator / (cvector)} fit
\begin{verbatim}
cvector operator / (const cvector& vB) const throw (cvmexception);
\end{verbatim}
returns solution $x$ of linear equation
$A*x=b$ where calling matrix is square hermitian matrix $A$
and a vector $b$ is passed in parameter \verb"vB".
This operator throws exception 
of type \GOT{cvmexception}{cvmexception}
in case of inappropriate sizes
of the objects or when  matrix $A$ is close to singular.
See also \GOT{cvector::solve}{cvector.solve}, 
\GOT{scmatrix::solve}{scmatrix.solve}, 
\GOT{cvector.operator~\%}{cvector.operator percent (scmatrix)}, 
\GOT{cvector}{cvector}, \GOT{scmatrix}{scmatrix}, \GOT{schmatrix}{schmatrix}.
Example:
\begin{Verbatim}
using namespace cvm;
std::cout.setf (std::ios::scientific | std::ios::showpos);
std::cout.precision (2);

double a[] = {1., 0., 2., 1., -1., 2., 2., -1., 2., 0.,
              0., 3., -1., -2., 0., -3., 3., 0.};
try {
    schmatrix ma((std::complex<double>*)a,3);
    cvector vb(3);
    cvector vx(3);
    vb.set(std::complex<double>(1.,2.));

    vx = ma / vb;

    std::cout << ma * vx - vb;
}
catch (std::exception& e) {
    std::cout << "Exception: " << e.what () << std::endl;
}
\end{Verbatim}
prints
\begin{Verbatim}
(+0.00e+000,+1.78e-015) (-8.88e-016,+3.55e-015) (-3.55e-015,-1.78e-015)
\end{Verbatim}
\newpage



\subsubsection{herk}
Functions%
\pdfdest name {schmatrix.herk} fit
\begin{verbatim}
schmatrix& 
schmatrix::herk (TR alpha, const cvector& v, TR beta) 
                 throw (cvmexception);
schmatrix& 
schmatrix::herk (bool bTransp, TR alpha, const cmatrix& m, TR beta)
                 throw (cvmexception);
\end{verbatim}
call \verb"ZHERK" routine of the
\URL{BLAS library}{http://www.netlib.org/blas}
performing  matrix-vector operation defined for the first version as
rank-1 update operation
\begin{equation*}
A=\alpha\,\begin{pmatrix}
v_1 \\
v_2 \\
\vdots \\
v_n
\end{pmatrix}
\begin{pmatrix}
v_1^* & v_2^* & \cdots & v_n^*
\end{pmatrix} + \beta A,
\end{equation*}
and for the second version as
\begin{equation*}
A=\alpha MM^H + \beta A\quad\text{or}\quad A=\alpha M^HM + \beta A.
\end{equation*}
Here $\alpha$ and $\beta$ are real numbers 
(parameters \verb"alpha" and \verb"beta"),
$M$ is  complex matrix (parameter \verb"m"),
$A$ is  calling hermitian matrix
and $v$ is  complex vector (parameter \verb"v").
First operation for the second version 
is performed if \verb"bTransp" passed
is \verb"false" and second one otherwise.
Function
returns a reference to the matrix changed and throws
 \GOT{cvmexception}{cvmexception}
in case of inappropriate sizes of the operands.
See also
\GOT{cvector}{cvector},
\GOT{cmatrix}{cmatrix} and \GOT{schmatrix}{schmatrix}
Example:
\begin{Verbatim}
using namespace cvm;
const treal a[] = {1., -1., 2., 2., 3., -3.};
const cvector v((std::complex<double>*)a, 3);
const treal alpha = 2.12;
const treal beta = -3.07;
schmatrix mh(3), mh2(3);
mh.randomize_real(-1.,2.);
mh.randomize_imag(-2.,3.);
mh2 = mh;
mh.herk (alpha, v, beta);
mh2 = alpha * schmatrix(v.rank1update_c(v)) + beta * mh2;
std::cout << mh - mh2;
\end{Verbatim}
prints
\begin{Verbatim}
(0,0) (0,0) (0,0)
(0,0) (0,0) (0,0)
(0,0) (0,0) (0,0)
\end{Verbatim}
Example:
\begin{Verbatim}
cmatrix m(3,3);
const treal alpha = 2.12;
const treal beta = -3.07;
schmatrix mh(3), mh2(3);
m.randomize_real(-1.,2.);
m.randomize_imag(-2.,3.);
mh.randomize_real(-1.,2.);
mh.randomize_imag(-2.,3.);
mh2 = mh;
mh.herk (false, alpha, m, beta);
mh2 = alpha * schmatrix (m * ~m) + beta * mh2;
std::cout << mh - mh2;
\end{Verbatim}
prints
\begin{Verbatim}
(0,0) (0,0) (0,0)
(0,0) (0,0) (0,0)
(0,0) (0,0) (0,0)
\end{Verbatim}
Example:
\begin{Verbatim}
cmatrix m(3,3);
const treal alpha = 2.12;
const treal beta = -3.07;
schmatrix mh(3), mh2(3);
m.randomize_real(-1.,2.);
m.randomize_imag(-2.,3.);
mh.randomize_real(-1.,2.);
mh.randomize_imag(-2.,3.);
mh2 = mh;
mh.herk (true, alpha, m, beta);
mh2 = alpha * schmatrix (~m * m) + beta * mh2;
std::cout << mh - mh2;
\end{Verbatim}
prints
\begin{Verbatim}
(3.55271e-015,0) (0,0) (0,0)
(0,0) (0,0) (0,0)
(0,0) (0,0) (1.77636e-015,0)
\end{Verbatim}
\newpage




\subsubsection{her2k}
Functions%
\pdfdest name {schmatrix.her2k} fit
\begin{verbatim}
schmatrix& 
schmatrix::her2k (TC alpha, const cvector& v1, 
                  const cvector& v2, TR beta) throw (cvmexception);
schmatrix& 
schmatrix::her2k (bool bTransp, TC alpha, const cmatrix& m1, 
                  const cmatrix& m2, TR beta) throw (cvmexception);
\end{verbatim}
call one of \verb"?HER2K" routines of the
\URL{BLAS library}{http://www.netlib.org/blas}
performing  matrix-vector operation defined for the first version as
rank-1 update operation
\begin{equation*}
A=\alpha v_1 v_2^* + \alpha^* v_2 v_1^* + \beta A,
\end{equation*}
and for the second version as
\begin{equation*}
A=\alpha M_1 M_2^H + \alpha^* M_2 M_1^H + \beta A\quad
\text{or}\quad 
A=\alpha M_1^H M_2 + \alpha^* M_2^H M_1 + \beta A.
\end{equation*}
Here $\alpha$ is complex and $\beta$ is real number
(parameters \verb"alpha" and \verb"beta"),
$M_1$ and $M_2$ are complex matrices (parameters \verb"m1" and \verb"m2"),
$A$ is  calling hermitian matrix
and $v_1$ and $v_2$ are complex vectors (parameters \verb"v1" and \verb"v2").
First operation for the second version 
is performed if \verb"bTransp" passed
is \verb"false" and second one otherwise.
Function
returns a reference to the matrix changed and throws
 \GOT{cvmexception}{cvmexception}
in case of inappropriate sizes of the operands.
See also
\GOT{cvector}{cvector},
\GOT{cmatrix}{cmatrix} and \GOT{schmatrix}{schmatrix}
Example:
\begin{Verbatim}
using namespace cvm;
std::cout.setf(std::ios::scientific | std::ios::showpos); 
std::cout.precision(2);
const tcomplex alpha(2.12,-0.14);
const tcomplex alphac(2.12,0.14);
const treal beta = -3.07;
cvector v1(3), v2(3);
schmatrix mh(3), mh2(3);
v1.randomize_real(-1.,2.);
v1.randomize_imag(-2.,3.);
v2.randomize_real(-1.,2.);
v2.randomize_imag(-2.,3.);
mh.randomize_real(-1.,2.);
mh.randomize_imag(-2.,3.);
mh2 = mh;
mh.her2k (alpha, v1, v2, beta);
mh2 = schmatrix(alpha * v1.rank1update_c(v2) + 
                alphac * v2.rank1update_c(v1)) + beta * mh2;
std::cout << mh - mh2;
\end{Verbatim}
prints
\begin{Verbatim}
(+1.78e-015,+0.00e+000) (+0.00e+000,+0.00e+000) (+0.00e+000,+0.00e+000)
(+0.00e+000,+0.00e+000) (-1.78e-015,+0.00e+000) (+0.00e+000,+1.78e-015)
(+0.00e+000,+0.00e+000) (+0.00e+000,-1.78e-015) (+4.44e-016,+0.00e+000)
\end{Verbatim}
Example:
\begin{Verbatim}
using namespace cvm;
std::cout.setf(std::ios::scientific | std::ios::showpos); 
std::cout.precision(2);
const tcomplex alpha(2.12,-0.14);
const tcomplex alphac(2.12,0.14);
const treal beta = -3.07;
cmatrix m1(3,3), m2(3,3);
schmatrix mh(3), mh2(3);
m1.randomize_real(-1.,2.);
m1.randomize_imag(-2.,3.);
m2.randomize_real(-1.,2.);
m2.randomize_imag(-2.,3.);
mh.randomize_real(-1.,2.);
mh.randomize_imag(-2.,3.);
mh2 = mh;
mh.her2k (false, alpha, m1, m2, beta);
mh2 = schmatrix(alpha * m1 * ~m2 + 
                alphac * m2 * ~m1, 1.e-14) + beta * mh2;
std::cout << mh - mh2;
\end{Verbatim}
prints
\begin{Verbatim}
(+0.00e+000,+4.44e-016) (+0.00e+000,+0.00e+000) (+0.00e+000,+1.78e-015)
(+3.55e-015,+0.00e+000) (+4.44e-016,-8.88e-016) (+0.00e+000,+0.00e+000)
(+8.88e-016,+8.88e-016) (+0.00e+000,+0.00e+000) (+8.88e-016,+1.78e-015)
\end{Verbatim}
Example:
\begin{Verbatim}
using namespace cvm;
std::cout.setf(std::ios::scientific | std::ios::showpos); 
std::cout.precision(2);
const tcomplex alpha(2.12,-0.14);
const tcomplex alphac(2.12,0.14);
const treal beta = -3.07;
cmatrix m1(3,3), m2(3,3);
schmatrix mh(3), mh2(3);
m1.randomize_real(-1.,2.);
m1.randomize_imag(-2.,3.);
m2.randomize_real(-1.,2.);
m2.randomize_imag(-2.,3.);
mh.randomize_real(-1.,2.);
mh.randomize_imag(-2.,3.);
mh2 = mh;
mh.her2k (true, alpha, m1, m2, beta);
mh2 = schmatrix(alpha * ~m1 * m2 + 
                alphac * ~m2 * m1, 1.e-14) + beta * mh2;
std::cout << mh - mh2;
\end{Verbatim}
prints
\begin{Verbatim}
(+3.55e-015,-8.88e-016) (+0.00e+000,+0.00e+000) (+0.00e+000,-1.78e-015)
(-8.88e-016,+1.78e-015) (+0.00e+000,-4.44e-016) (+8.88e-016,+0.00e+000)
(+0.00e+000,+3.55e-015) (-1.33e-015,-1.78e-015) (+0.00e+000,+0.00e+000)
\end{Verbatim}
\newpage





\subsubsection{inv}
Functions%
\pdfdest name {schmatrix.inv} fit
\begin{verbatim}
schmatrix& schmatrix::inv (const schmatrix& m) throw (cvmexception);
schmatrix schmatrix::inv () const throw (cvmexception);
\end{verbatim}
implement hermitian matrix inversion.
The first version sets  calling hermitian matrix to be equal to 
 hermitian matrix \verb"m" inverted and the second one
creates an object of type \verb"schmatrix" as inverted calling matrix.
Functions throw  \GOT{cvmexception}{cvmexception} 
in case of inappropriate
sizes of the operands or when  matrix to be inverted is close to
singular.
See also
\GOT{schmatrix}{schmatrix}.
Example:
\begin{Verbatim}
using namespace cvm;

std::cout.setf (std::ios::scientific | std::ios::left); 
std::cout.precision (2);

double a[] = {1., 0., 2., 1., -1., 2., 2., -1., 2., 0.,
              0., 3., -1., -2., 0., -3., 3., 0.};
schmatrix m((std::complex<double>*)a,3);
const schmatrix mi = m.inv();

std::cout << mi << std::endl;
std::cout << mi * m - eye_complex(3);
\end{Verbatim}
prints
\begin{Verbatim}
(-1.50e+000,0.00e+000) (-1.67e-016,8.33e-017) (-5.00e-001,-1.00e+000)
(-1.67e-016,-8.33e-017) (-1.00e+000,0.00e+000) (0.00e+000,-1.00e+000)
(-5.00e-001,1.00e+000) (0.00e+000,1.00e+000) (-1.50e+000,0.00e+000)

(2.22e-016,0.00e+000) (1.11e-016,0.00e+000) (2.78e-017,0.00e+000)
(2.78e-016,2.22e-016) (4.44e-016,0.00e+000) (0.00e+000,-4.44e-016)
(2.22e-016,-4.44e-016) (0.00e+000,-8.88e-016) (-2.22e-016,0.00e+000)
\end{Verbatim}
\newpage



\subsubsection{exp}
Functions%
\pdfdest name {schmatrix.exp} fit
\begin{verbatim}
schmatrix& schmatrix::exp (const schmatrix& m, TR tol = cvmMachSp ())
throw (cvmexception);

schmatrix schmatrix::exp (TR tol = cvmMachSp ()) const
throw (cvmexception);
\end{verbatim}
compute  exponent of a calling hermitian matrix using Pad\'e approximation
defined as
\begin{equation*}
R_{pq}(z)=D_{pq}(z)^{-1}N_{pq}(z)=1+z+\dots+z^p/p!\,,
\end{equation*}
where
\begin{align*}
N_{pq}(z)&=\sum_{k=0}^p\frac{(p+q-k)!p!}{(p+q)!k!(q-k)!}z^k,\\
D_{pq}(z)&=\sum_{k=0}^q\frac{(p+q-k)!p!}{(p+q)!k!(q-k)!}(-z)^k
\end{align*}
along with a matrix normalizing as described in
\GO{\cite{Golub}}{biblio}, p.~572.
Functions use \verb"ZMEXP" (or \verb"CMEXP" for \verb"float" version)
\FORTRAN subroutine implementing the algorithm.
The first version sets  calling hermitian matrix to be equal to
 exponent of  hermitian matrix \verb"m" and returns 
 reference to the matrix
changed. The second version
creates an object of type \verb"schmatrix" as 
exponent of a calling matrix.
The algorithm uses parameter \verb"tol"
as $\varepsilon(p,q)$ in order to choose constants $p$ and $q$
so that
\begin{equation*}
\varepsilon(p,q)\ge 2^{3-(p+q)}\frac{p!q!}{(p+q)!(p+q+1)!}.
\end{equation*}
This parameter is equal to the
\GO{largest relative spacing}{Utilities.cvmMachSp} by default.
Functions throw  \GOT{cvmexception}{cvmexception} 
in case of
inappropriate sizes of the operands or when LAPACK subroutine fails.
See also
\GOT{schmatrix}{schmatrix}.
Example:
\begin{Verbatim}
using namespace cvm;

std::cout.setf (std::ios::scientific | std::ios::left |
                std::ios::showpos);
std::cout.precision (15);
double a[] = {1., 0., 2., 1., -1., 2., 2., -1., 2., 0.,
              0., 3., -1., -2., 0., -3., 3., 0.};
schmatrix m((std::complex<double>*)a,3);
schmatrix me(3);
me.exp(m);
std::cout << "Column 1" << std::endl
<< me(1,1) << std::endl << me(2,1) << std::endl << me(3,1) << std::endl
          << "Column 2" << std::endl
<< me(1,2) << std::endl << me(2,2) << std::endl << me(3,2) << std::endl
          << "Column 3" << std::endl
<< me(1,3) << std::endl << me(2,3) << std::endl << me(3,3) << std::endl;
\end{Verbatim}
prints
\begin{Verbatim}
Column 1
(+2.673228708372002e+002,+1.091141066389412e-014)
(+3.071187567026803e+002,+1.535593783513402e+002)
(-1.749365628720766e+002,+3.498731257441531e+002)
Column 2
(+3.071187567026803e+002,-1.535593783513401e+002)
(+4.422594337092766e+002,+1.919736460939478e-015)
(-9.600094996571151e-015,+5.034325040954932e+002)
Column 3
(-1.749365628720765e+002,-3.498731257441531e+002)
(+6.184072406183948e-015,-5.034325040954932e+002)
(+5.744416275398805e+002,+1.540673883337074e-014)
\end{Verbatim}
Matlab output:
\begin{Verbatim}
  Column 1 

     2.673228708371998e+002 -7.105427357601002e-015i
     3.071187567026802e+002 +1.535593783513401e+002i
    -1.749365628720764e+002 +3.498731257441527e+002i

  Column 2 

     3.071187567026802e+002 -1.535593783513401e+002i
     4.422594337092769e+002 -5.489286670342458e-016i
     3.549798266275454e-015 +5.034325040954932e+002i

  Column 3 

    -1.749365628720763e+002 -3.498731257441526e+002i
    -1.776065298147746e-014 -5.034325040954931e+002i
     5.744416275398801e+002 -2.096383162906490e-014i
\end{Verbatim}
\newpage



\subsubsection{polynomial}
Functions%
\pdfdest name {schmatrix.polynom} fit
\begin{verbatim}
schmatrix& schmatrix::polynom (const schmatrix& m, const rvector& v)
throw (cvmexception);

schmatrix schmatrix::polynom (const rvector& v) const
throw (cvmexception);
\end{verbatim}
compute  hermitian matrix polynomial defined as
\begin{equation*}
p(A)=b_0I+b_1A+\dots+b_qA^q
\end{equation*}
using the Horner's rule:
\begin{equation*}
p(A)=\sum_{k=0}^r B_k(A^s)^k,\quad s=\floor(\!\sqrt{q}\,),\ r=\floor(q/s)
\end{equation*}
where
\begin{equation*}
B_k=\begin{cases}
\sum\limits_{i=0}^{s-1}b_{sk+i} A^i, & k=0,1,\dots,r-1\\
\sum\limits_{i=0}^{q-sr}b_{sr+i} A^i, & k=r.
\end{cases}
\end{equation*}
See also \GO{\cite{Golub}}{biblio}, p.~568.
\emph{Real} coefficients $b_0,b_1,\dots,b_q$ are 
passed in  parameter \verb"v",
where $q$ is equal to \verb"v.size()-1", so  functions
compute matrix polynomial equal to
\begin{equation*}
\mathtt{v[1]*I + v[2]*m +\cdots + v[v.size()]*m^{v.size()-1}}
\end{equation*}
The first version sets calling hermitian matrix to be equal 
to the polynomial of hermitian matrix \verb"m"
and the second one
creates an object of type \verb"schmatrix" as the 
polynomial of a calling symmetric matrix.
Functions use \verb"ZPOLY" (or \verb"CPOLY" for \verb"float" version)
\FORTRAN subroutine implementing the Horner's algorithm.
Functions throw  \GOT{cvmexception}{cvmexception} 
in case of inappropriate sizes of the operands.
See also
\GOT{schmatrix}{schmatrix}.
Example:
\begin{Verbatim}
using namespace cvm;
std::cout.setf (std::ios::scientific | std::ios::left |
                std::ios::showpos);
std::cout.precision (10);
double a[] = {1., 0., 2., 1., -1., 2., 2., -1., 2., 0.,
            0., 3., -1., -2., 0., -3., 3., 0.};
schmatrix m((std::complex<double>*)a,3);
double re[]={2.2,1.3,1.1,-0.9,0.2,-0.45,45.,-30.,10.,3.,1.13};
const rvector vr(re, 11);
schmatrix mp(3);

mp.polynom (m, vr);

std::cout << "Column 1" << std::endl
<< mp(1,1) << std::endl << mp(2,1) << std::endl << mp(3,1) << std::endl
          << "Column 2" << std::endl
<< mp(1,2) << std::endl << mp(2,2) << std::endl << mp(3,2) << std::endl
          << "Column 3" << std::endl
<< mp(1,3) << std::endl << mp(2,3) << std::endl << mp(3,3) << std::endl;
\end{Verbatim}
prints
\begin{Verbatim}
Column 1
(+1.2319548758e+008,+0.0000000000e+000)
(+1.4179323916e+008,+7.0896619580e+007)
(-8.0802738460e+007,+1.6160547692e+008)
Column 2
(+1.4179323916e+008,-7.0896619580e+007)
(+2.0399822604e+008,+0.0000000000e+000)
(+0.0000000000e+000,+2.3250209650e+008)
Column 3
(-8.0802738460e+007,-1.6160547692e+008)
(+0.0000000000e+000,-2.3250209650e+008)
(+2.6498872674e+008,+0.0000000000e+000)
\end{Verbatim}
Matlab output:
\begin{Verbatim}
  Column 1 

     1.231954875800000e+008                         
     1.417932391600000e+008 +7.089661958000000e+007i
    -8.080273845999999e+007 +1.616054769200000e+008i

  Column 2 

     1.417932391600000e+008 -7.089661958000000e+007i
     2.039982260400000e+008                         
                          0 +2.325020965000000e+008i

  Column 3 

    -8.080273845999999e+007 -1.616054769200000e+008i
                          0 -2.325020965000000e+008i
     2.649887267400000e+008                         
\end{Verbatim}
\newpage



\subsubsection{eig}
Functions%
\pdfdest name {schmatrix.eig} fit
\begin{verbatim}
rvector schmatrix::eig (scmatrix& mEigVect) const throw (cvmexception);
rvector schmatrix::eig () const throw (cvmexception);
\end{verbatim}
solve the
\GO{symmetric eigenvalue problem}{rvector.eig}
and return  real vector with eigenvalues
of a calling hermitian matrix.
The first version
sets  output parameter \verb"mEigVect" to be equal
to  square matrix containing orthogonal
eigenvectors as columns.
All  functions
throw 
 \GOT{cvmexception}{cvmexception} 
in case of inappropriate sizes of the operands
or in case of convergence error.
See also
\GOT{rvector}{rvector},
\GOT{scmatrix}{scmatrix} and
\GOT{schmatrix}{schmatrix}.
Example:
\begin{Verbatim}
using namespace cvm;

std::cout.setf (std::ios::scientific | std::ios::left); 
std::cout.precision (2);
double a[] = {1., 0., 2., 1., -1., 2., 2., -1., 2., 0.,
              0., 3., -1., -2., 0., -3., 3., 0.};
schmatrix m((std::complex<double>*)a,3);
scmatrix me(3);
rvector v(3);

v = m.eig(me);
std::cout << v << std::endl;
cvector vc(v);

std::cout << m * me(1) - me(1) * vc(1);
std::cout << m * me(2) - me(2) * vc(2);
std::cout << m * me(3) - me(3) * vc(3);
// orthogonality check:
std::cout << me(1) % me(2) << std::endl;
std::cout << me(2) % me(3) << std::endl;
std::cout << me(1) % me(3) << std::endl;
\end{Verbatim}
prints
\begin{Verbatim}
-8.13e-001 -3.44e-001 7.16e+000

(1.39e-016,2.22e-016) (5.25e-017,-1.11e-016) (1.94e-016,1.67e-016)
(4.86e-016,4.44e-016) (7.63e-017,0.00e+000) (3.33e-016,2.22e-016)
(-2.22e-016,-8.88e-016) (-5.55e-017,-8.88e-016) (8.88e-016,-5.55e-017)
(-5.17e-017,-9.74e-017)
(-5.81e-017,-5.40e-017)
(2.37e-018,-3.56e-017)
\end{Verbatim}
\newpage



\subsubsection{Cholesky}
Function%
\pdfdest name {schmatrix.cholesky} fit
\begin{verbatim}
scmatrix schmatrix::cholesky () const throw (cvmexception);
\end{verbatim}
forms the Cholesky factorization of  hermitian positive-definite 
matrix $A$ defined as
\begin{equation*}
A=U^H U,
\end{equation*}
where $U$ is upper triangular matrix.
It utilizes one of \verb"?POTRF" routines of the
\URL{LAPACK library}{http://www.netlib.org/lapack}.
Function
creates an object of type \verb"scmatrix" as 
 factorization of a calling matrix.
It throws \GOT{cvmexception}{cvmexception} 
in case of convergence error.
See also
\GOT{scmatrix}{scmatrix} and
\GOT{schmatrix}{schmatrix}.
Example:
\begin{Verbatim}
using namespace cvm;

try {
    double a[] = {3., 0., 2., 1., -1., 2., 2., -1., 3., 0.,
                0., 3., -1., -2., 0., -3., 5., 0.};
    const schmatrix m((std::complex<double>*)a,3);
    
    scmatrix h = m.cholesky();
    std::cout << h << std::endl;
    std::cout << ~h * h - m;
}
catch (std::exception& e) {
    std::cout << "Exception: " << e.what () << std::endl;
}
\end{Verbatim}
prints
\begin{Verbatim}
(1.73205,0) (1.1547,-0.57735) (-0.57735,-1.1547)
(0,0) (1.1547,0) (0,-1.1547)
(0,0) (0,0) (1.41421,0)

(-4.44089e-016,0) (0,0) (0,0)
(0,0) (0,0) (0,0)
(0,0) (0,0) (0,0)
\end{Verbatim}
\newpage



\subsubsection{Bunch-Kaufman}
Function%
\pdfdest name {schmatrix.bunch_kaufman} fit
\begin{verbatim}
scmatrix schmatrix::bunch_kaufman () throw (cvmexception);
\end{verbatim}
forms the Bunch-Kaufman factorization of a calling hermitian matrix 
(cited from the MKL library documentation):
\begin{equation*}
A=PUDU^HP^H,
\end{equation*}
where $A$ is calling matrix,
$P$ is  permutation matrix, $U$ and $L$ are upper and lower triangular
matrices with unit diagonal, and $D$ is  symmetric 
block-diagonal matrix with $1$-by-$1$ and $2$-by-$2$
diagonal blocks. $U$ and $L$ have $2$-by-$2$ unit diagonal 
blocks corresponding to the $2$-by-$2$ blocks of $D$.
It utilizes one of \verb"?HETRF" routines of the
\URL{LAPACK library}{http://www.netlib.org/lapack}.
Function
creates an object of type \verb"scmatrix" as 
 factorization of a calling matrix.
It throws \GOT{cvmexception}{cvmexception} 
in case of convergence error.
See also
\GOT{scmatrix}{scmatrix} and
\GOT{schmatrix}{schmatrix}.
Function is mostly designed to be used for subsequent calls
of \verb"?HETRS", \verb"?HECON" and \verb"?HETRI" routines of the
\URL{LAPACK library}{http://www.netlib.org/lapack}.
Currently it's used internally in \GOT{scmatrix::det}{scmatrix.det} flow
when argument is hermitian but not positive-definite matrix.
\newpage




\subsubsection{identity}
Function%
\pdfdest name {schmatrix.identity} fit
\begin{verbatim}
schmatrix& schmatrix::identity();
\end{verbatim}
sets  calling hermitian matrix to be equal to identity matrix
and returns a reference to
the matrix changed. 
See also \GOT{schmatrix}{schmatrix}.
Example:
\begin{Verbatim}
using namespace cvm;

schmatrix m(3);
m.randomize_real(0.,3.);
m.randomize_imag(-1.,2.);

std::cout << m << std::endl;
std::cout << m.identity();
\end{Verbatim}
prints
\begin{Verbatim}
(1.93548,0) (1.84027,1.08353) (0.429579,-0.614093)
(1.84027,-1.08353) (1.76922,0) (1.71364,1.82788)
(0.429579,0.614093) (1.71364,-1.82788) (0.824915,0)

(1,0) (0,0) (0,0)
(0,0) (1,0) (0,0)
(0,0) (0,0) (1,0)
\end{Verbatim}
\newpage



\subsubsection{vanish}
Function%
\pdfdest name {schmatrix.vanish} fit
\begin{verbatim}
schmatrix& schmatrix::vanish();
\end{verbatim}
sets every element of a calling hermitian 
matrix to be equal to zero
and returns a reference to
the matrix changed.
See also \GOT{schmatrix}{schmatrix}.
Example:
\begin{Verbatim}
using namespace cvm;

schmatrix m(3);
m.randomize_real(0.,3.);
m.randomize_imag(-1.,2.);

std::cout << m << std::endl;
std::cout << m.vanish();
\end{Verbatim}
prints
\begin{Verbatim}
(1.95499,0) (1.03925,0.830378) (0.951628,0.563677)
(1.03925,-0.830378) (0.150426,0) (2.29365,-0.580218)
(0.951628,-0.563677) (2.29365,0.580218) (0.0841395,0)

(0,0) (0,0) (0,0)
(0,0) (0,0) (0,0)
(0,0) (0,0) (0,0)
\end{Verbatim}
\newpage



\subsubsection{randomize\_real}
Function%
\pdfdest name {schmatrix.randomizereal} fit
\begin{verbatim}
schmatrix& schmatrix::randomize_real (TR dFrom, TR dTo);
\end{verbatim}
fills  real part of a calling hermitian matrix with 
pseudo-random numbers distributed between
\verb"dFrom" and \verb"dTo" keeping it to be hermitian.
Function
returns a reference to the matrix changed.
See also
\GOT{schmatrix}{schmatrix}.
Example:
\begin{Verbatim}
using namespace cvm;

schmatrix m(3);
m.randomize_real(0.,3.);
std::cout << m << std::endl;
\end{Verbatim}
prints
\begin{Verbatim}
(1.98245,0) (2.72103,0) (0.167272,0)
(2.72103,0) (0.0285653,0) (1.63765,0)
(0.167272,0) (1.63765,0) (1.15882,0)
\end{Verbatim}
\newpage


\subsubsection{randomize\_imag}
Function%
\pdfdest name {schmatrix.randomizeimag} fit
\begin{verbatim}
schmatrix& schmatrix::randomize_imag (TR dFrom, TR dTo);
\end{verbatim}
fills  imaginary part of a calling hermitian matrix with 
pseudo-random numbers distributed between
\verb"dFrom" and \verb"dTo" keeping it to be hermitian.
Function
returns a reference to the matrix changed.
See also
\GOT{schmatrix}{schmatrix}.
Example:
\begin{Verbatim}
using namespace cvm;

schmatrix m(3);
m.randomize_imag(0.,3.);
std::cout << m << std::endl;
\end{Verbatim}
prints
\begin{Verbatim}
(0,0) (0,0.13834) (0,2.39903)
(0,-0.13834) (0,0) (0,0.609577)
(0,-2.39903) (0,-0.609577) (0,0)
\end{Verbatim}
\newpage

