\subsection{Functions}
Classes%
\pdfdest name {SectionFunctions} fit%
\pdfdest name {Functions} fit{}
\verb"rfunction"  and \verb"cfunction" 
encapsulate elementary functions of real
       or complex numbers.
Typically they are used to convert strings entered by a user to
computable expressions without need to parse same expression
over and over again. Also might be used to simplify computable
expressions and to compute derivatives analytically, i.e.
without using numerical methods. Functions can have zero, one
or more than one variables and can be parameterized.
See online documentation for details.

Operators and functions supported in expressions:
\begin{align*}
+\ -\ *\ /\ \PVerb{^} \quad & \text{add, subtract, multiply, divide, power}\\
\texttt{exp(x)}\quad & e^x \\
\texttt{sqrt(x)}\quad & \sqrt x\\
\texttt{log(x)}\quad & \text{natural logarithm of}\ x\\
\texttt{log10(x)}\quad & \text{common (base 10) logarithm of}\ x\\
\texttt{sin(x)}\quad & \text{sine of}\ x\\
\texttt{cos(x)}\quad & \text{cosine of}\ x\\
\texttt{tan(x)}\quad & \text{tangent of}\ x\\
\texttt{asin(x)}\quad & \text{arc sine of}\ x\\
\texttt{acos(x)}\quad & \text{arc cosine of}\ x\\
\texttt{atan(x)}\quad & \text{arc tangent of}\ x\\
\texttt{sinh(x)}\quad & \text{hyperbolic sine of}\ x\\
\texttt{cosh(x)}\quad & \text{hyperbolic cosine of}\ x\\
\texttt{tanh(x)}\quad & \text{hyperbolic tangent of}\ x\\
\texttt{sinint(x)}\quad & \text{sine integral of}\ x,\quad
\int_0^x \frac{\sin t}{t}dt\\
\texttt{cosint(x)}\quad & \text{cosine integral of}\ x,\quad
-\int_x^\infty \frac{\cos t}{t}dt \\
\texttt{sign(x)}\quad & \text{sign of}\ x,\quad
 \left\{
  \begin{aligned}
-1 & \text{ if } x < 0\\
 0 & \text{ if } x = 0\\
 1 & \text{ if } x > 0
  \end{aligned}
 \right.\\
\texttt{abs(x)}\quad & |x| \\
\texttt{iif(x,expr1,expr2)}\quad & \text{immediate if}\quad
 \left\{
  \begin{aligned}
expr1 & \text{ if } x < 0\\
expr2 & \text{ if } x \ge 0
  \end{aligned}
 \right.\\
\texttt{sat(x,y)}\quad & \text{satellite function of}\ x\
\text{and}\ y,\quad
 \left\{
  \begin{aligned}
1 & \text{ if } x > |y|\\
0 & \text{ if } -|y| \le x \le |y|\\
-1 & \text{ if } x < -|y|
  \end{aligned}
 \right.\\
\texttt{i}\quad & 1\ \text{(one)}\\
 & 0\ \text{(empty field means zero)}
\end{align*}


Example:
\begin{Verbatim}
using namespace cvm;
try {
    rfunction rf("{x,y} cos(x)*cos(y) + x + x + x");
    std::cout << rf.simp() << std::endl;
    std::cout << rf.drv(0) << std::endl;
    std::cout << rf.drv(1) << std::endl;
    double vars[] = {0., 0.};
    std::cout << rf(vars) << std::endl;
}
catch (std::exception& e) {
    std::cout << "Exception " << e.what () << std::endl;
}
\end{Verbatim}
prints
\begin{Verbatim}
{x,y} x*3+cos(x)*cos(y)
{x,y} 3-sin(x)*cos(y)
{x,y} -sin(y)*cos(x)
1
\end{Verbatim}


\subsection{Function vectors and matrices}
Classes%
\pdfdest name {FunctionsVM} fit{}
\verb"rfvector", \verb"cfvector", 
\verb"rfmatrix" and \verb"cfmatrix" 
encapsulate vectors and matrices of elementary functions of real
       or complex numbers.
See online documentation for details.
Example:
\begin{Verbatim}
using namespace cvm;
try {
    rfunction f1("{x} x+sin(x/2)");
    rfunction f2(2.);
    // note: (f2^2) in parenthesis to follow priority:
    std::cout << (f1*f2 + (f2^2)).simp() << std::endl;

    string_array sa;
    sa.push_back ("{x,y} x+y");
    sa.push_back ("{x,y} x-y");
    rfvector fv1(sa), fv2(sa);
    
    rfunction f = fv1 * fv2;
    std::cout << f << std::endl;
    std::cout << f.drv(0) << std::endl; // derivative by x
    std::cout << f.drv(1) << std::endl; // derivative by y


    double vars[] = {1., 2.};
    std::cout << f(vars) << std::endl;
    
    string_array sa1;
    sa1.push_back ("{x,y} x");
    sa1.push_back ("{x,y} y");
    sa1.push_back ("{x,y} 1");
    sa1.push_back ("{x,y} 2");
    sa1.push_back ("{x,y} 3");
    sa1.push_back ("{x,y} 4");
    rfmatrix fm(2,3,sa1);

    rfvector fvm = fv1 * fm;
    std::cout << fvm << std::endl;

    std::cout << std::endl;
    string_array sa2;
    sa2.push_back ("{x,y} 2*x^2+3*y^3");
    sa2.push_back ("{x,y} 4*x^3-5*y^2");
    sa2.push_back ("{x,y} -x^4+6*y^2");
    rfvector fv(sa2);

    rfmatrix fmj = fv.jacobian();
    std::cout << fv << std::endl << fmj << std::endl;
    std::cout << fmj(vars);
}
catch (std::exception& e) {
    std::cout << "Exception " << e.what () << std::endl;
}
\end{Verbatim}
prints
\begin{Verbatim}
{x} (x+sin(x/2))*2+4
{x,y} (x+y)^2+(x-y)^2
{x,y} 4*x
{x,y} 2*(x+y)+(-2)*(x-y)
10
{x,y} (x+y)*x+(x-y)*y {x,y} x+y+(x-y)*2 {x,y} (x+y)*3+(x-y)*4


{x,y} 2*x^2+3*y^3 {x,y} 4*x^3-5*y^2 {x,y} -x^4+6*y^2

{x,y} 4*x {x,y} 9*y^2
{x,y} 12*x^2 {x,y} (-10)*y
{x,y} (-4)*x^3 {x,y} 12*y

4 36
12 -20
-4 24
\end{Verbatim}












\newpage
