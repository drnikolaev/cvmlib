\bigskip
\noindent
\verb"class iarray : public Array<int> {"\\
\verb"public:"\\
\verb"    explicit "\GOT{iarray}{iarray.iarray (int)}\verb" (const int nSize);"\\
\verb"    "\GOT{operator int*}{iarray.operator int*}\verb" ();"\\
\verb"    "\GOT{operator const int*}{iarray.operator const int*}\verb" () const;"\\
\verb"    int& "\GOT{operator []}{iarray.operator []}\verb" (const int nI) throw (cvmexception);"\\
\verb"    int "\GOT{operator []}{iarray.operator []}\verb" (const int nI) const throw (cvmexception);"\\
\verb"    int& "\GOT{operator ()}{iarray.operator ()}\verb" (const int nI) throw (cvmexception);"\\
\verb"    int "\GOT{operator ()}{iarray.operator ()}\verb" (const int nI) const throw (cvmexception);"\\
\verb"};"
\newpage





\subsubsection{iarray (int)}
Constructor%
\pdfdest name {iarray.iarray (int)} fit
\begin{verbatim}
explicit iarray::iarray (const int nSize);
\end{verbatim}
creates an \verb"iarray" object with the size equal to \verb"nSize".
See also \GOT{iarray}{iarray}.

Example of usage:
\begin{verbatim}
using namespace cvm;

iarray a(15);
cout << a;
\end{verbatim}
prints
\begin{verbatim}
0 0 0 0 0 0 0 0 0 0 0 0 0 0 0
\end{verbatim}
\newpage


\subsubsection{operator int*}
The%
\pdfdest name {iarray.operator int*} fit
operator
\begin{verbatim}
iarray::operator int* ();
\end{verbatim}
returns the pointer to the first element of an array.
See also \GOT{iarray}{iarray}.

Example of usage:
\begin{verbatim}
using namespace cvm;

void fill_squares (int n, int* p)
{
    for (int i = 1; i <= n; i++) {
        p[i-1] = i * i;
    }
}


.....


iarray a(15);

fill_squares (a.size(), a);
cout << a;
\end{verbatim}
prints
\begin{verbatim}
1 4 9 16 25 36 49 64 81 100 121 144 169 196 225
\end{verbatim}
\newpage





\subsubsection{operator const int*}
Operator%
\pdfdest name {iarray.operator const int*} fit
\begin{verbatim}
iarray::operator const int* () const;
\end{verbatim}
returns the pointer to the first
element of a constant array.
See also \GOT{iarray}{iarray}.

Example of usage:
\begin{verbatim}
using namespace cvm;

int sum (int n, const int* p)
{
    int i, s = 0;

    for (i = 0; i < n; i++) {
        s += p[i];
    }
    return s;
}

.....

iarray a(4);

a[1] = 1;
a[2] = 2;
a[3] = 3;
a[4] = 4;

cout << sum (a.size(), a) << endl;
\end{verbatim}
prints
\begin{verbatim}
10
\end{verbatim}
\newpage







\subsubsection{operator []}
Indexing operators%
\pdfdest name {iarray.operator []} fit
\begin{verbatim}
int& iarray::operator [] (const int nI)
throw (cvmexception);

int iarray::operator [] (const int nI) const
throw (cvmexception);
\end{verbatim}
provide access to a vector's element. The first version
of the operator is applicable to a non-constant object.
This version returns l-value
in order to make possible write access to an element.
The indexing begins with 1. The operator throws an exception
if \verb"nI" is outside of \verb"[1,size()]" range.
See also \GOT{iarray}{iarray}, \GOT{basic\_array::size()}{basicarray.size}.

Example of usage:
\begin{verbatim}
using namespace cvm;

try {
    iarray a(3);
    const iarray ac(3);

    a[1] = 1;
    a[2] = 2;
    a[3] = 3;

    cout << a << endl << ac[1] << endl;
}
catch (cvmexception& e) {
    cout << "Exception " << e.cause () <<
                    ": " << e.what () << endl;
}
\end{verbatim}
prints
\begin{verbatim}
1 2 3

0
\end{verbatim}
\newpage




\subsubsection{operator ()}
Indexing operators%
\pdfdest name {iarray.operator ()} fit
\begin{verbatim}
int& iarray::operator () (const int nI)
throw (cvmexception);

int iarray::operator () (const int nI) const
throw (cvmexception);
\end{verbatim}
do the same as \GOT{iarray::operator []}{iarray.operator []}.
See also \GOT{iarray}{iarray}.

Example of usage:
\begin{verbatim}
using namespace cvm;

try {
    iarray a(3);
    const iarray ac(3);

    a(1) = 1;
    a(2) = 2;
    a(3) = 3;

    cout << a << endl << ac(1) << endl;
}
catch (cvmexception& e) {
    cout << "Exception " << e.cause () <<
                    ": " << e.what () << endl;
}
\end{verbatim}
prints
\begin{verbatim}
1 2 3

0
\end{verbatim}
\newpage


