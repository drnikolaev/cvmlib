\bigskip
\noindent
\verb"template <typename TR, typename TC>"\\
\verb"class scmatrix : public cmatrix <TR,TC>, public SqMatrix <TR,TC> {"\\
\verb"public:"\\
\verb"    "\GOT{scmatrix}{scmatrix.scmatrix ()}\verb" ();"\\
\verb"    explicit "\GOT{scmatrix}{scmatrix.scmatrix (int)}\verb" (int nMN);"\\
\verb"    "\GOT{scmatrix}{scmatrix.scmatrix (TC*,int)}\verb" (TC* pD, int nMN);"\\
\verb"    "\GOT{scmatrix}{scmatrix.scmatrix (const TC*,int)}\verb" (const TC* pD, int nMN);"\\
\verb"    "\GOT{scmatrix}{scmatrix.scmatrix (const scmatrix&)}\verb" (const scmatrix& m);"\\
\verb"    explicit "\GOT{scmatrix}{scmatrix.scmatrix (const cmatrix&)}\verb" (const cmatrix& m);"\\
\verb"    explicit "\GOT{scmatrix}{scmatrix.scmatrix (const cvector&)}\verb" (const cvector& v);"\\
\verb"    explicit "\GOT{scmatrix}{scmatrix.scmatrix (const srmatrix&,bool)}\verb" (const srmatrix& m, bool bRealPart = true);"\\
\verb"    "\GOT{scmatrix}{scmatrix.scmatrix (TR*,TR*,int)}\verb" (const TR* pRe, const TR* pIm, int nMN);"\\
\verb"    "\GOT{scmatrix}{scmatrix.scmatrix (const srmatrix&, const srmatrix&)}\verb" (const srmatrix& mRe, const srmatrix& mIm);"\\
\verb"    "\GOT{scmatrix}{scmatrix.submatrixctr}\verb" (cmatrix& m, int nRow, int nCol, int size);"\\
\verb"    TC& "\GOT{operator ()}{scmatrix.operator (,)}\verb" (int im, int in) throw (cvmexception);"\\
\verb"    TC "\GOT{operator ()}{scmatrix.operator (,)}\verb" (int im, int in) const throw (cvmexception);"\\
\verb"    cvector "\GOT{operator ()}{scmatrix.operator ()}\verb" (int i) throw (cvmexception);"\\
\verb"    const cvector "\GOT{operator ()}{scmatrix.operator ()}\verb" (int i) const throw (cvmexception);"\\
\verb"    cvector "\GOT{operator []}{scmatrix.operator []}\verb" (int i) throw (cvmexception);"\\
\verb"    const cvector "\GOT{operator []}{scmatrix.operator []}\verb" (int i) const throw (cvmexception);"\\
\verb"    const srmatrix "\GOT{real}{scmatrix.real}\verb" () const;"\\
\verb"    const srmatrix "\GOT{imag}{scmatrix.imag}\verb" () const;"\\
\verb"    scmatrix& "\GOT{operator =}{scmatrix.operator = (const scmatrix&)}\verb" (const scmatrix& m) throw (cvmexception);"\\
\verb"    scmatrix& "\GOT{assign}{scmatrix.assign}\verb" (const cvector& v) throw (cvmexception);"\\
\verb"    scmatrix& "\GOT{assign}{scmatrix.assign}\verb" (const TC* pD);"\\
\verb"    scmatrix& "\GOT{assign}{scmatrix.assign (int, int, const cmatrix&)}\verb" (int nRow, int nCol, const cmatrix& m)"\\
\verb"                      throw (cvmexception);"\\
\verb"    scmatrix& "\GOT{set}{scmatrix.set}\verb" (TC x);"\\
\verb"    scmatrix& "\GOT{assign\_real}{scmatrix.assignreal}\verb" (const srmatrix& mRe) throw (cvmexception);"\\
\verb"    scmatrix& "\GOT{assign\_imag}{scmatrix.assignimag}\verb" (const srmatrix& mIm) throw (cvmexception);"\\
\verb"    scmatrix& "\GOT{set\_real}{scmatrix.setreal}\verb" (TR d);"\\
\verb"    scmatrix& "\GOT{set\_imag}{scmatrix.setimag}\verb" (TR d);"\\
\verb"    scmatrix& "\GOT{resize}{scmatrix.resize}\verb" (int nNewMN) throw (cvmexception);"\\
\verb"    scmatrix& "\GOT{operator <{}<}{scmatrix.operator <<}\verb" (const scmatrix& m) throw (cvmexception);"\\
\verb"    scmatrix "\GOT{operator +}{scmatrix.operator +}\verb" (const scmatrix& m) const"\\
\verb"                         throw (cvmexception);"\\
\verb"    scmatrix "\GOT{operator -}{scmatrix.operator -}\verb" (const scmatrix& m) const"\\
\verb"                         throw (cvmexception);"\\
\verb"    scmatrix& "\GOT{sum}{scmatrix.sum}\verb" (const scmatrix& m1,"\\
\verb"                   const scmatrix& m2) throw (cvmexception);"\\
\verb"    scmatrix& "\GOT{diff}{scmatrix.diff}\verb" (const scmatrix& m1,"\\
\verb"                    const scmatrix& m2) throw (cvmexception);"\\
\verb"    scmatrix& "\GOT{operator +=}{scmatrix.operator +=}\verb" (const scmatrix& m) throw (cvmexception);"\\
\verb"    scmatrix& "\GOT{operator -=}{scmatrix.operator -=}\verb" (const scmatrix& m) throw (cvmexception);"\\
\verb"    scmatrix "\GOT{operator -}{scmatrix.operator - ()}\verb" () const;"\\
\verb"    scmatrix& "\GOT{operator ++}{scmatrix.operator ++}\verb" ();"\\
\verb"    scmatrix& "\GOT{operator ++}{scmatrix.operator ++}\verb" (int);"\\
\verb"    scmatrix& "\GOT{operator -{}-}{scmatrix.operator --}\verb" ();"\\
\verb"    scmatrix& "\GOT{operator -{}-}{scmatrix.operator --}\verb" (int);"\\
\verb"    scmatrix "\GOT{operator *}{scmatrix.operator * (TR)}\verb" (TR d) const;"\\
\verb"    scmatrix "\GOT{operator /}{scmatrix.operator / (TR)}\verb" (TR d) const throw (cvmexception);"\\
\verb"    scmatrix "\GOT{operator *}{scmatrix.operator * (TC)}\verb" (TC z) const;"\\
\verb"    scmatrix "\GOT{operator /}{scmatrix.operator / (TC)}\verb" (TC z) const throw (cvmexception);"\\
\verb"    scmatrix& "\GOT{operator *=}{scmatrix.operator *= (TR)}\verb" (TR d);"\\
\verb"    scmatrix& "\GOT{operator /=}{scmatrix.operator /= (TR)}\verb" (TR d) throw (cvmexception);"\\
\verb"    scmatrix& "\GOT{operator *=}{scmatrix.operator *= (TC)}\verb" (TC z);"\\
\verb"    scmatrix& "\GOT{operator /=}{scmatrix.operator /= (TC)}\verb" (TC z) throw (cvmexception);"\\
\verb"    scmatrix& "\GOT{normalize}{scmatrix.normalize}\verb" ();"\\
\verb"    scmatrix "\GOT{operator \TildaT}{scmatrix.conj}\verb" () const;"\\
\verb"    scmatrix "\GOT{operator !}{scmatrix.transpose}\verb" () const;"\\
\verb"    scmatrix& "\GOT{conj}{scmatrix.conj}\verb" (const scmatrix& m) throw (cvmexception);"\\
\verb"    scmatrix& "\GOT{conj}{scmatrix.conj}\verb" ();"\\
\verb"    scmatrix& "\GOT{transpose}{scmatrix.transpose}\verb" (const scmatrix& m) throw (cvmexception);"\\
\verb"    scmatrix& "\GOT{transpose}{scmatrix.transpose}\verb" ();"\\
\verb"    cvector "\GOT{operator *}{scmatrix.operator * (const cvector&)}\verb" (const cvector& v) const"\\
\verb"                        throw (cvmexception);"\\
\verb"    cmatrix "\GOT{operator *}{scmatrix.operator * (const cmatrix&)}\verb" (const cmatrix& m) const"\\
\verb"                        throw (cvmexception);"\\
\verb"    scmatrix "\GOT{operator *}{scmatrix.operator * (const scmatrix&)}\verb" (const scmatrix& m) const"\\
\verb"                         throw (cvmexception);"\\
\verb"    scmatrix& "\GOT{operator *=}{scmatrix.operator *= (const scmatrix&)}\verb" (const scmatrix& m)"\\
\verb"                           throw (cvmexception);"\\
\verb"    scmatrix& "\GOT{swap\_rows}{scmatrix.swap_rows}\verb" (int n1, int n2) throw (cvmexception);"\\
\verb"    scmatrix& "\GOT{swap\_cols}{scmatrix.swap_cols}\verb" (int n1, int n2) throw (cvmexception);"\\
\verb"    cvector "\GOT{solve}{scmatrix.solve}\verb" (const cvector& vB) const throw (cvmexception);"\\
\verb"    cmatrix "\GOT{solve}{scmatrix.solve}\verb" (const cmatrix& mB) const throw (cvmexception);"\\
\verb"    cvector "\GOT{solve}{scmatrix.solve}\verb" (const cvector& vB, TR& dErr) const"\\
\verb"                   throw (cvmexception);"\\
\verb"    cmatrix "\GOT{solve}{scmatrix.solve}\verb" (const cmatrix& mB, TR& dErr) const"\\
\verb"                   throw (cvmexception);"\\
\verb"    cvector "\GOT{solve\_tran}{scmatrix.solvetran}\verb" (const cvector& vB) const throw (cvmexception);"\\
\verb"    cmatrix "\GOT{solve\_tran}{scmatrix.solvetran}\verb" (const cmatrix& mB) const throw (cvmexception);"\\
\verb"    cvector "\GOT{solve\_tran}{scmatrix.solvetran}\verb" (const cvector& vB, TR& dErr) const"\\
\verb"                        throw (cvmexception);"\\
\verb"    cmatrix "\GOT{solve\_tran}{scmatrix.solvetran}\verb" (const cmatrix& mB, TR& dErr) const"\\
\verb"                        throw (cvmexception);"\\
\verb"    cvector "\GOT{solve\_conj}{scmatrix.solveconj}\verb" (const cvector& vB) const throw (cvmexception);"\\
\verb"    cmatrix "\GOT{solve\_conj}{scmatrix.solveconj}\verb" (const cmatrix& mB) const throw (cvmexception);"\\
\verb"    cvector "\GOT{solve\_conj}{scmatrix.solveconj}\verb" (const cvector& vB, TR& dErr) const"\\
\verb"                        throw (cvmexception);"\\
\verb"    cmatrix "\GOT{solve\_conj}{scmatrix.solveconj}\verb" (const cmatrix& mB, TR& dErr) const"\\
\verb"                        throw (cvmexception);"\\
\verb"    cvector "\GOT{operator \%}{scmatrix.operator percent (cvector)}\verb" (const cvector& vB) const throw (cvmexception);"\\
\verb"    cvector "\GOT{operator /}{scmatrix.operator / (cvector)}\verb" (const cvector& vB) const throw (cvmexception);"\\
\verb"    cvector "\GOT{solve\_lu}{scmatrix.solvelu}\verb" (const scmatrix& mLU, const int* pPivots,"\\
\verb"                      const cvector& vB, TR& dErr) throw (cvmexception);"\\
\verb"    cvector "\GOT{solve\_lu}{scmatrix.solvelu}\verb" (const scmatrix& mLU, const int* pPivots,"\\
\verb"                      const cvector& vB) throw (cvmexception);"\\
\verb"    cmatrix "\GOT{solve\_lu}{scmatrix.solvelu}\verb" (const scmatrix& mLU, const int* pPivots,"\\
\verb"                      const cmatrix& mB, TR& dErr) throw (cvmexception);"\\
\verb"    cmatrix "\GOT{solve\_lu}{scmatrix.solvelu}\verb" (const scmatrix& mLU, const int* pPivots,"\\
\verb"                      const cmatrix& mB) throw (cvmexception);"\\
\verb"    TC "\GOT{det}{scmatrix.det}\verb" () const throw (cvmexception);"\\
\verb"    scmatrix& "\GOT{low\_up}{scmatrix.low_up}\verb" (const scmatrix& m,"\\
\verb"                      int* nPivots) throw (cvmexception);"\\
\verb"    scmatrix "\GOT{low\_up}{scmatrix.low_up}\verb" (int* nPivots) const throw (cvmexception);"\\
\verb"    TR "\GOT{cond}{scmatrix.cond}\verb" () const throw (cvmexception);"\\
\verb"    scmatrix& "\GOT{inv}{scmatrix.inv}\verb" (const scmatrix& mArg) throw (cvmexception);"\\
\verb"    scmatrix "\GOT{inv}{scmatrix.inv}\verb" () const throw (cvmexception);"\\
\verb"    scmatrix& "\GOT{exp}{scmatrix.exp}\verb" (const scmatrix& mArg, TR tol = cvmMachSp ())"\\
\verb"                   throw (cvmexception);"\\
\verb"    scmatrix "\GOT{exp}{scmatrix.exp}\verb" (TR tol = cvmMachSp ()) const throw (cvmexception);"\\
\verb"    scmatrix& "\GOT{polynom}{scmatrix.polynom}\verb" (const scmatrix& m, const cvector& v)"\\
\verb"                       throw (cvmexception);"\\
\verb"    scmatrix "\GOT{polynom}{scmatrix.polynom}\verb" (const cvector& v) const"\\
\verb"                      throw (cvmexception);"\\
\verb"    cvector "\GOT{eig}{scmatrix.eig}\verb" (scmatrix& mEigVect, bool bRightVect = true) const"\\
\verb"                 throw (cvmexception);"\\
\verb"    cvector "\GOT{eig}{scmatrix.eig}\verb" () const throw (cvmexception);"\\
\verb"    scmatrix& "\GOT{cholesky}{scmatrix.cholesky}\verb" (const schmatrix& m) throw (cvmexception);"\\
\verb"    scmatrix& "\GOT{bunch\_kaufman}{scmatrix.bunch_kaufman}\verb" (const schmatrix& m,"\\
\verb"                             int* pivots) throw (cvmexception);"\\
\verb"    void "\GOT{qr}{scmatrix.qr}\verb"(scmatrix& mQ, scmatrix& mR) const throw (cvmexception);"\\
\verb"    void "\GOT{lq}{scmatrix.lq}\verb"(scmatrix& mL, scmatrix& mQ) const throw (cvmexception);"\\
\verb"    void "\GOT{rq}{scmatrix.rq}\verb"(scmatrix& mR, scmatrix& mQ) const throw (cvmexception);"\\
\verb"    void "\GOT{ql}{scmatrix.ql}\verb"(scmatrix& mQ, scmatrix& mL) const throw (cvmexception);"\\
\verb"    scmatrix& "\GOT{identity}{scmatrix.identity} ();\\
\verb"    scmatrix& "\GOT{vanish}{scmatrix.vanish}\verb" ();"\\
\verb"    scmatrix& "\GOT{randomize\_real}{scmatrix.randomizereal}\verb" (TR dFrom, TR dTo);"\\
\verb"    scmatrix& "\GOT{randomize\_imag}{scmatrix.randomizeimag}\verb" (TR dFrom, TR dTo);"\\
\verb"};"
\newpage


\subsubsection{scmatrix ()}
Constructor%
\pdfdest name {scmatrix.scmatrix ()} fit
\begin{verbatim}
scmatrix::scmatrix ();
\end{verbatim}
creates  empty \verb"scmatrix" object.
See also \GOT{scmatrix}{scmatrix}.
Example:
\begin{Verbatim}
using namespace cvm;

scmatrix m;
std::cout << m.msize() << std::endl 
          << m.nsize() << std::endl
          << m.size() << std::endl;
m.resize(3);
std::cout << m;
\end{Verbatim}
prints
\begin{Verbatim}
0
0
0
(0,0) (0,0) (0,0)
(0,0) (0,0) (0,0)
(0,0) (0,0) (0,0)
\end{Verbatim}
\newpage



\subsubsection{scmatrix (int)}
Constructor%
\pdfdest name {scmatrix.scmatrix (int)} fit
\begin{verbatim}
explicit scmatrix::scmatrix (int nMN);
\end{verbatim}
creates  $n\times n$ \verb"scmatrix" object where $n$ is passed in
\verb"nMN" parameter.
Constructor throws  \GOT{cvmexception}{cvmexception}
in case of non-positive size passed or memory allocation failure.
See also \GOT{scmat\-rix}{scmatrix}.
Example:
\begin{Verbatim}
using namespace cvm;

scmatrix m (4);
std::cout << m.msize() << std::endl 
          << m.nsize() << std::endl
          << m.size() << std::endl << m;
\end{Verbatim}
prints
\begin{Verbatim}
4
4
16
(0,0) (0,0) (0,0) (0,0)
(0,0) (0,0) (0,0) (0,0)
(0,0) (0,0) (0,0) (0,0)
(0,0) (0,0) (0,0) (0,0)
\end{Verbatim}
\newpage




\subsubsection{scmatrix (TC*,int)}
Constructor%
\pdfdest name {scmatrix.scmatrix (TC*,int)} fit
\begin{verbatim}
scmatrix::scmatrix (TC* pD, int nMN);
\end{verbatim}
creates  $n\times n$ \verb"scmatrix" object where $n$ is passed in
\verb"nMN" parameter.
Unlike others, this constructor \textit{does not allocate  memory}.
It just shares  memory with an array pointed to by \verb"pD".
Constructor throws  \GOT{cvmexception}{cvmexception}
in case of non-positive size passed.
See also \GOT{scmatrix}{scmatrix},
\GOT{scmatrix (const TC*,int)}{scmatrix.scmatrix (const TC*,int)}.
Example:
\begin{Verbatim}
using namespace cvm;
double a[] = {1., 1., 1., 1., 1., 1., 1., 1., 1.,
              1., 1., 1., 1., 1., 1., 1., 1., 1.};
scmatrix m ((std::complex<double>*) a, 3);

m(1,1) = std::complex<double>(2.,3.);
std::cout << m << std::endl;
std::cout << a[0] << " " << a[1] << " " << a[2] << " "
          << a[3] << " " << a[4] << " " << a[5] << " " << std::endl;
\end{Verbatim}
prints
\begin{Verbatim}
(2,3) (1,1) (1,1)
(1,1) (1,1) (1,1)
(1,1) (1,1) (1,1)

2 3 1 1 1 1
\end{Verbatim}
\newpage


\subsubsection{scmatrix (const TC*,int)}
Constructor%
\pdfdest name {scmatrix.scmatrix (const TC*,int)} fit
\begin{verbatim}
scmatrix::scmatrix (const TC* pD, int nMN);
\end{verbatim}
creates  $n\times n$ \verb"scmatrix" object where $n$ is passed in
\verb"nMN" parameter
and copies $n*n$ elements of an array  \verb"pD" to it \GOT{by colums}{SubSubSectionStorage}.
Constructor throws  \GOT{cvmexception}{cvmexception}
in case of non-positive size passed.
See also \GOT{scmatrix}{scmatrix},
\GOT{scmatrix (TC*,int)}{scmatrix.scmatrix (TC*,int)}.
Example:
\begin{Verbatim}
using namespace cvm;
const double a[] = {1., 1., 1., 1., 1., 1., 1., 1., 1.,
                    1., 1., 1., 1., 1., 1., 1., 1., 1.};
scmatrix m ((const std::complex<double>*) a, 3);

m(1,1) = std::complex<double>(2.,3.);
std::cout << m << std::endl;
std::cout << a[0] << " " << a[1] << " " << a[2] << " "
          << a[3] << " " << a[4] << " " << a[5] << " " << std::endl;
\end{Verbatim}
prints
\begin{Verbatim}
(2,3) (1,1) (1,1)
(1,1) (1,1) (1,1)
(1,1) (1,1) (1,1)

1 1 1 1 1 1
\end{Verbatim}
\newpage




\subsubsection{scmatrix (const scmatrix\&)}
Copy constructor%
\pdfdest name {scmatrix.scmatrix (const scmatrix&)} fit
\begin{verbatim}
scmatrix::scmatrix (const scmatrix& m)
\end{verbatim}
creates  \verb"scmatrix" object as a copy of \verb"m".
Constructor throws  \GOT{cvmexception}{cvmexception}
in case of memory allocation failure.
See also \GOT{scmatrix}{scmatrix}.
Example:
\begin{Verbatim}
using namespace cvm;

double a[] = {1., 2., 3., 4., 5., 6., 7., 8.};
scmatrix m ((std::complex<double>*) a, 2);
scmatrix mc(m);

m(1,1) = std::complex<double>(7.77,7.77);
std::cout << m << std::endl << mc;
\end{Verbatim}
prints
\begin{Verbatim}
(7.77,7.77) (5,6)
(3,4) (7,8)

(1,2) (5,6)
(3,4) (7,8)
\end{Verbatim}
\newpage





\subsubsection{scmatrix (const cmatrix\&)}
Constructor%
\pdfdest name {scmatrix.scmatrix (const cmatrix&)} fit
\begin{verbatim}
explicit scmatrix::scmatrix (const cmatrix& m)
\end{verbatim}
creates  \verb"scmatrix" object as a copy of a matrix \verb"m".
It's assumed that $m\times n$ matrix \verb"m" must have equal
sizes, i.e. $m = n$ is satisfied.
Cnstructor throws  \GOT{cvmexception}{cvmexception}
if this is not true or in case of memory allocation failure.
Please note that this constructor is \emph{not explicit} anymore.
See also \GOT{scmatrix}{scmatrix}.
Example:
\begin{Verbatim}
using namespace cvm;

double a[] = {1., 2., 3., 4., 5., 6.,
              7., 8., 9., 10., 11., 12.};
cmatrix m((std::complex<double>*) a, 2, 3);
std::cout << m << std::endl;

m.resize(3, 3);
scmatrix ms (m);
std::cout << ms;
\end{Verbatim}
prints
\begin{Verbatim}
(1,2) (5,6) (9,10)
(3,4) (7,8) (11,12)

(1,2) (5,6) (9,10)
(3,4) (7,8) (11,12)
(0,0) (0,0) (0,0)
\end{Verbatim}
\newpage




\subsubsection{scmatrix (const cvector\&)}
Constructor%
\pdfdest name {scmatrix.scmatrix (const cvector&)} fit
\begin{verbatim}
explicit scmatrix::scmatrix (const cvector& v);
\end{verbatim}
creates  \verb"scmatrix" object
of size \verb"v.size()" by \verb"v.size()"
and assigns vector \verb"v" to its main diagonal.
Constructor throws  \GOT{cvmexception}{cvmexception}
in case of memory allocation failure.
See also \GOT{scmatrix}{scmatrix}, \GOT{cvector}{cvector}.
Example:
\begin{Verbatim}
using namespace cvm;

double a[] = {1., 2., 3., 4., 5., 6., 7., 8.};
cvector v((std::complex<double>*) a, 4);
scmatrix m(v);
std::cout << m;
\end{Verbatim}
prints
\begin{Verbatim}
(1,2) (0,0) (0,0) (0,0)
(0,0) (3,4) (0,0) (0,0)
(0,0) (0,0) (5,6) (0,0)
(0,0) (0,0) (0,0) (7,8)
\end{Verbatim}
\newpage



\subsubsection{scmatrix (const srmatrix\&,bool)}
Constructor%
\pdfdest name {scmatrix.scmatrix (const srmatrix&,bool)} fit
\begin{verbatim}
explicit scmatrix::scmatrix (const srmatrix& m, bool bRealPart = true);
\end{verbatim}
creates  \verb"scmatrix" object
having the same dimension as real matrix \verb"m"
and copies  matrix \verb"m" to its real part if
\verb"bRealPart" is \verb"true" or
to its imaginary part otherwise.
See also \GOT{scmatrix}{scmatrix}, \GOT{srmatrix}{srmatrix}.
Example:
\begin{Verbatim}
using namespace cvm;

double a[] = {1., 2., 3., 4., 5., 6., 7., 8., 9.};
const srmatrix m(a, 3);
scmatrix mr(m), mi(m, false);
std::cout << mr << std::endl << mi;
\end{Verbatim}
prints
\begin{Verbatim}
(1,0) (4,0) (7,0)
(2,0) (5,0) (8,0)
(3,0) (6,0) (9,0)

(0,1) (0,4) (0,7)
(0,2) (0,5) (0,8)
(0,3) (0,6) (0,9)
\end{Verbatim}
\newpage




\subsubsection{scmatrix (const TR*,const TR*,int)}
Constructor%
\pdfdest name {scmatrix.scmatrix (TR*,TR*,int)} fit
\begin{verbatim}
scmatrix::scmatrix (const TR* pRe, const TRl* pIm, int nMN);
\end{verbatim}
creates  \verb"scmatrix" object
of size \verb"nMN" by \verb"nMN" and copies every
element of arrays pointed to by \verb"pRe" and \verb"pIm"
to  real and imaginary part of the matrix created respectively.
Use \verb"NULL" pointer to fill up appropriate
part with zero values.
Constructor throws  \GOT{cvmexception}{cvmexception}
in case of memory allocation failure.
See also \GOT{scmatrix}{scmatrix}.
Example:
\begin{Verbatim}
using namespace cvm;

double re[] = {1., 2., 3., 4.};
double im[] = {4., 3., 2., 1.};
scmatrix m(re, im, 2);
std::cout << m << std::endl;
re[0] = 7.777;
std::cout << m << std::endl;

const double rec[] = {1., 2., 3., 4.};
const scmatrix mc (rec, NULL, 2);
std::cout << mc;
\end{Verbatim}
prints
\begin{Verbatim}
(1,4) (3,2)
(2,3) (4,1)

(1,4) (3,2)
(2,3) (4,1)

(1,0) (3,0)
(2,0) (4,0)
\end{Verbatim}
\newpage





\subsubsection{scmatrix (const srmatrix\&, const srmatrix\&)}
Constructor%
\pdfdest name {scmatrix.scmatrix (const srmatrix&, const srmatrix&)} fit
\begin{verbatim}
scmatrix::scmatrix (const srmatrix& mRe, const srmatrix& mIm);
\end{verbatim}
creates  \verb"scmatrix" object
of the same size as \verb"mRe" and \verb"mIm" has
(it throws \GOT{cvmexception}{cvmexception}
if \verb"mRe" and
\verb"mIm" have different sizes)
and copies matrices \verb"mRe" and \verb"mIm"
to  real and imaginary part of the matrix created.
Constructor throws  \GOT{cvmexception}{cvmexception}
in case of memory allocation failure.
See also \GOT{scmatrix}{scmatrix}, \GOT{srmatrix}{srmatrix}.
Example:
\begin{Verbatim}
using namespace cvm;

srmatrix mr(3), mi(3);
mr.set(1.);
mi.set(2.);
const scmatrix mc(mr, mi);
std::cout << mc;
\end{Verbatim}
prints
\begin{Verbatim}
(1,2) (1,2) (1,2)
(1,2) (1,2) (1,2)
(1,2) (1,2) (1,2)
\end{Verbatim}
\newpage




\subsubsection{submatrix}
Submatrix constructor%
\pdfdest name {scmatrix.submatrixctr} fit
\begin{verbatim}
scmatrix::scmatrix (cmatrix& m, int nRow, int nCol, int size);
\end{verbatim}
creates  \verb"scmatrix" object as  \emph{submatrix} of \verb"m".
It means that the matrix object created shares  memory with some part
of \verb"m". This part is defined by its upper left corner (parameters 
\verb"nRow" and \verb"nCol", both are \Based)
and its size (parameter \verb"size").
See also \GOT{scmatrix}{scmatrix}.
Example:
\begin{Verbatim}
using namespace cvm;

cmatrix m(4,5);
scmatrix subm(m,2,2,2);
subm.set(std::complex<double>(1.,2.));
std::cout << m;
\end{Verbatim}
prints
\begin{Verbatim}
(0,0) (0,0) (0,0) (0,0) (0,0)
(0,0) (1,2) (1,2) (0,0) (0,0)
(0,0) (1,2) (1,2) (0,0) (0,0)
(0,0) (0,0) (0,0) (0,0) (0,0)
\end{Verbatim}
\newpage




\subsubsection{operator (,)}
Indexing operators%
\pdfdest name {scmatrix.operator (,)} fit
\begin{verbatim}
TC& scmatrix::operator () (int im, int in) throw (cvmexception);
TC scmatrix::operator () (int im, int in) const throw (cvmexception);
\end{verbatim}
provide access to a particular element of a calling matrix. The first version
of  operator is applicable to  non-constant object.
This version returns \emph{l-value}
in order to make possible write access to an element.
Both operators are \Based.
Operators throw \GOT{cvmexception}{cvmexception}
if some of parameters passed
is outside of \verb"[1,msize()]" range.
Operators are \emph{inherited}
in the class
\GOT{scbmatrix}{scbmatrix} 
and \emph{redefined}
in the class \GOT{schmatrix}{schmatrix}.
See also \GOT{scmatrix}{scmatrix},
\GOT{Matrix::msize()}{Matrix.msize},
\GOT{Matrix::nsize()}{Matrix.nsize}.
Example:
\begin{Verbatim}
using namespace cvm;

scmatrix m (3);
m.set(std::complex<double>(1.,2.));
std::cout << m(1,1) << std::endl;

m(2,2) = std::complex<double>(7.77,7.77);
std::cout << m;
\end{Verbatim}
prints
\begin{Verbatim}
(1,2)
(1,2) (1,2) (1,2)
(1,2) (7.77,7.77) (1,2)
(1,2) (1,2) (1,2)
\end{Verbatim}
\newpage




\subsubsection{operator ()}
Indexing operators%
\pdfdest name {scmatrix.operator ()} fit
\begin{verbatim}
cvector scmatrix::operator () (int i) throw (cvmexception);
const rcector scmatrix::operator () (int i) const throw (cvmexception);
\end{verbatim}
provide access to \hbox{$i$-th} column of a calling matrix. The first version
of  operator is applicable to  non-constant object and
\emph{returns  l-value}, i.e.  vector returned shares  memory
with  \hbox{$i$-th} column of a calling matrix
in order to make possible write access to it.
The second version creates  \emph{copy} of a column and therefore is
\emph{not  l-value}.
Both operators are \Based.
Operators throw \GOT{cvmexception}{cvmexception}
if  parameter \verb"i" is outside of \verb"[1,nsize()]" range.
Operators are \emph{redefined}
in the classes \GOT{scbmatrix}{scbmatrix}
and \GOT{schmatrix}{schmatrix}.
See also \GOT{scmatrix}{scmatrix}.
Example:
\begin{Verbatim}
using namespace cvm;

try {
    double a[] = {1., 2., 3., 4., 5., 6., 7., 8., 9.,
                  10., 11., 12., 13., 14., 15., 16., 17., 18.};
    const scmatrix m ((std::complex<double>*)a, 3);
    scmatrix ms(3);
    std::cout << m(2) << std::endl;

    ms(2) = m(3);
    std::cout << ms;
}
catch (std::exception& e) {
    std::cout << "Exception " << e.what () << std::endl;
}
\end{Verbatim}
prints
\begin{Verbatim}
(7,8) (9,10) (11,12)

(0,0) (13,14) (0,0)
(0,0) (15,16) (0,0)
(0,0) (17,18) (0,0)
\end{Verbatim}
\newpage




\subsubsection{operator []}
Indexing operators%
\pdfdest name {scmatrix.operator []} fit
\begin{verbatim}
cvector scmatrix::operator [] (int i) throw (cvmexception);
const cvector scmatrix::operator [] (int i) const throw (cvmexception);
\end{verbatim}
provide access to \hbox{$i$-th} row of a calling matrix. The first version
is applicable to  non-constant object and
\emph{returns  l-value}, i.e.  vector returned shares  memory
with  \hbox{$i$-th} row of a calling matrix
in order to make possible write access to it.
The second version creates  \emph{copy} of a row and therefore is
\emph{not  l-value}.
Both operators are \Based.
Operators throw \GOT{cvmexception}{cvmexception}
if  parameter \verb"i" is outside of \verb"[1,msize()]" range.
Operators are \emph{redefined}
in  the classes \GOT{scbmatrix}{scbmatrix}
and \GOT{schmatrix}{schmatrix}.
See also \GOT{scmatrix}{scmatrix},
\GOT{Matrix::msize()}{Matrix.msize}.
Example:
\begin{Verbatim}
using namespace cvm;

try {
    double a[] = {1., 2., 3., 4., 5., 6., 7., 8., 9.,
                  10., 11., 12., 13., 14., 15., 16., 17., 18.};
    const scmatrix m ((std::complex<double>*)a, 3);
    scmatrix ms(3);
    std::cout << m[2] << std::endl;

    ms[2] = m[3];
    std::cout << ms;
}
catch (std::exception& e) {
    std::cout << "Exception " << e.what () << std::endl;
}
\end{Verbatim}
prints
\begin{Verbatim}
(3,4) (9,10) (15,16)

(0,0) (0,0) (0,0)
(5,6) (11,12) (17,18)
(0,0) (0,0) (0,0)
\end{Verbatim}
\newpage



\subsubsection{real}
Function%
\pdfdest name {scmatrix.real} fit
\begin{verbatim}
const srmatrix scmatrix::real () const;
\end{verbatim}
creates an object of type \verb"const srmatrix"
as  real part
of a calling matrix.
Please note that, unlike
\GO{cvector::real}{cvector.real}, this
function creates new object \emph{not sharing}  memory
with  real part of a calling matrix, i.e.
 matrix returned is \emph{not  l-value}.
Function is \emph{redefined} in the classes
\GOT{scbmatrix}{scbmatrix} and
\GOT{schmatrix}{schmatrix}.
See also
\GOT{srmatrix}{srmatrix},
\GOT{scmatrix}{scmatrix}.
Example:
\begin{Verbatim}
using namespace cvm;

double a[] = {1., 2., 3., 4., 5., 6., 7., 8.};
scmatrix m((std::complex<double>*) a, 2);
std::cout << m << std::endl << m.real();
\end{Verbatim}
prints
\begin{Verbatim}
(1,2) (5,6)
(3,4) (7,8)

1 5
3 7
\end{Verbatim}
\newpage




\subsubsection{imag}
Function%
\pdfdest name {scmatrix.imag} fit
\begin{verbatim}
const srmatrix scmatrix::imag () const;
\end{verbatim}
creates an object of type \verb"const srmatrix"
as  imaginary part
of a calling matrix.
Please note that, unlike
\GO{cvector::imag}{cvector.imag}, this
function creates new object \emph{not sharing}  memory
with  imaginary part of a calling matrix, i.e.
 matrix returned is \emph{not  l-value}.
Function is \emph{redefined} in the classes
\GOT{scbmatrix}{scbmatrix} and
\GOT{schmatrix}{schmatrix}.
See also
\GOT{srmatrix}{srmatrix},
\GOT{scmatrix}{scmatrix}.
Example:
\begin{Verbatim}
using namespace cvm;

double a[] = {1., 2., 3., 4., 5., 6., 7., 8.};
scmatrix m((std::complex<double>*) a, 2);
std::cout << m << std::endl << m.imag();
\end{Verbatim}
prints
\begin{Verbatim}
(1,2) (5,6)
(3,4) (7,8)

2 6
4 8
\end{Verbatim}
\newpage




\subsubsection{operator = (const scmatrix\&)}
Operator%
\pdfdest name {scmatrix.operator = (const scmatrix&)} fit
\begin{verbatim}
scmatrix& scmatrix::operator = (const scmatrix& m)
throw (cvmexception);
\end{verbatim}
sets  every element of a calling matrix to a value of
appropriate element of a matrix \verb"m"
and returns a reference to
the matrix changed.
Operator throws  \GOT{cvmexception}{cvmexception}
in case of different matrix sizes.
Operator is \emph{redefined} in the classes
\GOT{scbmatrix}{scbmatrix} 
and \GOT{schmatrix}{schmatrix}.
See also \GOT{scmatrix}{scmatrix}.
Example:
\begin{Verbatim}
using namespace cvm;

try {
    double a[] = {1., 2., 3., 4., 5., 6., 7., 8.};
    const scmatrix m1((std::complex<double>*) a, 2);
    scmatrix m2(2);

    m2 = m1;
    std::cout << m2;
}
catch (std::exception& e) {
    std::cout << "Exception " << e.what () << std::endl;
}
\end{Verbatim}
prints
\begin{Verbatim}
(1,2) (5,6)
(3,4) (7,8)
\end{Verbatim}
\newpage





\subsubsection{assign (const TC*)}
Function%
\pdfdest name {scmatrix.assign} fit
\begin{verbatim}
scmatrix& scmatrix::assign (const cvector& v) throw (cvmexception);
scmatrix& scmatrix::assign (const TC* pD);
\end{verbatim}
sets every element of a calling matrix to a value of
appropriate element of  vector~\verb'v'
or  array pointed to by~\verb"pD"
and returns a reference to
the matrix changed.
In first version it's assumed that vector passed is long
enough to fill calling matrix. Function throws \GOT{cvmexception}{cvmexception} 
otherwise.
Function is \emph{redefined} in the classes
\GOT{scbmatrix}{scbmatrix} 
and \GOT{schmatrix}{schmatrix}.
See also \GOT{scmatrix}{scmatrix}.
Example:
\begin{Verbatim}
using namespace cvm;

double a[] = {1., 2., 3., 4., 5., 6., 7., 8., 9.,
              10., 11., 12., 13., 14., 15., 16., 17., 18.};
scmatrix m(3);
m.assign((std::complex<double>*) a);
std::cout << m;
\end{Verbatim}
prints
\begin{Verbatim}
(1,2) (7,8) (13,14)
(3,4) (9,10) (15,16)
(5,6) (11,12) (17,18)
\end{Verbatim}
\newpage




\subsubsection{assign (int, int, const cmatrix\&)}
Function%
\pdfdest name {scmatrix.assign (int, int, const cmatrix&)} fit
\begin{verbatim}
scmatrix& scmatrix::assign (int nRow, int nCol, const cmatrix& m)
throw (cvmexception);
\end{verbatim}
sets sub-matrix of a calling matrix beginning with \Based row
\verb"nRow" and column \verb"nCol" to a matrix \verb"m" and
returns a reference to the matrix changed. Function throws 
 \GOT{cvmexception}{cvmexception} if \verb"nRow"
or \verb"nCol" are not positive or matrix \verb"m" doesn't fit.
Function is \emph{redefined} in the class
\GOT{schmatrix}{schmatrix}.
See also \GOT{cmatrix}{cmatrix}, \GOT{scmatrix}{scmatrix}.
Example:
\begin{Verbatim}
using namespace cvm;

scmatrix m1(5);
cmatrix m2(2,3);
m1.set(std::complex<double>(1.,1.));
m2.set(std::complex<double>(2.,2.));
m1.assign(2,3,m2);
std::cout << m1;
\end{Verbatim}
prints
\begin{Verbatim}
(1,1) (1,1) (1,1) (1,1) (1,1)
(1,1) (1,1) (2,2) (2,2) (2,2)
(1,1) (1,1) (2,2) (2,2) (2,2)
(1,1) (1,1) (1,1) (1,1) (1,1)
(1,1) (1,1) (1,1) (1,1) (1,1)
\end{Verbatim}
\newpage



\subsubsection{set (TC)}
Function%
\pdfdest name {scmatrix.set} fit
\begin{verbatim}
scmatrix& scmatrix::set (TC x);
\end{verbatim}
sets every element of a calling matrix to a value of
parameter \verb"x" and returns a reference to
the matrix changed.
Use \GOT{vanish}{scmatrix.vanish} to set every element
of a calling matrix to be equal to zero.
Function is \emph{redefined} in the classes
\GOT{scbmatrix}{scbmatrix} and
\emph{not applicable} to objects of the class
\GOT{srsmatrix}{srsmatrix} (i.e. \GOT{cvmexception}{cvmexception} would be thrown
in case of using it for objects of this class).
See also \GOT{scmatrix}{scmatrix}.
Example:
\begin{Verbatim}
using namespace cvm;

scmatrix m(3);
m.set(std::complex<double>(1.,2.));
std::cout << m;
\end{Verbatim}
prints
\begin{Verbatim}
(1,2) (1,2) (1,2)
(1,2) (1,2) (1,2)
(1,2) (1,2) (1,2)
\end{Verbatim}
\newpage




\subsubsection{assign\_real}
Function%
\pdfdest name {scmatrix.assignreal} fit
\begin{verbatim}
scmatrix& scmatrix::assign_real (const srmatrix& mRe)
throw (cvmexception);
\end{verbatim}
sets real part of every element of a calling matrix to a value of
appropriate element of a matrix \verb"mRe"
and returns a reference to
the matrix changed.
Function throws  \GOT{cvmexception}{cvmexception}
in case of different sizes of the operands.
Function is \emph{redefined} in the classes
\GOT{scbmatrix}{scbmatrix} 
and \GOT{schmatrix}{schmatrix}.
See also \GOT{scmatrix}{scmatrix} and \GOT{srmatrix}{srmatrix}.
Example:
\begin{Verbatim}
using namespace cvm;
scmatrix mc(3);
srmatrix m(3);
m.set(1.);
mc.assign_real(m);
std::cout << mc;
\end{Verbatim}
prints
\begin{Verbatim}
(1,0) (1,0) (1,0)
(1,0) (1,0) (1,0)
(1,0) (1,0) (1,0)
\end{Verbatim}
\newpage




\subsubsection{assign\_imag}
Function%
\pdfdest name {scmatrix.assignimag} fit
\begin{verbatim}
scmatrix& scmatrix::assign_imag (const srmatrix& mIm)
throw (cvmexception);
\end{verbatim}
sets imaginary part of every element of a calling matrix to a value of
appropriate element of a matrix \verb"mIm"
and returns a reference to
the matrix changed.
Function throws  \GOT{cvmexception}{cvmexception}
in case of different sizes of the operands.
Function is \emph{redefined} in the classes
\GOT{scbmatrix}{scbmatrix} 
and \GOT{schmatrix}{schmatrix}.
See also \GOT{scmatrix}{scmatrix} and \GOT{srmatrix}{srmatrix}.
Example:
\begin{Verbatim}
using namespace cvm;
scmatrix mc(3);
srmatrix m(3);
m.set(1.);
mc.assign_imag(m);
std::cout << mc;
\end{Verbatim}
prints
\begin{Verbatim}
(0,1) (0,1) (0,1)
(0,1) (0,1) (0,1)
(0,1) (0,1) (0,1)
\end{Verbatim}
\newpage




\subsubsection{set\_real}
Function%
\pdfdest name {scmatrix.setreal} fit
\begin{verbatim}
scmatrix& scmatrix::set_real (TR d);
\end{verbatim}
sets real part of every element of a calling matrix to a value of
parameter \verb"d"
and returns a reference to the matrix changed.
Function is \emph{redefined} in the classes
\GOT{scbmatrix}{scbmatrix} 
and \GOT{schmatrix}{schmatrix}.
See also \GOT{scmatrix}{scmatrix}.
Example:
\begin{Verbatim}
using namespace cvm;

scmatrix m(3);
m.set_real(1.);
std::cout << m;
\end{Verbatim}
prints
\begin{Verbatim}
(1,0) (1,0) (1,0)
(1,0) (1,0) (1,0)
(1,0) (1,0) (1,0)
\end{Verbatim}
\newpage





\subsubsection{set\_imag}
Function%
\pdfdest name {scmatrix.setimag} fit
\begin{verbatim}
scmatrix& scmatrix::set_imag (TR d);
\end{verbatim}
sets imaginary part of every element of a calling matrix to a value of
parameter \verb"d"
and returns a reference to the matrix changed.
Function is \emph{redefined} in the class
\GOT{scbmatrix}{scbmatrix} 
and \emph{not allowed} in \GOT{schmatrix}{schmatrix}.
See also \GOT{scmatrix}{scmatrix}.
Example:
\begin{Verbatim}
using namespace cvm;

scmatrix m(3);
m.set_imag(1.);
std::cout << m;
\end{Verbatim}
prints
\begin{Verbatim}
(0,1) (0,1) (0,1)
(0,1) (0,1) (0,1)
(0,1) (0,1) (0,1)
\end{Verbatim}
\newpage




\subsubsection{resize}
Function%
\pdfdest name {scmatrix.resize} fit
\begin{verbatim}
scmatrix& scmatrix::resize (int nNewMN);
throw (cvmexception);
\end{verbatim}
changes  size of a calling matrix to \verb"nNewMN" by \verb"nNewMN"
and returns a reference to
the matrix changed. In case of increasing of its size, the matrix
is filled up with zeroes.
Function throws  \GOT{cvmexception}{cvmexception}
in case of negative size passed or memory allocation failure.
Function is \emph{redefined} in the classes
\GOT{scbmatrix}{scbmatrix} 
and \GOT{schmatrix}{schmatrix}.
See also \GOT{scmatrix}{scmatrix}.
Example:
\begin{Verbatim}
using namespace cvm;

try {
    double a[] = {1., 2., 3., 4., 5., 6., 7., 8.};
    scmatrix m((std::complex<double>*) a, 2);

    std::cout << m << std::endl;
    m.resize (3);
    std::cout << m;
}
catch (std::exception& e) {
    std::cout << "Exception " << e.what () << std::endl;
}
\end{Verbatim}
prints
\begin{Verbatim}
(1,2) (5,6)
(3,4) (7,8)

(1,2) (5,6) (0,0)
(3,4) (7,8) (0,0)
(0,0) (0,0) (0,0)
\end{Verbatim}
\newpage





\subsubsection{operator <{}<}
Operator%
\pdfdest name {scmatrix.operator <<} fit
\begin{verbatim}
scmatrix& scmatrix::operator << (const scmatrix& m)
throw (cvmexception);
\end{verbatim}
destroys  calling matrix, creates  new one as a copy of \verb"m"
and returns a reference to
the matrix changed.
Operator throws  \GOT{cvmexception}{cvmexception}
in case of memory allocation failure.
Operator is \emph{redefined} in the classes
\GOT{scbmatrix}{scbmatrix} 
and \GOT{schmatrix}{schmatrix}.
See also \GOT{scmatrix}{scmatrix}.
Example:
\begin{Verbatim}
using namespace cvm;

try {
    scmatrix m(3);
    scmatrix mc(1);
    m(1,2) = 1.;
    m(2,3) = 2.;
    std::cout << m << std::endl << mc << std::endl;

    mc << m;
    std::cout << mc;
}
catch (std::exception& e) {
    std::cout << "Exception " << e.what () << std::endl;
}
\end{Verbatim}
prints
\begin{Verbatim}
(0,0) (1,0) (0,0)
(0,0) (0,0) (2,0)
(0,0) (0,0) (0,0)

(0,0)

(0,0) (1,0) (0,0)
(0,0) (0,0) (2,0)
(0,0) (0,0) (0,0)
\end{Verbatim}
\newpage




\subsubsection{operator +}
Operator%
\pdfdest name {scmatrix.operator +} fit
\begin{verbatim}
scmatrix scmatrix::operator + (const scmatrix& m) const
throw (cvmexception);
\end{verbatim}
creates an object of type \verb"scmatrix" as a sum of
a calling matrix and a matrix \verb"m".
Operator throws  \GOT{cvmexception}{cvmexception}
in case of different sizes of the operands.
Operator is \emph{redefined} in the classes
\GOT{scbmatrix}{scbmatrix} 
and \GOT{schmatrix}{schmatrix}.
See also \GOT{scmatrix::sum}{scmatrix.sum}, \GOT{scmatrix}{scmatrix}.
Example:
\begin{Verbatim}
using namespace cvm;

try {
    double a[] = {1., 2., 3., 4., 5., 6., 7., 8.};
    double b[] = {10., 20., 30., 40., 50., 60., 70., 80.};
    scmatrix m1((std::complex<double>*) a, 2);
    scmatrix m2((std::complex<double>*) b, 2);

    std::cout << m1 + m2 << std::endl << m1 + m1;
}
catch (std::exception& e) {
    std::cout << "Exception " << e.what () << std::endl;
}
\end{Verbatim}
prints
\begin{Verbatim}
(11,22) (55,66)
(33,44) (77,88)

(2,4) (10,12)
(6,8) (14,16)
\end{Verbatim}
\newpage






\subsubsection{operator -}
Operator%
\pdfdest name {scmatrix.operator -} fit
\begin{verbatim}
scmatrix scmatrix::operator - (const scmatrix& m) const
throw (cvmexception);
\end{verbatim}
creates an object of type \verb"scmatrix" as a difference of
a calling matrix and a matrix \verb"m".
It throws  \GOT{cvmexception}{cvmexception}
in case of different sizes of the operands.
Operator is \emph{redefined} in the classes
\GOT{scbmatrix}{scbmatrix} 
and \GOT{schmatrix}{schmatrix}.
See also \GOT{scmatrix::diff}{scmatrix.diff}, \GOT{scmatrix}{scmatrix}.
Example:
\begin{Verbatim}
using namespace cvm;

try {
    double a[] = {1., 2., 3., 4., 5., 6., 7., 8.};
    double b[] = {10., 20., 30., 40., 50., 60., 70., 80.};
    scmatrix m1((std::complex<double>*) a, 2);
    scmatrix m2((std::complex<double>*) b, 2);

    std::cout << m2 - m1 << std::endl << m1 - m1;
}
catch (std::exception& e) {
    std::cout << "Exception " << e.what () << std::endl;
}
\end{Verbatim}
prints
\begin{Verbatim}
(9,18) (45,54)
(27,36) (63,72)

(0,0) (0,0)
(0,0) (0,0)
\end{Verbatim}
\newpage





\subsubsection{sum}
Function%
\pdfdest name {scmatrix.sum} fit
\begin{verbatim}
scmatrix& scmatrix::sum (const scmatrix& m1, const scmatrix& m2)
throw (cvmexception);
\end{verbatim}
assigns the result of addition of
matrices \verb"m1" and \verb"m2"  to a calling matrix
and returns a reference to
the matrix changed.
It throws  \GOT{cvmexception}{cvmexception}
in case of different sizes of the operands.
Function is \emph{redefined} in the classes
\GOT{scbmatrix}{scbmatrix} 
and \GOT{schmatrix}{schmatrix}.
See also \GOT{scmatrix::operator +~}{scmatrix.operator +},
\GOT{scmatrix}{scmatrix}.
Example:
\begin{Verbatim}
using namespace cvm;

try {
    double a[] = {1., 2., 3., 4., 5., 6., 7., 8., 9.,
                  10., 11., 12., 13., 14., 15., 16., 17., 18.};
    const scmatrix m1((std::complex<double>*)a, 3);
    scmatrix m2(3);
    scmatrix m(3);
    m2.set(std::complex<double>(1.,1.));

    std::cout << m.sum(m1, m2) << std::endl;
    std::cout << m.sum(m, m2);
}
catch (std::exception& e) {
    std::cout << "Exception " << e.what () << std::endl;
}
\end{Verbatim}
prints
\begin{Verbatim}
(2,3) (8,9) (14,15)
(4,5) (10,11) (16,17)
(6,7) (12,13) (18,19)

(3,4) (9,10) (15,16)
(5,6) (11,12) (17,18)
(7,8) (13,14) (19,20)
\end{Verbatim}
\newpage



\subsubsection{diff}
Function%
\pdfdest name {scmatrix.diff} fit
\begin{verbatim}
scmatrix& scmatrix::diff (const scmatrix& m1, const scmatrix& m2)
throw (cvmexception);
\end{verbatim}
assigns the result of subtraction of
matrices \verb"m1" and \verb"m2" to a calling matrix
and returns a reference to
the matrix changed.
It throws  \GOT{cvmexception}{cvmexception}
in case of different sizes of the operands.
Function is \emph{redefined} in the classes
\GOT{scbmatrix}{scbmatrix} 
and \GOT{schmatrix}{schmatrix}.
See also \GOT{scmatrix::operator -~}{scmatrix.operator -},
\GOT{scmatrix}{scmatrix}.
Example:
\begin{Verbatim}
using namespace cvm;

try {
    double a[] = {1., 2., 3., 4., 5., 6., 7., 8., 9.,
                  10., 11., 12., 13., 14., 15., 16., 17., 18.};
    const scmatrix m1((std::complex<double>*)a, 3);
    scmatrix m2(3);
    scmatrix m(3);
    m2.set(std::complex<double>(1.,1.));

    std::cout << m.diff(m1, m2) << std::endl;
    std::cout << m.diff(m, m2);
}
catch (std::exception& e) {
    std::cout << "Exception " << e.what () << std::endl;
}
\end{Verbatim}
prints
\begin{Verbatim}
(0,1) (6,7) (12,13)
(2,3) (8,9) (14,15)
(4,5) (10,11) (16,17)

(-1,0) (5,6) (11,12)
(1,2) (7,8) (13,14)
(3,4) (9,10) (15,16)
\end{Verbatim}
\newpage



\subsubsection{operator +=}
Operator%
\pdfdest name {scmatrix.operator +=} fit
\begin{verbatim}
scmatrix& scmatrix::operator += (const scmatrix& m) throw (cvmexception);
\end{verbatim}
adds  matrix \verb"m" to a calling matrix 
and returns a reference to
the matrix changed.
It throws  \GOT{cvmexception}{cvmexception}
in case of different sizes of the operands.
Operator is \emph{redefined} in the classes
\GOT{scbmatrix}{scbmatrix} 
and \GOT{schmatrix}{schmatrix}.
See also \GOT{scmatrix::ope\-ra\-tor +~}{scmatrix.operator +},
\GOT{scmatrix::sum}{scmatrix.sum},
\GOT{scmatrix}{scmatrix}.
Example:
\begin{Verbatim}
using namespace cvm;

try {
    scmatrix m1(3);
    scmatrix m2(3);
    m1.set(std::complex<double>(1.,2.));
    m2.set(std::complex<double>(3.,4.));

    m1 += m2;
    std::cout << m1 << std::endl;

    // well, you can do this too, but temporary object would be created
    m2 += m2; 
    std::cout << m2;
}
catch (std::exception& e) {
    std::cout << "Exception " << e.what () << std::endl;
}
\end{Verbatim}
prints
\begin{Verbatim}
(4,6) (4,6) (4,6)
(4,6) (4,6) (4,6)
(4,6) (4,6) (4,6)

(6,8) (6,8) (6,8)
(6,8) (6,8) (6,8)
(6,8) (6,8) (6,8)
\end{Verbatim}
\newpage




\subsubsection{operator -=}
Operator%
\pdfdest name {scmatrix.operator -=} fit
\begin{verbatim}
scmatrix& scmatrix::operator -= (const scmatrix& m) throw (cvmexception);
\end{verbatim}
subtracts  matrix \verb"m" from  calling matrix
and returns a reference to
the matrix changed.
It throws  \GOT{cvmexception}{cvmexception}
in case of different sizes of the operands.
Operator is \emph{redefined} in the classes
\GOT{scbmatrix}{scbmatrix} 
and \GOT{schmatrix}{schmatrix}.
See also \GOT{scmatrix::ope\-ra\-tor -~}{scmatrix.operator -},
\GOT{scmatrix::diff}{scmatrix.diff},
\GOT{scmatrix}{scmatrix}.
Example:
\begin{Verbatim}
using namespace cvm;

try {
    scmatrix m1(3);
    scmatrix m2(3);
    m1.set(std::complex<double>(1.,2.));
    m2.set(std::complex<double>(3.,4.));

    m1 -= m2;
    std::cout << m1 << std::endl;

    // well, you can do this too, but temporary object would be created
    m2 -= m2; 
    std::cout << m2;
}
catch (std::exception& e) {
    std::cout << "Exception " << e.what () << std::endl;
}
\end{Verbatim}
prints
\begin{Verbatim}
(-2,-2) (-2,-2) (-2,-2)
(-2,-2) (-2,-2) (-2,-2)
(-2,-2) (-2,-2) (-2,-2)

(0,0) (0,0) (0,0)
(0,0) (0,0) (0,0)
(0,0) (0,0) (0,0)
\end{Verbatim}
\newpage




\subsubsection{operator - ()}
Operator%
\pdfdest name {scmatrix.operator - ()} fit
\begin{verbatim}
scmatrix scmatrix::operator - () const throw (cvmexception);
\end{verbatim}
creates an object of type \verb"scmatrix" as
a calling matrix multiplied by $-1$.
Operator is \emph{redefined} in the classes
\GOT{scbmatrix}{scbmatrix} 
and \GOT{schmatrix}{schmatrix}.
See also \GOT{scmatrix}{scmatrix}.
Example:
\begin{Verbatim}
using namespace cvm;

double a[] = {1., 2., 3., 4., 5., 6., 7., 8.};
scmatrix m((std::complex<double>*) a, 2);
std::cout << -m;
\end{Verbatim}
prints
\begin{Verbatim}
(-1,-2) (-5,-6)
(-3,-4) (-7,-8)
\end{Verbatim}
\newpage



\subsubsection{operator ++}
Operator%
\pdfdest name {scmatrix.operator ++} fit
\begin{verbatim}
scmatrix& scmatrix::operator ++ ();
scmatrix& scmatrix::operator ++ (int);
\end{verbatim}
adds identity matrix to a calling matrix
and returns a reference to
the matrix changed.
Operator is \emph{redefined} in the classes
\GOT{scbmatrix}{scbmatrix} and \GOT{schmatrix}{schmatrix}.
See also \GOT{scmatrix}{scmatrix}.
Example:
\begin{Verbatim}
using namespace cvm;

scmatrix m(3);
m.set(std::complex<double>(1.,1.));
m++;
std::cout << m << std::endl;
std::cout << ++m;
\end{Verbatim}
prints
\begin{Verbatim}
(2,1) (1,1) (1,1)
(1,1) (2,1) (1,1)
(1,1) (1,1) (2,1)

(3,1) (1,1) (1,1)
(1,1) (3,1) (1,1)
(1,1) (1,1) (3,1)
\end{Verbatim}
\newpage



\subsubsection{operator -{}-}
Operator%
\pdfdest name {scmatrix.operator --} fit
\begin{verbatim}
scmatrix& scmatrix::operator -- ();
scmatrix& scmatrix::operator -- (int);
\end{verbatim}
subtracts identity matrix from  calling matrix
and returns a reference to
the matrix changed.
Operator is \emph{redefined} in the classes
\GOT{scbmatrix}{scbmatrix} and \GOT{schmatrix}{schmatrix}.
See also \GOT{scmatrix}{scmatrix}.
Example:
\begin{Verbatim}
using namespace cvm;

scmatrix m(3);
m.set(std::complex<double>(1.,1.));
m--;
std::cout << m << std::endl;
std::cout << --m;
\end{Verbatim}
prints
\begin{Verbatim}
(0,1) (1,1) (1,1)
(1,1) (0,1) (1,1)
(1,1) (1,1) (0,1)

(-1,1) (1,1) (1,1)
(1,1) (-1,1) (1,1)
(1,1) (1,1) (-1,1)
\end{Verbatim}
\newpage



\subsubsection{operator * (TR)}
Operator%
\pdfdest name {scmatrix.operator * (TR)} fit
\begin{verbatim}
scmatrix scmatrix::operator * (TR d) const;
\end{verbatim}
creates an object of type \verb"scmatrix" as a product of
a calling matrix and  real number~\verb"d".
Operator is \emph{redefined} in the classes
\GOT{scbmatrix}{scbmatrix} 
and \GOT{schmatrix}{schmatrix}.
See also \GOT{scmatrix::ope\-ra\-tor *=~}{scmatrix.operator *= (TR)},
\GOT{scmatrix}{scmatrix}.
Example:
\begin{Verbatim}
using namespace cvm;

double a[] = {1., 2., 3., 4., 5., 6., 7., 8., 9.,
              10., 11., 12., 13., 14., 15., 16., 17., 18.};
scmatrix m((std::complex<double>*) a, 3);
std::cout << m * 5.;
\end{Verbatim}
prints
\begin{Verbatim}
(5,10) (35,40) (65,70)
(15,20) (45,50) (75,80)
(25,30) (55,60) (85,90)
\end{Verbatim}
\newpage



\subsubsection{operator / (TR)}
Operator%
\pdfdest name {scmatrix.operator / (TR)} fit
\begin{verbatim}
scmatrix scmatrix::operator / (TR d) const throw (cvmexception);
\end{verbatim}
creates an object of type \verb"scmatrix" as a quotient of
a calling matrix and  real number~\verb"d". It throws
 \GOT{cvmexception}{cvmexception}
if \verb"d" has  absolute value equal or less than the
\GO{smallest normalized positive number}{Utilities.cvmMachMin}.
Operator is \emph{redefined} in the classes
\GOT{scbmatrix}{scbmatrix} 
and \GOT{schmatrix}{schmatrix}.
See also \GOT{scmatrix::operator /=~}{scmatrix.operator /= (TR)},
\GOT{scmatrix}{scmatrix}.
Example:
\begin{Verbatim}
using namespace cvm;

double a[] = {1., 2., 3., 4., 5., 6., 7., 8., 9.,
              10., 11., 12., 13., 14., 15., 16., 17., 18.};
scmatrix m((std::complex<double>*) a, 3);
std::cout << m / 2.;
\end{Verbatim}
prints
\begin{Verbatim}
(0.5,1) (3.5,4) (6.5,7)
(1.5,2) (4.5,5) (7.5,8)
(2.5,3) (5.5,6) (8.5,9)
\end{Verbatim}
\newpage



\subsubsection{operator * (TC)}
Operator%
\pdfdest name {scmatrix.operator * (TC)} fit
\begin{verbatim}
scmatrix scmatrix::operator * (TC z) const;
\end{verbatim}
creates an object of type \verb"scmatrix" as a product of
a calling matrix and  complex number~\verb"z".
Operator is \emph{redefined} in the classes
\GOT{scbmatrix}{scbmatrix} 
and \GOT{schmatrix}{schmatrix}.
See also \GOT{scmatrix}{scmatrix}, 
\GOT{scmatrix::operator *=~}{scmatrix.operator *= (TC)}.
Example:
\begin{Verbatim}
using namespace cvm;

double a[] = {1., 2., 3., 4., 5., 6., 7., 8., 9.,
              10., 11., 12., 13., 14., 15., 16., 17., 18.};
scmatrix m((std::complex<double>*) a, 3);
std::cout << m * std::complex<double>(1.,1.);
\end{Verbatim}
prints
\begin{Verbatim}
(-1,3) (-1,15) (-1,27)
(-1,7) (-1,19) (-1,31)
(-1,11) (-1,23) (-1,35)
\end{Verbatim}
\newpage



\subsubsection{operator / (TC)}
Operator%
\pdfdest name {scmatrix.operator / (TC)} fit
\begin{verbatim}
scmatrix scmatrix::operator / (TC z) const throw (cvmexception);
\end{verbatim}
creates an object of type \verb"scmatrix" as a quotient of
a calling matrix and  complex number~\verb"z". 
It throws
 \GOT{cvmexception}{cvmexception}
if \verb"z" has  absolute value equal or less than the
\GO{smallest normalized positive number}{Utilities.cvmMachMin}.
Operator is \emph{redefined} in the classes
\GOT{scbmatrix}{scbmatrix} 
and \GOT{schmatrix}{schmatrix}.
See also \GOT{scmatrix::operator /=~}
{scmatrix.operator /= (TC)},
\GOT{scmatrix}{scmatrix}.
Example:
\begin{Verbatim}
using namespace cvm;

double a[] = {1., 2., 3., 4., 5., 6., 7., 8., 9.,
              10., 11., 12., 13., 14., 15., 16., 17., 18.};
scmatrix m((std::complex<double>*) a, 3);
std::cout << m / std::complex<double>(1.,1.);
\end{Verbatim}
prints
\begin{Verbatim}
(1.5,0.5) (7.5,0.5) (13.5,0.5)
(3.5,0.5) (9.5,0.5) (15.5,0.5)
(5.5,0.5) (11.5,0.5) (17.5,0.5)
\end{Verbatim}
\newpage



\subsubsection{operator *= (TR)}
Operator%
\pdfdest name {scmatrix.operator *= (TR)} fit
\begin{verbatim}
scmatrix& scmatrix::operator *= (TR d);
\end{verbatim}
multiplies  calling matrix by  real number~\verb"d"
and returns a reference to
the matrix changed.
It's \emph{redefined} in the classes
\GOT{scbmatrix}{scbmatrix} 
and \GOT{schmatrix}{schmatrix}.
See also \GOT{scmatrix::operator *~}{scmatrix.operator * (TR)},
\GOT{scmatrix}{scmatrix}.
Example:
\begin{Verbatim}
using namespace cvm;

double a[] = {1., 2., 3., 4., 5., 6., 7., 8., 9.,
              10., 11., 12., 13., 14., 15., 16., 17., 18.};
scmatrix m((std::complex<double>*) a, 3);
m *= 5.;
std::cout << m;
\end{Verbatim}
prints
\begin{Verbatim}
(5,10) (35,40) (65,70)
(15,20) (45,50) (75,80)
(25,30) (55,60) (85,90)
\end{Verbatim}
\newpage



\subsubsection{operator /= (TR)}
Operator%
\pdfdest name {scmatrix.operator /= (TR)} fit
\begin{verbatim}
scmatrix& scmatrix::operator /= (TR d) throw (cvmexception);
\end{verbatim}
divides  calling matrix by  real number~\verb"d"
and returns a reference to
the matrix changed.
It throws  \GOT{cvmexception}{cvmexception}
if \verb"d" has  absolute value equal or less
than the 
\GO{smallest normalized positive number}{Utilities.cvmMachMin}.
Operator is \emph{redefined} in the classes
\GOT{scbmatrix}{scbmatrix} 
and \GOT{schmatrix}{schmatrix}.
See also \GOT{scmatrix::operator /~}{scmatrix.operator / (TR)},
\GOT{scmatrix}{scmatrix}.
Example:
\begin{Verbatim}
using namespace cvm;

double a[] = {1., 2., 3., 4., 5., 6., 7., 8., 9.,
              10., 11., 12., 13., 14., 15., 16., 17., 18.};
scmatrix m((std::complex<double>*) a, 3);
m /= 2.;
std::cout << m;
\end{Verbatim}
prints
\begin{Verbatim}
(0.5,1) (3.5,4) (6.5,7)
(1.5,2) (4.5,5) (7.5,8)
(2.5,3) (5.5,6) (8.5,9)
\end{Verbatim}
\newpage




\subsubsection{operator *= (TC)}
Operator%
\pdfdest name {scmatrix.operator *= (TC)} fit
\begin{verbatim}
scmatrix& scmatrix::operator *= (TC z);
\end{verbatim}
multiplies  calling matrix by  complex number~\verb"z"
and returns a reference to
the matrix changed.
It's \emph{redefined} in the classes
\GOT{scbmatrix}{scbmatrix} 
and \GOT{schmatrix}{schmatrix}.
See also \GOT{scmatrix}{scmatrix},
\GOT{scmatrix::operator *~}{scmatrix.operator * (TC)}.
Example:
\begin{Verbatim}
using namespace cvm;

double a[] = {1., 2., 3., 4., 5., 6., 7., 8., 9.,
              10., 11., 12., 13., 14., 15., 16., 17., 18.};
scmatrix m((std::complex<double>*) a, 3);
m *= std::complex<double>(2.,1.);
std::cout << m;
\end{Verbatim}
prints
\begin{Verbatim}
(0,5) (6,23) (12,41)
(2,11) (8,29) (14,47)
(4,17) (10,35) (16,53)
\end{Verbatim}
\newpage



\subsubsection{operator /= (TC)}
Operator%
\pdfdest name {scmatrix.operator /= (TC)} fit
\begin{verbatim}
scmatrix& scmatrix::operator /= (TC z) throw (cvmexception);
\end{verbatim}
divides  calling matrix by  complex number~\verb"z"
and returns a reference to
the matrix changed.
It throws  \GOT{cvmexception}{cvmexception}
if \verb"z" has  absolute value equal or less
than the 
\GO{smallest normalized positive number}{Utilities.cvmMachMin}.
Operator is \emph{redefined} in the classes
\GOT{scbmatrix}{scbmatrix} 
and \GOT{schmatrix}{schmatrix}.
See also \GOT{scmatrix::operator /~}{scmatrix.operator / (TC)},
\GOT{scmatrix}{scmatrix}.
Example:
\begin{Verbatim}
using namespace cvm;

double a[] = {1., 2., 3., 4., 5., 6., 7., 8., 9.,
              10., 11., 12., 13., 14., 15., 16., 17., 18.};
scmatrix m((std::complex<double>*) a, 3);
m /= std::complex<double>(2.,1.);
std::cout << m;
\end{Verbatim}
prints
\begin{Verbatim}
(0.8,0.6) (4.4,1.8) (8,3)
(2,1) (5.6,2.2) (9.2,3.4)
(3.2,1.4) (6.8,2.6) (10.4,3.8)
\end{Verbatim}
\newpage



\subsubsection{normalize}
Function%
\pdfdest name {scmatrix.normalize} fit
\begin{verbatim}
scmatrix& scmatrix::normalize ();
\end{verbatim}
normalizes  calling matrix so its \GO{Euclidean norm}{Array.norm}
becomes equal to $1$ if it was greater than the 
\GO{smallest normalized positive number}{Utilities.cvmMachMin}
before the call (otherwise function does nothing).
Function is \emph{redefined} in the classes
\GOT{scbmatrix}{scbmatrix} 
and \GOT{schmatrix}{schmatrix}.
See also \GOT{scmatrix}{scmatrix}.
Example:
\begin{Verbatim}
using namespace cvm;

std::cout.setf (std::ios::scientific | std::ios::left); 
std::cout.precision (5);

double a[] = {1., 2., 3., 4., 5., 6., 7., 8.};
scmatrix m((std::complex<double>*) a, 2);
m.normalize();
std::cout << m << m.norm() << std::endl;
\end{Verbatim}
prints
\begin{Verbatim}
(7.00140e-002,1.40028e-001) (3.50070e-001,4.20084e-001)
(2.10042e-001,2.80056e-001) (4.90098e-001,5.60112e-001)
1.00000e+000
\end{Verbatim}
\newpage




\subsubsection{conjugation}
Operator and functions%
\pdfdest name {scmatrix.conj} fit
\begin{verbatim}
scmatrix scmatrix::operator ~ () const throw (cvmexception);
scmatrix& scmatrix::conj (const scmatrix& m) throw (cvmexception);
scmatrix& scmatrix::conj () throw (cvmexception);
\end{verbatim}
implement complex matrix conjugation.
First operator creates an object of type \verb"scmatrix" as
 conjugated calling matrix
(it throws  
\GOT{cvmexception}{cvmexception}
in case of memory allocation failure). 
Second function sets  calling matrix to be equal to  matrix
\verb"m" conjugated
(it throws  
\GOT{cvmexception}{cvmexception}
in case of not appropriate sizes of the operands), 
third one makes it to be equal to
conjugated itself (it also throws  
\GOT{cvmexception}{cvmexception}
in case of memory allocation failure). 
Functions are \emph{redefined} in the classes
\GOT{scbmatrix}{scbmatrix} 
and \GOT{schmatrix}{schmatrix}.
See also \GOT{scmatrix}{scmatrix}.
Example:
\begin{Verbatim}
using namespace cvm;

double a[] = {1., 2., 3., 4., 5., 6., 7., 8., 9.,
              10., 11., 12., 13., 14., 15., 16., 17., 18.};
scmatrix m((std::complex<double>*) a, 3);
scmatrix mc(3);
std::cout << m << std::endl << ~m << std::endl ;
mc.conj(m);
std::cout << mc << std::endl;
mc.conj();
std::cout << mc;
\end{Verbatim}
prints
\begin{Verbatim}
(1,2) (7,8) (13,14)
(3,4) (9,10) (15,16)
(5,6) (11,12) (17,18)

(1,-2) (3,-4) (5,-6)
(7,-8) (9,-10) (11,-12)
(13,-14) (15,-16) (17,-18)

(1,-2) (3,-4) (5,-6)
(7,-8) (9,-10) (11,-12)
(13,-14) (15,-16) (17,-18)

(1,2) (7,8) (13,14)
(3,4) (9,10) (15,16)
(5,6) (11,12) (17,18)
\end{Verbatim}
\newpage





\subsubsection{transposition}
Operator and functions%
\pdfdest name {scmatrix.transpose} fit
\begin{verbatim}
scmatrix scmatrix::operator ! () const throw (cvmexception);
scmatrix& scmatrix::transpose (const scmatrix& m) throw (cvmexception);
scmatrix& scmatrix::transpose () throw (cvmexception);
\end{verbatim}
implement complex matrix transposition (\emph{not} conjugation).
First operator creates an object of type \verb"scmatrix" as
 transposed calling matrix
(it throws  
\GOT{cvmexception}{cvmexception}
in case of memory allocation failure). 
Second function sets  calling matrix to be equal to  matrix
\verb"m" transposed
(it throws  
\GOT{cvmexception}{cvmexception}
in case of not appropriate sizes of the operands), 
third one makes it to be equal to
transposed itself (it also throws  
\GOT{cvmexception}{cvmexception}
in case of memory allocation failure). 
Functions are \emph{redefined} in the classes
\GOT{scbmatrix}{scbmatrix} 
and \GOT{schmatrix}{schmatrix}.
See also \GOT{scmatrix}{scmatrix}.
Example:
\begin{Verbatim}
using namespace cvm;

double a[] = {1., 2., 3., 4., 5., 6., 7., 8., 9.,
              10., 11., 12., 13., 14., 15., 16., 17., 18.};
scmatrix m((std::complex<double>*) a, 3);
scmatrix mc(3);
std::cout << m << std::endl << !m << std::endl ;
mc.transpose(m);
std::cout << mc << std::endl;
mc.transpose();
std::cout << mc;
\end{Verbatim}
prints
\begin{Verbatim}
(1,2) (7,8) (13,14)
(3,4) (9,10) (15,16)
(5,6) (11,12) (17,18)

(1,2) (3,4) (5,6)
(7,8) (9,10) (11,12)
(13,14) (15,16) (17,18)

(1,2) (3,4) (5,6)
(7,8) (9,10) (11,12)
(13,14) (15,16) (17,18)

(1,2) (7,8) (13,14)
(3,4) (9,10) (15,16)
(5,6) (11,12) (17,18)
\end{Verbatim}
\newpage







\subsubsection{operator * (const cvector\&)}
Operator%
\pdfdest name {scmatrix.operator * (const cvector&)} fit
\begin{verbatim}
cvector scmatrix::operator * (const cvector& v) const
throw (cvmexception);
\end{verbatim}
creates an object of type \verb"cvector"
as a product of a calling matrix and a vector \verb"v".
It throws  \GOT{cvmexception}{cvmexception}
if  number of columns of a calling matrix
differs from  size of a vector \verb"v".
Use \GOT{cvector::mult}{cvector.mult (const cmatrix&, const cvector&)}
in order to avoid new object creation.
Function is \emph{redefined} in the classes
\GOT{scbmatrix}{scbmatrix} 
and \GOT{schmatrix}{schmatrix}.
See also
\GOT{scmatrix}{scmatrix}, \GOT{cvector}{cvector}.
Example:
\begin{Verbatim}
using namespace cvm;

try {
    scmatrix m(3);
    cvector v(3);
    m.set(std::complex<double>(1.,1.));
    v.set(std::complex<double>(1.,1.));
    std::cout << m * v;
}
catch (std::exception& e) {
    std::cout << "Exception " << e.what () << std::endl;
}
\end{Verbatim}
prints
\begin{Verbatim}
(0,6) (0,6) (0,6)
\end{Verbatim}
\newpage



\subsubsection{operator * (const cmatrix\&)}
Operator%
\pdfdest name {scmatrix.operator * (const cmatrix&)} fit
\begin{verbatim}
cmatrix scmatrix::operator * (const cmatrix& m) const
throw (cvmexception);
\end{verbatim}
creates an object of type \verb"cmatrix"
as a product of a calling matrix and a matrix \verb"m".
It throws  \GOT{cvmexception}{cvmexception}
if  number of columns of a calling matrix
differs from  number of rows of a matrix \verb"m".
Use \GOT{cmatrix::mult}{cmatrix.mult} in order to avoid
 new object creation.
Operator is \emph{redefined} in the classes
\GOT{scbmatrix}{scbmatrix} 
and \GOT{schmatrix}{schmatrix}.
See also
\GOT{cmatrix}{cmatrix}, \GOT{scmatrix}{scmatrix}.
Example:
\begin{Verbatim}
using namespace cvm;

try {
    scmatrix ms(3);
    cmatrix m(3,2);
    ms.set(std::complex<double>(1.,1.));
    m.set(std::complex<double>(1.,1.));
    std::cout << ms * m;
}
catch (std::exception& e) {
    std::cout << "Exception " << e.what () << std::endl;
}
\end{Verbatim}
prints
\begin{Verbatim}
(0,6) (0,6)
(0,6) (0,6)
(0,6) (0,6)
\end{Verbatim}
\newpage



\subsubsection{operator * (const scmatrix\&)}
Operator%
\pdfdest name {scmatrix.operator * (const scmatrix&)} fit
\begin{verbatim}
scmatrix scmatrix::operator * (const scmatrix& m) const
throw (cvmexception);
\end{verbatim}
creates an object of type \verb"scmatrix"
as a product of a calling matrix and a matrix \verb"m".
It throws  \GOT{cvmexception}{cvmexception}
if the operands have different sizes.
Use \GOT{cmatrix::mult}{cmatrix.mult} in order to avoid
 new object creation.
Operator is \emph{inherited} in the class
\GOT{scbmatrix}{scbmatrix} 
and \emph{redefined} in \GOT{schmatrix}{schmatrix}.
See also
\GOT{scmatrix}{scmatrix}.
Example:
\begin{Verbatim}
using namespace cvm;

try {
    scmatrix m1(3), m2(3);
    m1.set(std::complex<double>(1.,1.));
    m2.set(std::complex<double>(1.,1.));
    std::cout << m1 * m2;
}
catch (std::exception& e) {
    std::cout << "Exception " << e.what () << std::endl;
}
\end{Verbatim}
prints
\begin{Verbatim}
(0,6) (0,6) (0,6)
(0,6) (0,6) (0,6)
(0,6) (0,6) (0,6)
\end{Verbatim}
\newpage




\subsubsection{operator *= (const scmatrix\&)}
Operator%
\pdfdest name {scmatrix.operator *= (const scmatrix&)} fit
\begin{verbatim}
scmatrix& scmatrix::operator *= (const scmatrix& m)
throw (cvmexception);
\end{verbatim}
sets  calling matrix to be equal to  product
of itself by  matrix \verb"m"
and returns a reference to
the object it changes.
Operator throws \GOT{cvmexception}{cvmexception}
in case of different sizes of the operands.
Operator is \emph{inherited} in the class
\GOT{scbmatrix}{scbmatrix} 
and \emph{redefined} in \GOT{schmatrix}{schmatrix}.
See also \GOT{scmatrix}{scmatrix}.
Example:
\begin{Verbatim}
using namespace cvm;

try {
    scmatrix m1(3), m2(3);
    m1.set(std::complex<double>(1.,2.));
    m2.set(std::complex<double>(2.,1.));
    m1 *= m2;
    m2 *= m2;
    std::cout << m1 << std::endl << m2;
}
catch (std::exception& e) {
    std::cout << "Exception " << e.what () << std::endl;
}
\end{Verbatim}
prints
\begin{Verbatim}
(0,15) (0,15) (0,15)
(0,15) (0,15) (0,15)
(0,15) (0,15) (0,15)

(9,12) (9,12) (9,12)
(9,12) (9,12) (9,12)
(9,12) (9,12) (9,12)
\end{Verbatim}
\newpage



\subsubsection{swap\_rows}
Function%
\pdfdest name {scmatrix.swap_rows} fit
\begin{verbatim}
scmatrix& scmatrix::swap_rows (int n1, int n2) throw (cvmexception);
\end{verbatim}
swaps two rows of a calling matrix and returns a reference to
the matrix changed. \verb"n1" and \verb"n2" are
 numbers of rows to be swapped, 
both are \Based).
Function throws  \GOT{cvmexception}{cvmexception}
if one of  parameters is outside of the range
\verb"[1,msize()]".
Function is 
\emph{not applicable} to objects of the classes
\GOT{scbmatrix}{scbmatrix} and
\GOT{schmatrix}{schmatrix} (i.e. \GOT{cvmexception}{cvmexception} would be thrown
in case of using it for objects of those classes).
See also \GOT{scmatrix}{scmatrix}.
Example:
\begin{Verbatim}
using namespace cvm;

try {
    double a[] = {1., 2., 3., 4., 5., 6., 7., 8., 9.,
                  10., 11., 12., 13., 14., 15., 16., 17., 18.};
    scmatrix m ((std::complex<double>*)a, 3);

    std::cout << m << std::endl;
    std::cout << m.swap_rows(2,3);
}
catch (std::exception& e) {
    std::cout << "Exception " << e.what () << std::endl;
}
\end{Verbatim}
prints
\begin{Verbatim}
(1,2) (7,8) (13,14)
(3,4) (9,10) (15,16)
(5,6) (11,12) (17,18)

(1,2) (7,8) (13,14)
(5,6) (11,12) (17,18)
(3,4) (9,10) (15,16)
\end{Verbatim}
\newpage


\subsubsection{swap\_cols}
Function%
\pdfdest name {scmatrix.swap_cols} fit
\begin{verbatim}
scmatrix& scmatrix::swap_cols (int n1, int n2) throw (cvmexception);
\end{verbatim}
swaps two columns of a calling matrix and returns a reference to
the matrix changed. \verb"n1" and \verb"n2" are
 numbers of columns to be swapped, 
both are \Based).
Function throws  \GOT{cvmexception}{cvmexception}
if one of  parameters is outside of the range
\verb"[1,nsize()]".
Function is 
\emph{not applicable} to objects of the classes
\GOT{scbmatrix}{scbmatrix} and
\GOT{schmatrix}{schmatrix} (i.e. \GOT{cvmexception}{cvmexception} would be thrown
in case of using it for objects of those classes).
See also \GOT{scmatrix}{scmatrix}.
Example:
\begin{Verbatim}
using namespace cvm;

try {
    double a[] = {1., 2., 3., 4., 5., 6., 7., 8., 9.,
                  10., 11., 12., 13., 14., 15., 16., 17., 18.};
    scmatrix m ((std::complex<double>*)a, 3);

    std::cout << m << std::endl;
    std::cout << m.swap_cols(2,3);
}
catch (std::exception& e) {
    std::cout << "Exception " << e.what () << std::endl;
}
\end{Verbatim}
prints
\begin{Verbatim}
(1,2) (7,8) (13,14)
(3,4) (9,10) (15,16)
(5,6) (11,12) (17,18)

(1,2) (13,14) (7,8)
(3,4) (15,16) (9,10)
(5,6) (17,18) (11,12)
\end{Verbatim}
\newpage



\subsubsection{solve}
Functions%
\pdfdest name {scmatrix.solve} fit
\begin{verbatim}
cvector scmatrix::solve (const cvector& vB) const throw (cvmexception);
cmatrix scmatrix::solve (const cmatrix& mB) const throw (cvmexception);
cvector scmatrix::solve (const cvector& vB, TR& dErr) const
throw (cvmexception);
cmatrix scmatrix::solve (const cmatrix& mB, TR& dErr) const
throw (cvmexception);
\end{verbatim}
return  solution of  linear equation of kind $A*x=b$ or $A*X=B$
where $A$ is  calling matrix. The first and third versions
solve the equation $A*x=b$ where vector $b$ is passed in 
parameter \verb"vB"
and the second and fourth versions
solve the equation $A*X=B$ where matrix $B$ is passed 
in parameter \verb"mB".
The last two versions also set output parameter \verb"dErr" to be equal
to the norm of computation error.
Functions throw \GOT{cvmexception}{cvmexception}
in case of inappropriate sizes of the operands.
Functions are \emph{inherited} in the classes
\GOT{scbmatrix}{scbmatrix}
and \GOT{schmatrix}{schmatrix}.
See also
\GOT{cvector::solve}{cvector.solve},
\GOT{cmatrix::solve}{cmatrix.solve},
\GOT{scmatrix}{scmatrix}.
Example:
\begin{Verbatim}
using namespace cvm;

std::cout.setf (std::ios::scientific | std::ios::showpos);
std::cout.precision (2);
try {
    double re[] = {1., 2., 3., 4., 5., 6., 7., 8., 9.};
    double im[] = {-1., 2., -3., -4., 5., -6., 7., -8., 9.};
    scmatrix ma(re, im, 3);
    cmatrix  mb(3,2);
    cmatrix  mx(3,2);
    double dErr;

    mb(1).set(std::complex<double>(1.,1.));
    mb(1,2) = std::complex<double>(1.,1.);
    mb(2,2) = std::complex<double>(2.,2.);
    mb(3,2) = std::complex<double>(3.,3.);

    mx.solve (ma, mb, dErr);
    std::cout << mx << dErr 
              << std::endl << ma * mx - mb << std::endl;

    cvector vb(3), vx(3);
    vb = mb(2);
    vx = ma.solve (vb, dErr);
    std::cout << vx << dErr << std::endl << ma * vx - vb;
}
catch (std::exception& e) {
    std::cout << "Exception " << e.what () << std::endl;
}
\end{Verbatim}
prints
\begin{Verbatim}
(-6.25e-002,-2.71e-001) (-1.25e-001,+1.13e+000)
(+1.25e-001,+2.08e-001) (+2.50e-001,-2.50e-001)
(+6.25e-002,+6.25e-002) (+1.25e-001,+1.25e-001)
+1.45e-014
(+0.00e+000,+0.00e+000) (+0.00e+000,+0.00e+000)
(-2.22e-016,+4.44e-016) (+0.00e+000,+0.00e+000)
(+2.22e-016,+0.00e+000) (+0.00e+000,+0.00e+000)

(-1.25e-001,+1.13e+000) (+2.50e-001,-2.50e-001) (+1.25e-001,+1.25e-001)
+7.41e-015
(+0.00e+000,+0.00e+000) (+0.00e+000,+0.00e+000) (+0.00e+000,+0.00e+000)
\end{Verbatim}
\newpage





\subsubsection{solve\_tran}
Functions%
\pdfdest name {scmatrix.solvetran} fit
\begin{verbatim}
cvector scmatrix::solve_tran (const cvector& vB)
const throw (cvmexception);
cmatrix scmatrix::solve_tran (const cmatrix& mB)
const throw (cvmexception);
cvector scmatrix::solve_tran (const cvector& vB, TR& dErr) const
throw (cvmexception);
cmatrix scmatrix::solve_tran (const cmatrix& mB, TR& dErr) const
throw (cvmexception);
\end{verbatim}
return  solution of  linear equation of kind $A^T*x=b$ or $A^T*X=B$
(equivalent to $x*A=b$ and $X*A=B$)
where $A$ is  calling matrix. The first and third versions
solve  equation $A^T*x=b$ where vector $b$ is passed in 
 parameter \verb"vB"
and the second and fourth versions
solve equation $A^T*X=B$ where matrix $B$ is passed 
in parameter \verb"mB".
The last two versions also set output parameter \verb"dErr" to be equal
to the norm of computation error.
Functions throw \GOT{cvmexception}{cvmexception}
in case of inappropriate sizes of the operands.
Function is \emph{inherited} in the classes
\GOT{scbmatrix}{scbmatrix}
and \GOT{schmatrix}{schmatrix}.
See also
\GOT{cvector::solve}{cvector.solve},
\GOT{cmatrix::solve}{cmatrix.solve},
\GOT{scmatrix::solve}{scmatrix.solve},
\GOT{scmatrix}{scmatrix}.
Example:
\begin{Verbatim}
using namespace cvm;

std::cout.setf (std::ios::scientific | std::ios::showpos);
std::cout.precision (2);
try {
    double re[] = {1., 2., 3., 4., 5., 6., 7., 8., 9.};
    double im[] = {-1., 2., -3., -4., 5., -6., 7., -8., 9.};
    scmatrix ma(re, im, 3);
    cmatrix  mb(3,2);
    cmatrix  mx(3,2);
    double dErr;

    mb(1).set(std::complex<double>(1.,1.));
    mb(1,2) = std::complex<double>(1.,1.);
    mb(2,2) = std::complex<double>(2.,2.);
    mb(3,2) = std::complex<double>(3.,3.);

    mx.solve_tran (ma, mb, dErr);
    std::cout << mx << dErr 
              << std::endl << !ma * mx - mb
              << std::endl << !mx * ma - !mb << std::endl;

    cvector vb(3), vx(3);
    vb = mb(2);
    vx = ma.solve_tran (vb, dErr);
    std::cout << vx << dErr << std::endl << vx * ma - vb;
}
catch (std::exception& e) {
    std::cout << "Exception " << e.what () << std::endl;
}
\end{Verbatim}
prints
\begin{Verbatim}
(+3.13e-002,-5.31e-001) (+9.38e-002,-9.38e-002)
(+6.25e-002,+6.25e-002) (+1.88e-001,+1.88e-001)
(+3.13e-002,+4.69e-001) (+9.38e-002,+2.40e-001)
+1.02e-014
(-2.22e-016,-2.22e-016) (+0.00e+000,+0.00e+000)
(-4.44e-016,-4.44e-016) (+0.00e+000,+0.00e+000)
(+8.88e-016,-8.88e-016) (+0.00e+000,+0.00e+000)

(-2.22e-016,-2.22e-016) (-4.44e-016,-4.44e-016) (+8.88e-016,-8.88e-016)
(+0.00e+000,+0.00e+000) (+0.00e+000,+0.00e+000) (+0.00e+000,+0.00e+000)

(+9.38e-002,-9.38e-002) (+1.88e-001,+1.88e-001) (+9.38e-002,+2.40e-001)
+1.02e-014
(+0.00e+000,+0.00e+000) (+0.00e+000,+0.00e+000) (+0.00e+000,+0.00e+000)
\end{Verbatim}
\newpage




\subsubsection{solve\_conj}
Functions%
\pdfdest name {scmatrix.solveconj} fit
\begin{verbatim}
cvector scmatrix::solve_conj (const cvector& vB)
const throw (cvmexception);
cmatrix scmatrix::solve_conj (const cmatrix& mB)
const throw (cvmexception);
cvector scmatrix::solve_conj (const cvector& vB, TR& dErr) const
throw (cvmexception);
cmatrix scmatrix::solve_conj (const cmatrix& mB, TR& dErr) const
throw (cvmexception);
\end{verbatim}
return  solution of  linear equation of kind $A^H*x=b$ or $A^H*X=B$
(which is equivalent to $X^H*A=B^H$), where $A^H$ is conjugated $A$.
Here $A$ is calling matrix. The first and third versions
solve  equation $A^H*x=b$ where vector $b$ is passed in 
 parameter \verb"vB"
and the second and fourth versions
solve equation $A^H*X=B$ where matrix $B$ is passed 
in parameter \verb"mB".
The last two versions also set output parameter \verb"dErr" to be equal
to the norm of computation error.
Functions throw \GOT{cvmexception}{cvmexception}
in case of inappropriate sizes of the operands.
Function is \emph{inherited} in the classes
\GOT{scbmatrix}{scbmatrix}
and \GOT{schmatrix}{schmatrix}.
See also
\GOT{cvector::solve}{cvector.solve},
\GOT{cmatrix::solve}{cmatrix.solve},
\GOT{scmatrix::solve}{scmatrix.solve},
\GOT{scmatrix}{scmatrix}.
Example:
\begin{Verbatim}
using namespace cvm;

std::cout.setf (std::ios::scientific | std::ios::showpos);
std::cout.precision (2);
try {
    double re[] = {1., 2., 3., 4., 5., 6., 7., 8., 9.};
    double im[] = {-1., 2., -3., -4., 5., -6., 7., -8., 9.};
    scmatrix ma(re, im, 3);
    cmatrix  mb(3,2);
    cmatrix  mx(3,2);
    double dErr;

    mb(1).set(std::complex<double>(1.,1.));
    mb(1,2) = std::complex<double>(1.,1.);
    mb(2,2) = std::complex<double>(2.,2.);
    mb(3,2) = std::complex<double>(3.,3.);

    mx.solve_conj (ma, mb, dErr);
    std::cout << mx << dErr 
              << std::endl << ~ma * mx - mb
              << std::endl << ~mx * ma - ~mb << std::endl;

    cvector vb(3), vx(3);
    vb = mb(2);
    vx = ma.solve_conj (vb, dErr);
    std::cout << vx << dErr << std::endl << ~ma * vx - vb;
}
catch (std::exception& e) {
    std::cout << "Exception " << e.what () << std::endl;
}
\end{Verbatim}
prints
\begin{Verbatim}
(-5.31e-001,+3.13e-002) (-9.38e-002,+9.38e-002)
(+6.25e-002,+6.25e-002) (+1.88e-001,+1.88e-001)
(+4.69e-001,+3.13e-002) (+2.40e-001,+9.38e-002)
+1.02e-014
(-2.22e-016,-2.22e-016) (+0.00e+000,+0.00e+000)
(-4.44e-016,-4.44e-016) (+0.00e+000,+0.00e+000)
(-8.88e-016,+8.88e-016) (+0.00e+000,+0.00e+000)

(-2.22e-016,+2.22e-016) (-4.44e-016,+4.44e-016) (-8.88e-016,-8.88e-016)
(+0.00e+000,+0.00e+000) (+0.00e+000,+0.00e+000) (+0.00e+000,+0.00e+000)

(-9.38e-002,+9.38e-002) (+1.88e-001,+1.88e-001) (+2.40e-001,+9.38e-002)
+1.02e-014
(+0.00e+000,+0.00e+000) (+0.00e+000,+0.00e+000) (+0.00e+000,+0.00e+000)
\end{Verbatim}
\newpage





\subsubsection{operator \% (const cvector\&)}
Operator%
\pdfdest name {scmatrix.operator percent (cvector)} fit
\begin{verbatim}
cvector operator % (const cvector& vB) const throw (cvmexception);
\end{verbatim}
returns solution $x$ of  linear equation
$A^T*x=b$ (which is equivalent to $x*A=b$) 
where calling matrix is square matrix $A$
and a vector $b$ is passed in parameter \verb"vB".
This operator throws exception 
of type \GOT{cvmexception}{cvmexception}
in case of inappropriate sizes
of the objects or when  matrix $A$ is close to singular.
See also \GOT{cvector::solve\_tran}{cvector.solvetran}, 
\GOT{scmatrix::solve\_tran}{scmatrix.solvetran}, 
\GOT{cvector.operator~/}{cvector.operator / (scmatrix)}, 
\GOT{cvector}{cvector}, \GOT{scmatrix}{scmatrix}.
Example:
\begin{Verbatim}
using namespace cvm;
std::cout.setf (std::ios::scientific | std::ios::showpos);
std::cout.precision (7);

try {
    double m[] = {1., -1., 1., 2., -2., 1., 3., -3.};
    double b[] = {1., 2., 5., -3.};
    scmatrix ma((std::complex<double>*) m, 2);
    cvector vb((std::complex<double>*) b, 2);
    cvector vx(2);

    vx = ma % vb;

    std::cout << vx * ma - vb;
}
catch (std::exception& e) {
    std::cout << "Exception " << e.what () << std::endl;
}
\end{Verbatim}
prints
\begin{Verbatim}
(+0.0000000e+000,+0.0000000e+000) (+0.0000000e+000,+4.4408921e-016)
\end{Verbatim}
\newpage





\subsubsection{operator / (const cvector\&)}
Operator%
\pdfdest name {scmatrix.operator / (cvector)} fit
\begin{verbatim}
cvector operator / (const cvector& vB) const throw (cvmexception);
\end{verbatim}
returns solution $x$ of linear equation
$A*x=b$ where calling matrix is square matrix $A$
and a vector $b$ is passed in parameter \verb"vB".
This operator throws exception 
of type \GOT{cvmexception}{cvmexception}
in case of inappropriate sizes
of the objects or when  matrix $A$ is close to singular.
It's \emph{redefined} in class \GOT{scbmatrix}{scbmatrix}.
See also \GOT{cvector::solve}{cvector.solve}, 
\GOT{scmatrix::solve}{scmatrix.solve}, 
\GOT{cvector.operator~\%}{cvector.operator percent (scmatrix)}, 
\GOT{cvector}{cvector}, \GOT{scmatrix}{scmatrix}.
Example:
\begin{Verbatim}
using namespace cvm;
std::cout.setf (std::ios::scientific | std::ios::showpos);
std::cout.precision (7);

try {
    double m[] = {1., -1., 1., 2., -2., 1., 3., -3.};
    double b[] = {1., 2., 5., -3.};
    scmatrix ma((std::complex<double>*) m, 2);
    cvector vb((std::complex<double>*) b, 2);
    cvector vx(2);

    vx = ma / vb;

    std::cout << ma * vx - vb;
}
catch (std::exception& e) {
    std::cout << "Exception " << e.what () << std::endl;
}
\end{Verbatim}
prints
\begin{Verbatim}
(-6.6613381e-016,+4.4408921e-016) (+0.0000000e+000,+0.0000000e+000)
\end{Verbatim}
\newpage





\subsubsection{solve\_lu}
Functions%
\pdfdest name {scmatrix.solvelu} fit
\begin{verbatim}
cvector
scmatrix::solve_lu (const scmatrix& mLU, const int* pPivots,
                    const cvector& vB, TR& dErr) throw (cvmexception);
cvector
scmatrix::solve_lu (const scmatrix& mLU, const int* pPivots,
                    const cvector& vB) throw (cvmexception);
cmatrix
scmatrix::solve_lu (const scmatrix& mLU, const int* pPivots,
                    const cmatrix& mB, TR& dErr) throw (cvmexception);
cmatrix
scmatrix::solve_lu (const scmatrix& mLU, const int* pPivots,
                    const cmatrix& mB) throw (cvmexception);
\end{verbatim}
create  object of type \verb"cvector" or \verb"cmatrix" as
 solution $x$ or $X$ of a matrix linear equation
$A*x=b$ or $A*X=B$ respectively. Here $A$ is  calling
matrix, 
parameter \verb"mLU" is \GO{$LU$ factorization}{scmatrix.low_up}
of the matrix $A$, parameter \verb"pPivots" is an array of pivot numbers
created while factorizing the matrix $A$
and parameters \verb"vB" and \verb"mB" are the vector $b$ and
matrix $B$ respectively.
The first and third version also set output parameter \verb"dErr" to be equal
to the norm of computation error.
These functions are useful when you need to solve few linear equations
of kind $A*x=b$ or $A*X=B$ with the same matrix $A$ and different vectors $b$ 
or matrices $B$.
In such case you save on matrix $A$ factorization since it's needed to be
performed just one time.
Functions throw \GOT{cvmexception}{cvmexception}
in case of inappropriate sizes
of the operands or when the matrix $A$ is close to singular.
Function is \emph{inherited} in the classes
\GOT{scbmatrix}{scbmatrix}
and \GOT{schmatrix}{schmatrix}.
See also
\GOT{cvector::solve}{cvector.solve},
\GOT{cmatrix}{cmatrix},
\GOT{scmatrix}{scmatrix}.
Example:
\begin{Verbatim}
using namespace cvm;
std::cout.setf (std::ios::scientific | std::ios::left); 
std::cout.precision (2);
try {
    scmatrix ma(3);
    scmatrix mLU(3);
    cmatrix  mb1(3,2); cvector vb1(3);
    cmatrix  mb2(3,2); cvector vb2(3);
    cmatrix  mx1(3,2); cvector vx1(3);
    cmatrix  mx2(3,2); cvector vx2(3);
    iarray   nPivots(3);
    double   dErr = 0.;
    ma.randomize_real(-1.1,3.); ma.randomize_imag(-3.7,3.);
    mb1.randomize_real(-1.,3.); mb1.randomize_imag(-1.,3.);
    vb1.randomize_real(-2.,3.); vb1.randomize_imag(-3.,1.);
    mb2.randomize_real(-5.,1.); mb2.randomize_imag(-4.,1.);
    vb2.randomize_real(-1.,6.); vb1.randomize_imag(-4.,4.);
    mLU.low_up(ma, nPivots);
    mx1 = ma.solve_lu (mLU, nPivots, mb1, dErr);
    std::cout << mx1 << dErr << std::endl;
    mx2 = ma.solve_lu (mLU, nPivots, mb2);
    std::cout << mx2 << std::endl;;
    std::cout << ma * mx1 - mb1 << std::endl << ma * mx2 - mb2;
    vx1 = ma.solve_lu (mLU, nPivots, vb1, dErr);
    std::cout << vx1 << dErr << std::endl;
    vx2 = ma.solve_lu (mLU, nPivots, vb2);
    std::cout << vx2 << std::endl;;
    std::cout << ma * vx1 - vb1 << std::endl << ma * vx2 - vb2;
}
catch (std::exception& e) {
    std::cout << "Exception " << e.what () << std::endl;
}
\end{Verbatim}
prints
\begin{Verbatim}
(-3.91e-001,-1.62e-001) (-3.17e-001,2.39e-001)
(-3.00e-001,-7.91e-001) (4.71e-001,-9.13e-001)
(-2.34e-001,1.09e+000) (1.10e-001,2.72e-001)
2.78e-015
(-2.60e-001,-5.48e-001) (-3.09e-002,-9.62e-001)
(8.77e-001,8.41e-001) (-6.02e-001,1.87e+000)
(4.20e-003,-9.72e-001) (6.18e-001,-5.64e-001)

(0.00e+000,-2.64e-016) (-1.11e-016,1.11e-016)
(0.00e+000,-2.22e-016) (2.22e-016,-4.44e-016)
(0.00e+000,0.00e+000) (-5.55e-017,0.00e+000)

(6.66e-016,-2.22e-016) (6.18e-016,0.00e+000)
(0.00e+000,1.11e-016) (0.00e+000,0.00e+000)
(0.00e+000,0.00e+000) (-4.44e-016,0.00e+000)
(2.61e-001,2.97e-001) (1.95e+000,-1.07e-001) (-5.51e-001,-1.03e-001)
1.96e-015
(1.26e-001,4.07e-001) (-4.82e-001,-1.14e-002) (2.59e-001,1.60e-001)

(1.11e-016,0.00e+000) (5.55e-017,0.00e+000) (1.11e-016,-2.22e-016)

(-1.11e-016,-7.61e-017) (-2.22e-016,-8.94e-017) (0.00e+000,4.07e-017)
\end{Verbatim}
\newpage



\subsubsection{det}
Function%
\pdfdest name {scmatrix.det} fit
\begin{verbatim}
TC scmatrix::det () const throw (cvmexception);
\end{verbatim}
returns  determinant of a calling matrix.
It uses the \GO{$LU$ factorization}{scmatrix.low_up} inside
and may throw the same exceptions as the factorizer.
Function is \emph{inherited} in the classes
\GOT{scbmatrix}{scbmatrix}
and \GOT{schmatrix}{schmatrix}.
See also
\GOT{scmatrix}{scmatrix}.
Example:
\begin{Verbatim}
using namespace cvm;

try {
    double re[] = {1., 2., 3., 4., 5., 6., 7., 8., 9.};
    double im[] = {-1., 2., -3., -4., 5., -6., 7., -8., 9.};
    const scmatrix m(re, im, 3);

    std::cout << m << std::endl << m.det() << std::endl;
}
catch (std::exception& e) {
    std::cout << "Exception " << e.what () << std::endl;
}
\end{Verbatim}
prints
\begin{Verbatim}
(1,-1) (4,-4) (7,7)
(2,2) (5,5) (8,-8)
(3,-3) (6,-6) (9,9)

(-192,-192)
\end{Verbatim}
\newpage



\subsubsection{low\_up}
Functions%
\pdfdest name {scmatrix.low_up} fit
\begin{verbatim}
scmatrix& scmatrix::low_up (const scmatrix& m, int* nPivots)
throw (cvmexception);

scmatrix scmatrix::low_up (int* nPivots) const
throw (cvmexception);
\end{verbatim}
compute $LU$ factorization of a calling matrix as
\begin{equation*}
A=PLU
\end{equation*}
where $P$ is  permutation matrix, $L$ is  lower
triangular matrix with unit diagonal
elements and $U$ is  upper triangular matrix.
All functions store result as the matrix $L$ without
main diagonal combined with $U$. All functions
return pivot indices as an array of integers
(it should support at least \verb"msize()" elements)
pointed to by \verb"nPivots" so \hbox{$i$-th} row
was interchanged with \hbox{\verb"nPivots["$i$\verb"]"-th} row.
The first version sets  calling matrix to be equal to the
\verb"m"'s $LU$ factorization and the second one
creates an object of type \verb"scmatrix" as  calling matrix
$LU$ factorization.
Functions throw \GOT{cvmexception}{cvmexception}
in case of inappropriate
sizes of the operands or when the matrix to be factorized is close to
singular. It is recommended to use \GOT{iarray}{iarray}
for pivot values.
Function is \emph{redefined} in the class
\GOT{scbmatrix}{scbmatrix}
and \emph{inherited} in \GOT{schmatrix}{schmatrix}.
See also
\GOT{scmatrix}{scmatrix}.
Example:
\begin{Verbatim}
using namespace cvm;

try {
    double re[] = {1., 2., 3., 4., 5., 6., 7., 8., 9.};
    double im[] = {1., 2., 3., 4., 5., 6., 7., 8., 9.1};
    scmatrix m(re, im, 3);
    scmatrix mLU(3), mLo(3), mUp(3);
    iarray naPivots(3);

    mLU.low_up (m, naPivots);

    mLo.identity ();
    mLo.diag(-2) = mLU.diag(-2);
    mLo.diag(-1) = mLU.diag(-1);
    mUp.diag(0) = mLU.diag(0);
    mUp.diag(1) = mLU.diag(1);
    mUp.diag(2) = mLU.diag(2);

    std::cout << mLo << std::endl << mUp 
              << std::endl << naPivots << std::endl;

    mLU = mLo * mUp;
    for (int i = 3; i >= 1; i--) {
        mLU.swap_rows (i, naPivots[i]);
    }
    std::cout << mLU << std::endl << m - mLU;
}
catch (std::exception& e) {
    std::cout << "Exception " << e.what () << std::endl;
}
\end{Verbatim}
prints
\begin{Verbatim}
(1,0) (0,0) (0,0)
(0.333333,0) (1,0) (0,0)
(0.666667,0) (0.5,0) (1,0)

(3,3) (6,6) (9,9.1)
(0,0) (2,2) (4,3.96667)
(0,0) (0,0) (-1.11022e-016,-0.05)

3 3 3

(1,1) (4,4) (7,7)
(2,2) (5,5) (8,8)
(3,3) (6,6) (9,9.1)

(0,0) (0,0) (0,0)
(0,0) (0,0) (0,0)
(0,0) (0,0) (0,0)
\end{Verbatim}
\newpage




\subsubsection{cond}
Function%
\pdfdest name {scmatrix.cond} fit
\begin{verbatim}
TR scmatrix::cond () const throw (cvmexception);
\end{verbatim}
returns  condition number reciprocal of 
a calling matrix $A$ in the \GO{infinity-norm}{Array.norminf}:
\begin{equation*}
\kappa_\infty=\|A\|_\infty \|A^{-1}\|_\infty.
\end{equation*}
Less value returned means that matrix $A$ is closer to singular.
Zero value returned means estimation underflow or
that matrix $A$ is singular.
The condition number is used for error analysis
of systems of linear equations.
Function throws \GOT{cvmexception}{cvmexception} in case of LAPACK
subroutines failure.
Function is \emph{inherited} in the classes
\GOT{scbmatrix}{scbmatrix}
and \GOT{schmatrix}{schmatrix}.
See also
\GOT{scmatrix::solve}{scmatrix.solve}, \GOT{scmatrix}{scmatrix}.
Example:
\begin{Verbatim}
using namespace cvm;

try {
    double re[] = {1., 2., 3., 4., 5., 6., 7., 8., 9.};
    double im[] = {1., 2., 3., 4., 5., 6., 7., 8., 9.};

    scmatrix m(re, im, 3);
    std::cout << m.cond() << std::endl 
              << m.det() << std::endl << std::endl;

    m(3,3) = std::complex<double>(9.,10.);
    std::cout << m.cond() << std::endl << m.det() << std::endl;
}
catch (std::exception& e) {
    std::cout << "Exception " << e.what () << std::endl;
}
\end{Verbatim}
prints
\begin{Verbatim}
1.54198e-018
(1.33227e-015,-1.33227e-015)

0.0050679
(6,-1.33227e-015)
\end{Verbatim}
\newpage




\subsubsection{inv}
Functions%
\pdfdest name {scmatrix.inv} fit
\begin{verbatim}
scmatrix& scmatrix::inv (const scmatrix& m) throw (cvmexception);
scmatrix scmatrix::inv () const throw (cvmexception);
\end{verbatim}
implement matrix inversion.
The first version sets  calling matrix to be equal to 
\verb"m" inverted and the second one
creates an object of type \verb"scmatrix" as inverted calling matrix.
Functions throw  \GOT{cvmexception}{cvmexception} 
in case of inappropriate
sizes of the operands or when the matrix to be inverted is close to
singular.
Function is \emph{redefined} in the class
\GOT{schmatrix}{schmatrix}
and \emph{inherited} in \GOT{scbmatrix}{scbmatrix}.
See also
\GOT{scmatrix}{scmatrix}.
Example:
\begin{Verbatim}
using namespace cvm;

std::cout.setf (std::ios::scientific | std::ios::left); 
std::cout.precision (2);
try {
    double re[] = {1., 2., 3., 4., 5., 6., 7., 8., 9.};
    double im[] = {1., 2., 3., 4., 5., 6., 7., 8., 10.};
    scmatrix m(re, im, 3);
    scmatrix mi(3);

    mi.inv (m);
    std::cout << mi << std::endl << mi * m - eye_complex(3);
    std::cout << std::endl << mi.inv() * mi - eye_complex(3);
}
catch (std::exception& e) {
    std::cout << "Exception " << e.what () << std::endl;
}
\end{Verbatim}
prints
\begin{Verbatim}
(-8.33e-001,-1.67e-001) (6.67e-001,1.33e+000) (4.81e-016,-1.00e+000)
(3.33e-001,1.67e+000) (-1.67e-001,-3.83e+000) (-5.37e-016,2.00e+000)
(2.22e-016,-1.00e+000) (-4.44e-016,2.00e+000) (2.22e-016,-1.00e+000)

(-1.11e-016,1.11e-016) (0.00e+000,2.22e-016) (8.33e-017,1.26e-015)
(4.44e-016,-5.00e-016) (0.00e+000,7.77e-016) (7.22e-016,-1.15e-015)
(-1.11e-016,1.11e-016) (2.22e-016,-6.66e-016) (0.00e+000,1.11e-016)

(4.44e-016,4.44e-016) (-1.18e-015,1.33e-015) (6.66e-016,-7.77e-016)
(-6.85e-016,8.88e-016) (1.33e-015,3.16e-030) (-8.33e-016,-1.77e-030)
(-5.09e-016,6.66e-016) (7.96e-016,-1.78e-015) (-4.44e-016,1.11e-016)
\end{Verbatim}
\newpage




\subsubsection{exp}
Functions%
\pdfdest name {scmatrix.exp} fit
\begin{verbatim}
scmatrix& scmatrix::exp (const scmatrix& m, TR tol = cvmMachSp ())
throw (cvmexception);
scmatrix scmatrix::exp (TR tol = cvmMachSp ()) const
throw (cvmexception);
\end{verbatim}
compute  exponent of a calling matrix using Pad\'e approximation
defined as
\begin{equation*}
R_{pq}(z)=D_{pq}(z)^{-1}N_{pq}(z)=1+z+\dots+z^p/p!\,,
\end{equation*}
where
\begin{align*}
N_{pq}(z)&=\sum_{k=0}^p\frac{(p+q-k)!p!}{(p+q)!k!(q-k)!}z^k,\\
D_{pq}(z)&=\sum_{k=0}^q\frac{(p+q-k)!p!}{(p+q)!k!(q-k)!}(-z)^k
\end{align*}
along with the matrix normalizing as described in
\GO{\cite{Golub}}{biblio}, p.~572.
Functions use \verb"ZMEXP" (or \verb"CMEXP" for \verb"float" version)
\FORTRAN subroutine implementing the algorithm.
The first version sets calling matrix to be equal to
 exponent of \verb"m" and returns a reference to the matrix
changed. The second version
creates an object of type \verb"scmatrix" as 
exponent of a calling matrix.
The algorithm uses parameter \verb"tol"
as $\varepsilon(p,q)$ in order to choose constants $p$ and $q$
so that
\begin{equation*}
\varepsilon(p,q)\ge 2^{3-(p+q)}\frac{p!q!}{(p+q)!(p+q+1)!}.
\end{equation*}
This parameter is equal to the
\GO{largest relative spacing}{Utilities.cvmMachSp} by default.
Functions throw  \GOT{cvmexception}{cvmexception} 
in case of
inappropriate sizes of the operands or when LAPACK subroutine fails.
Functions are \emph{inherited} in the classes
\GOT{scbmatrix}{scbmatrix}
and \GOT{schmatrix}{schmatrix}. The second version
is \emph{redefined} in \GOT{scbmatrix}{scbmatrix}.
See also
\GOT{scmatrix}{scmatrix}.
Example:
\begin{Verbatim}
using namespace cvm;

std::cout.setf (std::ios::scientific | std::ios::left); 
std::cout.precision (15);
try {
    scmatrix m(2);
    m(1,1) = std::complex<double>(-49.,1.);
    m(1,2) = std::complex<double>(24.,1.);
    m(2,1) = std::complex<double>(-64.,1.);
    m(2,2) = std::complex<double>(31.,1.);

    m = m.exp();
    std::cout << m(1,1) << std::endl << "   " 
              << m(1,2) << std::endl;
    std::cout << m(2,1) << std::endl << "   " 
              << m(2,2) << std::endl;
}
catch (std::exception& e) {
    std::cout << "Exception " << e.what () << std::endl;
}
\end{Verbatim}
prints
\begin{Verbatim}
(-4.508497580070061e-001,7.900659666739228e-001)
   (3.199576050798058e-001,-6.081804753524478e-001)
(-7.584316151932173e-001,1.666747485117903e+000)
   (5.295040786048336e-001,-1.278050361026397e+000)
\end{Verbatim}
MATLAB output:
\begin{Verbatim}
  Column 1

    -4.508497580070262e-001 +7.900659666739607e-001i
    -7.584316151932523e-001 +1.666747485117982e+000i

  Column 2

     3.199576050798204e-001 -6.081804753524764e-001i
     5.295040786048589e-001 -1.278050361026457e+000i
\end{Verbatim}
\newpage





\subsubsection{polynomial}
Functions%
\pdfdest name {scmatrix.polynom} fit
\begin{verbatim}
scmatrix& scmatrix::polynom (const scmatrix& m, const cvector& v)
throw (cvmexception);

scmatrix scmatrix::polynom (const cvector& v) const
throw (cvmexception);
\end{verbatim}
compute  matrix polynomial defined as
\begin{equation*}
p(A)=b_0I+b_1A+\dots+b_qA^q
\end{equation*}
using the Horner's rule:
\begin{equation*}
p(A)=\sum_{k=0}^r B_k(A^s)^k,\quad s=\floor(\!\sqrt{q}\,),\ r=\floor(q/s)
\end{equation*}
where
\begin{equation*}
B_k=\begin{cases}
\sum\limits_{i=0}^{s-1}b_{sk+i} A^i, & k=0,1,\dots,r-1\\
\sum\limits_{i=0}^{q-sr}b_{sr+i} A^i, & k=r.
\end{cases}
\end{equation*}
See also \GO{\cite{Golub}}{biblio}, p.~568.
The coefficients $b_0,b_1,\dots,b_q$ are passed in  parameter 
\verb"v",
where $q$ is equal to \verb"v.size()-1", so  functions
compute matrix polynomial equal to
\begin{equation*}
\mathtt{v[1]*I + v[2]*m +\cdots + v[v.size()]*m^{v.size()-1}}
\end{equation*}
The first version sets  calling matrix to be equal to the polynomial of
\verb"m"
and the second one
creates an object of type \verb"scmatrix" as the 
polynomial of a calling matrix.
Functions use \verb"ZPOLY" (or \verb"CPOLY" for \verb"float" version)
\FORTRAN subroutine implementing the Horner's algorithm.
Functions throw  \GOT{cvmexception}{cvmexception} 
in case of inappropriate sizes of the operands.
Functions are \emph{inherited} in the class
\GOT{scbmatrix}{scbmatrix}
and \emph{redefined} in \GOT{schmatrix}{schmatrix}.
See also
\GOT{scmatrix}{scmatrix}.
Example:
\begin{Verbatim}
using namespace cvm;

std::cout.setf (std::ios::scientific | std::ios::left); 
std::cout.precision (15);
try {
    const double re[] = {2.2, 1.3, 1.1, -0.9, 0.2,
                        -0.45, 45., -30., 10., 3., 0.};
    const double im[] = {0.5, -2, 0, 1, 3,
                        -3., 30., 0., -9., 0., 1.};
    const cvector v(re, im, 11);
    scmatrix m(2), mp(2);
    m(1,1) = std::complex<double>(0.1, -0.2);
    m(1,2) = std::complex<double>(0.1, -0.2);
    m(2,1) = std::complex<double>(0.5, -0.6);
    m(2,2) = std::complex<double>(0.3, -0.4);

    mp.polynom (m, v);
    std::cout << mp(1,1) << std::endl << "   " 
              << mp(1,2) << std::endl;
    std::cout << mp(2,1) << std::endl << "   " 
              << mp(2,2) << std::endl;
}
catch (std::exception& e) {
    std::cout << "Exception " << e.what () << std::endl;
}
\end{Verbatim}
prints
\begin{Verbatim}
(2.485652665600000e+000,3.791263308800001e+000)
   (2.817786176000004e-001,2.301942860800001e+000)
(-8.835069888000001e-001,8.052028620800002e+000)
   (1.903009862400001e+000,6.666306188800002e+000)
\end{Verbatim}
MATLAB output:
\begin{Verbatim}
  Column 1

     2.485652665600000e+000 +3.791263308800001e+000i
    -8.835069887999991e-001 +8.052028620800002e+000i

  Column 2

     2.817786176000000e-001 +2.301942860800001e+000i
     1.903009862399999e+000 +6.666306188800003e+000i
\end{Verbatim}
\newpage




\subsubsection{eig}
Functions%
\pdfdest name {scmatrix.eig} fit
\begin{verbatim}
cvector scmatrix::eig (scmatrix& mEigVect, bool bRightVect = true) const
throw (cvmexception);

cvector scmatrix::eig () const throw (cvmexception);
\end{verbatim}
solve the 
\GO{nonsymmetric eigenvalue problem}{cvector.eig}
and return  complex vector with eigenvalues
of a calling matrix.
The first version
sets  output parameter \verb"mEigVect" to be equal
to  square matrix containing right (if parameter
\verb"bRightVect" is true, which is default value)
or left (if parameter
\verb"bRightVect" is false)
eigenvectors as columns.
All  functions
throw 
 \GOT{cvmexception}{cvmexception} 
in case of inappropriate sizes of the operands
or in case of convergence error.
Functions are \emph{inherited} in the class
\GOT{scbmatrix}{scbmatrix}
and \emph{redefined} in \GOT{schmatrix}{schmatrix}.
See also
\GOT{cvector}{cvector},
\GOT{scmatrix}{scmatrix}.
Example:
\begin{Verbatim}
using namespace cvm;
try {
    double re[] = {1., 2., 3., 4., 5., 6., 7., 8., 9.};
    double im[] = {1., 2., 3., 4., 5., 6., 7., 8., 10.};
    scmatrix m(re, im, 3);
    scmatrix me(3);
    cvector vl(3);

    vl = m.eig (me);
    std::cout << vl << std::endl;
    std::cout.setf (std::ios::scientific | std::ios::left); 
    std::cout.precision (2);
    std::cout << m * me(1) - me(1) * vl(1);
    std::cout << m * me(2) - me(2) * vl(2);
    std::cout << m * me(3) - me(3) * vl(3);
}
catch (std::exception& e) {
    std::cout << "Exception " << e.what () << std::endl;
}
\end{Verbatim}
prints
\begin{Verbatim}
(16.1096,16.7004) (-1.09351,-0.88358) (-0.0161248,0.183218)

(-9.44e-016,-3.55e-015) (-1.11e-016,-1.07e-014) (2.66e-015,-1.42e-014)
(-5.55e-016,-4.44e-016) (-1.80e-015,9.44e-016) (-2.00e-015,7.22e-016)
(9.92e-016,1.24e-015) (1.05e-015,2.78e-017) (1.64e-015,9.30e-016)
\end{Verbatim}
\newpage




\subsubsection{Cholesky}
Function%
\pdfdest name {scmatrix.cholesky} fit
\begin{verbatim}
scmatrix& scmatrix::cholesky (const schmatrix& m)
throw (cvmexception);
\end{verbatim}
forms the Cholesky factorization of  hermitian positive-definite 
matrix $A$ defined as
\begin{equation*}
A=U^T U,
\end{equation*}
where $U$ is upper triangular matrix.
It utilizes one of \verb"?POTRF" routines of the
\URL{LAPACK library}{http://www.netlib.org/lapack}.
Function
sets  calling matrix to be equal to the factorization
of  hermitian
positive-definite matrix \verb"m".
Function throws 
 \GOT{cvmexception}{cvmexception} 
in case of inappropriate sizes of the operands
or in case of convergence error.
See also
\GOT{scmatrix}{scmatrix} and
\GOT{schmatrix}{schmatrix}.
Example:
\begin{Verbatim}
using namespace cvm;

try {
    double r[] = {1., 2., 1., 2., 15., -1., 1., -1., 20.};
    double i[] = {0., -1., 2., 1., 0., 3., -2., -3., 0.};
    const schmatrix m(r, i, 3);
    scmatrix c(3);

    c.cholesky(m);

    std::cout << c << std::endl;
    std::cout << ~c * c - m;
}
catch (std::exception& e) {
    std::cout << "Exception: " << e.what () << std::endl;
}
\end{Verbatim}
prints
\begin{Verbatim}
(1,0) (2,1) (1,-2)
(0,-0) (3.16228,0) (-0.316228,0.632456)
(0,0) (-0,0) (3.80789,0)

(0,0) (0,0) (0,0)
(0,0) (1.77636e-015,0) (0,0)
(0,0) (0,0) (0,0)
\end{Verbatim}
\newpage


\subsubsection{Bunch-Kaufman}
Function%
\pdfdest name {scmatrix.bunch_kaufman} fit
\begin{verbatim}
scmatrix& scmatrix::bunch_kaufman (const schmatrix& m, int* pivots)
throw (cvmexception);
\end{verbatim}
forms the Bunch-Kaufman factorization of  hermitian matrix 
(cited from the MKL library documentation):
\begin{equation*}
A=PUDU^TP^T,
\end{equation*}
where $A$ is the input matrix passed in parameter \verb"m",
$P$ is  permutation matrix, $U$ and $L$ are upper and lower triangular
matrices with unit diagonal, and $D$ is  symmetric 
block-diagonal matrix with $1$-by-$1$ and $2$-by-$2$
diagonal blocks. $U$ and $L$ have $2$-by-$2$ unit diagonal 
blocks corresponding to the $2$-by-$2$ blocks of $D$.
It utilizes one of \verb"?SYTRF" routines of the
\URL{LAPACK library}{http://www.netlib.org/lapack}.
Function
sets  calling matrix to be equal to the factorization
of  hermitian
positive-definite matrix \verb"m".
Function throws 
 \GOT{cvmexception}{cvmexception} 
in case of inappropriate sizes of the operands
or in case of convergence error.
See also
\GOT{scmatrix}{scmatrix} and
\GOT{schmatrix}{schmatrix}.
Function is mostly designed to be used for subsequent calls
of \verb"?SYTRS", \verb"?SYCON" and \verb"?SYTRI" routines of the
\URL{LAPACK library}{http://www.netlib.org/lapack}.
Currently it's used internally in \GOT{scmatrix::det}{scmatrix.det} flow
when argument is hermitian but not positive-definite matrix.
\newpage


\subsubsection{qr}
Function%
\pdfdest name {scmatrix.qr} fit
\begin{verbatim}
void scmatrix::qr (scmatrix& mQ, scmatrix& mR) const throw (cvmexception);
\end{verbatim}
computes QR factorization as
\begin{equation*}
M=QR
\end{equation*}
where 
$M$ is  calling square matrix, unitary matrix $Q$ 
and upper triangular matrix $R$ are \verb"mQ" and \verb"mR"
respectively. 
Function throws
 \GOT{cvmexception}{cvmexception}
in case of inappropriate sizes of the operands.
See also \GOT{cmatrix}{cmatrix}, \GOT{cmatrix::qr}{cmatrix.qr},
\GOT{scmatrix}{scmatrix}.
Example:
\begin{Verbatim}
using namespace cvm;

treal ar[] = {1., 2., 3., 4., 5., 6., 7., 8., 9.};
treal ai[] = {1., -1., 2., -2., 3., -3., 4., -4., 5.};
const cvm::scmatrix m(ar, ai, 3);
cvm::scmatrix q(3), r(3);

m.qr(q,r);
std::cout << (eye_complex(3) - ~q * q).norm()
          << " " << (m - q * r).norm() << std::endl;
\end{Verbatim}
prints
\begin{Verbatim}
+3.7957075e-16 +1.9641850e-15
\end{Verbatim}
\newpage



\subsubsection{lq}
Function%
\pdfdest name {scmatrix.lq} fit
\begin{verbatim}
void scmatrix::lq (scmatrix& mL, scmatrix& mQ) const throw (cvmexception);
\end{verbatim}
computes LQ factorization as
\begin{equation*}
M=LQ
\end{equation*}
where 
$M$ is  calling square matrix, lower triangular matrix $L$ 
and unitary matrix $Q$ are \verb"mL" and \verb"mQ"
respectively. 
Function throws
 \GOT{cvmexception}{cvmexception}
in case of inappropriate sizes of the operands.
See also \GOT{cmatrix}{cmatrix}, \GOT{cmatrix::lq}{cmatrix.lq},
\GOT{scmatrix}{scmatrix}.
Example:
\begin{Verbatim}
using namespace cvm;

treal ar[] = {1., 2., 3., 4., 5., 6., 7., 8., 9.};
treal ai[] = {1., -1., 2., -2., 3., -3., 4., -4., 5.};
const cvm::scmatrix m(ar, ai, 3);
cvm::scmatrix l(3), q(3);

m.lq(l,q);
std::cout << (eye_complex(3) - ~q * q).norm()
          << " " << (m - l * q).norm() << std::endl;
\end{Verbatim}
prints
\begin{Verbatim}
+3.5331153e-016 +3.0526086e-015
\end{Verbatim}
\newpage



\subsubsection{rq}
Function%
\pdfdest name {scmatrix.rq} fit
\begin{verbatim}
void scmatrix::rq (scmatrix& mR, scmatrix& mQ) const throw (cvmexception);
\end{verbatim}
computes RQ factorization as
\begin{equation*}
M=RQ
\end{equation*}
where 
$M$ is  calling square matrix, upper triangular matrix $R$ 
and unitary matrix $Q$ are \verb"mR" and \verb"mQ"
respectively. 
Function throws
 \GOT{cvmexception}{cvmexception}
in case if inappropriate sizes of the operands passed.
See also \GOT{cmatrix}{cmatrix}, \GOT{cmatrix::rq}{cmatrix.rq},
\GOT{srmatrix}{srmatrix}.
Example:
\begin{Verbatim}
using namespace cvm;

treal ar[] = {1., 2., 3., 4., 5., 6., 7., 8., 9.};
treal ai[] = {1., -1., 2., -2., 3., -3., 4., -4., 5.};
const cvm::scmatrix m(ar, ai, 3);
cvm::scmatrix r(3), q(3);

m.rq(r,q);
std::cout << (eye_complex(3) - ~q * q).norm()
          << " " << (m - r * q).norm() << std::endl;
\end{Verbatim}
prints
\begin{Verbatim}
+3.3357248e-016 +3.4488467e-015
\end{Verbatim}
\newpage




\subsubsection{ql}
Function%
\pdfdest name {scmatrix.ql} fit
\begin{verbatim}
void scmatrix::ql (scmatrix& mQ, scmatrix& mL) const throw (cvmexception);
\end{verbatim}
computes QL factorization as
\begin{equation*}
M=QL
\end{equation*}
where 
$M$ is  calling square matrix, unitary matrix $Q$ 
and lower triangular matrix $L$ are \verb"mQ" and \verb"mL"
respectively. 
Function throws
 \GOT{cvmexception}{cvmexception}
in case if inappropriate sizes of the operands passed.
See also \GOT{cmatrix}{cmatrix}, \GOT{cmatrix::ql}{cmatrix.ql},
\GOT{scmatrix}{scmatrix}.
Example:
\begin{Verbatim}
using namespace cvm;

treal ar[] = {1., 2., 3., 4., 5., 6., 7., 8., 9.};
treal ai[] = {1., -1., 2., -2., 3., -3., 4., -4., 5.};
const cvm::scmatrix m(ar, ai, 3);
cvm::scmatrix q(3), l(3);

m.ql(q,l);
std::cout << (eye_complex(3) - ~q * q).norm()
          << " " << (m - q * l).norm() << std::endl;
\end{Verbatim}
prints
\begin{Verbatim}
+8.8904536e-016 +4.5789129e-015
\end{Verbatim}
\newpage





\subsubsection{identity}
Function%
\pdfdest name {scmatrix.identity} fit
\begin{verbatim}
scmatrix& scmatrix::identity();
\end{verbatim}
sets  calling matrix to be equal to identity matrix
and returns a reference to
the matrix changed. 
Function is \emph{redefined} in the classes
\GOT{scbmatrix}{scbmatrix} and
\GOT{schmatrix}{schmatrix}.
See also \GOT{scmatrix}{scmatrix}.
Example:
\begin{Verbatim}
using namespace cvm;

scmatrix m(3);
m.randomize_real(0.,3.);
m.randomize_imag(-1.,2.);

std::cout << m << std::endl;
std::cout << m.identity();
\end{Verbatim}
prints
\begin{Verbatim}
(1.31162,-0.52501) (2.8612,-0.531144) (1.31849,0.547838)
(1.19929,1.48253) (0.535417,0.41316) (0.459883,1.7019)
(0.415937,-0.491134) (2.0969,-0.218024) (0.545305,1.17866)

(1,0) (0,0) (0,0)
(0,0) (1,0) (0,0)
(0,0) (0,0) (1,0)
\end{Verbatim}
\newpage



\subsubsection{vanish}
Function%
\pdfdest name {scmatrix.vanish} fit
\begin{verbatim}
scmatrix& scmatrix::vanish();
\end{verbatim}
sets every element of a calling matrix to be equal to zero
and returns a reference to
the matrix changed. This function is faster
than
\GOT{scmatrix::set(TC)}{scmatrix.set}
with zero operand passed.
Function is \emph{redefined} in the classes
\GOT{scbmatrix}{scbmatrix} and
\GOT{schmatrix}{schmatrix}.
See also \GOT{scmatrix}{scmatrix}.
Example:
\begin{Verbatim}
using namespace cvm;

scmatrix m(3);
m.randomize_real(0.,3.);
m.randomize_imag(-1.,2.);

std::cout << m << std::endl;
std::cout << m.vanish();
\end{Verbatim}
prints
\begin{Verbatim}
(1.34834,-0.758385) (0.837825,-0.225532) (0.367687,0.791833)
(2.23698,-0.183142) (2.6878,0.741111) (0.495865,0.698904)
(0.584124,0.00491348) (1.31574,0.687643) (0.482131,1.66482)

(0,0) (0,0) (0,0)
(0,0) (0,0) (0,0)
(0,0) (0,0) (0,0)
\end{Verbatim}
\newpage



\subsubsection{randomize\_real}
Function%
\pdfdest name {scmatrix.randomizereal} fit
\begin{verbatim}
scmatrix& scmatrix::randomize_real (TR dFrom, TR dTo);
\end{verbatim}
fills  real part of a calling matrix with 
pseudo-random numbers distributed between
\verb"dFrom" and \verb"dTo".
Function
returns a reference to the matrix changed.
Function is \emph{redefined} in the classes
\GOT{scbmatrix}{scbmatrix} and
\GOT{schmatrix}{schmatrix}.
See also
\GOT{scmatrix}{scmatrix}.
Example:
\begin{Verbatim}
using namespace cvm;

std::cout.setf (std::ios::scientific | std::ios::left); 
std::cout.precision (2);
scmatrix m(3);
m.randomize_real(0.,3.);
std::cout << m;
\end{Verbatim}
prints
\begin{Verbatim}
(1.56e+000,0.00e+000) (2.39e+000,0.00e+000) (2.41e+000,0.00e+000)
(3.73e-002,0.00e+000) (2.61e+000,0.00e+000) (1.36e+000,0.00e+000)
(2.71e+000,0.00e+000) (1.69e+000,0.00e+000) (2.68e+000,0.00e+000)
\end{Verbatim}
\newpage


\subsubsection{randomize\_imag}
Function%
\pdfdest name {scmatrix.randomizeimag} fit
\begin{verbatim}
scmatrix& scmatrix::randomize_imag (TR dFrom, TR dTo);
\end{verbatim}
fills  imaginary part of a calling matrix with 
pseudo-random numbers distributed between
\verb"dFrom" and \verb"dTo".
Function
returns a reference to the matrix changed.
Function is \emph{redefined} in the classes
\GOT{scbmatrix}{scbmatrix} and
\GOT{schmatrix}{schmatrix}.
See also
\GOT{scmatrix}{scmatrix}.
Example:
\begin{Verbatim}
using namespace cvm;

std::cout.setf (std::ios::scientific | std::ios::left); 
std::cout.precision (2);
scmatrix m(3);
m.randomize_imag(0.,3.);
std::cout << m;
\end{Verbatim}
prints
\begin{Verbatim}
(0.00e+000,1.58e+000) (0.00e+000,2.38e+000) (0.00e+000,2.64e+000)
(0.00e+000,1.62e-002) (0.00e+000,4.26e-002) (0.00e+000,2.21e+000)
(0.00e+000,2.39e+000) (0.00e+000,6.95e-001) (0.00e+000,4.30e-001)
\end{Verbatim}
\newpage

