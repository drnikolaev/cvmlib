\bigskip
\noindent
\verb"template <typename TR, typename TC>"\\
\verb"class scbmatrix : public scmatrix <TR,TC>, public BandMatrix <TR,TC> {"\\
\verb"public:"\\
\verb"    "\GOT{scbmatrix}{scbmatrix.scbmatrix ()}\verb" ();"\\
\verb"    explicit "\GOT{scbmatrix}{scbmatrix.scbmatrix (int)}\verb" (int nMN);"\\
\verb"    "\GOT{scbmatrix}{scbmatrix.scbmatrix (int,int,int)}\verb" (int nMN);"\\
\verb"    "\GOT{scbmatrix}{scbmatrix.scbmatrix (TC*,int,int,int)}\verb" (TC* pD, int nMN, int nKL, int nKU);"\\
\verb"    "\GOT{scbmatrix}{scbmatrix.scbmatrix (const TC*,int,int,int)}\verb" (const TC* pD, int nMN, int nKL, int nKU);"\\
\verb"    "\GOT{scbmatrix}{scbmatrix.scbmatrix (const scbmatrix&)}\verb" (const scbmatrix& m);"\\
\verb"    "\GOT{scbmatrix}{scbmatrix.scbmatrix (const cmatrix&,int,int)}\verb" (const cmatrix& m, int nKL, int nKU);"\\
\verb"    explicit "\GOT{scbmatrix}{scbmatrix.scbmatrix (const cvector&)}\verb" (const cvector& v);"\\
\verb"    explicit "\GOT{scbmatrix}{scbmatrix.scbmatrix (const srbmatrix&,bool)}\verb" (const srbmatrix& m, bool bRealPart = true);"\\
\verb"    "\GOT{scbmatrix}{scbmatrix.scbmatrix (const srbmatrix&, const srbmatrix&)}\verb" (const srbmatrix& mRe, const srbmatrix& mIm);"\\
\verb"    TC& "\GOT{operator ()}{scbmatrix.operator (,)}\verb" (int im, int in) throw (cvmexception);"\\
\verb"    TC "\GOT{operator ()}{scbmatrix.operator (,)}\verb" (int im, int in) const throw (cvmexception);"\\
\verb"    const cvector "\GOT{operator ()}{scbmatrix.operator ()}\verb" (int i) const throw (cvmexception);"\\
\verb"    const cvector "\GOT{operator []}{scbmatrix.operator []}\verb" (int i) const throw (cvmexception);"\\
\verb"    const srbmatrix "\GOT{real}{scbmatrix.real}\verb" () const;"\\
\verb"    const srbmatrix "\GOT{imag}{scbmatrix.imag}\verb" () const;"\\
\verb"    scbmatrix& "\GOT{operator =}{scbmatrix.operator = (const scbmatrix&)}\verb" (const scbmatrix& m) throw (cvmexception);"\\
\verb"    scbmatrix& "\GOT{assign}{scbmatrix.assign}\verb" (const cvector& v) throw (cvmexception);"\\
\verb"    scbmatrix& "\GOT{assign}{scbmatrix.assign}\verb" (const TC* pD);"\\
\verb"    scbmatrix& "\GOT{set}{scbmatrix.set}\verb" (TC z);"\\
\verb"    scbmatrix& "\GOT{assign\_real}{scbmatrix.assignreal}\verb" (const srbmatrix& mRe) throw (cvmexception);"\\
\verb"    scbmatrix& "\GOT{assign\_imag}{scbmatrix.assignimag}\verb" (const srbmatrix& mIm) throw (cvmexception);"\\
\verb"    scbmatrix& "\GOT{set\_real}{scbmatrix.setreal}\verb" (TR d);"\\
\verb"    scbmatrix& "\GOT{set\_imag}{scbmatrix.setimag}\verb" (TR d);"\\
\verb"    scbmatrix& "\GOT{resize}{scbmatrix.resize}\verb" (int nNewMN) throw (cvmexception);"\\
\verb"    scbmatrix& "\GOT{resize\_lu}{scbmatrix.resizelu}\verb" (int nNewKL, int nNewKU) throw (cvmexception);"\\
\verb"    bool "\GOT{operator ==}{scbmatrix.operator ==}\verb" (const scbmatrix& v) const;"\\
\verb"    bool "\GOT{operator !=}{scbmatrix.operator !=}\verb" (const scbmatrix& v) const;"\\
\verb"    scbmatrix& "\GOT{operator <{}<}{scbmatrix.operator <<}\verb" (const scbmatrix& m) throw (cvmexception);"\\
\verb"    scbmatrix "\GOT{operator +}{scbmatrix.operator +}\verb" (const scbmatrix& m) const"\\
\verb"                          throw (cvmexception);"\\
\verb"    scbmatrix "\GOT{operator -}{scbmatrix.operator -}\verb" (const scbmatrix& m) const"\\
\verb"                          throw (cvmexception);"\\
\verb"    scbmatrix& "\GOT{sum}{scbmatrix.sum}\verb" (const scbmatrix& m1,"\\
\verb"                    const scbmatrix& m2) throw (cvmexception);"\\
\verb"    scbmatrix& "\GOT{diff}{scbmatrix.diff}\verb" (const scbmatrix& m1,"\\
\verb"                     const scbmatrix& m2) throw (cvmexception);"\\
\verb"    scbmatrix& "\GOT{operator +=}{scbmatrix.operator +=}\verb" (const scbmatrix& m) throw (cvmexception);"\\
\verb"    scbmatrix& "\GOT{operator -=}{scbmatrix.operator -=}\verb" (const scbmatrix& m) throw (cvmexception);"\\
\verb"    scbmatrix "\GOT{operator -}{scbmatrix.operator - ()}\verb" () const;"\\
\verb"    scbmatrix& "\GOT{operator ++}{scbmatrix.operator ++}\verb" ();"\\
\verb"    scbmatrix& "\GOT{operator ++}{scbmatrix.operator ++}\verb" (int);"\\
\verb"    scbmatrix& "\GOT{operator -{}-}{scbmatrix.operator --}\verb" ();"\\
\verb"    scbmatrix& "\GOT{operator -{}-}{scbmatrix.operator --}\verb" (int);"\\
\verb"    scbmatrix "\GOT{operator *}{scbmatrix.operator * (TR)}\verb" (TR d) const;"\\
\verb"    scbmatrix "\GOT{operator /}{scbmatrix.operator / (TR)}\verb" (TR d) const throw (cvmexception);"\\
\verb"    scbmatrix "\GOT{operator *}{scbmatrix.operator * (TC)}\verb" (TC z) const;"\\
\verb"    scbmatrix "\GOT{operator /}{scbmatrix.operator / (TC)}\verb" (TC z) const throw (cvmexception);"\\
\verb"    scbmatrix& "\GOT{operator *=}{scbmatrix.operator *= (TR)}\verb" (TR d);"\\
\verb"    scbmatrix& "\GOT{operator /=}{scbmatrix.operator /= (TR)}\verb" (TR d) throw (cvmexception);"\\
\verb"    scbmatrix& "\GOT{operator *=}{scbmatrix.operator *= (TC)}\verb" (TC z);"\\
\verb"    scbmatrix& "\GOT{operator /=}{scbmatrix.operator /= (TC)}\verb" (TC z) throw (cvmexception);"\\
\verb"    scbmatrix& "\GOT{normalize}{scbmatrix.normalize}\verb" ();"\\
\verb"    scbmatrix "\GOT{operator \TildaT}{scbmatrix.conj}\verb" () const;"\\
\verb"    scbmatrix "\GOT{operator !}{scbmatrix.transpose}\verb" () const;"\\
\verb"    scbmatrix& "\GOT{conj}{scbmatrix.conj}\verb" (const scbmatrix& m) throw (cvmexception);"\\
\verb"    scbmatrix& "\GOT{conj}{scbmatrix.conj}\verb" ();"\\
\verb"    scbmatrix& "\GOT{transpose}{scbmatrix.transpose}\verb" (const scbmatrix& m) throw (cvmexception);"\\
\verb"    scbmatrix& "\GOT{transpose}{scbmatrix.transpose}\verb" ();"\\
\verb"    cvector "\GOT{operator *}{scbmatrix.operator * (const cvector&)}\verb" (const cvector& v) const throw (cvmexception);"\\
\verb"    cmatrix "\GOT{operator *}{scbmatrix.operator * (const cmatrix&)}\verb" (const cmatrix& m) const throw (cvmexception);"\\
\verb"    scmatrix "\GOT{operator *}{scbmatrix.operator * (const scmatrix&)}\verb" (const scmatrix& m) const throw (cvmexception);"\\
\verb"    scbmatrix "\GOT{operator *}{scbmatrix.operator * (const scbmatrix&)}\verb" (const scbmatrix& m) const throw (cvmexception);"\\
\verb"    scbmatrix& "\GOT{low\_up}{scbmatrix.low_up}\verb" (const scbmatrix& m,"\\
\verb"                       int* nPivots) throw (cvmexception);"\\
\verb"    scbmatrix "\GOT{low\_up}{scbmatrix.low_up}\verb" (int* nPivots) const throw (cvmexception);"\\
\verb"    cvector "\GOT{operator /}{scbmatrix.operator / (cvector)}\verb" (const cvector& vB) const throw (cvmexception);"\\
\verb"    scbmatrix& "\GOT{identity}{scbmatrix.identity} ();\\
\verb"    scbmatrix& "\GOT{vanish}{scbmatrix.vanish}\verb" ();"\\
\verb"    scbmatrix& "\GOT{randomize\_real}{scbmatrix.randomizereal}\verb" (TR dFrom, TR dTo);"\\
\verb"    scbmatrix& "\GOT{randomize\_imag}{scbmatrix.randomizeimag}\verb" (TR dFrom, TR dTo);"\\
\verb"};"
\newpage



\subsubsection{scbmatrix ()}
Constructor%
\pdfdest name {scbmatrix.scbmatrix ()} fit
\begin{verbatim}
scbmatrix::scbmatrix ();
\end{verbatim}
creates  empty \verb"scbmatrix" object.
See also \GOT{scbmatrix}{scbmatrix}.
Example:
\begin{Verbatim}
using namespace cvm;

scbmatrix m;
std::cout << m.msize() << " " << m.nsize() << " " << m.size() ;
std::cout << " " << m.lsize() << " " << m.usize() << std::endl;
m.resize(3);
m.resize_lu(1,0);
m.set(std::complex<double>(1.,2.));
std::cout << m;
\end{Verbatim}
prints
\begin{Verbatim}
0 0 0 0 0
(1,2) (0,0) (0,0)
(1,2) (1,2) (0,0)
(0,0) (1,2) (1,2)
\end{Verbatim}
\newpage




\subsubsection{scbmatrix (int)}
Constructor%
\pdfdest name {scbmatrix.scbmatrix (int)} fit
\begin{verbatim}
explicit scbmatrix::scbmatrix (int nMN);
\end{verbatim}
creates  $n\times n$ \verb"scbmatrix" object where $n$ is passed in
\verb"nMN" parameter. The matrix created is diagonal, i.e. $k_l=k_u=0$.
Constructor throws  \GOT{cvmexception}{cvmexception}
in case of non-positive size passed or memory allocation failure.
See also \GOT{scbmat\-rix}{scbmatrix}.
Example:
\begin{Verbatim}
using namespace cvm;

scbmatrix m(4);
std::cout << m.msize() << " " << m.nsize() << " " << m.size() ;
std::cout << " " << m.lsize() << " " << m.usize() << std::endl;
m.set(std::complex<double>(1.,2.));
std::cout << m;
\end{Verbatim}
prints
\begin{Verbatim}
4 4 4 0 0
(1,2) (0,0) (0,0) (0,0)
(0,0) (1,2) (0,0) (0,0)
(0,0) (0,0) (1,2) (0,0)
(0,0) (0,0) (0,0) (1,2)
\end{Verbatim}
\newpage




\subsubsection{scbmatrix (int,int,int)}
Constructor%
\pdfdest name {scbmatrix.scbmatrix (int,int,int)} fit
\begin{verbatim}
scbmatrix::scbmatrix (int nMN, int nKL, int nKU);
\end{verbatim}
creates  $n\times n$ \verb"scbmatrix" object where $n$ is passed in
\verb"nMN" parameter. The matrix created has \verb"nKL" 
sub-diagonals and \verb"nKU" super-diagonals.
Constructor throws  \GOT{cvmexception}{cvmexception}
in case of non-positive size or negative number 
of sub-diagonals or super-diagonals
passed or in case of memory allocation failure.
See also \GOT{scbmat\-rix}{scbmatrix}.
Example:
\begin{Verbatim}
using namespace cvm;

scbmatrix m(4,1,1);
m.set(std::complex<double>(1.,2.));
std::cout << m << std::endl
          << m.msize() << " " << m.nsize() << " " << m.size()
          << " " << m.lsize() << " " << m.usize() << std::endl;
\end{Verbatim}
prints
\begin{Verbatim}
(1,2) (1,2) (0,0) (0,0)
(1,2) (1,2) (1,2) (0,0)
(0,0) (1,2) (1,2) (1,2)
(0,0) (0,0) (1,2) (1,2)

4 4 12 1 1
\end{Verbatim}
\newpage




\subsubsection{scbmatrix (TC*,int,int,int)}
Constructor%
\pdfdest name {scbmatrix.scbmatrix (TC*,int,int,int)} fit
\begin{verbatim}
scbmatrix::scbmatrix (TC* pD, int nMN, int nKL, int nKU);
\end{verbatim}
creates  $n\times n$ \verb"scbmatrix" object where $n$ is passed in
\verb"nMN" parameter. The matrix created has $k_l$=\verb"nKL" 
sub-diagonals and $k_u$=\verb"nKU" super-diagonals.
Unlike others, this constructor \textit{does not allocate  memory}.
It just shares  memory with an array pointed to by \verb"pD".
Please note that this array must contain at least $(k_l + k_u + 1)*n$ elements.
Constructor throws  \GOT{cvmexception}{cvmexception}
in case of non-positive size or negative number of sub- or super-diagonals 
passed.
See also \GOT{scbmat\-rix}{scbmatrix},
\GOT{scbmatrix (const TC*,int,int,int)}{scbmatrix.scbmatrix (const TC*,int,int,int)}.
Example:
\begin{Verbatim}
using namespace cvm;
double a[] = {1., 1., 1., 1., 1., 1., 1., 1., 
              1., 1., 1., 1., 1., 1., 1., 1.};
scbmatrix ml ((std::complex<double>*)a,4,1,0);
scbmatrix mu ((std::complex<double>*)a,4,0,1);
ml(2,1) = std::complex<double>(5.,5.);
std::cout << ml << std::endl << mu << std::endl;
std::cout << a[0] << " " << a[1] << " " << a[2] << " "
          << a[3] << " " << a[4] << " " << a[5] << " " << std::endl;
\end{Verbatim}
prints
\begin{Verbatim}
(1,1) (0,0) (0,0) (0,0)
(5,5) (1,1) (0,0) (0,0)
(0,0) (1,1) (1,1) (0,0)
(0,0) (0,0) (1,1) (1,1)

(5,5) (1,1) (0,0) (0,0)
(0,0) (1,1) (1,1) (0,0)
(0,0) (0,0) (1,1) (1,1)
(0,0) (0,0) (0,0) (1,1)

1 1 5 5 1 1
\end{Verbatim}
\newpage


\subsubsection{scbmatrix (const TC*,int,int,int)}
Constructor%
\pdfdest name {scbmatrix.scbmatrix (const TC*,int,int,int)} fit
\begin{verbatim}
scbmatrix::scbmatrix (const TC* pD, int nMN, int nKL, int nKU);
\end{verbatim}
creates  $n\times n$ \verb"scbmatrix" object where $n$ is passed in
\verb"nMN" parameter. The matrix created has $k_l$=\verb"nKL" 
sub-diagonals and $k_u$=\verb"nKU" super-diagonals.
Then constructor copies $(k_l + k_u + 1)*n$ elements of an array  \verb"pD" to the 
matrix according to \GOT{band storage}{SubSubSectionStorage}.
Constructor throws  \GOT{cvmexception}{cvmexception}
in case of non-positive size or negative number of sub- or super-diagonals 
passed.
See also \GOT{scbmat\-rix}{scbmatrix},
\GOT{scbmatrix (TC*,int,int,int)}{scbmatrix.scbmatrix (TC*,int,int,int)}.
Example:
\begin{Verbatim}
using namespace cvm;
const double a[] = {1., 1., 1., 1., 1., 1., 1., 1., 
                    1., 1., 1., 1., 1., 1., 1., 1.};
scbmatrix ml ((const std::complex<double>*)a,4,1,0);
scbmatrix mu ((const std::complex<double>*)a,4,0,1);
ml(2,1) = std::complex<double>(5.,5.);
std::cout << ml << std::endl << mu << std::endl;
std::cout << a[0] << " " << a[1] << " " << a[2] << " "
          << a[3] << " " << a[4] << " " << a[5] << " " << std::endl;
\end{Verbatim}
prints
\begin{Verbatim}
(1,1) (0,0) (0,0) (0,0)
(5,5) (1,1) (0,0) (0,0)
(0,0) (1,1) (1,1) (0,0)
(0,0) (0,0) (1,1) (1,1)

(1,1) (1,1) (0,0) (0,0)
(0,0) (1,1) (1,1) (0,0)
(0,0) (0,0) (1,1) (1,1)
(0,0) (0,0) (0,0) (1,1)

1 1 1 1 1 1
\end{Verbatim}
\newpage


\subsubsection{scbmatrix (const scbmatrix\&)}
Copy constructor%
\pdfdest name {scbmatrix.scbmatrix (const scbmatrix&)} fit
\begin{verbatim}
scbmatrix::scbmatrix (const scbmatrix& m);
\end{verbatim}
creates  \verb"scbmatrix" object as a copy of \verb"m".
Constructor throws  \GOT{cvmexception}{cvmexception}
in case of memory allocation failure.
See also \GOT{scbmatrix}{scbmatrix}.
Example:
\begin{Verbatim}
using namespace cvm;

double a[] = {1., 2., 3., 4., 5., 6., 7., 8., 9.,
              10., 11., 12., 13., 14., 15., 16.};
scbmatrix m ((std::complex<double>*)a,4,1,0);
scbmatrix mc(m);
m(1,1) = 7.77;
std::cout << m << std::endl << mc;
\end{Verbatim}
prints
\begin{Verbatim}
(7.77,0) (0,0) (0,0) (0,0)
(3,4) (5,6) (0,0) (0,0)
(0,0) (7,8) (9,10) (0,0)
(0,0) (0,0) (11,12) (13,14)

(1,2) (0,0) (0,0) (0,0)
(3,4) (5,6) (0,0) (0,0)
(0,0) (7,8) (9,10) (0,0)
(0,0) (0,0) (11,12) (13,14)
\end{Verbatim}
\newpage




\subsubsection{scbmatrix (const cmatrix\&,int,int)}
Constructor%
\pdfdest name {scbmatrix.scbmatrix (const cmatrix&,int,int)} fit
\begin{verbatim}
scbmatrix::scbmatrix (const cmatrix& m, int nKL, int nKU);
\end{verbatim}
creates  \verb"scbmatrix" object as a copy of ``sliced'' 
matrix \verb"m", i.e. it copies main diagonal, \verb"nKL" 
sub-diagonals and \verb"nKU" super-diagonals of a matrix \verb"m".
It's assumed that $m\times n$ matrix \verb"m" must have equal
sizes, i.e. $m = n$ is satisfied.
Constructor throws  \GOT{cvmexception}{cvmexception}
if this is not true or in case of memory allocation failure.
See also \GOT{scbmatrix}{scbmatrix}.
Example:
\begin{Verbatim}
using namespace cvm;

double a[] = {1., 2., 3., 4., 5., 6., 7., 8., 9.,
              10., 11., 12., 13., 14., 15., 16., 17., 18.};
scmatrix m((std::complex<double>*)a,3);
scbmatrix mb(m,1,0);
std::cout << m << std::endl << mb;
\end{Verbatim}
prints
\begin{Verbatim}
(1,2) (7,8) (13,14)
(3,4) (9,10) (15,16)
(5,6) (11,12) (17,18)

(1,2) (0,0) (0,0)
(3,4) (9,10) (0,0)
(0,0) (11,12) (17,18)
\end{Verbatim}
\newpage




\subsubsection{scbmatrix (const cvector\&)}
Constructor%
\pdfdest name {scbmatrix.scbmatrix (const cvector&)} fit
\begin{verbatim}
explicit scbmatrix::scbmatrix (const cvector& v);
\end{verbatim}
creates  \verb"scbmatrix" object
of size \verb"v.size()" by \verb"v.size()"
and assigns vector \verb"v" to its main diagonal.
Constructor throws  \GOT{cvmexception}{cvmexception}
in case of memory allocation failure.
See also \GOT{scbmatrix}{scbmatrix}, \GOT{cvector}{cvector}.
Example:
\begin{Verbatim}
using namespace cvm;

double a[] = {1., 2., 3., 4., 5., 6.};
cvector v((std::complex<double>*)a,3);
scbmatrix m(v);
std::cout << m;
\end{Verbatim}
prints
\begin{Verbatim}
(1,2) (0,0) (0,0)
(0,0) (3,4) (0,0)
(0,0) (0,0) (5,6)
\end{Verbatim}
\newpage



\subsubsection{scbmatrix (const srbmatrix\&,bool)}
Constructor%
\pdfdest name {scbmatrix.scbmatrix (const srbmatrix&,bool)} fit
\begin{verbatim}
explicit scbmatrix::scbmatrix (const srbmatrix& m, bool bRealPart = true);
\end{verbatim}
creates  \verb"scbmatrix" object
having the same dimension and the same numbers
of sub- and super-diagonals as real matrix \verb"m"
and copies the matrix \verb"m" to its real part if
\verb"bRealPart" is \verb"true" or
to its imaginary part otherwise.
See also \GOT{scbmatrix}{scbmatrix}, \GOT{srbmatrix}{srbmatrix}.
Example:
\begin{Verbatim}
using namespace cvm;

double a[] = {1., 2., 3., 4., 5., 6., 7., 8.};
const srbmatrix m(a,4,1,0);
scbmatrix mr(m), mi(m, false);
std::cout << mr << std::endl << mi;
\end{Verbatim}
prints
\begin{Verbatim}
(1,0) (0,0) (0,0) (0,0)
(2,0) (3,0) (0,0) (0,0)
(0,0) (4,0) (5,0) (0,0)
(0,0) (0,0) (6,0) (7,0)

(0,1) (0,0) (0,0) (0,0)
(0,2) (0,3) (0,0) (0,0)
(0,0) (0,4) (0,5) (0,0)
(0,0) (0,0) (0,6) (0,7)
\end{Verbatim}
\newpage




\subsubsection{scbmatrix (const srbmatrix\&, const srbmatrix\&)}
Constructor%
\pdfdest name {scbmatrix.scbmatrix (const srbmatrix&, const srbmatrix&)} fit
\begin{verbatim}
scbmatrix::scbmatrix (const srbmatrix& mRe, const srbmatrix& mIm);
\end{verbatim}
creates  \verb"scbmatrix" object
of the same size as \verb"mRe" and \verb"mIm" has
(it throws \GOT{cvmexception}{cvmexception}
if \verb"mRe" and
\verb"mIm" have different sizes or different numbers
of sub- or super-diagonals)
and copies matrices \verb"mRe" and \verb"mIm"
to  real and imaginary part of the matrix created respectively.
Constructor throws  \GOT{cvmexception}{cvmexception}
in case of memory allocation failure.
See also \GOT{scbmatrix}{scbmatrix}, \GOT{srbmatrix}{srbmatrix}.
Example:
\begin{Verbatim}
using namespace cvm;

srbmatrix mr(4,1,0), mi(4,1,0);
mr.set(1.);
mi.set(2.);
const scbmatrix m(mr,mi);
std::cout << m;
\end{Verbatim}
prints
\begin{Verbatim}
(1,2) (0,0) (0,0) (0,0)
(1,2) (1,2) (0,0) (0,0)
(0,0) (1,2) (1,2) (0,0)
(0,0) (0,0) (1,2) (1,2)
\end{Verbatim}
\newpage




\subsubsection{operator (,)}
Indexing operators%
\pdfdest name {scbmatrix.operator (,)} fit
\begin{verbatim}
TC& scbmatrix::operator () (int im, int in) throw (cvmexception);
TC scbmatrix::operator () (int im, int in) const throw (cvmexception);
\end{verbatim}
provide access to a particular element of a calling band matrix. The first version
of operator is applicable to non-constant object.
This version returns  \emph{l-value}
in order to make possible write access to an element.
Only elements located on main diagonal or on non-zero
sub- or super-diagonals are l-values. All other values
located outside this area are not writable.
Both operators are \Based.
Operators throw \GOT{cvmexception}{cvmexception}
if some of parameters passed
is outside of \verb"[1,msize()]" range or
in case of attempt to write to  non-writable element%
\footnote{Here I use \verb"type_proxy<T>" class originally
described in \GO{\cite{Meyers}}{biblio}, p.~217.}.
See also \GOT{scbmatrix}{scbmatrix},
\GOT{BandMatrix::lsize()}{BandMatrix.lsize} and
\GOT{BandMatrix::usize()}{BandMatrix.usize}.
Example:
\begin{Verbatim}
using namespace cvm;

std::cout.setf (std::ios::scientific | std::ios::left); 
std::cout.precision (2);
try {
    double a[] = {1., 2., 3., 4., 5., 6., 7., 8., 9., 10., 11., 12.};
    srbmatrix m (a,3,1,0);

    m(2,1) = 7.77;
    std::cout << m << std::endl;
    std::cout << m(3,2) << " " << m(1,3) << std::endl;

    m(1,3) = 7.77;
}
catch (std::exception& e) {
    std::cout << "Exception: " << e.what () << std::endl;
}
\end{Verbatim}
prints
\begin{Verbatim}
1.00e+00 0.00e+00 0.00e+00
7.77e+00 3.00e+00 0.00e+00
0.00e+00 4.00e+00 5.00e+00

4.00e+00 0.00e+00
Exception: Attempt to change a read-only element
\end{Verbatim}
\newpage




\subsubsection{operator ()}
Indexing operator%
\pdfdest name {scbmatrix.operator ()} fit
\begin{verbatim}
const cvector scbmatrix::operator () (int i) const throw (cvmexception);
\end{verbatim}
provides access to \hbox{$i$-th} column of a calling band matrix.
Unlike \GO{scmatrix::operator~()}{scmatrix.operator ()},
this operator creates only  \emph{copy} of a column and therefore 
it returns
\emph{not  l-value}.
Operator is \Based.
It throws \GOT{cvmexception}{cvmexception}
if  parameter \verb"i" is outside of \verb"[1,nsize()]" range.
See also \GOT{scbmatrix}{scbmatrix}.
Example:
\begin{Verbatim}
using namespace cvm;

double a[] = {1., 2., 3., 4., 5., 6., 7., 8., 9., 10., 11., 12.};
scbmatrix m ((std::complex<double>*)a,3,1,0);
std::cout << m << std::endl;
std::cout << m(2);
\end{Verbatim}
prints
\begin{Verbatim}
(1,2) (0,0) (0,0)
(3,4) (5,6) (0,0)
(0,0) (7,8) (9,10)

(0,0) (5,6) (7,8)
\end{Verbatim}
\newpage



\subsubsection{operator []}
Indexing operator%
\pdfdest name {scbmatrix.operator []} fit
\begin{verbatim}
const cvector scbmatrix::operator [] (int i) const throw (cvmexception);
\end{verbatim}
provides access to  \hbox{$i$-th} row of a calling band matrix.
Unlike \GO{scmatrix::operator~[]}{scmatrix.operator []},
this operator creates only  \emph{copy} of a column and therefore 
it returns
\emph{not  l-value}.
Operator is \Based.
It throws \GOT{cvmexception}{cvmexception}
if parameter \verb"i" is outside of \verb"[1,nsize()]" range.
See also \GOT{scbmatrix}{scbmatrix}.
Example:
\begin{Verbatim}
using namespace cvm;

double a[] = {1., 2., 3., 4., 5., 6., 7., 8., 9., 10., 11., 12.};
scbmatrix m ((std::complex<double>*)a,3,1,0);
std::cout << m << std::endl;
std::cout << m[3];
\end{Verbatim}
prints
\begin{Verbatim}
(1,2) (0,0) (0,0)
(3,4) (5,6) (0,0)
(0,0) (7,8) (9,10)

(0,0) (7,8) (9,10)
\end{Verbatim}
\newpage




\subsubsection{real}
Function%
\pdfdest name {scbmatrix.real} fit
\begin{verbatim}
const srbmatrix scbmatrix::real () const;
\end{verbatim}
creates an object of type \verb"const srbmatrix"
as  real part
of a calling band matrix.
Please note that, unlike
\GO{cvector::real}{cvector.real}, this
function creates new object \emph{not sharing}  memory
with  real part of a calling matrix, i.e.
the matrix returned is \emph{not  l-value}.
See also
\GOT{srbmatrix}{srbmatrix},
\GOT{scbmatrix}{scbmatrix}.
Example:
\begin{Verbatim}
using namespace cvm;

double a[] = {1., 2., 3., 4., 5., 6., 7., 8., 9., 10., 11., 12.};
scbmatrix m ((std::complex<double>*)a,3,1,0);
std::cout << m << std::endl;
std::cout << m.real();
\end{Verbatim}
prints
\begin{Verbatim}
(1,2) (0,0) (0,0)
(3,4) (5,6) (0,0)
(0,0) (7,8) (9,10)

1 0 0
3 5 0
0 7 9
\end{Verbatim}
\newpage




\subsubsection{imag}
Function%
\pdfdest name {scbmatrix.imag} fit
\begin{verbatim}
const srbmatrix scbmatrix::imag () const;
\end{verbatim}
creates an object of type \verb"const srbmatrix"
as imaginary part
of a calling band matrix.
Please note that, unlike
\GO{cvector::imag}{cvector.imag}, this
function creates new object \emph{not sharing}  memory
with  imaginary part of a calling matrix, i.e.
the matrix returned is \emph{not  l-value}.
See also
\GOT{srbmatrix}{srbmatrix},
\GOT{scbmatrix}{scbmatrix}.
Example:
\begin{Verbatim}
using namespace cvm;

double a[] = {1., 2., 3., 4., 5., 6., 7., 8., 9., 10., 11., 12.};
scbmatrix m ((std::complex<double>*)a,3,1,0);
std::cout << m << std::endl;
std::cout << m.imag();
\end{Verbatim}
prints
\begin{Verbatim}
(1,2) (0,0) (0,0)
(3,4) (5,6) (0,0)
(0,0) (7,8) (9,10)

2 0 0
4 6 0
0 8 10
\end{Verbatim}
\newpage



\subsubsection{operator = (const scbmatrix\&)}
Operator%
\pdfdest name {scbmatrix.operator = (const scbmatrix&)} fit
\begin{verbatim}
scbmatrix& scbmatrix::operator = (const scbmatrix& m)
throw (cvmexception);
\end{verbatim}
sets every element of a calling band matrix to a value of
appropriate element of  band matrix \verb"m"
and returns a reference to
the matrix changed.
Operator throws  \GOT{cvmexception}{cvmexception}
in case of different matrix sizes or in case of different numbers
of sub- or super-diagonals.
See also \GOT{scbmatrix}{scbmatrix}.
Example:
\begin{Verbatim}
using namespace cvm;

std::cout.setf (std::ios::scientific | std::ios::left); 
std::cout.precision (2);
try {
    double a[] = {1., 2., 3., 4., 5., 6., 7., 8., 9., 10., 11., 12.};
    scbmatrix m1((std::complex<double>*)a,3,1,0);
    scbmatrix m2(3,1,0);

    m2 = m1;
    std::cout << m2;
}
catch (std::exception& e) {
    std::cout << "Exception: " << e.what () << std::endl;
}
\end{Verbatim}
prints
\begin{Verbatim}
(1.00e+00,2.00e+00) (0.00e+00,0.00e+00) (0.00e+00,0.00e+00)
(3.00e+00,4.00e+00) (5.00e+00,6.00e+00) (0.00e+00,0.00e+00)
(0.00e+00,0.00e+00) (7.00e+00,8.00e+00) (9.00e+00,1.00e+01)
\end{Verbatim}
\newpage



\subsubsection{assign (const TC*)}
Function%
\pdfdest name {scbmatrix.assign} fit
\begin{verbatim}
scbmatrix& scbmatrix::assign (const cvector& v) throw (cvmexception);
scbmatrix& scbmatrix::assign (const TC* pD);
\end{verbatim}
sets every element of a calling band matrix to a value of
appropriate element of  vector~\verb'v'
or array pointed to by~\verb"pD"
and returns a reference to
the matrix changed.
In first version it's assumed that vector passed is long
enough to fill calling matrix. Function throws \GOT{cvmexception}{cvmexception} 
otherwise.
In other words this array must contain at least $(k_l + k_u + 1)n$ elements.
See also \GOT{scbmatrix}{scbmatrix}.
Example:
\begin{Verbatim}
using namespace cvm;

const double a[] = {1., 2., 3., 4., 5., 6., 7., 8., 9., 10., 11., 12.};
scbmatrix m(3,0,1);
m.assign((const std::complex<double>*)a);
std::cout << m;
\end{Verbatim}
prints
\begin{Verbatim}
(3,4) (5,6) (0,0)
(0,0) (7,8) (9,10)
(0,0) (0,0) (11,12)
\end{Verbatim}
\newpage



\subsubsection{set (TC)}
Function%
\pdfdest name {scbmatrix.set} fit
\begin{verbatim}
scbmatrix& scbmatrix::set (TC z);
\end{verbatim}
sets every element of a calling band matrix to a value of
parameter \verb"z" and returns a reference to
the matrix changed.
Use \GOT{vanish}{scbmatrix.vanish} to set every element
of a calling matrix to be equal to zero.
See also \GOT{scbmatrix}{scbmatrix}.
Example:
\begin{Verbatim}
using namespace cvm;

scbmatrix m(4,1,0);
m.set(std::complex<double>(1.,2.));
std::cout << m;
\end{Verbatim}
prints
\begin{Verbatim}
(1,2) (0,0) (0,0) (0,0)
(1,2) (1,2) (0,0) (0,0)
(0,0) (1,2) (1,2) (0,0)
(0,0) (0,0) (1,2) (1,2)
\end{Verbatim}
\newpage




\subsubsection{assign\_real}
Function%
\pdfdest name {scbmatrix.assignreal} fit
\begin{verbatim}
scbmatrix& scbmatrix::assign_real (const srbmatrix& mRe)
throw (cvmexception);
\end{verbatim}
sets real part of every element of a calling band matrix to a value of
appropriate element of  band matrix \verb"mRe"
and returns a reference to
the matrix changed.
It throws  \GOT{cvmexception}{cvmexception}
in case of different sizes of the operands.
See also \GOT{scbmatrix}{scbmatrix} and \GOT{srbmatrix}{srbmatrix}.
Example:
\begin{Verbatim}
using namespace cvm;

std::cout.setf (std::ios::scientific | std::ios::left); 
std::cout.precision (2);
srbmatrix m (3,0,1);
scbmatrix mc(3,0,1);
m.randomize (0., 1.);

mc.assign_real(m);
std::cout << mc;
\end{Verbatim}
prints
\begin{Verbatim}
(5.44e-01,0.00e+00) (5.48e-02,0.00e+00) (0.00e+00,0.00e+00)
(0.00e+00,0.00e+00) (3.66e-01,0.00e+00) (3.49e-01,0.00e+00)
(0.00e+00,0.00e+00) (0.00e+00,0.00e+00) (8.00e-01,0.00e+00)
\end{Verbatim}
\newpage




\subsubsection{assign\_imag}
Function%
\pdfdest name {scbmatrix.assignimag} fit
\begin{verbatim}
scbmatrix& scbmatrix::assign_imag (const srbmatrix& mIm)
throw (cvmexception);
\end{verbatim}
sets imaginary part of every element of a calling band matrix to a value of
appropriate element of  bandmatrix \verb"mIm"
and returns a reference to
the matrix changed.
Function throws  \GOT{cvmexception}{cvmexception}
in case of different sizes of the operands.
See also \GOT{scbmatrix}{scbmatrix} and \GOT{srbmatrix}{srbmatrix}.
Example:
\begin{Verbatim}
using namespace cvm;

std::cout.setf (std::ios::scientific | std::ios::left); 
std::cout.precision (2);
srbmatrix m (3,0,1);
scbmatrix mc(3,0,1);
m.randomize (0., 1.);

mc.assign_imag(m);
std::cout << mc;
\end{Verbatim}
prints
\begin{Verbatim}
(0.00e+00,5.53e-01) (0.00e+00,2.16e-01) (0.00e+00,0.00e+00)
(0.00e+00,0.00e+00) (0.00e+00,1.57e-01) (0.00e+00,1.12e-01)
(0.00e+00,0.00e+00) (0.00e+00,0.00e+00) (0.00e+00,7.03e-01)
\end{Verbatim}
\newpage




\subsubsection{set\_real}
Function%
\pdfdest name {scbmatrix.setreal} fit
\begin{verbatim}
scbmatrix& scbmatrix::set_real (TR d);
\end{verbatim}
sets real part of every element of a calling band matrix to a value of
parameter \verb"d"
and returns a reference to the matrix changed.
See also \GOT{scbmatrix}{scbmatrix}.
Example:
\begin{Verbatim}
using namespace cvm;

scbmatrix m(4,0,1);
m.set_real(1.);
std::cout << m;
\end{Verbatim}
prints
\begin{Verbatim}
(1,0) (1,0) (0,0) (0,0)
(0,0) (1,0) (1,0) (0,0)
(0,0) (0,0) (1,0) (1,0)
(0,0) (0,0) (0,0) (1,0)
\end{Verbatim}
\newpage





\subsubsection{set\_imag}
Function%
\pdfdest name {scbmatrix.setimag} fit
\begin{verbatim}
scbmatrix& scbmatrix::set_imag (TR d);
\end{verbatim}
sets imaginary part of every element of a calling band matrix to a value of
parameter \verb"d"
and returns a reference to the matrix changed.
See also \GOT{scbmatrix}{scbmatrix}.
Example:
\begin{Verbatim}
using namespace cvm;

scbmatrix m(4,0,1);
m.set_imag(1.);
std::cout << m;
\end{Verbatim}
prints
\begin{Verbatim}
(0,1) (0,1) (0,0) (0,0)
(0,0) (0,1) (0,1) (0,0)
(0,0) (0,0) (0,1) (0,1)
(0,0) (0,0) (0,0) (0,1)
\end{Verbatim}
\newpage



\subsubsection{resize}
Function%
\pdfdest name {scbmatrix.resize} fit
\begin{verbatim}
scbmatrix& scbmatrix::resize (int nNewMN) throw (cvmexception);
\end{verbatim}
changes  size of a calling band matrix to \verb"nNewMN" by \verb"nNewMN"
and returns a reference to
the matrix changed. In case of increasing of its size, calling matrix
is filled up with zeroes. This function doesn't change
 number of sub- ore super-diagonals. Like any band matrix 
class member function, this function doesn't change 
\GO{non-referred elements}{SubSubSectionStorage}. 
See number \verb"(11,12)" appearing after
resize in example below.
Function throws  \GOT{cvmexception}{cvmexception}
in case of non-positive size passed or memory allocation failure.
See also \GOT{scbmatrix.resize\_lu}{scbmatrix.resizelu},
\GOT{scbmatrix}{scbmatrix}.
Example:
\begin{Verbatim}
using namespace cvm;

try {
    double a[] = {1., 2., 3., 4., 5., 6., 7., 8., 9., 10., 11., 12.};
    scbmatrix m((std::complex<double>*)a,3,1,0);
    std::cout << m << std::endl;
    m.resize (4);
    std::cout << m;
}
catch (std::exception& e) {
    std::cout << "Exception: " << e.what () << std::endl;
}
\end{Verbatim}
prints
\begin{Verbatim}
(1,2) (0,0) (0,0)
(3,4) (5,6) (0,0)
(0,0) (7,8) (9,10)

(1,2) (0,0) (0,0) (0,0)
(3,4) (5,6) (0,0) (0,0)
(0,0) (7,8) (9,10) (0,0)
(0,0) (0,0) (11,12) (0,0)
\end{Verbatim}
\newpage





\subsubsection{resize\_lu}
Function%
\pdfdest name {scbmatrix.resizelu} fit
\begin{verbatim}
scbmatrix& scbmatrix::resize_lu (int nNewKL, int nNewKU) 
throw (cvmexception);
\end{verbatim}
changes  number of sub- and super-diagonals 
of a calling band matrix to \verb"nNewKL" by \verb"nNewKU" respectively
and returns a reference to
the matrix changed. In case of increasing of the numbers, calling matrix
is filled up with zeroes. 
Function throws  \GOT{cvmexception}{cvmexception}
in case of negative number passed or memory allocation failure.
See also \GOT{scbmatrix::resize}{scbmatrix.resize},
\GOT{scbmatrix}{scbmatrix}.
Example:
\begin{Verbatim}
using namespace cvm;

try {
    double a[] = {1., 2., 3., 4., 5., 6., 7., 8., 9., 10., 11., 12.};
    scbmatrix m((std::complex<double>*)a,3,1,0);
    std::cout << m << std::endl;
    m.resize_lu (0,1);
    m.diag(1).set(std::complex<double>(9.,9.));
    std::cout << m;
}
catch (std::exception& e) {
    std::cout << "Exception: " << e.what () << std::endl;
}
\end{Verbatim}
prints
\begin{Verbatim}
(1,2) (0,0) (0,0)
(3,4) (5,6) (0,0)
(0,0) (7,8) (9,10)

(1,2) (9,9) (0,0)
(0,0) (5,6) (9,9)
(0,0) (0,0) (9,10)
\end{Verbatim}
\newpage





\subsubsection{operator ==}
Operator%
\pdfdest name {scbmatrix.operator ==} fit
\begin{verbatim}
bool scbmatrix::operator == (const scbmatrix& m) const;
\end{verbatim}
compares  calling band matrix with  band matrix \verb"m"
and returns \verb"true" if they have the same sizes, the same
numbers of sub- and super-diagonals 
and their appropriate elements differ by not more than the
\GO{smallest normalized positive number}{Utilities.cvmMachMin}.
Returns \verb"false" otherwise.
See also \GOT{scbmatrix}{scbmatrix}.
Example:
\begin{Verbatim}
using namespace cvm;

double a[] = {1., 2., 3., 4., 5., 6., 7., 8.};
scbmatrix m1((std::complex<double>*)a,2,1,0);
scbmatrix m2(2,1,0);
std::cout << m1 << std::endl;

m2(1,1) = std::complex<double>(1.,2.); 
m2(2,1) = std::complex<double>(3.,4.); 
m2(2,2) = std::complex<double>(5.,6.);

std::cout << (m1 == m2) << std::endl;
\end{Verbatim}
prints
\begin{Verbatim}
(1,2) (0,0)
(3,4) (5,6)

1
\end{Verbatim}
\newpage




\subsubsection{operator !=}
Operator%
\pdfdest name {scbmatrix.operator !=} fit
\begin{verbatim}
bool scbmatrix::operator != (const scbmatrix& m) const;
\end{verbatim}
compares  calling band matrix with  band matrix \verb"m"
and returns \verb"true" if they have different sizes,
different numbers of sub- or super-diagonals
or at least one of their appropriate elements
differs by more than the
\GO{smallest normalized positive number}{Utilities.cvmMachMin}.
Returns \verb"false" otherwise.
See also \GOT{scbmatrix}{scbmatrix}.
Example:
\begin{Verbatim}
using namespace cvm;

double a[] = {1., 2., 3., 4., 5., 6., 7., 8.};
scbmatrix m1((std::complex<double>*)a,2,1,0);
scbmatrix m2(2,1,0);
std::cout << m1 << std::endl;

m2(1,1) = std::complex<double>(1.,2.); 
m2(2,1) = std::complex<double>(3.,4.); 
m2(2,2) = std::complex<double>(5.,6.00001);

std::cout << (m1 != m2) << std::endl;
\end{Verbatim}
prints
\begin{Verbatim}
(1,2) (0,0)
(3,4) (5,6)

1
\end{Verbatim}
\newpage




\subsubsection{operator <{}<}
Operator%
\pdfdest name {scbmatrix.operator <<} fit
\begin{verbatim}
scbmatrix& scbmatrix::operator << (const scbmatrix& m)
throw (cvmexception);
\end{verbatim}
destroys  calling band matrix, creates  new one as a copy of \verb"m"
and returns a reference to
the matrix changed.
Operator throws  \GOT{cvmexception}{cvmexception}
in case of memory allocation failure.
See also \GOT{scbmatrix}{scbmatrix}.
Example:
\begin{Verbatim}
using namespace cvm;

try {
    scbmatrix m(3,1,0);
    scbmatrix mc(1);
    m(2,1) = std::complex<double>(1.,2.);
    m(2,2) = std::complex<double>(3.,4.);
    std::cout << m << std::endl << mc << std::endl;

    mc << m;
    std::cout << mc;
}
catch (std::exception& e) {
    std::cout << "Exception: " << e.what () << std::endl;
}
\end{Verbatim}
prints
\begin{Verbatim}
(0,0) (0,0) (0,0)
(1,2) (3,4) (0,0)
(0,0) (0,0) (0,0)

(0,0)

(0,0) (0,0) (0,0)
(1,2) (3,4) (0,0)
(0,0) (0,0) (0,0)
\end{Verbatim}
\newpage



\subsubsection{operator +}
Operator%
\pdfdest name {scbmatrix.operator +} fit
\begin{verbatim}
scbmatrix scbmatrix::operator + (const scbmatrix& m) const
throw (cvmexception);
\end{verbatim}
creates an object of type \verb"scbmatrix" as a sum of
a calling band matrix and  band matrix \verb"m".
It throws  \GOT{cvmexception}{cvmexception}
in case of different sizes or different numbers of sub- or super-diagonals
of the operands.
See also \GOT{scbmatrix::sum}{scbmatrix.sum}, \GOT{scbmatrix}{scbmatrix}.
Example:
\begin{Verbatim}
using namespace cvm;

try {
    double a[] = {1., 2., 3., 4., 5., 6., 7., 8.,
                  9., 10., 11., 12.};
    double b[] = {10., 20., 30., 40., 50., 60., 
                  70., 80., 90., 100., 110., 120.};
    scbmatrix m1((std::complex<double>*)a,3,0,1);
    scbmatrix m2((std::complex<double>*)b,3,0,1);

    std::cout << m1 << std::endl << m2 << std::endl;
    std::cout << m1 + m2 << std::endl << m1 + m1;
}
catch (std::exception& e) {
    std::cout << "Exception: " << e.what () << std::endl;
}
\end{Verbatim}
prints
\begin{Verbatim}
(3,4) (5,6) (0,0)
(0,0) (7,8) (9,10)
(0,0) (0,0) (11,12)

(30,40) (50,60) (0,0)
(0,0) (70,80) (90,100)
(0,0) (0,0) (110,120)

(33,44) (55,66) (0,0)
(0,0) (77,88) (99,110)
(0,0) (0,0) (121,132)

(6,8) (10,12) (0,0)
(0,0) (14,16) (18,20)
(0,0) (0,0) (22,24)
\end{Verbatim}
\newpage



\subsubsection{operator -}
Operator%
\pdfdest name {scbmatrix.operator -} fit
\begin{verbatim}
scbmatrix scbmatrix::operator - (const scbmatrix& m) const
throw (cvmexception);
\end{verbatim}
creates an object of type \verb"scbmatrix" as a difference of
a calling band matrix and  band matrix \verb"m".
It throws  \GOT{cvmexception}{cvmexception}
in case of different sizes or different numbers of sub- or super-diagonals
of the operands.
See also \GOT{scbmatrix::diff}{scbmatrix.diff}, \GOT{scbmatrix}{scbmatrix}.
Example:
\begin{Verbatim}
using namespace cvm;

try {
    double a[] = {1., 2., 3., 4., 5., 6., 7., 8.,
                  9., 10., 11., 12.};
    double b[] = {10., 20., 30., 40., 50., 60., 
                  70., 80., 90., 100., 110., 120.};
    scbmatrix m1((std::complex<double>*)a,3,0,1);
    scbmatrix m2((std::complex<double>*)b,3,0,1);

    std::cout << m1 << std::endl << m2 << std::endl;
    std::cout << m1 - m2 << std::endl << m1 - m1;
}
catch (std::exception& e) {
    std::cout << "Exception: " << e.what () << std::endl;
}
\end{Verbatim}
prints
\begin{Verbatim}
(3,4) (5,6) (0,0)
(0,0) (7,8) (9,10)
(0,0) (0,0) (11,12)

(30,40) (50,60) (0,0)
(0,0) (70,80) (90,100)
(0,0) (0,0) (110,120)

(-27,-36) (-45,-54) (0,0)
(0,0) (-63,-72) (-81,-90)
(0,0) (0,0) (-99,-108)

(0,0) (0,0) (0,0)
(0,0) (0,0) (0,0)
(0,0) (0,0) (0,0)
\end{Verbatim}
\newpage





\subsubsection{sum}
Function%
\pdfdest name {scbmatrix.sum} fit
\begin{verbatim}
scbmatrix& scbmatrix::sum (const scbmatrix& m1, const scbmatrix& m2)
throw (cvmexception);
\end{verbatim}
assigns the result of addition of
band matrices \verb"m1" and \verb"m2"  to a calling band matrix
and returns a reference to
the matrix changed.
It throws  \GOT{cvmexception}{cvmexception}
in case of different sizes or different numbers of sub- or super-diagonals
of the operands.
See also \GOT{scbmatrix::operator +~}{scbmatrix.operator +},
\GOT{scbmatrix}{scbmatrix}.
Example:
\begin{Verbatim}
using namespace cvm;

double a[] = {1., 2., 3., 4., 5., 6., 7., 8.,
              9., 10., 11., 12.};
const scbmatrix m1((std::complex<double>*)a,3,1,0);
scbmatrix m2(3,1,0);
scbmatrix m(3,1,0);
m2.set(std::complex<double>(1.,1.));
std::cout << m1 << std::endl << m2 << std::endl;
std::cout << m.sum(m1, m2) << std::endl;
std::cout << m.sum(m, m2);
\end{Verbatim}
prints
\begin{Verbatim}
(1,2) (0,0) (0,0)
(3,4) (5,6) (0,0)
(0,0) (7,8) (9,10)

(1,1) (0,0) (0,0)
(1,1) (1,1) (0,0)
(0,0) (1,1) (1,1)

(2,3) (0,0) (0,0)
(4,5) (6,7) (0,0)
(0,0) (8,9) (10,11)

(3,4) (0,0) (0,0)
(5,6) (7,8) (0,0)
(0,0) (9,10) (11,12)
\end{Verbatim}
\newpage



\subsubsection{diff}
Function%
\pdfdest name {scbmatrix.diff} fit
\begin{verbatim}
scbmatrix& scbmatrix::diff (const scbmatrix& m1, const scbmatrix& m2)
throw (cvmexception);
\end{verbatim}
assigns the result of subtraction of
band matrices \verb"m1" and \verb"m2" to a calling band matrix
and returns a reference to
the matrix changed.
It throws  \GOT{cvmexception}{cvmexception}
in case of different sizes or different numbers of sub- or super-diagonals
of the operands.
See also \GOT{scbmatrix::operator -~}{scbmatrix.operator -},
\GOT{scbmatrix}{scbmatrix}.
Example:
\begin{Verbatim}
using namespace cvm;

double a[] = {1., 2., 3., 4., 5., 6., 7., 8.,
              9., 10., 11., 12.};
const scbmatrix m1((std::complex<double>*)a,3,1,0);
scbmatrix m2(3,1,0);
scbmatrix m(3,1,0);
m2.set(std::complex<double>(1.,1.));
std::cout << m1 << std::endl << m2 << std::endl;
std::cout << m.diff(m1, m2) << std::endl;
std::cout << m.diff(m, m2);
\end{Verbatim}
prints
\begin{Verbatim}
(1,2) (0,0) (0,0)
(3,4) (5,6) (0,0)
(0,0) (7,8) (9,10)

(1,1) (0,0) (0,0)
(1,1) (1,1) (0,0)
(0,0) (1,1) (1,1)

(0,1) (0,0) (0,0)
(2,3) (4,5) (0,0)
(0,0) (6,7) (8,9)

(-1,0) (0,0) (0,0)
(1,2) (3,4) (0,0)
(0,0) (5,6) (7,8)
\end{Verbatim}
\newpage




\subsubsection{operator +=}
Operator%
\pdfdest name {scbmatrix.operator +=} fit
\begin{verbatim}
scbmatrix& scbmatrix::operator += (const scbmatrix& m) 
throw (cvmexception);
\end{verbatim}
adds  band matrix \verb"m" to a calling band matrix 
and returns a reference to
the matrix changed.
Operator throws  \GOT{cvmexception}{cvmexception}
in case of different sizes or different numbers of sub- or super-diagonals
of the operands.
See also \GOT{scbmatrix::ope\-ra\-tor +~}{scbmatrix.operator +},
\GOT{scbmatrix::sum}{scbmatrix.sum},
\GOT{scbmatrix}{scbmatrix}.
Example:
\begin{Verbatim}
using namespace cvm;

try {
    scbmatrix m1(4,0,1);
    scbmatrix m2(4,0,1);
    m1.set(std::complex<double>(1.,2.));
    m2.set(std::complex<double>(3.,4.));

    m1 += m2;
    std::cout << m1 << std::endl;

    // well, you can do this too, but temporary object would be created
    m2 += m2; 
    std::cout << m2;
}
catch (std::exception& e) {
    std::cout << "Exception: " << e.what () << std::endl;
}
\end{Verbatim}
prints
\begin{Verbatim}
(4,6) (4,6) (0,0) (0,0)
(0,0) (4,6) (4,6) (0,0)
(0,0) (0,0) (4,6) (4,6)
(0,0) (0,0) (0,0) (4,6)

(6,8) (6,8) (0,0) (0,0)
(0,0) (6,8) (6,8) (0,0)
(0,0) (0,0) (6,8) (6,8)
(0,0) (0,0) (0,0) (6,8)
\end{Verbatim}
\newpage




\subsubsection{operator -=}
Operator%
\pdfdest name {scbmatrix.operator -=} fit
\begin{verbatim}
scbmatrix& scbmatrix::operator -= (const scbmatrix& m) 
throw (cvmexception);
\end{verbatim}
subtracts  band matrix \verb"m" from  calling band matrix
and returns a reference to
the matrix changed.
It throws  \GOT{cvmexception}{cvmexception}
in case of different sizes or different numbers of sub- or super-diagonals
of the operands.
See also \GOT{scbmatrix::ope\-ra\-tor -~}{scbmatrix.operator -},
\GOT{scbmatrix::diff}{scbmatrix.diff},
\GOT{scbmatrix}{scbmatrix}.
Example:
\begin{Verbatim}
using namespace cvm;

try {
    scbmatrix m1(4,0,1);
    scbmatrix m2(4,0,1);
    m1.set(std::complex<double>(1.,2.));
    m2.set(std::complex<double>(3.,4.));

    m1 -= m2;
    std::cout << m1 << std::endl;

    // well, you can do this too, but temporary object would be created
    m2 -= m2; 
    std::cout << m2;
}
catch (std::exception& e) {
    std::cout << "Exception: " << e.what () << std::endl;
}
\end{Verbatim}
prints
\begin{Verbatim}
(-2,-2) (-2,-2) (0,0) (0,0)
(0,0) (-2,-2) (-2,-2) (0,0)
(0,0) (0,0) (-2,-2) (-2,-2)
(0,0) (0,0) (0,0) (-2,-2)

(0,0) (0,0) (0,0) (0,0)
(0,0) (0,0) (0,0) (0,0)
(0,0) (0,0) (0,0) (0,0)
(0,0) (0,0) (0,0) (0,0)
\end{Verbatim}
\newpage



\subsubsection{operator - ()}
Operator%
\pdfdest name {scbmatrix.operator - ()} fit
\begin{verbatim}
scbmatrix scbmatrix::operator - () const throw (cvmexception);
\end{verbatim}
creates an object of type \verb"scbmatrix" as
a calling band matrix multiplied by $-1$.
See also \GOT{scbmatrix}{scbmatrix}.
Example:
\begin{Verbatim}
using namespace cvm;

std::cout.setf (std::ios::scientific | 
                std::ios::left | 
                std::ios::showpos); 
std::cout.precision (2);

double a[] = {1., 2., 3., 4., 5., 6., 7., 8.,
              9., 10., 11., 12.};
scbmatrix m((std::complex<double>*)a,3,1,0);

std::cout << -m;
\end{Verbatim}
prints
\begin{Verbatim}
(-1.00e+000,-2.00e+000) (+0.00e+000,+0.00e+000) (+0.00e+000,+0.00e+000)
(-3.00e+000,-4.00e+000) (-5.00e+000,-6.00e+000) (+0.00e+000,+0.00e+000)
(+0.00e+000,+0.00e+000) (-7.00e+000,-8.00e+000) (-9.00e+000,-1.00e+001)
\end{Verbatim}
\newpage



\subsubsection{operator ++}
Operator%
\pdfdest name {scbmatrix.operator ++} fit
\begin{verbatim}
scbmatrix& scbmatrix::operator ++ ();
scbmatrix& scbmatrix::operator ++ (int);
\end{verbatim}
adds identity matrix to a calling band matrix
and returns a reference to
the matrix changed.
See also \GOT{scbmatrix}{scbmatrix}.
Example:
\begin{Verbatim}
using namespace cvm;

scbmatrix m(4,1,0);
m.set(std::complex<double>(1.,1.));

m++;
std::cout << m << std::endl;
std::cout << ++m;
\end{Verbatim}
prints
\begin{Verbatim}
(2,1) (0,0) (0,0) (0,0)
(1,1) (2,1) (0,0) (0,0)
(0,0) (1,1) (2,1) (0,0)
(0,0) (0,0) (1,1) (2,1)

(3,1) (0,0) (0,0) (0,0)
(1,1) (3,1) (0,0) (0,0)
(0,0) (1,1) (3,1) (0,0)
(0,0) (0,0) (1,1) (3,1)
\end{Verbatim}
\newpage



\subsubsection{operator -{}-}
Operator%
\pdfdest name {scbmatrix.operator --} fit
\begin{verbatim}
scbmatrix& scbmatrix::operator -- ();
scbmatrix& scbmatrix::operator -- (int);
\end{verbatim}
subtracts identity matrix from  calling band matrix
and returns a reference to
the matrix changed.
See also \GOT{scbmatrix}{scbmatrix}.
Example:
\begin{Verbatim}
using namespace cvm;

scbmatrix m(4,1,0);
m.set(std::complex<double>(1.,1.));

m--;
std::cout << m << std::endl;
std::cout << --m;
\end{Verbatim}
prints
\begin{Verbatim}
(0,1) (0,0) (0,0) (0,0)
(1,1) (0,1) (0,0) (0,0)
(0,0) (1,1) (0,1) (0,0)
(0,0) (0,0) (1,1) (0,1)

(-1,1) (0,0) (0,0) (0,0)
(1,1) (-1,1) (0,0) (0,0)
(0,0) (1,1) (-1,1) (0,0)
(0,0) (0,0) (1,1) (-1,1)
\end{Verbatim}
\newpage




\subsubsection{operator * (TR)}
Operator%
\pdfdest name {scbmatrix.operator * (TR)} fit
\begin{verbatim}
scbmatrix scbmatrix::operator * (TR d) const;
\end{verbatim}
creates an object of type \verb"scbmatrix" as a product of
a calling band matrix and  real number~\verb"d".
See also \GOT{scbmatrix::ope\-ra\-tor *=~}{scbmatrix.operator *= (TR)},
\GOT{scbmatrix}{scbmatrix}.
Example:
\begin{Verbatim}
using namespace cvm;

double a[] = {1., 2., 3., 4., 5., 6., 7., 8., 9.,
              10., 11., 12.};
scbmatrix m((std::complex<double>*)a,3,0,1);
std::cout << m * 5.;
\end{Verbatim}
prints
\begin{Verbatim}
(15,20) (25,30) (0,0)
(0,0) (35,40) (45,50)
(0,0) (0,0) (55,60)
\end{Verbatim}
\newpage



\subsubsection{operator / (TR)}
Operator%
\pdfdest name {scbmatrix.operator / (TR)} fit
\begin{verbatim}
scbmatrix scbmatrix::operator / (TR d) const throw (cvmexception);
\end{verbatim}
creates an object of type \verb"scbmatrix" as a quotient of
a calling band matrix and  real number~\verb"d". It throws
 \GOT{cvmexception}{cvmexception}
if \verb"d" has  absolute value equal or less than the
\GO{smallest normalized positive number}{Utilities.cvmMachMin}.
See also \GOT{scbmatrix::operator /=~}{scbmatrix.operator /= (TR)},
\GOT{scbmatrix}{scbmatrix}.
Example:
\begin{Verbatim}
using namespace cvm;

double a[] = {1., 2., 3., 4., 5., 6., 7., 8., 9.,
              10., 11., 12.};
scbmatrix m((std::complex<double>*)a,3,0,1);
std::cout << m / 2.;
\end{Verbatim}
prints
\begin{Verbatim}
(1.5,2) (2.5,3) (0,0)
(0,0) (3.5,4) (4.5,5)
(0,0) (0,0) (5.5,6)
\end{Verbatim}
\newpage



\subsubsection{operator * (TC)}
Operator%
\pdfdest name {scbmatrix.operator * (TC)} fit
\begin{verbatim}
scbmatrix scbmatrix::operator * (TC z) const;
\end{verbatim}
creates an object of type \verb"scbmatrix" as a product of
a calling band matrix and  complex number~\verb"z".
See also \GOT{scbmatrix::operator *=~}
{scbmatrix.operator *= (TC)},
\GOT{scbmatrix}{scbmatrix}.
Example:
\begin{Verbatim}
using namespace cvm;

double a[] = {1., 2., 3., 4., 5., 6., 7., 8., 9.,
              10., 11., 12.};
scbmatrix m((std::complex<double>*)a,3,0,1);
std::cout << m * std::complex<double>(1.,1.);
\end{Verbatim}
prints
\begin{Verbatim}
(-1,7) (-1,11) (0,0)
(0,0) (-1,15) (-1,19)
(0,0) (0,0) (-1,23)
\end{Verbatim}
\newpage



\subsubsection{operator / (TC)}
Operator%
\pdfdest name {scbmatrix.operator / (TC)} fit
\begin{verbatim}
scbmatrix scbmatrix::operator / (TC z) const throw (cvmexception);
\end{verbatim}
creates an object of type \verb"scbmatrix" as a quotient of
a calling band matrix and  complex number~\verb"z". 
It throws
 \GOT{cvmexception}{cvmexception}
if \verb"z" has  absolute value equal or less than the
\GO{smallest normalized positive number}{Utilities.cvmMachMin}.
See also \GOT{scbmatrix::operator /=~}
{scbmatrix.operator /= (TC)},
\GOT{scbmatrix}{scbmatrix}.
Example:
\begin{Verbatim}
using namespace cvm;

double a[] = {1., 2., 3., 4., 5., 6., 7., 8., 9.,
              10., 11., 12.};
scbmatrix m((std::complex<double>*)a,3,0,1);
std::cout << m / std::complex<double>(1.,1.);
\end{Verbatim}
prints
\begin{Verbatim}
(3.5,0.5) (5.5,0.5) (0,0)
(0,0) (7.5,0.5) (9.5,0.5)
(0,0) (0,0) (11.5,0.5)
\end{Verbatim}
\newpage



\subsubsection{operator *= (TR)}
Operator%
\pdfdest name {scbmatrix.operator *= (TR)} fit
\begin{verbatim}
scbmatrix& scbmatrix::operator *= (TR d);
\end{verbatim}
multiplies  calling band matrix by  real number~\verb"d"
and returns a reference to
the matrix changed.
See also \GOT{scbmatrix::operator *~}{scbmatrix.operator * (TR)},
\GOT{scbmatrix}{scbmatrix}.
Example:
\begin{Verbatim}
using namespace cvm;

double a[] = {1., 2., 3., 4., 5., 6., 7., 8., 9.,
              10., 11., 12.};
scbmatrix m((std::complex<double>*)a,3,0,1);
m *= 5.;
std::cout << m;
\end{Verbatim}
prints
\begin{Verbatim}
(15,20) (25,30) (0,0)
(0,0) (35,40) (45,50)
(0,0) (0,0) (55,60)
\end{Verbatim}
\newpage



\subsubsection{operator /= (TR)}
Operator%
\pdfdest name {scbmatrix.operator /= (TR)} fit
\begin{verbatim}
scbmatrix& scbmatrix::operator /= (TR d) throw (cvmexception);
\end{verbatim}
divides  calling band matrix by  real number~\verb"d"
and returns a reference to
the matrix changed.
It throws  \GOT{cvmexception}{cvmexception}
if \verb"d" has  absolute value equal or less
than the 
\GO{smallest normalized positive number}{Utilities.cvmMachMin}.
See also \GOT{scbmatrix::operator /~}{scbmatrix.operator / (TR)},
\GOT{scbmatrix}{scbmatrix}.
Example:
\begin{Verbatim}
using namespace cvm;

double a[] = {1., 2., 3., 4., 5., 6., 7., 8., 9.,
              10., 11., 12.};
scbmatrix m((std::complex<double>*)a,3,0,1);
m /= 2.;
std::cout << m;
\end{Verbatim}
prints
\begin{Verbatim}
(1.5,2) (2.5,3) (0,0)
(0,0) (3.5,4) (4.5,5)
(0,0) (0,0) (5.5,6)
\end{Verbatim}
\newpage




\subsubsection{operator *= (TC)}
Operator%
\pdfdest name {scbmatrix.operator *= (TC)} fit
\begin{verbatim}
scbmatrix& scbmatrix::operator *= (TC z);
\end{verbatim}
multiplies  calling band matrix by  complex number~\verb"z"
and returns a reference to
the matrix changed.
See also \GOT{scbmatrix::operator *~}{scbmatrix.operator * (TC)},
\GOT{scbmatrix}{scbmatrix}.
Example:
\begin{Verbatim}
using namespace cvm;

double a[] = {1., 2., 3., 4., 5., 6., 7., 8., 9.,
              10., 11., 12.};
scbmatrix m((std::complex<double>*)a,3,0,1);
m *= std::complex<double>(1.,1.);
std::cout << m;
\end{Verbatim}
prints
\begin{Verbatim}
(-1,7) (-1,11) (0,0)
(0,0) (-1,15) (-1,19)
(0,0) (0,0) (-1,23)
\end{Verbatim}
\newpage



\subsubsection{operator /= (TC)}
Operator%
\pdfdest name {scbmatrix.operator /= (TC)} fit
\begin{verbatim}
scbmatrix& scbmatrix::operator /= (TC z) throw (cvmexception);
\end{verbatim}
divides  calling band matrix by  complex number~\verb"z"
and returns a reference to
the matrix changed.
It throws  \GOT{cvmexception}{cvmexception}
if \verb"z" has  absolute value equal or less
than the 
\GO{smallest normalized positive number}{Utilities.cvmMachMin}.
See also \GOT{scbmatrix::operator /~}{scbmatrix.operator / (TC)},
\GOT{scbmatrix}{scbmatrix}.
Example:
\begin{Verbatim}
using namespace cvm;

double a[] = {1., 2., 3., 4., 5., 6., 7., 8., 9.,
              10., 11., 12.};
scbmatrix m((std::complex<double>*)a,3,0,1);
m /= std::complex<double>(1.,1.);
std::cout << m;
\end{Verbatim}
prints
\begin{Verbatim}
(3.5,0.5) (5.5,0.5) (0,0)
(0,0) (7.5,0.5) (9.5,0.5)
(0,0) (0,0) (11.5,0.5)
\end{Verbatim}
\newpage



\subsubsection{normalize}
Function%
\pdfdest name {scbmatrix.normalize} fit
\begin{verbatim}
scbmatrix& scbmatrix::normalize ();
\end{verbatim}
normalizes  calling band matrix so its \GO{Euclidean norm}{Array.norm}
becomes equal to $1$ if it was greater than the 
\GO{smallest normalized positive number}{Utilities.cvmMachMin}
before the call (otherwise function does nothing).
See also \GOT{scbmatrix}{scbmatrix}.
Example:
\begin{Verbatim}
using namespace cvm;

std::cout.setf (std::ios::scientific | std::ios::left); 
std::cout.precision (2);
double a[] = {1., 2., 3., 4., 5., 6., 7., 8., 9.,
              10., 11., 12.};
scbmatrix m((std::complex<double>*)a,3,0,1);

m.normalize();
std::cout << m << m.norm() << std::endl;
\end{Verbatim}
prints
\begin{Verbatim}
(1.18e-001,1.57e-001) (1.97e-001,2.36e-001) (0.00e+000,0.00e+000)
(0.00e+000,0.00e+000) (2.76e-001,3.15e-001) (3.54e-001,3.94e-001)
(0.00e+000,0.00e+000) (0.00e+000,0.00e+000) (4.33e-001,4.72e-001)
1.00e+000
\end{Verbatim}
\newpage



\subsubsection{conjugation}
Operator and functions%
\pdfdest name {scbmatrix.conj} fit
\begin{verbatim}
scbmatrix scbmatrix::operator ~ () const;
scbmatrix& scbmatrix::conj (const scbmatrix& m);
scbmatrix& scbmatrix::conj ();
\end{verbatim}
implement complex band matrix conjugation.
First operator creates \verb"scbmatrix" object as
 conjugated calling band matrix
(it throws  
\GOT{cvmexcep\-tion}{cvmexception}
in case of memory allocation failure). 
Second function sets  calling matrix to be equal to  matrix
\verb"m" conjugated
(it throws  
\GOT{cvmexception}{cvmexception}
in case of not appropriate sizes or numbers of sub- or
super-diagonals of the operands), 
third one makes it to be equal to
conjugated itself (it also throws  
\GOT{cvmexception}{cvmexception}
in case of memory allocation failure). 
See also \GOT{scbmatrix}{scbmatrix}.
Example:
\begin{Verbatim}
using namespace cvm;

double a[] = {1., 2., 3., 4., 5., 6., 7., 8., 9.,
              10., 11., 12., 13., 14., 15., 16.};
scbmatrix m((std::complex<double>*)a,4,1,0);
scbmatrix mc(4,0,1);
std::cout << m << std::endl << ~m << std::endl ;
mc.conj(m);
std::cout << mc << std::endl;
mc.conj();
std::cout << mc;
\end{Verbatim}
prints
\begin{Verbatim}
(1,2) (0,0) (0,0) (0,0)
(3,4) (5,6) (0,0) (0,0)
(0,0) (7,8) (9,10) (0,0)
(0,0) (0,0) (11,12) (13,14)

(1,-2) (3,-4) (0,0) (0,0)
(0,0) (5,-6) (7,-8) (0,0)
(0,0) (0,0) (9,-10) (11,-12)
(0,0) (0,0) (0,0) (13,-14)

(1,-2) (3,-4) (0,0) (0,0)
(0,0) (5,-6) (7,-8) (0,0)
(0,0) (0,0) (9,-10) (11,-12)
(0,0) (0,0) (0,0) (13,-14)

(1,2) (0,0) (0,0) (0,0)
(3,4) (5,6) (0,0) (0,0)
(0,0) (7,8) (9,10) (0,0)
(0,0) (0,0) (11,12) (13,14)
\end{Verbatim}
\newpage




\subsubsection{transposition}
Operator and functions%
\pdfdest name {scbmatrix.transpose} fit
\begin{verbatim}
scbmatrix scbmatrix::operator ! () const throw (cvmexception);
scbmatrix& scbmatrix::transpose (const scbmatrix& m) throw (cvmexception);
scbmatrix& scbmatrix::transpose () throw (cvmexception);
\end{verbatim}
implement complex band matrix transposition (\emph{not} conjugation).
First operator creates an object of type \verb"scbmatrix" as
 transposed calling matrix
(it throws  
\GOT{cvmexception}{cvmexception}
in case of memory allocation failure). 
Second function sets  calling matrix to be equal to  matrix
\verb"m" transposed
(it throws  
\GOT{cvmexception}{cvmexception}
in case of not appropriate sizes of the operands), 
third one makes it to be equal to
transposed itself (it also throws  
\GOT{cvmexception}{cvmexception}
in case of memory allocation failure). 
See also \GOT{scbmatrix}{scbmatrix}.
Example:
\begin{Verbatim}
using namespace cvm;

double a[] = {1., 2., 3., 4., 5., 6., 7., 8., 9.,
              10., 11., 12., 13., 14., 15., 16.};
scbmatrix m((std::complex<double>*)a,4,1,0);
scbmatrix mc(4,0,1);
std::cout << m << std::endl << !m << std::endl ;
mc.transpose(m);
std::cout << mc << std::endl;
mc.transpose();
std::cout << mc;
\end{Verbatim}
prints
\begin{Verbatim}
(1,2) (0,0) (0,0) (0,0)
(3,4) (5,6) (0,0) (0,0)
(0,0) (7,8) (9,10) (0,0)
(0,0) (0,0) (11,12) (13,14)

(1,2) (3,4) (0,0) (0,0)
(0,0) (5,6) (7,8) (0,0)
(0,0) (0,0) (9,10) (11,12)
(0,0) (0,0) (0,0) (13,14)

(1,2) (3,4) (0,0) (0,0)
(0,0) (5,6) (7,8) (0,0)
(0,0) (0,0) (9,10) (11,12)
(0,0) (0,0) (0,0) (13,14)

(1,2) (0,0) (0,0) (0,0)
(3,4) (5,6) (0,0) (0,0)
(0,0) (7,8) (9,10) (0,0)
(0,0) (0,0) (11,12) (13,14)
\end{Verbatim}
\newpage









\subsubsection{operator * (const cvector\&)}
Operator%
\pdfdest name {scbmatrix.operator * (const cvector&)} fit
\begin{verbatim}
cvector scbmatrix::operator * (const cvector& v) const
throw (cvmexception);
\end{verbatim}
creates an object of type \verb"cvector"
as a product of a calling band matrix and a vector \verb"v".
It throws  \GOT{cvmexception}{cvmexception}
if  number of columns of a calling matrix
differs from  size of a vector \verb"v".
Use \GOT{cvector::mult}{cvector.mult (const cmatrix&, const cvector&)}
in order to avoid new object creation.
See also
\GOT{scbmatrix}{scbmatrix} and \GOT{cvector}{cvector}.
Example:
\begin{Verbatim}
using namespace cvm;

try {
    scbmatrix m (4,1,0);
    cvector v(4);
    m.set(std::complex<double>(1.,1.));
    v.set(std::complex<double>(1.,1.));

    std::cout << m * v;
}
catch (std::exception& e) {
    std::cout << "Exception: " << e.what () << std::endl;
}
\end{Verbatim}
prints
\begin{Verbatim}
(0,2) (0,4) (0,4) (0,4)
\end{Verbatim}
\newpage



\subsubsection{operator * (const cmatrix\&)}
Operator%
\pdfdest name {scbmatrix.operator * (const cmatrix&)} fit
\begin{verbatim}
cmatrix scbmatrix::operator * (const cmatrix& m) const
throw (cvmexception);
\end{verbatim}
creates an object of type \verb"cmatrix"
as a product of a calling band matrix and a matrix \verb"m".
It throws  \GOT{cvmexception}{cvmexception}
if  number of columns of a calling matrix
differs from  number of rows of a matrix \verb"m".
Use \GOT{cmatrix::mult}{cmatrix.mult} in order to avoid
 new object creation.
See also
\GOT{cmatrix}{cmatrix} and \GOT{scbmatrix}{scbmatrix}.
Example:
\begin{Verbatim}
using namespace cvm;

try {
    scbmatrix mb(4,1,0);
    cmatrix m(4,2);
    mb.set(std::complex<double>(1.,1.));
    m.set(std::complex<double>(1.,1.));

    std::cout << mb * m;
}
catch (std::exception& e) {
    std::cout << "Exception: " << e.what () << std::endl;
}
\end{Verbatim}
prints
\begin{Verbatim}
(0,2) (0,2)
(0,4) (0,4)
(0,4) (0,4)
(0,4) (0,4)
\end{Verbatim}
\newpage




\subsubsection{operator * (const scmatrix\&)}
Operator%
\pdfdest name {scbmatrix.operator * (const scmatrix&)} fit
\begin{verbatim}
scmatrix scbmatrix::operator * (const scmatrix& m) const
throw (cvmexception);
\end{verbatim}
creates an object of type \verb"scmatrix"
as a product of a calling band matrix and a matrix \verb"m".
It throws  \GOT{cvmexception}{cvmexception}
if the operands have different sizes.
Use \GOT{cmatrix::mult}{cmatrix.mult} in order to avoid
 new object creation.
See also
\GOT{scmatrix}{scmatrix} and \GOT{scbmatrix}{scbmatrix}.
Example:
\begin{Verbatim}
using namespace cvm;

try {
    scbmatrix mb(4,1,0);
    scmatrix m(4);
    mb.set(std::complex<double>(1.,1.));
    m.set(std::complex<double>(1.,1.));

    std::cout << mb * m;
}
catch (std::exception& e) {
    std::cout << "Exception: " << e.what () << std::endl;
}
\end{Verbatim}
prints
\begin{Verbatim}
(0,2) (0,2) (0,2) (0,2)
(0,4) (0,4) (0,4) (0,4)
(0,4) (0,4) (0,4) (0,4)
(0,4) (0,4) (0,4) (0,4)
\end{Verbatim}
\newpage



\subsubsection{operator * (const scbmatrix\&)}
Operator%
\pdfdest name {scbmatrix.operator * (const scbmatrix&)} fit
\begin{verbatim}
scbmatrix scbmatrix::operator * (const scbmatrix& m) const
throw (cvmexception);
\end{verbatim}
creates an object of type \verb"scbmatrix"
as a product of a calling band matrix and  band matrix \verb"m".
It throws  \GOT{cvmexception}{cvmexception}
if the operands have different sizes.
Use \GOT{cmatrix::mult}{cmatrix.mult} in order to avoid
 new object creation.
See also
\GOT{scbmatrix}{scbmatrix}.
Example:
\begin{Verbatim}
using namespace cvm;

try {
    scbmatrix m1(5,1,0);
    scbmatrix m2(5,1,1);
    m1.set(std::complex<double>(1.,1.));
    m2.set(std::complex<double>(1.,1.));

    std::cout << m1 * m2;
}
catch (std::exception& e) {
    std::cout << "Exception: " << e.what () << std::endl;
}
\end{Verbatim}
prints
\begin{Verbatim}
(0,2) (0,2) (0,0) (0,0) (0,0)
(0,4) (0,4) (0,2) (0,0) (0,0)
(0,2) (0,4) (0,4) (0,2) (0,0)
(0,0) (0,2) (0,4) (0,4) (0,2)
(0,0) (0,0) (0,2) (0,4) (0,4)
\end{Verbatim}
\newpage


\subsubsection{low\_up}
Functions%
\pdfdest name {scbmatrix.low_up} fit
\begin{verbatim}
scbmatrix& 
scbmatrix::low_up (const scbmatrix& m, int* nPivots) throw (cvmexception);
scbmatrix
scbmatrix::low_up (int* nPivots) const throw (cvmexception);
\end{verbatim}
compute  $LU$ factorization of a calling band matrix as
\begin{equation*}
A=PLU
\end{equation*}
where $P$ is  permutation matrix, $L$ is  lower
triangular matrix with unit diagonal
elements and $U$ is  upper triangular matrix.
All  functions store the result as  matrix $L$ without
main diagonal combined with $U$. All  functions
return pivot indices as  array of integers
(it should support at least \verb"msize()" elements)
pointed to by \verb"nPivots" so \hbox{$i$-th} row
was interchanged with \hbox{\verb"nPivots["$i$\verb"]"-th} row.
The first version sets  calling matrix to be equal to matrix
\verb"m"'s $LU$ factorization and the second one
creates an object of type \verb"scbmatrix" as calling matrix's
$LU$ factorization.
Functions throw \GOT{cvmexception}{cvmexception}
in case of inappropriate
sizes of the operands or when  matrix to be factorized is close to
singular. 
The first version also changes numbers of 
super-diagonals to be equal to $k_l+k_u$
in order to keep  result of factorization.
It is recommended to use \GOT{iarray}{iarray}
for pivot values.
This function is provided mostly for solving multiple
systems of linear equations using 
\GOT{scmatrix::solve\_lu}{scmatrix.solvelu} function,
See also
\GOT{scbmatrix}{scbmatrix}.
Example:
\begin{Verbatim}
using namespace cvm;

std::cout.setf (std::ios::scientific | std::ios::left); 
std::cout.precision (2);
try {
    double a[] = {1., 2., 3., 4., 5., 6., 7., 8., 9.,
                  10., 11., 12.};
    scbmatrix ma((std::complex<double>*)a,3,1,0);
    scbmatrix mLU(3,1,0);
    cmatrix  mb1(3,2); cvector vb1(3);
    cmatrix  mb2(3,2); cvector vb2(3);
    cmatrix  mx1(3,2); cvector vx1(3);
    cmatrix  mx2(3,2); cvector vx2(3);
    iarray   nPivots(3);
    double   dErr = 0.;
    mb1.randomize_real(-1.,3.); mb1.randomize_imag(1.,5.);
    mb2.randomize_real(-2.,5.); mb2.randomize_imag(-3.,0.);
    vb1.randomize_real(-2.,4.); vb1.randomize_imag(-4.,1.);
    vb2.randomize_real(-3.,1.); vb2.randomize_imag(4.,5.);

    mLU.low_up(ma, nPivots);
    mx1 = ma.solve_lu (mLU, nPivots, mb1, dErr);
    std::cout << mx1 << dErr << std::endl << std::endl;
    mx2 = ma.solve_lu (mLU, nPivots, mb2);
    std::cout << mx2 << std::endl;;
    std::cout << ma * mx1 - mb1 << std::endl << ma * mx2 - mb2;

    vx1 = ma.solve_lu (mLU, nPivots, vb1, dErr);
    std::cout << vx1 << dErr << std::endl;
    vx2 = ma.solve_lu (mLU, nPivots, vb2);
    std::cout << vx2 << std::endl;;
    std::cout << ma * vx1 - vb1 << std::endl << ma * vx2 - vb2;
}
catch (std::exception& e) {
    std::cout << "Exception: " << e.what () << std::endl;
}
\end{Verbatim}
prints
\begin{Verbatim}
(1.20e+000,4.02e-002) (1.82e+000,1.23e+000)
(-6.55e-001,1.37e-001) (-6.41e-001,-8.72e-001)
(7.75e-001,4.70e-002) (5.35e-001,8.11e-001)
1.45e-015

(-4.52e-001,-2.68e-002) (-1.09e+000,2.01e-001)
(6.08e-001,-4.76e-001) (5.48e-001,-1.95e-001)
(-3.46e-001,1.57e-001) (-3.38e-001,-7.54e-002)

(0.00e+000,4.44e-016) (-2.22e-016,8.88e-016)
(-2.22e-016,2.22e-016) (0.00e+000,0.00e+000)
(-1.11e-016,0.00e+000) (-3.33e-016,-6.66e-016)

(0.00e+000,0.00e+000) (2.22e-016,2.22e-016)
(0.00e+000,0.00e+000) (4.44e-016,-2.22e-016)
(8.88e-016,5.55e-016) (0.00e+000,0.00e+000)
(-1.28e+000,-5.12e-001) (8.22e-001,1.59e-001) (-6.45e-001,-3.74e-001)
1.31e-015
(1.26e+000,1.50e+000) (-5.13e-001,-4.66e-001) (5.97e-001,7.01e-001)

(0.00e+000,8.88e-016) (-4.44e-016,4.44e-016) (-8.88e-016,0.00e+000)

(2.22e-016,-8.88e-016) (4.44e-016,-8.88e-016) (-2.22e-016,8.88e-016)
\end{Verbatim}
\newpage




\subsubsection{operator / (const cvector\&)}
Operator%
\pdfdest name {scbmatrix.operator / (cvector)} fit
\begin{verbatim}
cvector operator / (const cvector& vB) const throw (cvmexception);
\end{verbatim}
returns solution $x$ of linear equation
$A*x=b$ where calling matrix is square band matrix $A$
and a vector $b$ is passed in parameter \verb"vB".
This operator throws exception 
of type \GOT{cvmexception}{cvmexception}
in case of inappropriate sizes
of the objects or when  matrix $A$ is close to singular.
See also \GOT{cvector::solve}{cvector.solve}, 
\GOT{scmatrix::solve}{scmatrix.solve}, 
\GOT{cvector.operator~\%}{cvector.operator percent (scmatrix)}, 
\GOT{cvector}{cvector}, \GOT{scmatrix}{scmatrix}, \GOT{scbmatrix}{scbmatrix}.
Example:
\begin{Verbatim}
using namespace cvm;
std::cout.setf (std::ios::scientific | std::ios::showpos);
std::cout.precision (12);

try {
    scbmatrix ma(4,2,1);
    cvector  vb(4);
    cvector  vx(4);
    ma.randomize_real(-1.,1.);
    ma.randomize_imag(-1.,1.);
    vb.randomize_real(-2.,2.);
    vb.randomize_imag(-2.,2.);

    vx = ma / vb;

    std::cout << (ma * vx - vb).norm() << std::endl;
}
catch (std::exception& e) {
    std::cout << "Exception " << e.what () << std::endl;
}
\end{Verbatim}
prints
\begin{Verbatim}
+8.082545620881e-016
\end{Verbatim}
\newpage






\subsubsection{identity}
Function%
\pdfdest name {scbmatrix.identity} fit
\begin{verbatim}
scbmatrix& scbmatrix::identity();
\end{verbatim}
sets  calling band matrix to be equal to identity matrix
and returns a reference to
the matrix changed. Function doesn't change
numbers of sub- and super-diagonals.
See also \GOT{scbmatrix}{scbmatrix}.
Example:
\begin{Verbatim}
using namespace cvm;

srbmatrix m(4);
m.randomize(0.,1.);
std::cout << m << std::endl;
std::cout << m.identity();
\end{Verbatim}
prints
\begin{Verbatim}
(0.576128,1.42595) (0,0) (0,0) (0,0)
(0.956359,-0.919523) (0.869716,-0.704093) (0,0) (0,0)
(0,0) (0.0959807,0.0616779) (0.632618,1.1793) (0,0)
(0,0) (0,0) (0.532182,-0.870724) (0.338023,1.22892)

(1,0) (0,0) (0,0) (0,0)
(0,0) (1,0) (0,0) (0,0)
(0,0) (0,0) (1,0) (0,0)
(0,0) (0,0) (0,0) (1,0)
\end{Verbatim}
\newpage





\subsubsection{vanish}
Function%
\pdfdest name {scbmatrix.vanish} fit
\begin{verbatim}
scbmatrix& scbmatrix::vanish();
\end{verbatim}
sets every element of a calling band matrix to be equal to zero
and returns a reference to
the matrix changed. This function is faster
than
\GOT{scbmatrix::set(TR)}{scbmatrix.set}
with zero operand passed.
See also \GOT{scbmatrix}{scbmatrix}.
Example:
\begin{Verbatim}
using namespace cvm;

scbmatrix m(4,1,0);
m.randomize_real(0.,1.);
m.randomize_imag(-1.,2.);
std::cout << m << std::endl;
std::cout << m.vanish();
\end{Verbatim}
prints
\begin{Verbatim}
(0.584094,0.985931) (0,0) (0,0) (0,0)
(0.197546,0.0150761) (0.483413,-0.733848) (0,0) (0,0)
(0,0) (0.844356,1.97848) (0.814692,1.50194) (0,0)
(0,0) (0,0) (0.118931,-0.720756) (0.936796,-0.582232)

(0,0) (0,0) (0,0) (0,0)
(0,0) (0,0) (0,0) (0,0)
(0,0) (0,0) (0,0) (0,0)
(0,0) (0,0) (0,0) (0,0)
\end{Verbatim}
\newpage


\subsubsection{randomize\_real}
Function%
\pdfdest name {scbmatrix.randomizereal} fit
\begin{verbatim}
scbmatrix& scbmatrix::randomize_real (TR dFrom, TR dTo);
\end{verbatim}
fills  real part of a calling band matrix with 
pseudo-random numbers distributed between
\verb"dFrom" and \verb"dTo".
Function
returns a reference to the matrix changed.
See also
\GOT{scbmatrix}{scbmatrix}.
Example:
\begin{Verbatim}
using namespace cvm;

std::cout.setf (std::ios::scientific | std::ios::left); 
std::cout.precision (2);
scbmatrix m(3,0,1);
m.randomize_real(0.,3.);
std::cout << m;
\end{Verbatim}
prints
\begin{Verbatim}
(1.78e+000,0.00e+000) (1.17e+000,0.00e+000) (0.00e+000,0.00e+000)
(0.00e+000,0.00e+000) (1.09e-002,0.00e+000) (6.05e-001,0.00e+000)
(0.00e+000,0.00e+000) (0.00e+000,0.00e+000) (2.49e+000,0.00e+000)
\end{Verbatim}
\newpage


\subsubsection{randomize\_imag}
Function%
\pdfdest name {scbmatrix.randomizeimag} fit
\begin{verbatim}
scbmatrix& scbmatrix::randomize_imag (TR dFrom, TR dTo);
\end{verbatim}
fills  imaginary part of a calling band matrix with 
pseudo-random numbers distributed between
\verb"dFrom" and \verb"dTo".
Function
returns a reference to the matrix changed.
See also
\GOT{scbmatrix}{scbmatrix}.
Example:
\begin{Verbatim}
using namespace cvm;

std::cout.setf (std::ios::scientific | std::ios::left);
std::cout.precision (2);
scbmatrix m(3,0,1);
m.randomize_imag(0.,3.);
std::cout << m;
\end{Verbatim}
prints
\begin{Verbatim}
(0.00e+000,1.80e+000) (0.00e+000,1.68e-001) (0.00e+000,0.00e+000)
(0.00e+000,0.00e+000) (0.00e+000,1.05e+000) (0.00e+000,1.40e+000)
(0.00e+000,0.00e+000) (0.00e+000,0.00e+000) (0.00e+000,1.98e+000)
\end{Verbatim}
\newpage


