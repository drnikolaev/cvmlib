\bigskip
\noindent
\verb"template <class t_type>"\\
\verb"class SqMatrix : public virtual Matrix<t_type> {"\\
\verb"public:"\\
\verb"    SqMatrix& "\GOT{operator ++}{SqMatrix.operator ++}\verb" ();"\\
\verb"    SqMatrix& "\GOT{operator --}{SqMatrix.operator --}\verb" ();"\\
\verb"    SqMatrix& "\GOT{identity}{SqMatrix.identity}\verb" ();"\\
\verb"    SqMatrix& "\GOT{transpose}{SqMatrix.transpose}\verb" ();"\\
\verb"};"
\newpage




\subsubsection{operator ++}
Operator%
\pdfdest name {SqMatrix.operator ++} fit
\begin{verbatim}
SqMatrix<t_type>&
SqMatrix<t_type>::operator ++ ();
\end{verbatim}
adds to a calling object identity matrix of the same size
and returns a reference to
the object changed.
Operator is inherited in the classes
\GOT{srmatrix}{srmatrix}, \GOT{scmatrix}{scmatrix}.
See also \GOT{SqMatrix}{SqMatrix}.

Example of usage:
\begin{verbatim}
using namespace cvm;

treal re[] = {1., 2., 3., 4.};
treal im[] = {1., -1., 2., -2.};
srmatrix mr(2);
scmatrix mc(2);

mr = re;
mc.assign(re, im);

cout << mr << mc << endl;

++mr;
++mc;
cout << mr << mc;
\end{verbatim}
prints
\begin{verbatim}
1 3
2 4
(1,1) (3,2)
(2,-1) (4,-2)

2 3
2 5
(2,1) (3,2)
(2,-1) (5,-2)
\end{verbatim}
\newpage



\subsubsection{operator --}
Operator%
\pdfdest name {SqMatrix.operator --} fit
\begin{verbatim}
SqMatrix<t_type>&
SqMatrix<t_type>::operator -- ();
\end{verbatim}
subtracts from  calling object identity matrix of the same size
and returns a reference to
the object changed.
Operator is inherited in the classes
\GOT{srmatrix}{srmatrix}, \GOT{scmatrix}{scmatrix}.
See also \GOT{SqMatrix}{SqMatrix}.

Example of usage:
\begin{verbatim}
using namespace cvm;

treal re[] = {1., 2., 3., 4.};
treal im[] = {1., -1., 2., -2.};
srmatrix mr(2);
scmatrix mc(2);

mr = re;
mc.assign(re, im);

cout << mr << mc << endl;

--mr;
--mc;
cout << mr << mc;
\end{verbatim}
prints
\begin{verbatim}
1 3
2 4
(1,1) (3,2)
(2,-1) (4,-2)

0 3
2 3
(0,1) (3,2)
(2,-1) (3,-2)
\end{verbatim}
\newpage





\subsubsection{identity}
Function%
\pdfdest name {SqMatrix.identity} fit
\begin{verbatim}
SqMatrix<t_type>&
SqMatrix<t_type>::identuty ();
\end{verbatim}
sets  calling object to be equal to identity matrix of the same size
and returns a reference to
the object changed.
Function is inherited in the classes
\GOT{srmatrix}{srmatrix}, \GOT{scmatrix}{scmatrix}.
See also \GOT{SqMatrix}{SqMatrix}.

Example of usage:
\begin{verbatim}
using namespace cvm;

srmatrix mr(4);
scmatrix mc(3);

mr(1,2) = 1.;

mr.identity();
mc.identity();

cout << mr << mc;
\end{verbatim}
prints
\begin{verbatim}
1 0 0 0
0 1 0 0
0 0 1 0
0 0 0 1
(1,0) (0,0) (0,0)
(0,0) (1,0) (0,0)
(0,0) (0,0) (1,0)
\end{verbatim}
\newpage



\subsubsection{transpose}
Function%
\pdfdest name {SqMatrix.transpose} fit
\begin{verbatim}
SqMatrix<t_type>& SqMatrix<t_type>::transpose ();
\end{verbatim}
sets  calling object to be equal
to itself transposed
and returns a reference to
the object changed.
Function is inherited in the classes
\GOT{srmatrix}{srmatrix} and \GOT{scmatrix}{scmatrix}.
See also \GOT{SqMatrix}{SqMatrix}.

Example of usage:
\begin{verbatim}
using namespace cvm;

treal a[] = {1., 2., 3., 4., 5., 6., 7., 8., 9.};
srmatrix m(a, 3);

cout << m << endl;

m.transpose();
cout << m;
\end{verbatim}
prints
\begin{verbatim}
1 4 7
2 5 8
3 6 9

1 2 3
4 5 6
7 8 9
\end{verbatim}
\newpage

